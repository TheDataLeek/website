\documentclass[11pt]{article}\usepackage[]{graphicx}\usepackage[]{xcolor}
%% maxwidth is the original width if it is less than linewidth
%% otherwise use linewidth (to make sure the graphics do not exceed the margin)
\makeatletter
\def\maxwidth{ %
  \ifdim\Gin@nat@width>\linewidth
    \linewidth
  \else
    \Gin@nat@width
  \fi
}
\makeatother

\definecolor{fgcolor}{rgb}{0.345, 0.345, 0.345}
\newcommand{\hlnum}[1]{\textcolor[rgb]{0.686,0.059,0.569}{#1} }%
\newcommand{\hlstr}[1]{\textcolor[rgb]{0.192,0.494,0.8}{#1} }%
\newcommand{\hlcom}[1]{\textcolor[rgb]{0.678,0.584,0.686}{\textit{#1} } }%
\newcommand{\hlopt}[1]{\textcolor[rgb]{0,0,0}{#1} }%
\newcommand{\hlstd}[1]{\textcolor[rgb]{0.345,0.345,0.345}{#1} }%
\newcommand{\hlkwa}[1]{\textcolor[rgb]{0.161,0.373,0.58}{\textbf{#1} } }%
\newcommand{\hlkwb}[1]{\textcolor[rgb]{0.69,0.353,0.396}{#1} }%
\newcommand{\hlkwc}[1]{\textcolor[rgb]{0.333,0.667,0.333}{#1} }%
\newcommand{\hlkwd}[1]{\textcolor[rgb]{0.737,0.353,0.396}{\textbf{#1} } }%

\usepackage{framed}
\makeatletter
\newenvironment{kframe}{%
 \def\at@end@of@kframe{}%
 \ifinner\ifhmode%
  \def\at@end@of@kframe{\end{minipage} }%
  \begin{minipage}{\columnwidth}%
 \fi\fi%
 \def\FrameCommand##1{\hskip\@totalleftmargin \hskip-\fboxsep
 \colorbox{shadecolor}{##1}\hskip-\fboxsep
     % There is no \\@totalrightmargin, so:
     \hskip-\linewidth \hskip-\@totalleftmargin \hskip\columnwidth}%
 \MakeFramed {\advance\hsize-\width
   \@totalleftmargin\z@ \linewidth\hsize
   \@setminipage} }%
 {\par\unskip\endMakeFramed%
 \at@end@of@kframe}
\makeatother

\definecolor{shadecolor}{rgb}{.97, .97, .97}
\definecolor{messagecolor}{rgb}{0, 0, 0}
\definecolor{warningcolor}{rgb}{1, 0, 1}
\definecolor{errorcolor}{rgb}{1, 0, 0}
\newenvironment{knitrout}{}{} % an empty environment to be redefined in TeX

\usepackage{alltt}

%%%%%%%%%%%%%%%%%%%%%%%%%%%%%%%%%%%%%%%%%%%%%%%%%%%%%%%%%%%%%%%%%%%%%%%%%%%%%%%%
% LaTeX Imports
%%%%%%%%%%%%%%%%%%%%%%%%%%%%%%%%%%%%%%%%%%%%%%%%%%%%%%%%%%%%%%%%%%%%%%%%%%%%%%%%
\usepackage{amsfonts}                                                   % Math fonts
\usepackage{amsmath}                                                    % Math formatting
\usepackage{amssymb}                                                    % Math formatting
\usepackage{amsthm}                                                     % Math Theorems
\usepackage{arydshln}                                                   % Dashed hlines
\usepackage{attachfile}                                                 % AttachFiles
\usepackage{cancel}                                                     % Cancelled math
\usepackage{caption}                                                    % Figure captioning
\usepackage{color}                                                      % Nice Colors
\input{./lib/dragon.inp}                                                % Tikz dragon curve
\usepackage[ampersand]{easylist}                                        % Easy lists
\usepackage{fancyhdr}                                                   % Fancy Header
\usepackage[T1]{fontenc}                                                % Specific font-encoding
%\usepackage[margin=1in, marginparwidth=2cm, marginparsep=2cm]{geometry} % Margins
\usepackage{graphicx}                                                   % Include images
\usepackage{hyperref}                                                   % Referencing
\usepackage[none]{hyphenat}                                             % Don't allow hyphenation
\usepackage{lipsum}                                                     % Lorem Ipsum Dummy Text
\usepackage{listings}                                                   % Code display
\usepackage{marginnote}                                                 % Notes in the margin
\usepackage{microtype}                                                  % Niceness
\usepackage{lib/minted}                                                 % Code display
\usepackage{multirow}                                                   % Multirow tables
\usepackage{pdfpages}                                                   % Include pdfs
\usepackage{pgfplots}                                                   % Create Pictures
\usepackage{rotating}                                                   % Figure rotation
\usepackage{setspace}                                                   % Allow double spacing
\usepackage{subcaption}                                                 % Figure captioning
\usepackage{tikz}                                                       % Create Pictures
\usepackage{tocloft}                                                    % List of Equations
%%%%%%%%%%%%%%%%%%%%%%%%%%%%%%%%%%%%%%%%%%%%%%%%%%%%%%%%%%%%%%%%%%%%%%%%%%%%%%%%
% Package Setup
%%%%%%%%%%%%%%%%%%%%%%%%%%%%%%%%%%%%%%%%%%%%%%%%%%%%%%%%%%%%%%%%%%%%%%%%%%%%%%%%
\hypersetup{%                                                           % Setup linking
    colorlinks=true,
    linkcolor=black,
    citecolor=black,
    filecolor=black,
    urlcolor=black,
}
\RequirePackage[l2tabu, orthodox]{nag}                                  % Nag about bad syntax
\renewcommand*\thesection{\arabic{section} }                             % Reset numbering
\renewcommand{\theFancyVerbLine}{ {\arabic{FancyVerbLine} } }              % Needed for code display
\renewcommand{\footrulewidth}{0.4pt}                                    % Footer hline
\setcounter{secnumdepth}{3}                                             % Include subsubsections in numbering
\setcounter{tocdepth}{3}                                                % Include subsubsections in toc
%%%%%%%%%%%%%%%%%%%%%%%%%%%%%%%%%%%%%%%%%%%%%%%%%%%%%%%%%%%%%%%%%%%%%%%%%%%%%%%%
% Custom commands
%%%%%%%%%%%%%%%%%%%%%%%%%%%%%%%%%%%%%%%%%%%%%%%%%%%%%%%%%%%%%%%%%%%%%%%%%%%%%%%%
\newcommand{\nvec}[1]{\left\langle #1 \right\rangle}                    %  Easy to use vector
\newcommand{\ma}[0]{\mathbf{A} }                                         %  Easy to use vector
\newcommand{\mb}[0]{\mathbf{B} }                                         %  Easy to use vector
\newcommand{\abs}[1]{\left\lvert #1 \right\rvert}                       %  Easy to use abs
\newcommand{\pren}[1]{\left( #1 \right)}                                %  Big parens
\let\oldvec\vec
\renewcommand{\vec}[1]{\oldvec{\mathbf{#1} } }                            %  Vector Styling
\newtheorem{thm}{Theorem}                                               %  Define the theorem name
\newtheorem{definition}{Definition}                                     %  Define the definition name
\definecolor{bg}{rgb}{0.95,0.95,0.95}
\newcommand{\java}[4]{\vspace{10pt}\inputminted[firstline=#2,
                                 lastline=#3,
                                 firstnumber=#2,
                                 gobble=#4,
                                 frame=single,
                                 label=#1,
                                 bgcolor=bg,
                                 linenos]{java}{#1} }
\newcommand{\python}[4]{\vspace{10pt}\inputminted[firstline=#2,
                                 lastline=#3,
                                 firstnumber=#2,
                                 gobble=#4,
                                 frame=single,
                                 label=#1,
                                 bgcolor=bg,
                                 linenos]{python}{#1} }
\newcommand{\js}[4]{\vspace{10pt}\inputminted[firstline=#2,
                                 lastline=#3,
                                 firstnumber=#2,
                                 gobble=#4,
                                 frame=single,
                                 label=#1,
                                 bgcolor=bg,
                                 linenos]{js}{#1} }
%%%%%%%%%%%%%%%%%%%%%%%%%%%%%%%%%%%%%%%%%%%%%%%%%%%%%%%%%%%%%%%%%%%%%%%%%%%%%%%%
% Beginning of document items - headers, title, toc, etc...
%%%%%%%%%%%%%%%%%%%%%%%%%%%%%%%%%%%%%%%%%%%%%%%%%%%%%%%%%%%%%%%%%%%%%%%%%%%%%%%%
\pagestyle{fancy}                                                       %  Establishes that the headers will be defined
\fancyhead[LE,LO]{Computer Systems Notes}                                  %  Adds header to left
\fancyhead[RE,RO]{Zoe Farmer}                                       %  Adds header to right
\cfoot{ \thepage }
\lfoot{CSCI 2400}
\rfoot{Han}
\title{Computer Systems Notes}
\author{Zoe Farmer}

%%%%%%%%%%%%%%%%%%%%%%%%%%%%%%%%%%%%%%%%%%%%%%%%%%%%%%%%%%%%%%%%%%%%%%%%%%%%%%%%
% Beginning of document items - headers, title, toc, etc...
%%%%%%%%%%%%%%%%%%%%%%%%%%%%%%%%%%%%%%%%%%%%%%%%%%%%%%%%%%%%%%%%%%%%%%%%%%%%%%%%
\pagestyle{fancy}                                                       %  Establishes that the headers will be defined
\fancyhead[LE,LO]{Homework 14}                                  %  Adds header to left
\fancyhead[RE,RO]{Zoe Farmer}                                       %  Adds header to right
\cfoot{ \thepage }
\lfoot{APPM 3570}
\rfoot{Kleiber}
\title{Homework 14}
\date{Kleiber}
\author{Zoe Farmer - 101446930}
%%%%%%%%%%%%%%%%%%%%%%%%%%%%%%%%%%%%%%%%%%%%%%%%%%%%%%%%%%%%%%%%%%%%%%%%%%%%%%%%
% Beginning of document items - headers, title, toc, etc...
%%%%%%%%%%%%%%%%%%%%%%%%%%%%%%%%%%%%%%%%%%%%%%%%%%%%%%%%%%%%%%%%%%%%%%%%%%%%%%%%
\IfFileExists{upquote.sty}{\usepackage{upquote} }{}
\begin{document}



\maketitle

\begin{table}[H]
    \centering
    \scalebox{1.5}{%
    \begin{tabular}{|l|l|l|l||l|}
        \hline
        1 & 2 & 3 & 4 & T\\
        \hline
        & & & &\\
        \hline
    \end{tabular}
    }
\end{table}

\begin{easylist}[enumerate]
    \ListProperties(Hide1=50)
    @ \textbf{Chapter 7, \#6} A fair die is rolled 10 times. Calculate the expected sum of the 10 rolls.\newline

    Assuming that we're using a six-sided die, the expectation of the sum is the same as the sum of the expectation. A
    die has uniform distribution.

    \begin{equation*}
        \begin{aligned}
            E\left[ \sum X_i \right] &=& \sum E\left[ X_i \right]\\
            &=& \sum_{i=1}^{10} E[X_i]\\
            &=& \sum_{i=1}^{10} \left( \sum_{j=1}^6 j \cdot \frac{1}{6} \right)\\
            &=& \sum_{i=1}^{10} 3.5\\
            &=& 35
        \end{aligned}
    \end{equation*}

    @ \textbf{Chapter 7, \#16} Let $Z$ be a standard normal random variable, and, for a fixed $x$, set

    \begin{equation*}
        \begin{aligned}
            X = \begin{cases}
                Z \quad if Z > x\\
                0 \quad Otherwise
            \end{cases}
        \end{aligned}
    \end{equation*}

    Show that $E[X]=\exp\left( -x^2/2 \right)/\sqrt{2 \pi}$.\newline

    First we define $Z$ to be the standard normal variable.

    \begin{equation*}
        \begin{aligned}
            Z = f(x, \mu \to 1, \sigma \to 1) = \frac{1}{\sqrt{2\pi} } e^{ -\frac{x^2}{2} }
        \end{aligned}
    \end{equation*}

    Using the definition of expected value we see that the expected value of $X$ is equal to the probability of the
    outcome times the outcome itself. Therefore our expected value is simply equal to $Z$, which we've already defined
    as the standard normal distribution.

    @ \textbf{Chapter 7, \#21} For a group of 100 people, compute

    @@ the expected number of days of the year that are birthdays of exactly 3 people:\newline

    Assuming that each individual's birthday is independent of every other, and that the probability of a birthday
    falling on any day is uniformly distributed, we can define a new random variable $X$ to be the number of days of the
    year that there are birthdays of exactly 3 people.\newline

    The probability that there exactly three birthdays on any given day is

    \begin{equation*}
        \begin{aligned}
            \frac{1}{365^3} = \frac{1}{\ensuremath{4.8627\times 10^{7} } }
        \end{aligned}
    \end{equation*}

    We can now define $X$ to be a binomial distribution with $p=\ensuremath{2.0565\times 10^{-8} }$ and $n=365$, and expected value of any
    given day equal to

    \begin{equation*}
        \begin{aligned}
            {100 \choose 3} \left( \frac{1}{365^3} \right) {\left( \frac{364}{365} \right)}^{97}
        \end{aligned}
    \end{equation*}

    With yearly expected value equal to

    \begin{equation*}
        \begin{aligned}
            E[X] &=& \sum_{i=1}^{365} {100 \choose 3} \left( \frac{1}{365^3} \right) {\left( \frac{364}{365} \right)}^{97}\\
            &=& 0.9301
        \end{aligned}
    \end{equation*}

    @@ the expected number of distinct birthdays.\newline

    This can also be phrased as the number of days that a birthday occurs. We can find the expected value of $X_i$ if
    $X_i$ is a Bernoulli random variable equalling 1 if an individual has a birthday on day $i$.

    \begin{equation*}
        \begin{aligned}
            E[X_i] = 1 - {\left( \frac{364}{265} \right)}^{100}
        \end{aligned}
    \end{equation*}

    Therefore the expected value equals

    \begin{equation*}
        \begin{aligned}
            E[X] &=& \sum_{i=1}^{365} \left( 1 - {\left( \frac{364}{265} \right)}^{100}\right)\\
            &=& 87.5755
        \end{aligned}
    \end{equation*}

    @ \textbf{Chapter 7, \#31} In Problem 6, calculate the variance of the sum of the rolls.\newline

    This simply follows the formula

    \begin{equation*}
        \begin{aligned}
            \Var\left( \sum_{i=1}^N X_i \right) = \sum_{i=1}^N \Var(X_i) + \sum_{i\neq j} \Cov(X_i, X_j)
        \end{aligned}
    \end{equation*}

    We know the variance of $X_i$ is $E[X^2] - \left( E[X] \right)^2 = 2.9167$. We also know that the covariance between any two $X_i$ will equal zero, which lets us establish a new
    equation

    \begin{equation*}
        \begin{aligned}
            \Var\left( \sum_{i=1}^N X_i \right) &=& N \cdot \Var(X_i) + \cancel{\sum_{i\neq j} \Cov(X_i, X_j)}\\
            &=& N \cdot \Var(X_i)\\
            &=& 10 \cdot 2.9167\\
            &=& 29.1667\\
        \end{aligned}
    \end{equation*}

    @ \textbf{Chapter 7, \#36} Let $X$ be the number of ones and $Y$ the number of twos that occur in $n$ rolls of a
    fair die. Compute $\Cov(X, Y)$.\newline

    Covariance is defined as

    \begin{equation*}
        \begin{aligned}
            \Cov(X, Y) = E[XY] - E[X] E[Y]
        \end{aligned}
    \end{equation*}

    We can define $X$ and $Y$ to be Bernoulli random variables where $X_i=1$ if the $i$th roll is 1, and $Y_i=1$ if the
    $i$th roll equals 2. Since these random variables deal with the same die, they are not independent. If $X_i=1$, then
    $Y_i=0$ and vice versa, which indicates that $XY=0$, since if $X$ or $Y$ is 1, then the other is zero. Since $X$ and
    $Y$ follow a binomial distribution, we know the expected value for each is the probability of success times the
    number of trials, in this case $1/6$ and $n$ respectively.

    \begin{equation*}
        \begin{aligned}
            \Cov(X, Y) &=& -E[X]E[Y]\\
            &=& -\frac{n^2}{36}
        \end{aligned}
    \end{equation*}


    @ \textbf{Chapter 7, \#37} A die is rolled twice. Let $X$ equal the sum of the outcomes, and let $Y$ equal the first
    outcome minus the second. Compute $\Cov(X, Y)$.\newline

    Since we roll the die twice, each outcome is independent of the other. Let $A$ and $B$ be the first and second roll,
    respectively. This let's us establish $X=A+B$ and $Y=A-B$. Using our previous definition of Covariance we need the
    expected value of $X, Y,$ and $XY$. Again, since each outcome is independent, $A$ and $B$ are independent, meaning
    $E[A+B] = E[A] + E[B] = 7$, $E[A-B] = E[A] - E[B] = 0$, and $E[XY] = E[X]E[Y] = 0$.

    \begin{equation*}
        \begin{aligned}
            \Cov(X, Y) &=& E[XY] - E[X]E[Y]\\
            &=& 0
        \end{aligned}
    \end{equation*}

    @ \textbf{Chapter 7, \#38} The random variables $X$ and $Y$ have a joint density function given by

    \begin{equation*}
        \begin{aligned}
            f(x, y) =
            \begin{cases}
                \frac{2 e^{-2x} }{x} &\quad 0 \le x < \infty, 0 \le y \le x\\
                0 &\quad Otherwise
            \end{cases}
        \end{aligned}
    \end{equation*}

    Compute $\Cov(X, Y)$.\newline

    Since our joint distribution does not have $Y$, this means that as $Y$ changes, $X$ remains the same, and
    vice-versa. This lack of change in the other variable indicates a complete lack of covariance between the two,
    therefore $\Cov(X, Y) = 0$.

\end{easylist}

\end{document}
