\documentclass[10pt]{article}

%%%%%%%%%%%%%%%%%%%%%%%%%%%%%%%%%%%%%%%%%%%%%%%%%%%%%%%%%%%%%%%%%%%%%%%%%%%%%%%%
% LaTeX Imports
%%%%%%%%%%%%%%%%%%%%%%%%%%%%%%%%%%%%%%%%%%%%%%%%%%%%%%%%%%%%%%%%%%%%%%%%%%%%%%%%
\usepackage{amsfonts}                                                   % Math fonts
\usepackage{amsmath}                                                    % Math formatting
\usepackage{amssymb}                                                    % Math formatting
\usepackage{amsthm}                                                     % Math Theorems
\usepackage{arydshln}                                                   % Dashed hlines
\usepackage{attachfile}                                                 % AttachFiles
\usepackage{cancel}                                                     % Cancelled math
\usepackage{caption}                                                    % Figure captioning
\usepackage{color}                                                      % Nice Colors
\input{./lib/dragon.inp}                                                % Tikz dragon curve
\usepackage[ampersand]{easylist}                                        % Easy lists
\usepackage{fancyhdr}                                                   % Fancy Header
\usepackage[T1]{fontenc}                                                % Specific font-encoding
%\usepackage[margin=1in, marginparwidth=2cm, marginparsep=2cm]{geometry} % Margins
\usepackage{graphicx}                                                   % Include images
\usepackage{hyperref}                                                   % Referencing
\usepackage[none]{hyphenat}                                             % Don't allow hyphenation
\usepackage{lipsum}                                                     % Lorem Ipsum Dummy Text
\usepackage{listings}                                                   % Code display
\usepackage{marginnote}                                                 % Notes in the margin
\usepackage{microtype}                                                  % Niceness
\usepackage{lib/minted}                                                 % Code display
\usepackage{multirow}                                                   % Multirow tables
\usepackage{pdfpages}                                                   % Include pdfs
\usepackage{pgfplots}                                                   % Create Pictures
\usepackage{rotating}                                                   % Figure rotation
\usepackage{setspace}                                                   % Allow double spacing
\usepackage{subcaption}                                                 % Figure captioning
\usepackage{tikz}                                                       % Create Pictures
\usepackage{tocloft}                                                    % List of Equations
%%%%%%%%%%%%%%%%%%%%%%%%%%%%%%%%%%%%%%%%%%%%%%%%%%%%%%%%%%%%%%%%%%%%%%%%%%%%%%%%
% Package Setup
%%%%%%%%%%%%%%%%%%%%%%%%%%%%%%%%%%%%%%%%%%%%%%%%%%%%%%%%%%%%%%%%%%%%%%%%%%%%%%%%
\hypersetup{%                                                           % Setup linking
    colorlinks=true,
    linkcolor=black,
    citecolor=black,
    filecolor=black,
    urlcolor=black,
}
\RequirePackage[l2tabu, orthodox]{nag}                                  % Nag about bad syntax
\renewcommand*\thesection{\arabic{section} }                             % Reset numbering
\renewcommand{\theFancyVerbLine}{ {\arabic{FancyVerbLine} } }              % Needed for code display
\renewcommand{\footrulewidth}{0.4pt}                                    % Footer hline
\setcounter{secnumdepth}{3}                                             % Include subsubsections in numbering
\setcounter{tocdepth}{3}                                                % Include subsubsections in toc
%%%%%%%%%%%%%%%%%%%%%%%%%%%%%%%%%%%%%%%%%%%%%%%%%%%%%%%%%%%%%%%%%%%%%%%%%%%%%%%%
% Custom commands
%%%%%%%%%%%%%%%%%%%%%%%%%%%%%%%%%%%%%%%%%%%%%%%%%%%%%%%%%%%%%%%%%%%%%%%%%%%%%%%%
\newcommand{\nvec}[1]{\left\langle #1 \right\rangle}                    %  Easy to use vector
\newcommand{\ma}[0]{\mathbf{A} }                                         %  Easy to use vector
\newcommand{\mb}[0]{\mathbf{B} }                                         %  Easy to use vector
\newcommand{\abs}[1]{\left\lvert #1 \right\rvert}                       %  Easy to use abs
\newcommand{\pren}[1]{\left( #1 \right)}                                %  Big parens
\let\oldvec\vec
\renewcommand{\vec}[1]{\oldvec{\mathbf{#1} } }                            %  Vector Styling
\newtheorem{thm}{Theorem}                                               %  Define the theorem name
\newtheorem{definition}{Definition}                                     %  Define the definition name
\definecolor{bg}{rgb}{0.95,0.95,0.95}
\newcommand{\java}[4]{\vspace{10pt}\inputminted[firstline=#2,
                                 lastline=#3,
                                 firstnumber=#2,
                                 gobble=#4,
                                 frame=single,
                                 label=#1,
                                 bgcolor=bg,
                                 linenos]{java}{#1} }
\newcommand{\python}[4]{\vspace{10pt}\inputminted[firstline=#2,
                                 lastline=#3,
                                 firstnumber=#2,
                                 gobble=#4,
                                 frame=single,
                                 label=#1,
                                 bgcolor=bg,
                                 linenos]{python}{#1} }
\newcommand{\js}[4]{\vspace{10pt}\inputminted[firstline=#2,
                                 lastline=#3,
                                 firstnumber=#2,
                                 gobble=#4,
                                 frame=single,
                                 label=#1,
                                 bgcolor=bg,
                                 linenos]{js}{#1} }
%%%%%%%%%%%%%%%%%%%%%%%%%%%%%%%%%%%%%%%%%%%%%%%%%%%%%%%%%%%%%%%%%%%%%%%%%%%%%%%%
% Beginning of document items - headers, title, toc, etc...
%%%%%%%%%%%%%%%%%%%%%%%%%%%%%%%%%%%%%%%%%%%%%%%%%%%%%%%%%%%%%%%%%%%%%%%%%%%%%%%%
\pagestyle{fancy}                                                       %  Establishes that the headers will be defined
\fancyhead[LE,LO]{Computer Systems Notes}                                  %  Adds header to left
\fancyhead[RE,RO]{Zoe Farmer}                                       %  Adds header to right
\cfoot{ \thepage }
\lfoot{CSCI 2400}
\rfoot{Han}
\title{Computer Systems Notes}
\author{Zoe Farmer}

%%%%%%%%%%%%%%%%%%%%%%%%%%%%%%%%%%%%%%%%%%%%%%%%%%%%%%%%%%%%%%%%%%%%%%%%%%%%%%%%
% Beginning of document items - headers, title, toc, etc...
%%%%%%%%%%%%%%%%%%%%%%%%%%%%%%%%%%%%%%%%%%%%%%%%%%%%%%%%%%%%%%%%%%%%%%%%%%%%%%%%
\pagestyle{fancy}                                                 %  Establishes that the headers will be defined
\fancyhead[LE,LO]{Problem Set 4}                                  %  Adds header to left
\fancyhead[RE,RO]{Zoe Farmer, Jeremy Granger, Ryan Roden}     %  Adds header to right
\cfoot{\mlptikz[size=0.25in, text=on, textposx=0, textposy=0, textvalue=\thepage, textscale=0.75in]{applejack} }
\lfoot{CSCI 3104}
\rfoot{Clauset}
\title{Problem Set Four}
\author{Zoe Farmer\\Jeremy Granger\\Ryan Roden}
%%%%%%%%%%%%%%%%%%%%%%%%%%%%%%%%%%%%%%%%%%%%%%%%%%%%%%%%%%%%%%%%%%%%%%%%%%%%%%%%
% Beginning of document items - headers, title, toc, etc...
%%%%%%%%%%%%%%%%%%%%%%%%%%%%%%%%%%%%%%%%%%%%%%%%%%%%%%%%%%%%%%%%%%%%%%%%%%%%%%%%
\begin{document}

\maketitle

\begin{easylist}[enumerate]
    @ Recall that Huffman codes are constructed in a greedy fashion.
    @@ What is an optimal Huffman code for the following set of frequencies, based on the first 8 Fibonacci numbers?

        \[
            \left[
            \text{
            \ttfamily
            a:1,
            b:1,
            c:2,
            d:3,
            e:5,
            f:8,
            g:13,
            h:21}
            \right]
        \]

    @@@ 
        \begin{figure}[!ht]
            \centering
            \scalebox{0.8}{%
            \begin{tikzpicture}
            \node[circle,draw](A){\ttfamily abcdefgh:54}
              child{%
                node[circle,draw](C)[below left=0.3cm and 0.5cm of A]{\ttfamily abcdefg:33}
                  child{%
                    node[circle,draw](E)[below left=0.3cm and 0.5cm of C]{\ttfamily abcdef:20}
                      child{%
                        node[circle,draw](G)[below left=0.3cm and 0.5cm of E]{\ttfamily abcde:12}
                          child{%
                            node[circle,draw](I)[below left=0.3cm and 0.5cm of G]{\ttfamily abcd:7}
                              child{%
                                node[circle,draw](K)[below left=0.3cm and 0.5cm of I]{\ttfamily abc:4}
                                  child{%
                                    node[circle,draw](M)[below left=0.3cm and 0.5cm of K]{\ttfamily ab:2}
                                      child{%
                                        node[circle,draw,label={[label distance=0.1cm]-90:{\ttfamily 1111111} }](O)[below left=0.8cm and 0.8cm of M]{\ttfamily a:1}
                                          child[missing]
                                          child[missing]
                                      }
                                      child{%
                                        node[circle,draw,label={[label distance=0.1cm]-90:{\ttfamily 1111110} }](N)[below=0.5cm of M]{\ttfamily b:1}
                                          child[missing]
                                          child[missing]
                                      }
                                  }
                                  child{%
                                    node[circle,draw,label={[label distance=0.1cm]-90:{\ttfamily 111110} }](L)[right=0.55cm of N]{\ttfamily c:2}
                                      child[missing]
                                      child[missing]
                                  }
                              }
                              child{%
                                node[circle,draw,label={[label distance=0.1cm]-90:{\ttfamily 11110} }](J)[right=0.65cm of L]{\ttfamily d:3}
                                  child[missing]
                                  child[missing]
                              }
                          }
                          child{%
                            node[circle,draw,label={[label distance=0.1cm]-90:{\ttfamily 1110} }](H)[right=0.75cm of J]{\ttfamily e:5}
                              child[missing]
                              child[missing]
                          }
                      }
                      child{%
                        node[circle,draw,label={[label distance=0.1cm]-90:{\ttfamily 110} }](F)[right=0.85cm of H]{\ttfamily f:8}
                          child[missing]
                          child[missing]
                      }
                  }
                  child{%
                    node[circle,draw,label={[label distance=0.1cm]-90:{\ttfamily 10} }](D)[right=0.95cm of F]{\ttfamily g:13}
                    child[missing]
                    child[missing]
                  }
                }
              child{%
                node[circle,draw,label={[label distance=0.1cm]-90:{\ttfamily 0} }](B)[right=1.05cm of D]{\ttfamily h:21}
                child[missing]
                child[missing]
              };
            \end{tikzpicture}
        }
        \end{figure}

    @@ How many optimal Huffman codes are there for this set of frequencies? Justify your answer.
    @@@ There are three different cases to produce a unique Huffman code. One, if the symbols {\ttfamily a} \&
    {\ttfamily b} are switched in the tree. Two, if the parent symbol {\ttfamily ab} and symbol {\ttfamily c} are
    switched in the tree. Or three, if the symmetry is flipped, i.e.\ the tree goes up and left instead of up and right,
    switching the positions of all 0's with 1's and vice versa.\newline

    Each case creates 2 different outcomes and none are dependent on any others, so the Huffman code outcomes are represented by:

        \[ \begin{aligned}
            1,2,3\\
            1,2,\neg 3\\
            1,\neg 2,3\\
            1,\neg 2,\neg 3\\
            \neg1,2,3\\
            \neg 1,2,\neg 3\\
            \neg 1,\neg 2, 3\\
            \neg 1,\neg 2,\neg 3\\
        \end{aligned} \]

    Therefore this set of symbols and frequencies has 8 different but equally optimal Huffman Codes

    @@ Generalize your answer to find an optimal code when the frequencies are the first $n$ Fibonacci numbers.
    @@@ We can generalize our answer as is shown in Table~\ref{table:codesymbols}

    \begin{table}[!ht]
        \centering
        \begin{tabular}{|l|l|}
             \hline
             \textit{Symbol} & \textit{Code}\\
             \hline
             $a$             & $1^{n-2} \cdot 0$\\
             $b$             & $1^{n-2} \cdot 1$ or $1^{n-1}$\\
             $c$             & $1^{n-3} \cdot 0$\\
             \vdots          & \vdots\\
             $nth$           & $1^{n-n} \cdot 0$ or $0$\\
             \hline
        \end{tabular}
        \caption{Generalize Fibonacci Huffman Codes}
        \label{table:codesymbols}
    \end{table}

    @ Professor Hagrid is struggling with the problem of making change for $n$ cents using the smallest number of coins.
    Let the coin values be $v_1 > v_2 > \cdots > v_r$ for $r$ coins types, and let each coin's value $v_i$ be a positive
    integer. The output will be a set of counts $\{d_i\}$, one for each coin type, such that $\sum^n_{i=1} d_i = n$ and
    where $k$ is minimized. (Note: $v_ 1$ is the most valuable coin.)
    @@ Give a greedy algorithm, that takes $O(n)$ time, to make change consisting of quarters (worth 25 cents), dimes
    (10 cents), nickels (5 cents) and pennies (1 cent). Prove that your algorithm yields an optimal solution.
    @@@ The strategy for this problem is defined as the following. At each step, choose the coin of largest denomination
    possible without exceeding the total.

        \begin{pythoncode*}{gobble=12, xleftmargin=1in}
            # Input: array with coin values (vals)
            #        change value (n)
            # Output: Array of coin counts
            def change(vals=[c1, c2, ..., cn], n):
                c = [0 for item in vals]   # Number of coins we have
                for i in range(0, len(c)): # Check each denomination
                    if n - vals[i] > 0: # If we can subtract
                        c[i] += 1  # Add to count
                        return [x + y for x,y in zip(c, # Recurse
                                change(n - vals[i], vals))] # combine
                    elif n - vals[i] == 0: # Base case when we get to 0
                        c[i] += 1 # Add to denomination count
                        return c # return count up the tree
        \end{pythoncode*}

        This algorithm runs in $O(n)$ time. The {\ttfamily for}-loop runs in constant time, which is the number of coins
        we have, {\ttfamily len(c)}. The function is then called as many times as it takes until $n$ goes to zero, which
        by nature is dependent on $n$. Therefore the total runtime can be found with the recurrence relation

        \[ T(n) = T(n - vals[i]) + O(1) \Rightarrow O(n) \]

        The reason this algorithm provides the optimal solution is demonstrated as follows. For any coin denomination,
        there is at least one way to equivalently equal that single coin with several smaller coins, but \textit{no}
        ways to equate a smaller coin with a larger one. To demonstrate,

            \[ \begin{aligned}
                25 \cents =& \underbrace{10 \cents + 10 \cents + 5 \cents}_{\text{3 Coins} }\\
                10 \cents =& \underbrace{5 \cents + 5 \cents}_{\text{2 Coins} }\\
                5 \cents  =& \underbrace{1 \cents + 1 \cents + 1 \cents + 1 \cents + 1 \cents}_{\text{5 Coins} }\\
            \end{aligned} \]

        Given the above statement, we can state several facts about our algorithm. One, any solution will have less than
        3 dimes. One quarter is better than 3 dimes, therefore we will always have 2 or less. Two, any solution will
        have less than 2 nickels. One dime is better than 2 nickels, therefore the most nickels we can have is 1. Three,
        any solution will have less than 5 pennies. One nickel is better than 5 pennies, therefore the most pennies we
        can have is 4. The facts hold true about any optimal solution.

    @@ Suppose that the available coins are in the denominations that are powers of $c$, i.e., denominations of $c^0,
    c^1, \ldots, c^l$ for some integers $c > 1$ and $l \ge 1$. Prove that the greedy algorithm always yields an optimal
    solution in this case.
    @@@ To start, we know that every value of $c$ has $c^0$ (the ``penny'') so every positive integer $n$ has a solution with
        the greedy algorithm.\newline

        Similarly to above every denomination can be created with some combination of lower denominations. An
        illustrative example is when $c=2$ and we'll let $l=3$.

            \[ \begin{aligned}
                c^3 = 8 \cents =& \underbrace{4 \cents + 4 \cents}_{\text{2 Coins} }\\
                c^2 = 4 \cents =& \underbrace{2 \cents + 2 \cents}_{\text{2 Coins} }\\
                c^1 = 2 \cents =& \underbrace{1 \cents + 1 \cents}_{\text{2 Coins} }\\
            \end{aligned} \]

        Using similar logic we can state several facts about our algorithm. Any solution will have at most one $4
        \cents$ coin, and at most one $2 \cents$ coin, because any two $4 \cents$ coins could be replaced with one $8
        \cents$ coin, and two $2 \cents$ coins could be replaced with one $4 \cents$ coin.
        the optimal solution in this case.\newline

        Using a more extensible approach, we refer to our algorithm above. If we take the largest $i^{th}$ denomination,
        such that $n - c^i \ge  0$ we now know that every coin $c^i$ where $i > 0$ (our base case) can be comprised of
        no fewer than $c$ coins of value $c^{(i-1)}$. Therefore one $c^i$ coin will always yield a smaller coin count
        than $c$ of $c^(i-1)$ for any value of $n$.


    @@ Give a set of coin denominations for which the greedy algorithm does not yield an optimal solution, and explain
    why. The set should include a penny so that there is a solution for every value of $n$.
    @@@ One such situation is when we simply exclude the nickel. For this set of coin denominations, we can look at when
    $n=30$, and our algorithm will yield

        \[ \begin{aligned}
            30 \cents &=& 25 \cents + 1 \cents + 1 \cents + 1 \cents + 1 \cents + 1 \cents &\Rightarrow \text{ Algorithm Solution}\\
            30 \cents &=& 10 \cents + 10 \cents + 10 \cents &\Rightarrow \text{ Optimal Solution}\\
        \end{aligned} \]

        In this case our algorithm cannot provide the optimal solution, and is stuck with an unoptimal solution. This
        occurs because the algorithm is stuck choosing what it believes to be the ``best'' option at each step of the
        process. In this case it starts with a quarter, and then the \textit{only} remaining denomination that can fit
        (since we've removed the nickel) is the penny, of which we need 5. Now if we had removed the quarter as well our
        algorithm would see that the dime would fit at the first, second, and third steps, yielding three dimes
        optimally instead of our unoptimal solution with a quarter.\newline

        To be fair, our algorithm is still pretty good. This case does not fail every scenario, and still works for
        many, including the case where $n=36$. This yields one quarter, one dime, and one penny; which is in other words
        the optimal solution.

    @ Let $A$ and $B$ be arrays of integers. Each array contains $n$ elements, and each array is in sorted order
    (ascending). $A$ and $B$ do not share any elements in common. Give a $O(\lg(n))$-time algorithm which finds the
    median of $A \cup B$ and prove that it is correct. This algorithm will thus find the median of the $2n$ elements
    that would result from putting $A$ and $B$ together into one array.
    @@ Let's start out by supposing that the (lower) median is in $X$. Let's call this median value $m$, and let's
    suppose that it is in $X[k]$. Then $k$ elements of $X$ are less than or equal to $m$ and $n - k$ elements of $X$ are
    greater than or equal to $m$. We know that in the two arrays combined there must be $n$ elements less than or equal
    to $m$ and $n$ elements greater than or equal to $m$. So there must be $n - k$ elements of $Y$ that are less than or
    equal to $m$ and $n - (n - k) = k$ elements of $Y$ that are greater than or equal to $n$.\newline

    Thus, we can check that $X[k]$ is the lower median by checking $Y[n - k] \le X[k] \le Y[n - k + 1]$. A boundary case
    occurs for $k = n$. Then $n - k = 0$ and there is no array entry $Y[0]$, so we only need to check $X[n] \le Y[1]$.
    Now if the median is in $X$, but is not in $X[k]$, then the above condition will not hold. If the median is in
    $X[k\prime]$ where $k\prime < k$, then $X[k]$ is above the median and $Y[n - k + 1] < X[k]$. Conversely, if the
    median is in $X[k\prime\prime]$ where $k\prime\prime > k$ then $X[k]$ is below the median and $X[k] < Y[n -
    k]$.\newline

    Thus we can use a binary search tree to determine whether there is an $X[k]$ such that either $k < n$ and $Y[n - k]
    \le X[k] \le Y[n - k + 1]$ or $k = n$ and $X[k] \le Y[n - k + 1]$. If we find such an $X[k]$, then it is the median.
    Otherwise we know that the median is in $Y$, and we use a binary search to find $Y[k]$ such that either $k < n$ and
    $X[n-k] \le Y[k] \le X[n - k + 1]$ or $k = n$ and $Y[k] \le X[n - k + 1]$. Such $Y[k]$ is the median.

    \begin{pythoncode*}{gobble=8}
        n = len(X) # $|X|=|Y|$
        def findk(X, Y, k):
            if k == n:  # If we've reached the end of the set
                if X[n] <= Y[1]: # If the end of X is less than the first of Y
                    return k  # We've found our median
            else:
                if ((Y[n - k] <= X[k]) and  # If we've correctly identified k
                    (X[k] <= Y[n - k + 1])):
                    return k
                else:
                    X1 = X[1:k - 1]  # Endpoints inclusive
                    X2 = X[k + 1: n] # Endpoints inclusive
                    if X1 != [] and X2 != []: # If $X_{1|2} \neq \emptyset$, search both
                        return findk(X1, Y, n / 4) or findk(X2, Y, (3 * n) / 4)
                    elif X1 != []: # Otherwise search the non-empty sets
                        return findk(X1, Y, n / 4)
                    elif X2 != []:
                        return findk(X2, Y, (3 * n) / 4)
                    else: # $X1=X2=\emptyset$
                        return None   # Otherwise if both are empty we failed.
        flag = False        # Did we guess correctly?
        k = n / 2           # Initial Guess
        k = findk(X, Y, k)  # Attempt to find $k$, the index of the lower median
        if k is None:
            flag = True     # Picked the wrong starting set
            k = findk(Y, X, n/2)  # Use the other one
        if flag:  # If we resorted to Y
            if k == n: # if lower median is at upper bound
                median = (Y[k] + X[1]) / 2 # take k and X[1]
            else: # Otherwise k and k + 1
                median = (Y[k] + Y[k + 1]) / 2
        else:  # If X worked
            if k == n: # As above, if at end take k and start of other
                median = (X[k] + Y[1]) / 2
            else: # Otherwise take k and k+1
                median = (X[k] + X[k + 1]) / 2
    \end{pythoncode*}

\end{easylist}

\end{document}
