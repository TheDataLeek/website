\documentclass[10pt]{article}

%%%%%%%%%%%%%%%%%%%%%%%%%%%%%%%%%%%%%%%%%%%%%%%%%%%%%%%%%%%%%%%%%%%%%%%%%%%%%%%%
% LaTeX Imports
%%%%%%%%%%%%%%%%%%%%%%%%%%%%%%%%%%%%%%%%%%%%%%%%%%%%%%%%%%%%%%%%%%%%%%%%%%%%%%%%
\usepackage{amsfonts}                                                   % Math fonts
\usepackage{amsmath}                                                    % Math formatting
\usepackage{amssymb}                                                    % Math formatting
\usepackage{amsthm}                                                     % Math Theorems
\usepackage{arydshln}                                                   % Dashed hlines
\usepackage{attachfile}                                                 % AttachFiles
\usepackage{cancel}                                                     % Cancelled math
\usepackage{caption}                                                    % Figure captioning
\usepackage{color}                                                      % Nice Colors
\input{./lib/dragon.inp}                                                % Tikz dragon curve
\usepackage[ampersand]{easylist}                                        % Easy lists
\usepackage{fancyhdr}                                                   % Fancy Header
\usepackage[T1]{fontenc}                                                % Specific font-encoding
%\usepackage[margin=1in, marginparwidth=2cm, marginparsep=2cm]{geometry} % Margins
\usepackage{graphicx}                                                   % Include images
\usepackage{hyperref}                                                   % Referencing
\usepackage[none]{hyphenat}                                             % Don't allow hyphenation
\usepackage{lipsum}                                                     % Lorem Ipsum Dummy Text
\usepackage{listings}                                                   % Code display
\usepackage{marginnote}                                                 % Notes in the margin
\usepackage{microtype}                                                  % Niceness
\usepackage{lib/minted}                                                 % Code display
\usepackage{multirow}                                                   % Multirow tables
\usepackage{pdfpages}                                                   % Include pdfs
\usepackage{pgfplots}                                                   % Create Pictures
\usepackage{rotating}                                                   % Figure rotation
\usepackage{setspace}                                                   % Allow double spacing
\usepackage{subcaption}                                                 % Figure captioning
\usepackage{tikz}                                                       % Create Pictures
\usepackage{tocloft}                                                    % List of Equations
%%%%%%%%%%%%%%%%%%%%%%%%%%%%%%%%%%%%%%%%%%%%%%%%%%%%%%%%%%%%%%%%%%%%%%%%%%%%%%%%
% Package Setup
%%%%%%%%%%%%%%%%%%%%%%%%%%%%%%%%%%%%%%%%%%%%%%%%%%%%%%%%%%%%%%%%%%%%%%%%%%%%%%%%
\hypersetup{%                                                           % Setup linking
    colorlinks=true,
    linkcolor=black,
    citecolor=black,
    filecolor=black,
    urlcolor=black,
}
\RequirePackage[l2tabu, orthodox]{nag}                                  % Nag about bad syntax
\renewcommand*\thesection{\arabic{section} }                             % Reset numbering
\renewcommand{\theFancyVerbLine}{ {\arabic{FancyVerbLine} } }              % Needed for code display
\renewcommand{\footrulewidth}{0.4pt}                                    % Footer hline
\setcounter{secnumdepth}{3}                                             % Include subsubsections in numbering
\setcounter{tocdepth}{3}                                                % Include subsubsections in toc
%%%%%%%%%%%%%%%%%%%%%%%%%%%%%%%%%%%%%%%%%%%%%%%%%%%%%%%%%%%%%%%%%%%%%%%%%%%%%%%%
% Custom commands
%%%%%%%%%%%%%%%%%%%%%%%%%%%%%%%%%%%%%%%%%%%%%%%%%%%%%%%%%%%%%%%%%%%%%%%%%%%%%%%%
\newcommand{\nvec}[1]{\left\langle #1 \right\rangle}                    %  Easy to use vector
\newcommand{\ma}[0]{\mathbf{A} }                                         %  Easy to use vector
\newcommand{\mb}[0]{\mathbf{B} }                                         %  Easy to use vector
\newcommand{\abs}[1]{\left\lvert #1 \right\rvert}                       %  Easy to use abs
\newcommand{\pren}[1]{\left( #1 \right)}                                %  Big parens
\let\oldvec\vec
\renewcommand{\vec}[1]{\oldvec{\mathbf{#1} } }                            %  Vector Styling
\newtheorem{thm}{Theorem}                                               %  Define the theorem name
\newtheorem{definition}{Definition}                                     %  Define the definition name
\definecolor{bg}{rgb}{0.95,0.95,0.95}
\newcommand{\java}[4]{\vspace{10pt}\inputminted[firstline=#2,
                                 lastline=#3,
                                 firstnumber=#2,
                                 gobble=#4,
                                 frame=single,
                                 label=#1,
                                 bgcolor=bg,
                                 linenos]{java}{#1} }
\newcommand{\python}[4]{\vspace{10pt}\inputminted[firstline=#2,
                                 lastline=#3,
                                 firstnumber=#2,
                                 gobble=#4,
                                 frame=single,
                                 label=#1,
                                 bgcolor=bg,
                                 linenos]{python}{#1} }
\newcommand{\js}[4]{\vspace{10pt}\inputminted[firstline=#2,
                                 lastline=#3,
                                 firstnumber=#2,
                                 gobble=#4,
                                 frame=single,
                                 label=#1,
                                 bgcolor=bg,
                                 linenos]{js}{#1} }
%%%%%%%%%%%%%%%%%%%%%%%%%%%%%%%%%%%%%%%%%%%%%%%%%%%%%%%%%%%%%%%%%%%%%%%%%%%%%%%%
% Beginning of document items - headers, title, toc, etc...
%%%%%%%%%%%%%%%%%%%%%%%%%%%%%%%%%%%%%%%%%%%%%%%%%%%%%%%%%%%%%%%%%%%%%%%%%%%%%%%%
\pagestyle{fancy}                                                       %  Establishes that the headers will be defined
\fancyhead[LE,LO]{Computer Systems Notes}                                  %  Adds header to left
\fancyhead[RE,RO]{Zoe Farmer}                                       %  Adds header to right
\cfoot{ \thepage }
\lfoot{CSCI 2400}
\rfoot{Han}
\title{Computer Systems Notes}
\author{Zoe Farmer}

%%%%%%%%%%%%%%%%%%%%%%%%%%%%%%%%%%%%%%%%%%%%%%%%%%%%%%%%%%%%%%%%%%%%%%%%%%%%%%%%
% Beginning of document items - headers, title, toc, etc...
%%%%%%%%%%%%%%%%%%%%%%%%%%%%%%%%%%%%%%%%%%%%%%%%%%%%%%%%%%%%%%%%%%%%%%%%%%%%%%%%
\pagestyle{fancy}                                                       %  Establishes that the headers will be defined
\fancyhead[LE,LO]{Fourier Series}                                  %  Adds header to left
\fancyhead[RE,RO]{Zoe Farmer}                                       %  Adds header to right
\cfoot{\thepage}
\lfoot{APPM 4350}
\rfoot{Mark Hoefer}
\title{Fourier Series Homework Six}
\author{Zoe Farmer}
%%%%%%%%%%%%%%%%%%%%%%%%%%%%%%%%%%%%%%%%%%%%%%%%%%%%%%%%%%%%%%%%%%%%%%%%%%%%%%%%
% Beginning of document items - headers, title, toc, etc...
%%%%%%%%%%%%%%%%%%%%%%%%%%%%%%%%%%%%%%%%%%%%%%%%%%%%%%%%%%%%%%%%%%%%%%%%%%%%%%%%
\begin{document}



\maketitle

\section{}

Let $v(x, t)$ denote the solution of the following problem.

\begin{align*}
    v_t &= k v_{xx} + \alpha v + Q(x, t) \qquad & 0 < x < L \qquad & t > 0\\
    v(0, t) &= f_0(t), v(L, t) = f_L(t) \qquad && t > 0\\
    v(x, 0) &= V(x) \qquad & 0 < x < L \qquad & t = 0\\
\end{align*}

Where $k > 0$, $L > 0$ and $\alpha$ are fixed constants $f_0(t)$ and $f_L(t)$ are known functions of $t$, $V(x)$ is a
known function of $x$, and $Q(x, t)$ is a known function of both variables. We seek to prove that this problem has at
most one solution, for any choice of the functions $\{ f_0(t), f_L(t), V(x), Q(x, t) \}$.\\

\NewList
\begin{easylist}[enumerate]
    @ By mimicking the proof given in the Outline of Method, in the Homework Set, show that this problem has at most one
    solution for a range of real values of $\alpha$. State clearly the range of values of $\alpha$ for which the problem
    has at most one solution. If there is a restriction on the space of functions for which the uniqueness proof
    applies, state that restriction clearly. Support your conclusions with the appropriate work.

    @@ \textit{Assume that Two Solutions Exist}

    As with the outline, we assume that the problem has two solutions, $v_1(x, t)$ and $v_2(x, t)$. We will now prove
    that they are the same using contradiction.

    @@ \textit{Define the Difference}

    We define the difference between the two solutions to be

    \[
        \delta(x, t) = v_1(x, t) - v_2(x, t)
    \]

    This equation has its own set of equations determined by subtracting the problem above from itself.

    \begin{align}\label{1:delta}
        \delta_t &= k \delta_{xx} + \alpha \delta \qquad & 0 < x < L \qquad & t > 0 \nonumber \\
        \delta(0, t) &= 0, \delta(L, t) = 0 \qquad && t > 0\\
        \delta(x, 0) &= 0 \qquad & 0 < x < L \qquad & t = 0 \nonumber
    \end{align}

    It is obvious by inspection that $\delta(x, t) \equiv 0$ satisfies~\eqref{1:delta}. We now must show that this is
    the only solution to~\eqref{1:delta}.

    @@ \textit{Multiply the PDE and Integrate}

    We will multiply equation~\eqref{1:delta} by $\delta(x, t)$ (following the pattern in the notes). This yields the
    following equation.

    \begin{align*}
        \delta \delta_t &= k \delta \delta_{xx} + \alpha \delta^2\\
        \pren{\frac{\delta^2}{2}}_t &= k \delta \delta_{xx} + \alpha \delta^2
    \end{align*}

    We can integrate over the interval $[0, L]$.

    \begin{align*}
        \mathcal{L}(t) &= \int_0^L \pren{\frac{\delta^2}{2}} \, dx & \text{Note, this is strictly} \ge 0\\
        &= \int_0^L k \delta \delta_{xx} \, dx + \int_0^L \alpha \delta^2 \, dx\\
        &= \bren{k \delta \delta_x}\bigg|_0^L - \int_0^L k \delta_x^2 \, dx + \int_0^L \alpha \delta^2 \, dx\\
    \end{align*}

    @@ \textit{Examine the Integration}

    We can determine by inspection using the boundary conditions that $\mathcal{L}(0) = 0$. We also can take the time
    derivative.

    \begin{align*}
        \frac{d\mathcal{L}}{dt} &= - \bren{\int_0^L k \delta_x^2 \, dx}_t + \bren{\int_0^L \alpha \delta^2 \, dx}_t
    \end{align*}

    For any $\alpha \le 0$, $\mathcal{L}_t(t)$ is less than zero. Therefore we can determine that for any $\alpha \le
    0$, $\mathcal{L} \equiv 0$ for all $t \ge 0$.

    @@ \textit{Examine Possible Removable Discontinuities}

    Almost done, but we need to examine the possibility of removable discontinuities. Shadowing the notes, assume
    $\delta(x, t)$ is continuous in $x \in (0, L)$, $t \ge 0$. Assume at some time $t^* \ge 0$, there exists $x^*, x \in
    (0, L)$, where $\delta^2(x^*, t^*) = p^2 > 0$, where $p$ is a fixed real number. Then because $\delta^2(x, t)$ is
    continuous in $x$, there must be an open interval $(x^* - \epsilon < x < x^* + \epsilon)$ in which $\delta^2(x, t^*)
    > p^2 / 2 > 0$. But this implies

    \ms{%
        \mathcal{L}(t^*) = \int_0^L \bren{\frac{\delta^2(x, t^*)}{2}} \, dx \ge
            \int_{x^* - \epsilon}^{x^* + \epsilon} \bren{\frac{\delta^2(x, t^*)}{2}} \, dx \ge
            \frac{p^2}{4} \cdot 2 \epsilon > 0.
    }

    Therefore $\mathcal{L}(t^*)$ does not satisfy our initial assumption. Therefore no such $(x^*, t^*)$ exists.

    @@ \textit{Conclusion}

    Since $\delta$ is zero for all $t$, there exists at most one solution that is continuous in $x \in (0, L)$, for $t >
    0$.

    @ For $\alpha$ in the range for which the problem has a unique solution, set $Q(x, t) = 0$, $f_0(t)=0$, $f_L(t)=0$,
    $V(x,0)=1$, and find $v(x, t)$ in the form of a Fourier Series.\\

    Assume $\alpha \le 0$, and

    \begin{align*}
        v_t &= k v_{xx} + \alpha v \qquad & 0 < x < L \qquad & t > 0\\
        v(0, t) &= 0, v(L, t) = 0 \qquad && t > 0\\
        v(x, 0) &= 1 \qquad & 0 < x < L \qquad & t = 0\\
    \end{align*}

    We can find the Fourier series solution.

    @@ Steady State

    Let $v_t = 0$,

    \ms{%
        0 &= k v_{xx} + \alpha v\\
        v(x, t) &= A \cosh\pren{x\sqrt{\frac{\alpha}{k}}}
    }

    Using the Boundary Conditions we find $A$ and $B$.

    \ms{%
        v(0) &= 0 = A\\
        v(L) &= 0
    }

    Therefore the steady state solution is 0.

    @@ Separation of Variables

    Now we can apply separation of variables. Assume $v(x, t) = g(x)f(t)$.

    \ms{%
        g(x)f^\prime(t) &= k\ddot{g}(x)f(t) + \alpha g(x) f(t)\\
        \frac{f^\prime(t)}{kf(t)} - \frac{\alpha}{k} &= \frac{\ddot{g}(x)}{g(x)} = \mu
    }

    Yielding two ODEs.

    \ms{%
        f^\prime(t) &= f(t) \pren{k\mu + \alpha}\\
        \ddot{g}(x) &= \mu g(x)
    }

    @@ Finding the first is easy,

    \ms{%
        f(t) = e^{t \pren{k\mu + \alpha}}
    }

    @@ Now we need to check the three cases:

    @@@ $\mu > 0$

    \ms{%
        g(x) = A \sinh\pren{x\sqrt{mu}} + B \cosh\pren{x\sqrt{mu}}
    }

    With Boundary Conditions, $A = B = 0$, therefore $\mu \not> 0$.

    @@@ $\mu = 0$

    \ms{%
        g(x) = Ax + B
    }

    With Boundary Conditions, $A = B = 0$, therefore $\mu \neq 0$.

    @@@ $\mu < 0$. Let $\mu = -\lambda^2$.

    This yields

    \ms{%
        g(x) = A \sin\pren{\lambda x} + B \cos\pren{\lambda x}
    }

    With Boundary conditions, it is trivial to see that $B = 0$. This yields

    \ms{%
        g(L) &= 0 = A \sin\pren{\lambda L}
    }

    Which is satisfied when

    \ms{%
        \lambda L = n \pi \Rightarrow \lambda_n = \frac{n \pi}{L}
    }

    Therefore the eigenfunctions and values are

    \ms{%
        \phi_n(x) &= \sin\pren{\lambda_n x}\\
        \lambda_n &= \frac{n \pi}{L}
    }

    @@ Full Solution

    Yielding full solution:

    \ms{%
        v(x, t) = \sum_{n=1}^\infty \bren{a_n \sin\pren{\frac{n\pi x}{L}} e^{t \pren{-k \pren{\frac{n\pi}{L}}^2 + \alpha}}}
    }

    @@ Find $a_n$

    We can find $a_n$ using the Fourier Series equations.

    \ms{%
        v(x, 0) = 1 &= \sum_{n=1}^\infty \bren{a_n \sin\pren{\frac{n\pi x}{L}}}\\
        a_n &= \frac{1}{2L} \int_0^L \sin\pren{\frac{n\pi x}{L}} \, dx = \frac{L}{n \pi} \pren{-\cos\pren{n\pi}+1}
    }

    @@ With final solution

    \ms{%
        v(x, t) = \sum_{n=1}^\infty \bren{\frac{L}{n \pi} \pren{-\cos\pren{n\pi}+1} \sin\pren{\frac{n\pi x}{L}} e^{t \pren{-k \pren{\frac{n\pi}{L}}^2 + \alpha}}}
    }

    @@ We can plot this

\simpleweave

\includegraphics[width= 6in]{/home/zoe/classwork/2015b/appm4350/homeworks/figures/hw6_figure1_1.pdf}

\nosimpleweave

    @ For $\alpha$ in the range in which the uniqueness proof using Lyapunov functionals fails, let $\delta(x, t)$
    denote the difference between two solutions, and steady state. This steady-state problem has non-zero solutions for
    several values of $\alpha$. Find the values of $\alpha$ for which this problem has a non-zero steady-state solution,
    and give the most general solution of the problem for each such value.

    @@ The system for which it fails is where $\alpha > 0$, shown below.

    \begin{align}\label{1:delta}
        \delta_t &= k \delta_{xx} + \alpha \delta \qquad & 0 < x < L \qquad & t > 0 \nonumber \\
        \delta(0, t) &= 0, \delta(L, t) = 0 \qquad && t > 0\\
        \delta(x, 0) &= 0 \qquad & 0 < x < L \qquad & t = 0 \nonumber
    \end{align}

    @@ Finding its steady state, we get

    \ms{%
        \delta_t(x, t) = 0 = k \delta_{xx}(x, t) + \alpha \delta(x, t)
    }

    Yielding solution

    \ms{%
        \delta(x, t) = A e^{-x \sqrt{- \frac{\alpha}{k}}} + B e^{x \sqrt{- \frac{\alpha}{k}}}
    }

    Using BCs we get

    \ms{%
        \delta(0, t) &= A + B = 0 \Rightarrow A = -B\\
        \delta(L, t) &= A e^{-L \sqrt{- \frac{\alpha}{k}}} - A e^{L \sqrt{- \frac{\alpha}{k}}} = 0\\
        &= A \pren{e^{-L \sqrt{- \frac{\alpha}{k}}} - e^{L \sqrt{- \frac{\alpha}{k}}}} = 0\\
        &= A \pren{-2 \sinh\pren{L \sqrt{- \frac{\alpha}{k}}}} = 0\\
        &= A \pren{-2 \sinh\pren{L i \sqrt{\frac{\alpha}{k}}}} = 0\\
        &= A \pren{-2 i \sin\pren{L \sqrt{\frac{\alpha}{k}}}} = 0\\
    }

    With some simple manipulation we can see that the above expression holds whenever

    \ms{%
        \alpha = k \pren{\frac{n\pi}{L}}^2
    }

    Applying this result to the original system, we get

    \begin{align}\label{1:delta}
        \delta_t &= k \delta_{xx} + k \pren{\frac{n\pi}{L}}^2 \delta \qquad & 0 < x < L \qquad & t > 0 \nonumber \\
        \delta(0, t) &= 0, \delta(L, t) = 0 \qquad && t > 0\\
        \delta(x, 0) &= 0 \qquad & 0 < x < L \qquad & t = 0 \nonumber
    \end{align}

    Where fore both $n$ is some integer greater than or equal to zero.

    @ What is the relation between (1) and (3)?\\

    Whenever 

    \ms{%
        \alpha = k \pren{\frac{n\pi}{L}}^2
    }

    The original system has more than one solution. In other words, any solution that is found will not be unique.
\end{easylist}

\newpage
\section{}

Let $z(x, t)$ denote the solution of the following problem.

\begin{align*}
    z_t &= k z_{xx} + Q(x, t) \qquad & 0 < x < L \qquad & t > 0\\
    z(0, t) &= f_0(t), z(L, t) + \beta z_x(L, t) = f_L(t)\qquad && t > 0\\
    z(x, 0) &= Z(x) \qquad & 0 < x < L \qquad & t = 0\\
\end{align*}

where $k >0, L>0$, and $\beta$ are fixed constants, $f_0(t)$ and $f_L(t)$ are known functions of $t$, $Z(x)$ is a known
function of $x$, and $Q(x, t)$ is a known function of both variables. The boundary condition at $x=L$ models convection
of a fluid past the end of the rod using Newton's law of cooling. We seek to prove that this problem has at most one
solution, for any choice of the functions $\{f_0(t), f_L(t), Z(x), Q(x, t)\}$.

\begin{easylist}[enumerate]
    @ Show that this problem has at most one solution for a range of values of $\beta$. State clearly the range of
    values of $\beta$ for which the problem has at most one solution.

    @@ Assume that two solutions exist.

    @@ Define $\delta(x, t) = z_1(x, t) - z_2(x, t)$.

    @@ Establish the initial and boundary conditions for $\delta$:

    \ms{%
        \delta_t(x, t) &= k \delta_{xx}(x, t)\\
        \delta(0, t) &= 0\\
        \delta(L, t) + \beta \delta_x(L, t) &= 0\\
        \delta(x, 0) &= 0
    }

    @@ By inspection, $\delta(x, t) = 0$ satisfies the system. Is this solution unique?

    @@ Multiply the PDE by $\delta(x, t)$.

    \ms{%
        \delta(x, t) \delta_t(x, t) &= k \delta(x, t) \delta_{xx}(x, t)\\
        \pren{\frac{\delta(x, t)}{2}}_t &= k \delta(x, t) \delta_{xx}(x, t)\\
    }

    @@ Since the integration is on $x$, we can derive by $t$ later.

    \ms{%
        \mathcal{L}(t) = \int_0^L \pren{\frac{\delta(x, t)}{2}}_t \, dx
    }

    @@ Note, $\mathcal{L} \ge 0$ for all $t$.

    @@

    \ms{%
        \mathcal{L}(0) = \int_0^L \pren{\frac{\delta(x, 0)}{2}}_t \, dx = 0
    }

    @@ Find $\mathcal{L}_t(t)$

    \ms{%
        \mathcal{L}(t) &= \int_0^L \pren{\frac{\delta(x, t)}{2}}_t \, dx\\
        &= \pren{k \delta(x, t) \delta_x(x, t)}\bigg|_0^L - \int_0^L k {\delta_x(x, t)}^2 \, dx\\
        &= \pren{k \delta(L, t) \delta_x(L, t) - k \delta(0, t) \delta_x(0, t)} - \int_0^L k {\delta_x(x, t)}^2 \, dx\\
        &= \pren{k \delta(L, t) \delta_x(L, t)} - \int_0^L k {\delta_x(x, t)}^2 \, dx\\
        \mathcal{L}_t(t) &= \bren{k \delta(L, t) \delta_x(L, t)}_t - \bren{\int_0^L k {\delta_x(x, t)}^2 \, dx}_t\\
        &= \bren{-k \beta {\delta_x(x, t)}^2}_t - \bren{\int_0^L k {\delta_x(x, t)}^2 \, dx}_t \le 0
    }

    @@ We can use the same proof in the homework set to prove that there are no discontinuities present.

    @@ Since $\mathcal{L}(t) \ge 0$, $\mathcal{L}(0) = 0$, and $\mathcal{L}_t(t) \le 0$, this implies that
    $\mathcal{L}(t) \equiv 0$ for all $t \ge 0$, and all $\beta \ge 0$.

    @ For $Q=0, f_0(t)=0,f_L(t)=0$, find separated solutions in the form $z(x, t) = g(x)g(t)$. What is the time
    dependence for $h(t)$? For all $\beta$, what eigenvalues and eigenfunctions $g(x)$ can you obtain. Be careful and
    check every possible case. How do the eigenvalues you find and the associated time dependence $h(t)$ relate to the
    values of $\beta$ for which the energy method works or fails? These separated solutions can be used to solve the
    initial/boundary value problem using generalized Fourier series. Do not do this.

    @@ This new system becomes

    \begin{align*}
        z_t &= k z_{xx} \qquad & 0 < x < L \qquad & t > 0\\
        z(0, t) &= 0, z(L, t) + \beta z_x(L, t) = 0 \qquad && t > 0\\
        z(x, 0) &= Z(x) \qquad & 0 < x < L \qquad & t = 0\\
    \end{align*}

    @@ Assume $z(x, t) = g(x)h(t)$, yielding

    \ms{%
        z_t(x, t) = kz_{xx}(x, t)\\
        g(x)f^\prime(t) = k \ddot{g}(x) h(t)\\
        \frac{h^\prime(t)}{kh(t)} = \frac{\ddot{g}(x)}{g(x)} = \mu
    }

    @@ We can solve for $h(t)$.

    \ms{%
        h(t) = e^{\mu kt}
    }

    @@ Examining the boundary conditions we see that

    \ms{%
        g(0) = 0\\
        g(L) + \beta \dot{g}(L) = 0
    }

    @@ We now need to examine the three cases.

    @@@ $\mu > 0$

    \ms{%
        g(x) = A \sinh\pren{x \sqrt{\mu}} + B \cosh\pren{x \sqrt{\mu}}
    }

    With boundary conditions we clearly see that $B = 0$, and

    \ms{%
        A \sinh\pren{L \sqrt{\mu}} + \beta A \sqrt{\mu} \cosh\pren{L \sqrt{\mu}} = 0\\
    }

    The only value of $A$ for which this holds is $A = 0$, therefore there are no non-trivial solutions, and
    $\mu\not>0$.

    @@@ $\mu = 0$

    \ms{%
        g(x) = Ax + B\\
    }

    With boundary conditions it is apparent that again, $A = B = 0$, and $\mu\neq0$.

    @@@ $\mu < 0$. Let $\mu = -\lambda^2$.

    This yields

    \ms{%
        g(x) = A \sin\pren{\lambda x} + B \cos\pren{\lambda x}\\
        \dot{g}(x) = A \lambda \cos\pren{\lambda x} - B \lambda \sin\pren{\lambda x}\\
    }

    With boundary conditions,

    \ms{%
        g(0) = 0 = A \cdot 0 + B \cdot 1\\
    }

    Therefore $B = 0$.

    \ms{%
        \sin\pren{\lambda L} + \beta \lambda \cos\pren{\lambda L} = 0\\
        \sin\pren{\lambda L} = - \beta \lambda \cos\pren{\lambda L}\\
        \tan\pren{\lambda L} = - \beta \lambda
    }

\simpleweave

\includegraphics[width= 6in]{/home/zoe/classwork/2015b/appm4350/homeworks/figures/hw6_figure2_1.pdf}

\nosimpleweave

    I don't know how to solve this.

\end{easylist}

\newpage
\section{}

The wave equation derived in class for the motion of a stretched guitar string without damping is

\begin{align*}
    w_{tt} &= c^2 w_{xx} + Q(x, t) \qquad & 0 < x < L \qquad & t > 0\\
    w(0, t) &= f_0(t), w(L, t) = f_L(t) \qquad && t > 0\\
    w(x, 0) &= W(x), w_t(x, 0) = S(x) \qquad & 0 < x < L \qquad & t = 0\\
\end{align*}

Prove uniqueness.

    \begin{easylist}[enumerate]
        @ Find $\delta(x, t)$, multiply by $\delta_t(x, t)$, and integrate by parts. What is $\mathcal{L}(t)$? What
        differential equation does it satisfy?

        @@ First we assume two solutions exist.

        @@ We obtain $\delta$ from these:

        \ms{%
            \delta(x, t) = w_1(x, t) - w_2(x, t)
        }

        @@ Yielding full system

        \ms{%
            \delta_t(x, t) = c^2 \delta_{xx}(x, t)\\
            \delta(0, t) = \delta(L, t) = \delta(x, 0) = \delta(x, 0) = 0
        }

        @@ We multiply by $\delta_t(x, t)$, yielding

        \ms{%
            \delta_t^2(x, t) = c^2 \delta_{xx}(x, t) \delta_t(x, t)
        }

        @@ Yielding

        \ms{%
            \mathcal{L}(t) &= \int_0^L \bren{\delta_t^2(x, t)} \, dx \ge 0\\
            &= \int_0^L \bren{c^2 \delta_{xx}(x, t) \delta_t(x, t)} \, dx\\
            &= \bren{\delta_t \delta_x}\bigg|_0^L - c^2 \int_0^L \bren{\delta_{x} \delta_{tx}} \, dx\\
            &= \bren{\delta_t \delta_x}\bigg|_0^L - c^2 \int_0^L \bren{\frac{\delta_x^2}{2}}_t \, dx\\
            &= \bren{\delta_t(L, t) \delta_x(L, t) - \delta_t(0, t) \delta_x(0, t)} - c^2 \int_0^L \bren{\frac{\delta_x^2}{2}}_t \, dx\\
            &= \int_0^L c^2 \bren{\frac{\delta_x^2}{2}}_t \, dx
        }

        $\mathcal{L}(t)$ satisfies the differential equation

        \ms{%
            v_x = \frac{c^2}{2} \pren{\delta^2_x}_t
        }

        @ Prove the rest. What conditions must be imposed on $w(x, t)$?

        @@ By inspection, $\mathcal{L}(0) = 0$.

        @@ Examining the $t$ derivative,

        \ms{%
            \frac{d\mathcal{L}(t)}{dt} &= \int_0^L c^2 \bren{\frac{\delta_x^2}{2}}_{tt} \, dx\\
            &= \frac{c^2}{2} \bren{\int_0^L \delta_x^2 \, dx}_{tt}\\
            &= \frac{c^2}{2} \bren{\delta^2(L, t) - \delta^2(0, t)}_{tt}\\
            &= 0
        }

        @@ Therefore since $\mathcal{L}(t) \ge 0$, $\mathcal{L}(0) = 0$, and $\mathcal{L}_t(t) = 0$, we can determine
        that $\mathcal{L}(t) \equiv 0$ for all $t \ge 0$.

        @@ In order for this to hold, $w(x, t)$ must be twice $x$ and $t$ differentiable.

        @ Is there any relation between $\mathcal{L}(t)$ and energy?

        The larger this value is the more energy is in the system.

        @ Is there a physical meaning of the ODE?

        The solution to this equation gives the energy over time.
    \end{easylist}

\newpage
\section{}

The beam equation is neat.

\begin{align*}
    \rho w_{tt} &= -EI w_{xxxx} + Q(x, t) \qquad & 0 < x < L \qquad & t > 0\\
    w(0, t) &= w_x(0, t) = w_{xx}(L, t) = w_{xxx}(L, t) = 0 \qquad && t > 0\\
    w(x, 0) &= W(x), w_t(x, 0) = S(x) \qquad & 0 < x < L \qquad & t = 0\\
\end{align*}

Where $\rho$ is the linear mass density, $E$ is Young's modulus, and $I$ is the second moment of area are fixed positive
constants.

\begin{easylist}[enumerate]
    @ For the Beam equation, one implements the procedure to prove uniqueness in the same way for the wave equation.
    What is $\mathcal{L}(t)$? What differential equation does it satisfy?\\

    @@ First we assume two solutions exist.

    @@ We obtain $\delta$ from these:

    \ms{%
        \delta(x, t) = w_1(x, t) - w_2(x, t)
    }

    @@ Yielding full system

    \ms{%
        \rho \delta_{tt}(x, t) = -E I \delta_{xxxx}\\
        \delta(0, t) = \delta_x(0, t) = \delta_{xx}(L, t) = \delta_{xxx}(L, t) = 0\\
        \delta(x, 0) = \delta_t(x, 0) = 0
    }

    @@ We multiply by $\delta_t(x, t)$, yielding

    \ms{%
        \rho \delta_{tt}(x, t) = -E I \delta_{xxxx}\\
        \Rightarrow \pren{\frac{\delta_t^2}{2}}_t = -E I \delta_{xxxx} \delta_t
    }

    @@ This gives us $\mathcal{L}(t)$,

    \ms{%
        \mathcal{L}(t) &= \int_0^L \pren{\frac{\delta_t^2}{2}} \, dx
    }

    @@ Which satisfies the following differential equation

    \ms{%
        2 v_x &= \delta_t^2
    }

    @ Prove the rest. What conditions must be imposed on $w(x, t)$ in order to complete the proof?

    @@ From above, we perform integration by parts on $\mathcal{L}(t)$.\\

    \ms{%
        \mathcal{L}(t) &= \int_0^L \pren{\frac{\delta_t^2}{2}} \, dx\\
        &= \int_0^L \pren{-EI \delta_{xxxx} \delta_t} \, dx\\
        &= \bren{\delta_t \delta_{xxx}}\bigg|_0^L + EI \int_0^L \pren{\delta_{xxx} \delta_{tx}} \, dx\\
        &= \bren{\delta_t(L, t) \delta_{xxx}(L, t) - \delta_t(0, t) \delta_{xxx}(0, t)} +
            EI \int_0^L \pren{\delta_{xxx} \delta_{tx}} \, dx\\
        &= \bren{- \delta_t(0, t) \delta_{xxx}(0, t)} + EI \int_0^L \pren{\delta_{xxx} \delta_{tx}} \, dx\\
        \mathcal{L}(0) &= 0\\
        \mathcal{L}_t(t) &= -\bren{\delta_t(0, t) \delta_{xxx}(0, t)}_t +
            EI \int_0^L \pren{\delta_{xxx} \delta_{tx}}_t \, dx\\
        &= -\bren{\delta_t(0, t) \delta_{xxx}(0, t)}_t +
            EI \int_0^L \pren{\delta_{xt}\delta_{xxxt} + \delta_{xtt}\delta_{xxx}} \, dx
    }

    @@ In order for uniqueness to hold, 

    \ms{%
        EI \int_0^L \pren{\delta_{xt}\delta_{xxxt} + \delta_{xtt}\delta_{xxx}} \, dx
    }

    must be negative.

    @@ We also need $w(x, t)$ differentiable four times.

    @@ If the above conditions are met, since $\mathcal{L}(t) \ge 0$, $\mathcal{L}(0) = 0$, and $\mathcal{L}_t(t) \le
    0$, this implies that $\mathcal{L}(t) \equiv 0$.
\end{easylist}
\end{document}
