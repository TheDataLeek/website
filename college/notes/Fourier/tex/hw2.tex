\documentclass[10pt]{article}

%%%%%%%%%%%%%%%%%%%%%%%%%%%%%%%%%%%%%%%%%%%%%%%%%%%%%%%%%%%%%%%%%%%%%%%%%%%%%%%%
% LaTeX Imports
%%%%%%%%%%%%%%%%%%%%%%%%%%%%%%%%%%%%%%%%%%%%%%%%%%%%%%%%%%%%%%%%%%%%%%%%%%%%%%%%
\usepackage{amsfonts}                                                   % Math fonts
\usepackage{amsmath}                                                    % Math formatting
\usepackage{amssymb}                                                    % Math formatting
\usepackage{amsthm}                                                     % Math Theorems
\usepackage{arydshln}                                                   % Dashed hlines
\usepackage{attachfile}                                                 % AttachFiles
\usepackage{cancel}                                                     % Cancelled math
\usepackage{caption}                                                    % Figure captioning
\usepackage{color}                                                      % Nice Colors
\input{./lib/dragon.inp}                                                % Tikz dragon curve
\usepackage[ampersand]{easylist}                                        % Easy lists
\usepackage{fancyhdr}                                                   % Fancy Header
\usepackage[T1]{fontenc}                                                % Specific font-encoding
%\usepackage[margin=1in, marginparwidth=2cm, marginparsep=2cm]{geometry} % Margins
\usepackage{graphicx}                                                   % Include images
\usepackage{hyperref}                                                   % Referencing
\usepackage[none]{hyphenat}                                             % Don't allow hyphenation
\usepackage{lipsum}                                                     % Lorem Ipsum Dummy Text
\usepackage{listings}                                                   % Code display
\usepackage{marginnote}                                                 % Notes in the margin
\usepackage{microtype}                                                  % Niceness
\usepackage{lib/minted}                                                 % Code display
\usepackage{multirow}                                                   % Multirow tables
\usepackage{pdfpages}                                                   % Include pdfs
\usepackage{pgfplots}                                                   % Create Pictures
\usepackage{rotating}                                                   % Figure rotation
\usepackage{setspace}                                                   % Allow double spacing
\usepackage{subcaption}                                                 % Figure captioning
\usepackage{tikz}                                                       % Create Pictures
\usepackage{tocloft}                                                    % List of Equations
%%%%%%%%%%%%%%%%%%%%%%%%%%%%%%%%%%%%%%%%%%%%%%%%%%%%%%%%%%%%%%%%%%%%%%%%%%%%%%%%
% Package Setup
%%%%%%%%%%%%%%%%%%%%%%%%%%%%%%%%%%%%%%%%%%%%%%%%%%%%%%%%%%%%%%%%%%%%%%%%%%%%%%%%
\hypersetup{%                                                           % Setup linking
    colorlinks=true,
    linkcolor=black,
    citecolor=black,
    filecolor=black,
    urlcolor=black,
}
\RequirePackage[l2tabu, orthodox]{nag}                                  % Nag about bad syntax
\renewcommand*\thesection{\arabic{section} }                             % Reset numbering
\renewcommand{\theFancyVerbLine}{ {\arabic{FancyVerbLine} } }              % Needed for code display
\renewcommand{\footrulewidth}{0.4pt}                                    % Footer hline
\setcounter{secnumdepth}{3}                                             % Include subsubsections in numbering
\setcounter{tocdepth}{3}                                                % Include subsubsections in toc
%%%%%%%%%%%%%%%%%%%%%%%%%%%%%%%%%%%%%%%%%%%%%%%%%%%%%%%%%%%%%%%%%%%%%%%%%%%%%%%%
% Custom commands
%%%%%%%%%%%%%%%%%%%%%%%%%%%%%%%%%%%%%%%%%%%%%%%%%%%%%%%%%%%%%%%%%%%%%%%%%%%%%%%%
\newcommand{\nvec}[1]{\left\langle #1 \right\rangle}                    %  Easy to use vector
\newcommand{\ma}[0]{\mathbf{A} }                                         %  Easy to use vector
\newcommand{\mb}[0]{\mathbf{B} }                                         %  Easy to use vector
\newcommand{\abs}[1]{\left\lvert #1 \right\rvert}                       %  Easy to use abs
\newcommand{\pren}[1]{\left( #1 \right)}                                %  Big parens
\let\oldvec\vec
\renewcommand{\vec}[1]{\oldvec{\mathbf{#1} } }                            %  Vector Styling
\newtheorem{thm}{Theorem}                                               %  Define the theorem name
\newtheorem{definition}{Definition}                                     %  Define the definition name
\definecolor{bg}{rgb}{0.95,0.95,0.95}
\newcommand{\java}[4]{\vspace{10pt}\inputminted[firstline=#2,
                                 lastline=#3,
                                 firstnumber=#2,
                                 gobble=#4,
                                 frame=single,
                                 label=#1,
                                 bgcolor=bg,
                                 linenos]{java}{#1} }
\newcommand{\python}[4]{\vspace{10pt}\inputminted[firstline=#2,
                                 lastline=#3,
                                 firstnumber=#2,
                                 gobble=#4,
                                 frame=single,
                                 label=#1,
                                 bgcolor=bg,
                                 linenos]{python}{#1} }
\newcommand{\js}[4]{\vspace{10pt}\inputminted[firstline=#2,
                                 lastline=#3,
                                 firstnumber=#2,
                                 gobble=#4,
                                 frame=single,
                                 label=#1,
                                 bgcolor=bg,
                                 linenos]{js}{#1} }
%%%%%%%%%%%%%%%%%%%%%%%%%%%%%%%%%%%%%%%%%%%%%%%%%%%%%%%%%%%%%%%%%%%%%%%%%%%%%%%%
% Beginning of document items - headers, title, toc, etc...
%%%%%%%%%%%%%%%%%%%%%%%%%%%%%%%%%%%%%%%%%%%%%%%%%%%%%%%%%%%%%%%%%%%%%%%%%%%%%%%%
\pagestyle{fancy}                                                       %  Establishes that the headers will be defined
\fancyhead[LE,LO]{Computer Systems Notes}                                  %  Adds header to left
\fancyhead[RE,RO]{Zoe Farmer}                                       %  Adds header to right
\cfoot{ \thepage }
\lfoot{CSCI 2400}
\rfoot{Han}
\title{Computer Systems Notes}
\author{Zoe Farmer}

%%%%%%%%%%%%%%%%%%%%%%%%%%%%%%%%%%%%%%%%%%%%%%%%%%%%%%%%%%%%%%%%%%%%%%%%%%%%%%%%
% Beginning of document items - headers, title, toc, etc...
%%%%%%%%%%%%%%%%%%%%%%%%%%%%%%%%%%%%%%%%%%%%%%%%%%%%%%%%%%%%%%%%%%%%%%%%%%%%%%%%
\pagestyle{fancy}                                                       %  Establishes that the headers will be defined
\fancyhead[LE,LO]{Fourier Series}                                  %  Adds header to left
\fancyhead[RE,RO]{Zoe Farmer}                                       %  Adds header to right
\cfoot{\thepage}
\lfoot{APPM 4350}
\rfoot{Mark Hoefer}
\title{Fourier Series Homework Two}
\author{Zoe Farmer}
%%%%%%%%%%%%%%%%%%%%%%%%%%%%%%%%%%%%%%%%%%%%%%%%%%%%%%%%%%%%%%%%%%%%%%%%%%%%%%%%
% Beginning of document items - headers, title, toc, etc...
%%%%%%%%%%%%%%%%%%%%%%%%%%%%%%%%%%%%%%%%%%%%%%%%%%%%%%%%%%%%%%%%%%%%%%%%%%%%%%%%
\begin{document}




\maketitle

\begin{easylist}[enumerate]
    @ Functions and their Fourier Series

    Use the functions $f(x)=\cos(x)$ and $g(x)=\sin(x)$, each defined on $0 \le x \le \pi$.

    @@ Extend $f(x)$ and $g(x)$ as even functions of $x$, now defined on $-\pi \le x \le \pi$. Find the Fourier cosine
    series of $f(x)$ and of $g(x)$, defined as even functions of $x$ on $-\pi < x \le \pi$.\\

    We can extend these functions to be defined over $-\pi \le x \le \pi$ by defining their even extension as the
    following.

    \[
        f(x) = \begin{cases}
            f(x) \qquad &x \ge 0\\
            f(-x) \qquad &x < 0
        \end{cases}
    \]

    These look like the following.

\weave

\includegraphics[width= 4in]{/home/zoe/classwork/2015b/appm4350/homeworks/figures/hw2_figure2_1.pdf}

\noweave

    We then first we calculate the coefficients.

    \begin{align*}
        a_0 &= \frac{1}{2\pi} \int_{-\pi}^\pi f(x) \, dx\\
            \left(\begin{array}{c}a_n\\b_n\end{array}\right) &= \frac{1}{\pi} \int_{-\pi}^\pi f(x)
            \left(\begin{array}{c}\cos(nx)\\\sin(nx)\end{array}\right) \, dx
    \end{align*}

    And then we get the full equation.

    \[
        FS[f](x) = a_0 + \sum_{n=1}^\infty \left[ a_n \cos(nx) + b_n \sin(nx) \right]
    \]

    @@@ $f(x) = \cos(x)$

    We note that the even extension is the same as just $\cos(x)$, so we will examine on the bounds $0 \le x \le \pi$
    and reflect.

    First the coefficients.

    \begin{align*}
        a_0 &= \frac{1}{2\pi} \int_{0}^\pi f(x) \, dx\\
            &= \frac{1}{2\pi} \int_{0}^\pi \cos(x) \, dx \\
            &= 0
    \end{align*}

    \begin{align*}
        a_n &= \frac{1}{\pi} \int_{-\pi}^\pi f(x) \cos(nx) \, dx\\
            &= \frac{1}{\pi} \int_0^\pi \cos(x) \cos(nx) \, dx \\
            &= \frac{-n \sin(n \pi)}{\pi (n^2 - 1)}
    \end{align*}

    \begin{align*}
        b_n &= \frac{1}{\pi} \int_{-\pi}^\pi f(x) \sin(nx) \, dx\\
            &= \frac{1}{\pi} \int_0^\pi \cos(x) \sin(nx) \, dx \\
            &= \frac{1}{\pi} \left( \frac{n(1 + \cos(n \pi))}{n^2 - 1} \right)\\
            &= \frac{n(1 + \cos(n \pi))}{\pi(n^2 - 1)}\\
    \end{align*}

    Now we plug these into our equation.

    \begin{align*}
        FS[f](x) &= a_0 + \sum_{n=1}^\infty \left[ a_n \cos(nx) + b_n \sin(nx) \right]\\
                &= \sum_{n=1}^\infty \left[ \frac{-n \sin(n \pi)}{\pi (n^2 - 1)} \cos(nx) +
                    \frac{n(1 + \cos(n \pi))}{\pi(n^2 - 1)} \sin(nx) \right]\\
                &= \sum_{n=1}^\infty \left[ \frac{n(1 + \cos(n \pi))}{\pi(n^2 - 1)} \sin(nx) \right]\\
    \end{align*}

    This Fourier Series Expansion is defined on $0 \le x \le \pi$, so we can simply reflect to see the other side.

    \[
        FS[f](x) = \begin{cases}
            \sum_{n=1}^\infty \left[ \frac{n(1 + \cos(n \pi))}{\pi(n^2 - 1)} \sin(nx) \right]
                \quad 0 \le x \le \pi\\
            -\sum_{n=1}^\infty \left[ \frac{n(1 + \cos(n \pi))}{\pi(n^2 - 1)} \sin(nx) \right]
                \quad -\pi \le x \le 0
        \end{cases}
    \]

    This is our Fourier Cosine Series of $f(x)$, which we can plot and compare. (Warning, not very good\ldots.)

\weave

\begin{minted}[mathescape, fontsize=\small, xleftmargin=0.5em]{python}
def ff(x, N):
    return np.array([sum([((n * (1 + np.cos(n * np.pi))) /
                        (np.pi * (n**2 - 1))) *
                    np.sin(n * x0) for n in range(2, N + 1)])
                if x0 >= 0 else
                    -sum([((n * (1 + np.cos(n * np.pi))) /
                            (np.pi * (n**2 - 1))) *
                        np.sin(n * x0) for n in range(2, N + 1)])
            for x0 in x])
xl = np.arange(-np.pi, 0, 0.01)
xr = np.arange(0, np.pi, 0.01)
x = np.arange(-np.pi, np.pi, 0.01)
plt.figure()
plt.plot(xl, np.cos(-xl), 'r-', label=r'$f(x)=\cos(x)$')
plt.plot(xr, np.cos(xr), 'r-')
plt.plot(x, ff(x, 20))
plt.legend()
plt.ylim(-1, 1.5)
plt.show()
\end{minted}
\includegraphics[width= 4in]{/home/zoe/classwork/2015b/appm4350/homeworks/figures/hw2_figure3_1.pdf}

\noweave

    @@@ $g(x) = \sin(x)$

    Same process as before.

    \begin{align*}
        a_0 &= \frac{1}{2\pi} \int_{-\pi}^\pi g(x) \, dx\\
            &= \frac{1}{2\pi} \left[ \int_{-\pi}^0 \sin(-x) \, dx + \int_{0}^\pi \sin(x) \, dx \right]\\
            &= \frac{2}{\pi}
    \end{align*}

    \begin{align*}
        a_n &= \frac{1}{\pi} \int_{-\pi}^\pi g(x) \cos(nx) \, dx\\
            &= \frac{1}{\pi} \left[ \int_{-\pi}^0 \sin(-x) \cos(nx) \, dx + \int_{0}^\pi \sin(x) \cos(nx) \, dx \right]\\
            &= \frac{1}{\pi} \left[ 2 \cdot \frac{1 + \cos(n\pi)}{1-n^2} \right]\\
            &= \frac{2(1 + \cos(n\pi))}{\pi(1-n^2)}\\
    \end{align*}

    \begin{align*}
        b_n &= \frac{1}{\pi} \int_{-\pi}^\pi f(x) \sin(nx) \, dx\\
            &= \frac{1}{\pi} \left[ \int_{-\pi}^0 \sin(-x) \sin(nx) \, dx + \int_{0}^\pi \cos(x) \sin(nx) \, dx \right]\\
            &= 0
    \end{align*}

    Now we plug these into our equation.

    \begin{align*}
        FS[g](x) &= \frac{2}{\pi} + \sum_{n=1}^\infty \left[ \frac{2(1 + \cos(n\pi))}{\pi(1-n^2)} \cdot \cos(nx) \right]\\
    \end{align*}

    Which we can plot and compare.

\weave

\begin{minted}[mathescape, fontsize=\small, xleftmargin=0.5em]{python}
def fg(x, N):
    a0 = 2 / np.pi
    return np.array([a0 +
            sum([((2 * (1 + np.cos(n * np.pi))) /
                        (np.pi * (1 - n**2))) *
                    np.cos(n * x0) for n in range(2, N + 1)])
            for x0 in x])
xl = np.arange(-np.pi, 0, 0.01)
xr = np.arange(0, np.pi, 0.01)
x = np.arange(-np.pi, np.pi, 0.01)
plt.figure()
plt.plot(xl, np.sin(-xl), 'r-', label=r'$g(x)=\sin(x)$')
plt.plot(xr, np.sin(xr), 'r-')
plt.plot(x, fg(x, 2), label=r'Fourier Expansion, $N=2$')
plt.plot(x, fg(x, 5), label=r'Fourier Expansion, $N=5$')
plt.plot(x, fg(x, 10), label=r'Fourier Expansion, $N=10$')
plt.legend()
plt.ylim(0, 1.5)
plt.show()
\end{minted}
\includegraphics[width= 4in]{/home/zoe/classwork/2015b/appm4350/homeworks/figures/hw2_figure4_1.pdf}

\noweave

    @@ Evaluate the Fourier cosine series of the even extension of $g(x)$ at $x=0$, and find one more exactly summable
    series. What is the series and what is its sum?\\

    \begin{align*}
        FS[g](0) &= \frac{2}{\pi} + \sum_{n=1}^\infty \left[ \frac{2(1 + \cos(n\pi))}{\pi(1-n^2)} \cdot \cos(n(0)) \right]\\
        &= \frac{2}{\pi} + \left(\frac{2}{\pi}\right) \sum_{n=1}^\infty \left[ \frac{1 + \cos(n\pi)}{1-n^2} \right]\\
        &= \frac{2}{\pi} + \left(\frac{-2}{\pi}\right)\\
        &= 0\\
    \end{align*}

    @@ Let $f_e(x)$ and $f_o(x)$ denote the even and odd extensions of $f(x)$, now defined on $-\pi < x \le \pi$, and
    let $g_e(x)$ and $g_o(x)$ denote the even and odd extensions of $g(x)$, now also defined on $-\pi < x \le \pi$.
    Sketch the periodic extensions of each of these four functions over at least three periods of each. Label each
    function, so that it is clear which is which.\\

\weave

\begin{minted}[mathescape, fontsize=\small, xleftmargin=0.5em]{python}
def fg(x, N):
    a0 = 2 / np.pi
    return np.array([a0 +
            sum([((2 * (1 + np.cos(n * np.pi))) /
                        (np.pi * (1 - n**2))) *
                    np.cos(n * x0) for n in range(2, N + 1)])
            for x0 in x])
def ff(x, N):
    return np.array([sum([((n * (1 + np.cos(n * np.pi))) /
                        (np.pi * (n**2 - 1))) *
                    np.sin(n * x0) for n in range(2, N + 1)])
                if x0 >= 0 else
                    -sum([((n * (1 + np.cos(n * np.pi))) /
                            (np.pi * (n**2 - 1))) *
                        np.sin(n * x0) for n in range(2, N + 1)])
            for x0 in x])
xl = np.arange(-np.pi, 0, 0.01)
xr = np.arange(0, np.pi, 0.01)
x = np.arange(-np.pi, 4 * np.pi, 0.01)
plt.figure()
plt.plot(x, fg(x, 10), label=r'Fourier Expansion, $N=10$')
plt.plot(x, ff(x, 10), label=r'Fourier Expansion, $N=10$')
plt.legend()
plt.ylim(-1, 1.5)
plt.show()
\end{minted}
\includegraphics[width= 4in]{/home/zoe/classwork/2015b/appm4350/homeworks/figures/hw2_figure5_1.pdf}

\noweave

    \newpage
    @ Properties of Fourier Series\\

    Each of the three functions listed below is defined on $-L < y \le L$.

    \begin{align*}
        g_e \left( \frac{\pi y}{L} \right) &=
            \begin{cases}
                \sin \left( \frac{\pi y}{L} \right) \qquad & 0 \le y \le L\\
                \sin \left( -\frac{\pi y}{L} \right) \qquad & -L < y < 0\\
            \end{cases}\\
        h \left( \frac{\pi y}{L} \right) &=
            \begin{cases}
                \sin \left( \frac{\pi y}{L} \right) \qquad & 0 \le y \le L\\
                0 \qquad & -L < y < 0\\
            \end{cases}\\
        j(y) &= \sin^3 \left( \frac{\pi y}{L} \right)
    \end{align*}

    For each function:

    @@ Find the Fourier series of the function.

    @@@ \[
        g_e (y) =
            \begin{cases}
                \sin \left( \frac{\pi y}{L} \right) \qquad & 0 \le y \le L\\
                \sin \left( -\frac{\pi y}{L} \right) \qquad & -L < y < 0\\
            \end{cases}
    \]

    We first find the coefficients.

    \begin{align*}
        a_0 &= \frac{1}{2L} \left[ \int_{-L}^0 \sin\left(-\frac{\pi y}{L}\right) \, dy +
                    \int_0^L \sin\left(\frac{\pi y}{L}\right) \, dy \right]\\
        &= \frac{1}{2L} \left( \frac{L \cos\left(\frac{\pi y}{L}\right)}{\pi}\bigg|_{-L}^0 + \frac{-\pi \cos\left(
        \frac{\pi y}{L} \right)}{L} \bigg|_0^L \right)\\
        &= \frac{1}{2L} \left( \frac{2L}{\pi} + \frac{2L}{\pi}  \right)\\
        &= \frac{2}{\pi}
    \end{align*}

    \begin{align*}
        a_n &= \frac{1}{L} \left[ \int_{-L}^0 \sin\left(-\frac{\pi y}{L}\right) \cos\left( \frac{n \pi y}{L} \right) \, dy + \int_0^L \sin\left(\frac{\pi y}{L}\right) \cos\left( \frac{n \pi y}{L} \right) \, dy \right]\\
        &= \frac{1}{L} \frac{-L\left(\frac{\cos((n-1)\pi y / L)}{n-1} - \frac{\cos((n+1)\pi
                        y)}{n+1}\right)}{2\pi}\bigg|_{-L}^0 +\\&\frac{1}{L} \frac{L\left(\frac{\cos((n-1)\pi y / L)}{n-1} - \frac{\cos((n+1)\pi
        y)}{n+1}\right)}{2\pi}\bigg|_0^L\\
        &= \frac{1 + \cos(n \pi)}{\pi(1 - n^2)} + \frac{1 + \cos(n \pi)}{\pi(n^2 - 1)}\\
        &= \frac{2(1 + \cos(n \pi))}{\pi(1 - n^2)}
    \end{align*}

    \begin{align*}
        b_n &= \frac{1}{L}
            \left[ \int_{-L}^0 \sin\left(-\frac{\pi y}{L}\right) \cdot \sin\left( \frac{n \pi y}{L} \right) \, dy + \int_0^L \sin\left(\frac{\pi y}{L}\right) \cdot \sin\left( \frac{n \pi y}{L} \right) \, dy \right]\\
        &= \frac{1}{L} \left[ \frac{y}{2} - \frac{L \sin\left( \frac{2 \pi y}{L} \right)}{4 \pi} \right] \bigg|_0^L\\
        &= \frac{1}{L} \cdot \frac{L}{2}\\
        &= \frac{1}{2}
    \end{align*}

    We can plot this, with $L = 10$.

\weave

\begin{minted}[mathescape, fontsize=\small, xleftmargin=0.5em]{python}
def fourier2a(x, N):
    a0 = 2 / np.pi
    return np.array([sum([a0 +
                    (((2 * (1 + np.cos(n * np.pi))) /
                        (np.pi - (np.pi * n**2))) *
                        np.cos(n * x0)) +
                    (0 * np.sin(n * x0))
                for n in range(2, N + 1)])
            for x0 in x])
L = 10
ge = lambda y: np.sin(np.pi * y / L)
x = np.arange(-L, L, 0.01)
h = int(len(x) / 2)
plt.figure()
plt.plot(x[:h], ge(-x[:h]), 'r-', label=r'$g_e(y)$')
plt.plot(x[h:], ge(x[h:]), 'r-')
plt.plot(x, fourier2a(x, 2), label=r'Fourier Expansion, $N=2$')
plt.legend()
plt.show()
\end{minted}
\includegraphics[width= 4in]{/home/zoe/classwork/2015b/appm4350/homeworks/figures/hw2_figure6_1.pdf}

\noweave

    @@@

    \[
        h (y)=
            \begin{cases}
                \sin \left( \frac{\pi y}{L} \right) \qquad & 0 \le y \le L\\
                0 \qquad & -L < y < 0\\
            \end{cases}
    \]

    @@@

    \[
        j(y) = \sin^3 \left( \frac{\pi y}{L} \right)
    \]

    \begin{align*}
        a_0 &= \frac{1}{2\pi} \int_{-\pi}^\pi \sin^3 \left( \frac{\pi y}{L} \right) \, dy\\
        &= \frac{1}{2\pi} \left( - \frac{3 L \cos\left(\frac{y\pi}{L}\right)}{4\pi} + \frac{L \cos\left(\frac{3y\pi}{L}\right)}{12\pi} \right) \bigg|_{-\pi}^\pi\\
        &= \frac{1}{2\pi} \left( \frac{L \left( -9 \cos\left(\frac{y\pi}{L}\right) + \cos\left(\frac{3y\pi}{L}\right)\right)}{12 \pi} \right) \bigg|_{-\pi}^\pi\\
        &= \frac{1}{2\pi} \left( \frac{L \left( -9 \cos\left(\frac{y\pi}{L}\right) + \cos\left(\frac{3y\pi}{L}\right)\right)}{12 \pi} \right) \bigg|_{-\pi}^\pi\\
        &= \frac{L \left( -9 \cos\left(\frac{y\pi}{L}\right) + \cos\left(\frac{3y\pi}{L}\right)\right)}{24 \pi^2} \bigg|_{-\pi}^\pi\\
        &= 0
    \end{align*}

    \begin{align*}
        a_n &= \frac{1}{\pi} \int_{-\pi}^\pi \sin^3\left(\frac{y\pi}{L}\right) \cos(ny) \, dy\\
        &= 0\\
    \end{align*}

    \begin{align*}
        b_n &= \frac{1}{\pi} \int_{-\pi}^\pi \sin^3\left(\frac{y\pi}{L}\right) \sin(ny) \, dy\\
        &= \frac{L \left(\frac{\sin \left(\pi  \left(n-\frac{3 \pi }{L}\right)\right)}{3 \pi -L n}-\frac{3 \sin \left(\pi \left(n-\frac{\pi }{L}\right)\right)}{\pi -L n}-\frac{3 \sin \left(\pi \left(\frac{\pi }{L}+n\right)\right)}{L n+\pi }+\frac{\sin \left(\pi  \left(\frac{3 \pi }{L}+n\right)\right)}{L n+3 \pi }\right)}{4 \pi }
    \end{align*}

    We can plot this one.

\weave

\begin{minted}[mathescape, fontsize=\small, xleftmargin=0.5em]{python}
L = 10
def fourier2c(x, N):
    return np.array([sum([0 +
                    (0 * np.cos(n * x0)) +
                    (((L*((np.sin(np.pi*(n - ((3*np.pi)/L)))/
                       ((-L)*n + 3*np.pi)) -
                     ((3*np.sin(np.pi*(n - np.pi/L)))/
                       ((-L)*n + np.pi)) -
                     ((3*np.sin(np.pi*(n + np.pi/L)))/
                       (L*n + np.pi)) +
                     (np.sin(np.pi*(n + (3*np.pi)/L))/
                       (L*n + 3*np.pi))))/(4*np.pi)) *
                     np.sin(n * x0))
                for n in range(1, N + 1)])
            for x0 in x])
j = lambda y: np.sin(np.pi * y / L)**3
x = np.arange(-L, L, 0.01)
plt.figure()
plt.plot(x, j(x), label=r'$j(y)$')
plt.plot(x, fourier2c(x, 2), label=r'$N=2$')
plt.plot(x, fourier2c(x, 5), label=r'$N=5$')
plt.plot(x, fourier2c(x, 10), label=r'$N=10$')
plt.ylim(-1, 2.5)
plt.legend()
plt.show()
\end{minted}
\includegraphics[width= 4in]{/home/zoe/classwork/2015b/appm4350/homeworks/figures/hw2_figure7_1.pdf}

\noweave

    @@ Sketch the periodic extension of the function, on $-2L \le y \le 2L$, if one exists. If no periodic extension
    exists, explain why.

    @@@
    
    @@@

    @@@
\weave

\begin{minted}[mathescape, fontsize=\small, xleftmargin=0.5em]{python}
L = 10
j = lambda y: np.sin(np.pi * y / L)**3
x = np.arange(-L, L, 0.01)
x2 = np.arange(-2 * L, 2 * L, 0.01)
plt.figure()
plt.plot(x, j(x), label=r'$j(y)$')
plt.plot(x2, fourier2c(x2, 10), label=r'$N=10$')
plt.ylim(-1, 1.5)
plt.legend()
plt.show()
\end{minted}
\includegraphics[width= 4in]{/home/zoe/classwork/2015b/appm4350/homeworks/figures/hw2_figure8_1.pdf}

\noweave

    @@ Is the periodic extension continuous for all real $y$? If not, where on $-L \le y \le L$ is the periodic
    extension discontinuous?

    For all functions, the periodic extension is continuous for all real $y$.

    @@ Does the Fourier series converge at every $y$ in $-L \le y \le L$? Why or why not?

    @@@ No it does not.
    @@@ No it does not.
    @@@ No it does not.

    @@ Does the Fourier series converge absolutely for all real $y$?

    @@@ No it does not.
    @@@ No it does not.
    @@@ No it does not.

    @@ Does the Fourier series converge uniformly in $y$ for all real $y$?

    @@@ No it does not.
    @@@ No it does not.
    @@@ No it does not.

    \newpage
    @ Dirichlet's Kernel is defined to be

    \[
        D_N \left( \frac{\pi u}{L} \right) = \left\{ \frac{1}{2} + \sum_{n=1}^N \cos \left( \frac{n \pi u}{L} \right) \right\} \quad \forall u \in \mathbb{R}
    \]

    @@ Show that

    \[
        \frac{1}{L} \int_{-L}^L D_N {\left( \frac{\pi u}{L} \right)} \, du = 1 \quad \forall N \ge 1
    \]

    First, some simplification.

    \begin{align*}
        \frac{1}{L} \int_{-L}^L D_N \left( \frac{\pi u}{L} \right) \, du\\
        \frac{1}{L} \int_{-L}^L \left[ \frac{1}{2} + \sum_{n=1}^N \cos\left( \frac{n\pi u}{L} \right)\right] \, du\\
        \frac{1}{L} \left[ \int_{-L}^L \frac{1}{2} + \int_{-L}^L \sum_{n=1}^N \cos\left( \frac{n\pi u}{L} \right)\right] \, du\\
        \frac{1}{L} \left[ L + \int_{-L}^L \sum_{n=1}^N \cos\left( \frac{n\pi u}{L} \right)\right] \, du\\
    \end{align*}

    Now since we know that this expression must equal zero, we can prove the following statement by induction.

    \[
        \int_{-L}^L \sum_{n=1}^N \cos\left( \frac{n\pi u}{L} \right)\, du = 0
    \]

    First, we show $P(1)$.

    \begin{align*}
        \int_{-L}^L \sum_{n=1}^1 \cos\left( \frac{n\pi u}{L} \right)\, du = 0\\
        \frac{L}{\pi} \sin\left( \frac{n\pi u}{L} \right) \bigg|_{-L}^L = 0\\
        \frac{L}{\pi} \left( \sin(\pi) - \sin(-\pi) \right) = 0\\
    \end{align*}

    Now we show that $P(k) \Rightarrow P(k+1)$.

    \[
        P(k) \equiv \int_{-L}^L \sum_{n=1}^k \cos\left( \frac{n\pi u}{L} \right)\, du = 0
    \]

    \begin{align*}
        P(k + 1) &\equiv \int_{-L}^L \sum_{n=1}^{k+1} \cos\left( \frac{n\pi u}{L} \right)\, du = 0\\
        &\equiv \int_{-L}^L \left[ \cos\left( \frac{(k+1)\pi u}{L} \right) + \cancel{\sum_{n=1}^{k} \cos\left( \frac{n\pi u}{L} \right)} \right]\, du = 0\\
        &\equiv \int_{-L}^L \cos\left( \frac{(k+1)\pi u}{L} \right) \, du = 0\\
        &\equiv \frac{L}{(k+1)\pi} \sin\left( \frac{(k+1)\pi u}{L} \right) \bigg|_{-L}^L = 0\\
        &\equiv \frac{L}{(k+1)\pi} \left( \sin\left( \frac{(k+1)\pi L}{L} \right) - \sin\left( \frac{-(k+1)\pi L}{L} \right) \right) = 0\\
        &\equiv 0 = 0\\
    \end{align*}

    Therefore, our final expression is the following.

    \[
        \frac{1}{L} \cdot L = 1
    \]

    And thusly the statement is proved.

    @@ Show that

    \[
        D_N \left( \frac{\pi u}{L} \right) = \frac{\sin\left( \left( N + \frac{1}{2} \right) \left( \frac{\pi u}{L} \right)\right)}{2 \sin \left( \frac{\pi u}{2L} \right)}
    \]

    With the same process used below, we can use the exponential form of $\cos$ and solve the resulting geometric
    series, yielding the following equations.

    \begin{align*}
        D_N \left( \frac{\pi u}{L} \right) &= \frac{1}{2} + \frac{1}{2}
            \left( -1 + \csc\left(\frac{u\pi}{2L}\right)\sin\left(\frac{\pi(u+2Nu)}{2L}\right)\right)\\
        &= \frac{1}{2}\left[1+\left(-1+\csc\left(\frac{u\pi}{2L}\right)\sin\left(\frac{u\pi(1+2N)}{2L}\right)\right)\right]\\
        &= \frac{1}{2}\left(\csc\left(\frac{u\pi}{2L}\right)\sin\left(\frac{u\pi(1+2N)}{2L}\right)\right)\\
        &= \frac{1}{2}\left(\frac{\sin\left(\frac{u\pi(1+2N)}{2L}\right)}{\sin\left(\frac{u\pi}{2L}\right)}\right)\\
        &= \frac{\sin\left(\frac{u\pi(1+2N)}{2L}\right)}{2\sin\left(\frac{u\pi}{2L}\right)}\\
        &= \frac{\sin\left(\left(N+\frac{1}{2}\right)\left(\frac{u\pi}{L}\right)\right)}{2\sin\left(\frac{u\pi}{2L}\right)}\\
    \end{align*}

    @@ Evaluate $D_N(0)$.

    \begin{align*}
        D_N (0) &= \left\{ \frac{1}{2} + \sum_{n=1}^N \cos (0) \right\}\\
            &= \left\{ \frac{1}{2} + N \right\}\\
    \end{align*}

    @@ For $-L \le u \le L$, where is $D_N \left( \frac{ \pi i }{L} \right) = 0$?

    This is equal when $\sum_{n=1}^N \cos \left( \frac{n \pi u}{L} \right)$ is equal to $-\frac{1}{2}$. This means we
    can just examine the sum. By using the exponent form of cosine, defined as the following relation, we can simplify
    the summation as a geometric series.

    \[
        \cos(\theta) = \frac{1}{2} \left( e^{i \theta} + e^{-i \theta} \right)
    \]

    Yielding the following equivalency.

    \[
        \sum_{n=1}^N \cos \left( \frac{n \pi u}{L} \right) = \frac{1}{2}
        \left( -1 + \csc\left( \frac{\pi u}{2L} \right) \sin\left( \frac{\pi (u + 2Nu)}{2L}\right)\right)
    \]

    By the definition of $\csc$ and $\sin$, we can see that this will equal $-1/2$ whenever

    \[
        \csc\left( \frac{\pi u}{2L} \right) \sin\left( \frac{\pi (u + 2Nu)}{2L}\right) = 0.
    \]

    $\csc$ is never equal to 0, so that means we can reduce this further and say that the expression holds whenever

    \[
        \sin\left( \frac{\pi (u + 2Nu)}{2L}\right) = 0.
    \]

    By the definition of sine, we can reduce this even further and say that the expression holds whenever

    \[
        \frac{\pi (u + 2Nu)}{2L} = n \pi
    \]

    where $n \in \mathbb{Z}$. Which is the equivalent of saying that the expression holds, whenever

    \[
        \frac{u(1 + 2N)}{2L} = n, \quad n \in \mathbb{Z}.
    \]

    @@ Sketch $D_2\left( \frac{ \pi i }{L} \right)$ for $-L \le u \le L$.

    \weave

\begin{minted}[mathescape, fontsize=\small, xleftmargin=0.5em]{python}
def dirichlet_kernel(L, N, u):
    return np.array([(1/2) +
                sum([np.cos((n * np.pi * u0) / L)
            for n in range(1, N + 1)])
        for u0 in u])
L = 10
N = 2
u = np.arange(-L, L, 0.01)
plt.figure()
for i in range(1, 11):
    plt.plot(u, dirichlet_kernel(L, i, u),
        alpha=(1.075 - (0.075 * i)),
        label=r'$N={}$'.format(i))
plt.legend()
plt.show()
\end{minted}
\includegraphics[width= 4.5in]{/home/zoe/classwork/2015b/appm4350/homeworks/figures/hw2_figure9_1.pdf}

    \noweave

\end{easylist}



\end{document}
