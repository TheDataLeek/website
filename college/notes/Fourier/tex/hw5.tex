\documentclass[10pt]{article}

%%%%%%%%%%%%%%%%%%%%%%%%%%%%%%%%%%%%%%%%%%%%%%%%%%%%%%%%%%%%%%%%%%%%%%%%%%%%%%%%
% LaTeX Imports
%%%%%%%%%%%%%%%%%%%%%%%%%%%%%%%%%%%%%%%%%%%%%%%%%%%%%%%%%%%%%%%%%%%%%%%%%%%%%%%%
\usepackage{amsfonts}                                                   % Math fonts
\usepackage{amsmath}                                                    % Math formatting
\usepackage{amssymb}                                                    % Math formatting
\usepackage{amsthm}                                                     % Math Theorems
\usepackage{arydshln}                                                   % Dashed hlines
\usepackage{attachfile}                                                 % AttachFiles
\usepackage{cancel}                                                     % Cancelled math
\usepackage{caption}                                                    % Figure captioning
\usepackage{color}                                                      % Nice Colors
\input{./lib/dragon.inp}                                                % Tikz dragon curve
\usepackage[ampersand]{easylist}                                        % Easy lists
\usepackage{fancyhdr}                                                   % Fancy Header
\usepackage[T1]{fontenc}                                                % Specific font-encoding
%\usepackage[margin=1in, marginparwidth=2cm, marginparsep=2cm]{geometry} % Margins
\usepackage{graphicx}                                                   % Include images
\usepackage{hyperref}                                                   % Referencing
\usepackage[none]{hyphenat}                                             % Don't allow hyphenation
\usepackage{lipsum}                                                     % Lorem Ipsum Dummy Text
\usepackage{listings}                                                   % Code display
\usepackage{marginnote}                                                 % Notes in the margin
\usepackage{microtype}                                                  % Niceness
\usepackage{lib/minted}                                                 % Code display
\usepackage{multirow}                                                   % Multirow tables
\usepackage{pdfpages}                                                   % Include pdfs
\usepackage{pgfplots}                                                   % Create Pictures
\usepackage{rotating}                                                   % Figure rotation
\usepackage{setspace}                                                   % Allow double spacing
\usepackage{subcaption}                                                 % Figure captioning
\usepackage{tikz}                                                       % Create Pictures
\usepackage{tocloft}                                                    % List of Equations
%%%%%%%%%%%%%%%%%%%%%%%%%%%%%%%%%%%%%%%%%%%%%%%%%%%%%%%%%%%%%%%%%%%%%%%%%%%%%%%%
% Package Setup
%%%%%%%%%%%%%%%%%%%%%%%%%%%%%%%%%%%%%%%%%%%%%%%%%%%%%%%%%%%%%%%%%%%%%%%%%%%%%%%%
\hypersetup{%                                                           % Setup linking
    colorlinks=true,
    linkcolor=black,
    citecolor=black,
    filecolor=black,
    urlcolor=black,
}
\RequirePackage[l2tabu, orthodox]{nag}                                  % Nag about bad syntax
\renewcommand*\thesection{\arabic{section} }                             % Reset numbering
\renewcommand{\theFancyVerbLine}{ {\arabic{FancyVerbLine} } }              % Needed for code display
\renewcommand{\footrulewidth}{0.4pt}                                    % Footer hline
\setcounter{secnumdepth}{3}                                             % Include subsubsections in numbering
\setcounter{tocdepth}{3}                                                % Include subsubsections in toc
%%%%%%%%%%%%%%%%%%%%%%%%%%%%%%%%%%%%%%%%%%%%%%%%%%%%%%%%%%%%%%%%%%%%%%%%%%%%%%%%
% Custom commands
%%%%%%%%%%%%%%%%%%%%%%%%%%%%%%%%%%%%%%%%%%%%%%%%%%%%%%%%%%%%%%%%%%%%%%%%%%%%%%%%
\newcommand{\nvec}[1]{\left\langle #1 \right\rangle}                    %  Easy to use vector
\newcommand{\ma}[0]{\mathbf{A} }                                         %  Easy to use vector
\newcommand{\mb}[0]{\mathbf{B} }                                         %  Easy to use vector
\newcommand{\abs}[1]{\left\lvert #1 \right\rvert}                       %  Easy to use abs
\newcommand{\pren}[1]{\left( #1 \right)}                                %  Big parens
\let\oldvec\vec
\renewcommand{\vec}[1]{\oldvec{\mathbf{#1} } }                            %  Vector Styling
\newtheorem{thm}{Theorem}                                               %  Define the theorem name
\newtheorem{definition}{Definition}                                     %  Define the definition name
\definecolor{bg}{rgb}{0.95,0.95,0.95}
\newcommand{\java}[4]{\vspace{10pt}\inputminted[firstline=#2,
                                 lastline=#3,
                                 firstnumber=#2,
                                 gobble=#4,
                                 frame=single,
                                 label=#1,
                                 bgcolor=bg,
                                 linenos]{java}{#1} }
\newcommand{\python}[4]{\vspace{10pt}\inputminted[firstline=#2,
                                 lastline=#3,
                                 firstnumber=#2,
                                 gobble=#4,
                                 frame=single,
                                 label=#1,
                                 bgcolor=bg,
                                 linenos]{python}{#1} }
\newcommand{\js}[4]{\vspace{10pt}\inputminted[firstline=#2,
                                 lastline=#3,
                                 firstnumber=#2,
                                 gobble=#4,
                                 frame=single,
                                 label=#1,
                                 bgcolor=bg,
                                 linenos]{js}{#1} }
%%%%%%%%%%%%%%%%%%%%%%%%%%%%%%%%%%%%%%%%%%%%%%%%%%%%%%%%%%%%%%%%%%%%%%%%%%%%%%%%
% Beginning of document items - headers, title, toc, etc...
%%%%%%%%%%%%%%%%%%%%%%%%%%%%%%%%%%%%%%%%%%%%%%%%%%%%%%%%%%%%%%%%%%%%%%%%%%%%%%%%
\pagestyle{fancy}                                                       %  Establishes that the headers will be defined
\fancyhead[LE,LO]{Computer Systems Notes}                                  %  Adds header to left
\fancyhead[RE,RO]{Zoe Farmer}                                       %  Adds header to right
\cfoot{ \thepage }
\lfoot{CSCI 2400}
\rfoot{Han}
\title{Computer Systems Notes}
\author{Zoe Farmer}

%%%%%%%%%%%%%%%%%%%%%%%%%%%%%%%%%%%%%%%%%%%%%%%%%%%%%%%%%%%%%%%%%%%%%%%%%%%%%%%%
% Beginning of document items - headers, title, toc, etc...
%%%%%%%%%%%%%%%%%%%%%%%%%%%%%%%%%%%%%%%%%%%%%%%%%%%%%%%%%%%%%%%%%%%%%%%%%%%%%%%%
\pagestyle{fancy}                                                       %  Establishes that the headers will be defined
\fancyhead[LE,LO]{Fourier Series}                                  %  Adds header to left
\fancyhead[RE,RO]{Zoe Farmer}                                       %  Adds header to right
\cfoot{\thepage}
\lfoot{APPM 4350}
\rfoot{Mark Hoefer}
\title{Fourier Series Homework Five}
\author{Zoe Farmer}
%%%%%%%%%%%%%%%%%%%%%%%%%%%%%%%%%%%%%%%%%%%%%%%%%%%%%%%%%%%%%%%%%%%%%%%%%%%%%%%%
% Beginning of document items - headers, title, toc, etc...
%%%%%%%%%%%%%%%%%%%%%%%%%%%%%%%%%%%%%%%%%%%%%%%%%%%%%%%%%%%%%%%%%%%%%%%%%%%%%%%%
\begin{document}



\maketitle

\section{Transient Solutions of the Heat Equation}
\begin{easylist}[enumerate]
    @ Let $u(x, t)$ satisfy the following problem.

    \begin{align*}
        u_t = ku_{xx} \qquad & 0 < x < L, t > 0, k > 0\\
        u_x(0, t) = u_x(L, t) = 0 \qquad & t > 0\\
        u(x, 0) = 100 \left( 1 - \frac{x}{L} \right) \qquad & 0 \le x \le L\\
    \end{align*}

    Find $u(x, t)$ in the form of an infinite series. Plot $u(x, t)$, evaluated at $x = L / 2$, as a function of $t$.

    @@ We first find the steady-state solution, by setting $u_t = 0$.

    \begin{align*}
        0 &= ku_{xx}\\
        u &= Ax + B\\
        u_x(0, t) &= 0 = A\\
    \end{align*}

    Therefore the steady state is some constant $C$.

    @@ We now find the transient.

    Using separation of variables, we get the two Ordinary Differential Equations:

    \begin{align*}
        f^\prime(t) &= \mu k f(t)\\
        \ddot{g}(x) &= \mu f(t)\\
    \end{align*}

    Where $\dot{g}(0) = \dot{g}(L) = 0$.

    Solving for $f(t)$ is straightforward.

    \[
        f(t) = e^{\mu k t}
    \]

    Solving for $g(x)$ is a little more complicated. We need to examine three cases, where $\mu > 0, \mu = 0$, and
    $\mu < 0$.

    @@@ $\mu > 0$

    In this case, we get the following equation for $g(x)$.

    \begin{align*}
        g(x) &= A \sinh(\sqrt{\mu}x) + B \cosh(\sqrt{\mu}x)\\
        \dot{g}(x) &= A \sqrt{\mu} \cosh{\left (\sqrt{\mu} x \right )} + B \sqrt{\mu} \sinh{\left (\sqrt{\mu} x \right)}
    \end{align*}

    Using the boundary conditions, we see the following.

    \begin{align*}
        \dot{g}(0) &= 0 = A\\
        \dot{g}(L) &= 0 = B \sqrt{\mu} \sinh{\left (\sqrt{\mu} L \right)}
    \end{align*}

    This holds iff $B=0$, therefore $\mu \not> 0$.

    @@@ $\mu = 0$

    This is easy, as $g$ is simply

    \begin{align*}
        g(x) &= Ax + B\\
        \dot{g}(x) &= A\\
    \end{align*}

    With Boundary Conditions:

    \begin{align*}
        \dot{g}(0) &= 0 = A\\
        \dot{g}(L) &= 0 = 0\\
    \end{align*}

    Using the boundary conditions we see that $A = 0$, and $B$ can be any real number.

    @@@ $\mu < 0$

    Let $\mu = -\lambda^2$. Now $g$ is the following.

    \begin{align*}
        g(x) &= A \sin\pren{\lambda x} + B \cos\pren{\lambda x}\\
        \dot{g}(x) &= A \lambda \cos\pren{\lambda x} - B \lambda \sin\pren{\lambda x}\\
    \end{align*}

    Using the boundary conditions, we see

    \begin{align*}
        \dot{g}(0) &= 0 = A \lambda\\
        \dot{g}(L) &= 0 = - B \lambda \sin\pren{\lambda L}\\
    \end{align*}

    Therefore, $A = 0$, and $B$ is some real number. The above equation holds when

    \[
        \lambda_n L = n \pi \Rightarrow \lambda_n = \frac{n \pi}{L}
    \]

    These are our eigenvalues, and our eigenfunctions are

    \[
        \phi_n(x) = \cos\pren{\lambda_n x}
    \]

    @@ Therefore, our final equation is the following.

    \begin{align*}
        u(x, t) &= c + \sum_{n = 1}^\infty a_n \phi_n(x) e^{-k\lambda^2t}\\
        &= c + \sum_{n = 1}^\infty a_n \cos\pren{\frac{n\pi x}{L}} e^{-k{\pren{\frac{n \pi}{L}}}^2t}
    \end{align*}

    @@ Now we can find the coefficients. Since $u(x, t)$ is in the form of a Fourier Series, we can write the
    coefficients as

    \begin{align*}
        u(x, 0) = 100 \pren{1 - \frac{x}{L}} = c + \sum_{n = 1}^\infty a_n \cos\pren{\frac{n\pi x}{L}}\\
    \end{align*}

    Yielding,

    \begin{align*}
        c &= \frac{1}{L} \int_0^L 100 \pren{1 - \frac{x}{L}} \, dx\\
        &= 50\\
        a_n &= \frac{2}{L} \int_0^L 100 \pren{1 - \frac{x}{L}} \cos\pren{\frac{n\pi x}{L}} \, dx\\
        &= \frac{1}{n^{2} \pi^{2}} \left(- 200 \cos{\left (n \pi \right )} + 200\right)
    \end{align*}

    @@ Finally, the whole thing with coefficients.

    \begin{align*}
        u(x, t) &= 50 + \sum_{n = 1}^\infty \pren{\frac{1}{n^{2} \pi^{2}} \left(- 200 \cos{\left (n \pi \right )} + 200\right)} \cos\pren{\frac{n\pi x}{L}} e^{-k{\pren{\frac{n \pi}{L}}}^2t}
    \end{align*}

    @@ Which we can plot.

\weave

\begin{minted}[mathescape, fontsize=\small, xleftmargin=0.5em]{python}
L = 10
k = 5
def an_gen(n):
    an = (1 / (n**2 * np.pi**2)) * (-200 * np.cos(n * np.pi) + 200)
    return an

def fourier(x, t, N):
    f = np.zeros((len(x), len(t)))
    for i in range(len(x)):
        for j in range(len(t)):
            f[i][j] += 100
            for n in range(1, N + 1):
                f[i][j] += (an_gen(n) * np.cos(n * np.pi * x[i] / L)
                    * np.exp(-k * t[j] * (n * np.pi / L)**2))
    return f

x = np.arange(0, L, 0.01)
t = np.arange(0, 5, 0.1)
points = fourier(x, t, 10)
points_1 = points

fig = plt.figure(figsize=(8, 8))
ax = fig.add_axes([0.1, 0.1, 0.8, 0.8], projection='3d')
for i in range(len(t)):
    ax.plot(x, points[:, i], t[i])
ax.view_init(elev=-70, azim=90)
ax.set_xlabel(r'x')
ax.set_ylabel(r'u(x, t)')
ax.set_zlabel(r't')
plt.show()
\end{minted}
\includegraphics[width= 4in]{/home/zoe/classwork/2015b/appm4350/homeworks/figures/hw5_figure1_1.pdf}

\noweave

    @@ We can specifically examine $u(x, t)$ at $x = L / 2$.

\weave

\begin{minted}[mathescape, fontsize=\small, xleftmargin=0.5em]{python}
plt.figure()
x0 = int(len(x) / 2)
plt.plot(t, points[x0, :])
plt.show()
\end{minted}
\includegraphics[width= 4in]{/home/zoe/classwork/2015b/appm4350/homeworks/figures/hw5_figure2_1.pdf}

\noweave

    \newpage
    @ Let $v(x, t)$ satisfy the following problem.

    \begin{align*}
        v_t = kv_{xx} \qquad & 0 < x < L, t > 0, k > 0\\
        v(0, t) = v(L, t) = 0 \qquad & t > 0\\
        v(x, 0) = 100 \left( 1 - \frac{x}{L} \right) \qquad & 0 \le x \le L\\
    \end{align*}

    Find $v(x, t)$ in the form of an infinite series.

    @@ First the steady-state solution.

    \begin{align*}
        v_t &= 0 = kv_{xx}\\
        v(x, t) &= Ax + B\\
        A = B &= 0\\
    \end{align*}

    Therefore the steady state is zero.

    @@ Now the transient. Again, we use separation of variables to get two ODEs.

    \begin{align*}
        f^\prime(t) &= \mu k f(t)\\
        \ddot{g}(x) &= \mu f(t)
    \end{align*}

    Where $g(0) = g(L) = 0$.

    $f(t)$ is a straightforward ODE.

    \[
        f(t) = e^{\mu k t}
    \]

    We can now find $g(t)$. We need to again examine three cases: $\mu > 0, \mu = 0, \mu < 0$.

    @@@ $\mu > 0$

    As before with this case we obtain the following form for $g(x)$.

    \[
        g(x) = A \sinh\pren{\sqrt{\mu} x} + B \cosh\pren{\sqrt{\mu} x}
    \]

    Plugging in our initial conditions yields the following.

    \begin{align*}
        g(0) &= 0 = B\\
        g(L) &= 0 = A \sinh\pren{\sqrt{\mu} L}
    \end{align*}

    So both $A$ and $B$ must be zero, yielding no non-trivial solution for $\mu > 0$.

    @@@ $\mu = 0$

    Same process:

    \[
        g(x) = Ax + B
    \]

    With intial conditions:

    \[
        A = B = 0
    \]

    And again, no non-trivial solutions exist, therefore $\mu \neq 0$.

    @@@ $\mu < 0$. Let $\mu = -\lambda^2$.

    Now we get $g(x)$ in the following form:

    \[
        g(x) = A \sin(\lambda x) + B \cos(\lambda x)
    \]

    With initial conditions we get the following.

    \begin{align*}
        g(0) &= 0 = B\\
        g(L) &= 0 = A \sin(\lambda L)
    \end{align*}

    That last equation is true (thus yielding our eigenvalues) when

    \[
        \lambda L = n \pi \Rightarrow \lambda_n = \frac{n \pi}{L}
    \]

    With corresponding eigenfunctions

    \[
        \phi_n(x) = \sin\pren{\frac{n \pi x}{L}}
    \]

    @@ Finally, our solution is the following.

    \[
        v(x, t) = \sum_{n=1}^\infty a_n \phi_n(\lambda_n x) e^{-kt\lambda^2}
    \]

    With the eigenvalues and eigenfunctions, this becomes

    \[
        v(x, t) = \sum_{n=1}^\infty a_n \sin\pren{\frac{n\pi x}{L}} e^{-kt \pren{\frac{n \pi}{L}}}
    \]

    @@ We now examine the coefficients using the intial heat distribution function.

    \[
        v(x, 0) = 100 \pren{1 - \frac{x}{L}} = \sum_{n=1}^\infty a_n \sin\pren{\frac{n\pi x}{L}}
    \]

    This is a Fourier Series, and we can find $a_n$ with the following formula.

    \begin{align*}
        a_n &= \frac{2}{L} \int_0^L \pren{100 \pren{1 - \frac{x}{L}}} \sin\pren{\frac{n \pi x}{L}} \, dx\\
        &= \frac{1}{n^{2} \pi^{2}} \left(200 n \pi - 200 \sin{\left (n \pi \right )}\right)
    \end{align*}

    @@ Finally, we get the whole equation.

    \[
        v(x, t) = \sum_{n=1}^\infty \pren{\frac{1}{n^{2} \pi^{2}} \left(200 n \pi - 200 \sin{\left (n \pi \right)}\right)} \sin\pren{\frac{n\pi x}{L}} e^{-kt \pren{\frac{n \pi}{L}}^2}
    \]

    @@ Which we can plot.

\weave

\begin{minted}[mathescape, fontsize=\small, xleftmargin=0.5em]{python}
L = 10
k = 5
def an_gen(n):
    an = (1 / (n**2 * np.pi**2)) * (200 * n * np.pi - 200 * np.sin(n *
np.pi))
    return an

def fourier(x, t, N):
    f = np.zeros((len(x), len(t)))
    for i in range(len(x)):
        for j in range(len(t)):
            for n in range(1, N + 1):
                f[i][j] += an_gen(n) * np.sin(n * np.pi * x[i] / L) *
np.exp(-k * t[j] * (n * np.pi / L)**2)
    return f

x = np.arange(0, L, 0.01)
t = np.arange(0, 5, 0.1)
points = fourier(x, t, 100)
points_2 = points

fig = plt.figure(figsize=(8, 8))
ax = fig.add_axes([0.1, 0.1, 0.8, 0.8], projection='3d')
for i in range(len(t)):
    ax.plot(x, points[:, i], t[i])
ax.view_init(elev=-90, azim=90)
ax.set_xlabel(r'x')
ax.set_ylabel(r'u(x, t)')
ax.set_zlabel(r't')
plt.show()
\end{minted}
\includegraphics[width= 4in]{/home/zoe/classwork/2015b/appm4350/homeworks/figures/hw5_figure3_1.pdf}

\noweave

    \newpage
    @ Let $w(x, t)$ satisfy the following problem.

    \begin{align*}
        w_t = kv_{xx} \qquad & 0 < x < L, t > 0, k > 0\\
        w_x(0, t) = w(L, t) = 0 \qquad & t > 0\\
        w(x, 0) = 100 \left( 1 - \frac{x}{L} \right) \qquad & 0 \le x \le L\\
    \end{align*}

    Find $w(x, t)$ in the form of an infinite series.

    @@ And one last time, let's do the steady state solution.

    \begin{align*}
        w_t &= 0 = kv_{xx}\\
        w_x(x, t) &= A\\
        w(x, t) &= Ax + B\\
        w_x(0, t) &= 0 = A\\
        w(L, t) &= 0 = B\\
    \end{align*}

    Therefore the steady-state solution is $0$.

    @@ Now we can use separation of variables to obtain two different ODEs. This yields the following.

    \begin{align*}
        f^\prime(t) &= \mu k f(t)\\
        \ddot{g}(x) &= \mu f(t)
    \end{align*}

    Where $\dot{g}(0) = g(L) = 0$.

    $f(t)$ is a straightforward ODE.

    \[
        f(t) = e^{\mu k t}
    \]

    We can now find $g(t)$. We need to again examine three cases: $\mu > 0, \mu = 0, \mu < 0$.

    @@@ $\mu > 0$

    In this case we obtain the following form for $g(x)$.

    \begin{align*}
        g(x) &= A \sinh(\sqrt{\mu}x) + B \cosh(\sqrt{\mu}x)\\
        \dot{g}(x) &= A \sqrt{\mu} \cosh{\left (\sqrt{\mu} x \right )} + B \sqrt{\mu} \sinh{\left (\sqrt{\mu} x \right)}
    \end{align*}

    We can use our boundary conditions.

    \begin{align*}
        \dot{g}(0) &= 0 = A \sqrt{\mu} \Rightarrow A = 0\\
        g(L) &= 0 = B \cosh(\sqrt{\mu}L) \Rightarrow B = 0\\
    \end{align*}

    No non-trivial solutions exist, therefore $\mu \not> 0$.

    @@@ $\mu = 0$

    As before we get the form for $g(x)$ to be the following.

    \begin{align*}
        g(x) &= Ax + B\\
        \dot{g}(x) &= A\\
    \end{align*}

    Using the boundary conditions we see

    \begin{align*}
        \dot{g}(0) &= 0 = B\\
        g(L) &= 0 = AL \Rightarrow A = 0\\
    \end{align*}

    No non trivial solutions exist.

    @@@ $\mu < 0$. Let $\mu = - \lambda^2$.

    This yields the following form of $g(x)$.

    \begin{align*}
        g(x) &= A \sin\pren{\lambda x} + B \cos\pren{\lambda x}\\
        \dot{g}(x) &= A \lambda \cos\pren{\lambda x} - B \lambda \sin\pren{\lambda x}\\
    \end{align*}

    Using our boundary conditions we see the following.

    \begin{align*}
        \dot{g}(0) &= 0 = A \lambda \Rightarrow A = 0\\
        g(L) &= 0 = B \cos\pren{\lambda L}\\
    \end{align*}

    That last equation holds when the following is true (yielding our eigenvalues).

    \[
        \lambda L = n \pi \Rightarrow \lambda_n = \frac{n \pi}{L}
    \]

    With corresponding eigenfunctions

    \[
        \phi_n(x) = B \cos\pren{\frac{n\pi x}{L}}
    \]

    @@ Yielding the final solution in the form

    \[
        w(x, t) = \sum_{n=1}^\infty a_n \phi_n(\lambda_n x)e^{-kt\lambda^2}
    \]

    @@ With our eigenvalues and eigenfunctions plugged in we get the following.

    \[
        w(x, t) = \sum_{n=1}^\infty a_n \cos\pren{\frac{nx\pi}{L}}e^{-kt\pren{\frac{n\pi}{L}}^2}
    \]

    @@ We can find $a_n$ using the initial condition.

    \[
        w(x, 0) = 100 \pren{1 - \frac{x}{L}} = \sum_{n=1}^\infty a_n \cos\pren{\frac{nx\pi}{L}}
    \]

    This is simply a Fourier Series, and we can use the following formula to determine this coefficient.

    \begin{align*}
        a_n &= \frac{2}{L} \int_0^L \pren{100 \pren{1 - \frac{x}{L}}} \cos\pren{\frac{nx\pi}{L}} \, dx\\
        &= \frac{1}{n^2\pi^2} \pren{-200\cos(n\pi) + 200}
    \end{align*}

    @@ Now the full equation:

    \[
        w(x, t) = \sum_{n=1}^\infty \pren{\frac{1}{n^2\pi^2} \pren{-200\cos(n\pi) + 200}} \cos\pren{\frac{nx\pi}{L}}e^{-kt\pren{\frac{n\pi}{L}}^2}
    \]

    @@ Which we can plot.

\weave

\begin{minted}[mathescape, fontsize=\small, xleftmargin=0.5em]{python}
L = 10
k = 5
def an_gen(n):
    an = (1 / (n**2 * np.pi**2)) * (- 200 * np.cos(n * np.pi) + 200)
    return an

def fourier(x, t, N):
    f = np.zeros((len(x), len(t)))
    for i in range(len(x)):
        for j in range(len(t)):
            for n in range(1, N + 1):
                f[i][j] += an_gen(n) * np.cos(n * np.pi * x[i] / L) *
np.exp(-k * t[j] * (n * np.pi / L)**2)
    return f

x = np.arange(0, L, 0.01)
t = np.arange(0, 5, 0.1)
points = fourier(x, t, 30)
points_3 = points

fig = plt.figure(figsize=(8, 8))
ax = fig.add_axes([0.1, 0.1, 0.8, 0.8], projection='3d')
for i in range(len(t)):
    ax.plot(x, points[:, i], t[i])
ax.view_init(elev=-80, azim=90)
ax.set_xlabel(r'x')
ax.set_ylabel(r'u(x, t)')
ax.set_zlabel(r't')
plt.show()
\end{minted}
\includegraphics[width= 4in]{/home/zoe/classwork/2015b/appm4350/homeworks/figures/hw5_figure4_1.pdf}

\noweave

    \newpage
    @ Answer each of the following questions for $u(x, t)$ from problem 1, for $v(x, t)$ from problem 2, and for $w(x,
    t)$ from problem 3. Three answers for each question.

    @@ To summarize:

    @@@

    \begin{align*}
        u(x, t) &= 50 + \sum_{n = 1}^\infty \pren{\frac{1}{n^{2} \pi^{2}} \left(- 200 \cos{\left (n \pi \right )} + 200\right)} \cos\pren{\frac{n\pi x}{L}} e^{-k{\pren{\frac{n \pi}{L}}}^2t}
    \end{align*}

    @@@

    \[
        v(x, t) = \sum_{n=1}^\infty \pren{\frac{1}{n^{2} \pi^{2}} \left(200 n \pi - 200 \sin{\left (n \pi \right)}\right)} \sin\pren{\frac{n\pi x}{L}} e^{-kt \pren{\frac{n \pi}{L}}}
    \]

    @@@

    \[
        w(x, t) = \sum_{n=1}^\infty \pren{\frac{1}{n^2\pi^2} \pren{-200\cos(n\pi) + 200}} \cos\pren{\frac{nx\pi}{L}}e^{-kt\pren{\frac{n\pi}{L}}^2}
    \]

    @@ What is the steady-state temperature distribution.

    @@@ $u(x, t)$

    Steady state is a straight line, $y=50$.

    @@@ $v(x, t)$

    Steady State is zero.

    @@@ $w(x, t)$

    Steady State is also zero.

    @@ Write down the Fourier mode that decays the slowest, as an explicit function of $(x, t)$ with no undefined
    constants.

    @@@ $u(x, t)$

    For this we can see that when $n=1$ we get the following equation.

    \begin{align*}
        u_1(x, t) &= 50 + \pren{\frac{400}{\pi^{2}}} \cos\pren{\frac{\pi x}{L}} e^{-k{\pren{\frac{\pi}{L}}}^2t}
    \end{align*}

    For this solution there is no such $t$ and $x$ such that $u_1(x, t)$ is equal to 30, and is also a maximum in $x$.

    @@@ $v(x, t)$

    And again, when $n = 1$

    \[
        v_1(x, t) = \pren{\frac{200}{\pi^{2}}} \sin\pren{\frac{\pi x}{L}} e^{-kt \pren{\frac{\pi}{L}}^2}
    \]

    And again there will be no maximum of 30.

    @@@ $w(x, t)$

    When $n = 1$,

    \[
        w_1(x, t) = \pren{\frac{400}{\pi^2}} \cos\pren{\frac{x\pi}{L}} e^{-kt \pren{\frac{\pi}{L}}^2}
    \]

    This however does have a solution. Solving $w_1(x, t)$ for $t$ yields the time at which any given $x$ is equal to
    $30$. We just need to find the maximum and we're set.

\simpleweave

\includegraphics[width= 4in]{/home/zoe/classwork/2015b/appm4350/homeworks/figures/hw5_figure5_1.pdf}

\nosimpleweave

    This maximum is at $t=0$.

    @@ At $t = 0$, the maximum temperature in the rod is $100$, at $x = 0$. At what time does the slowest decaying
    Fourier mode attain a maximum (in $x$) magnitude of $30$, and where (in $x$) does this maximum occur? Your answer
    should come out in the form $t = M \left( \frac{L^2}{k} \right)$, for some constant $M$. You need to find $M$.

    For each we will derivate with respect to $x$ and set equal to zero to solve.

    @@@ $u(x, t)$
    @@@ $v(x, t)$
    @@@ $w(x, t)$

    @@ In part two, you found the slowest decaying mode. Identify the second slowest mode and evaluate its maximum
    magnitude of each mode at the time you found in the third part.

    @@@ $u(x, t)$
    @@@ $v(x, t)$
    @@@ $w(x, t)$

    @@ Based on your last results, plus a constant term if the series has one, sketch each solution as a function of
    $x$, over $0 < x < L$ at the time you identified in part 3.

    @@@ $u(x, t)$
    @@@ $v(x, t)$
    @@@ $w(x, t)$

    @@ At each fixed $t \ge 0$, the solution of each problem attains a maximum value in $x$. Sketch the approximate
    trajectory of the maximum value of the solution as a function of time.

    @@@ $u(x, t)$
    @@@ $v(x, t)$
    @@@ $w(x, t)$

\end{easylist}

\newpage
\section{Two Culinary Problems}
    \subsection{Potato Slices}

    How long does it take to cook a potato? Slice a potato into slices of thickness $L$, and drop the slices into a
    large pot of boiling water. If $L$ is much smaller than the diameter of the potato, then almost all of the heat
    entering a slice of potato comes through its two cut surfaces, and we can use a one-dimensional model to describe
    the cooking process, at least approximately. Let $u(x, t)$ be the temperature in the potato, measured in $C$. If the
    potato was taken from the refrigerator just before cooking, and if the slices are not too close to each other in the
    boiling water, then

    \begin{align*}
        u_t = ku_{xx} \qquad & 0 < x < L, t > 0, k > 0\\
        u(0, t) = u(L, t) = 100 \qquad & t > 0\\
        u(x, 0) = 5 \qquad & 0 \le x \le L\\
    \end{align*}

    We need two things to answer this. The thermal diffusivity of a potato is $k = 0.0017 \, cm^2/s$. Let's call the
    potato cooked when it has reached $75^\circ C$ everywhere in the potato.

    \begin{easylist}[enumerate]
    @ Solve the problem above, for the temperature in the potato. If you cut the potato into slices $1cm$ thick,
    approximately how long do the slices take to cook?

    @@ Steady State

    Let $u_t = 0$. This yields the equation $0 = k u_{xx}$. Solving for $\overline{u}(x, t)$ we get the steady-state
    solution.

    \begin{align*}
        \overline{u}(x, t) &= Ax + B\\
    \end{align*}

    Using our boundary conditions we can determine the values for $A$ and $B$.

    \begin{align*}
        \overline{u}(0, t) &= 100 = B\\
        \overline{u}(L, t) &= 100 = AL + 100 \Rightarrow A = 0\\
    \end{align*}

    Yielding the steady-state solution,

    \begin{align*}
        \overline{u}(x, t) = 100
    \end{align*}

    Intuitively this makes sense. The pot of water is at $100^\circ \, C$. The potato bits should get to that
    temperature after a certain period of time.

    @@ Now we can determine the transient solution.

    Let us assume that the solution of $u(x, t)$ has the following form.

    \begin{align*}
        u(x, t) = \overline{u}(x, t) + v(x, t)
    \end{align*}

    Substituting back into the original equation we get

    \begin{align*}
        {[\overline{u} + v]}_t = k {[\overline{u} + v]}_{xx}
    \end{align*}

    Distributing the Linear Operator $\partial$ we get

    \begin{align*}
        \overline{u}_t + v_t = k\overline{u}_{xx} + kv_{xx}
    \end{align*}

    We know the form of the steady state solution, therefore the equation we can solve is 

    \begin{align*}
        v_t = kv_{xx}
    \end{align*}

    In order to apply separation of variables we need to adjust our Initial Condition and Boundary Conditions a little.
    We can rewrite these conditions as the following.

    \begin{align*}
        u(0, t) = u(L, t) = 0\\
        u(x, 0) = f(x) - 100 = -95
    \end{align*}

    Using separation of variables, we get the following two ODEs.

    \begin{align*}
        f^\prime(t) &= \mu k f(t)\\
        \ddot{g}(x) &= \mu g(x)
    \end{align*}

    Where we can use the initial conditions and get $g(0) = g(L) = 100$.

    The solution to $f(t)$ is trivial.

    \begin{align*}
        f(t) = e^{\mu k t}
    \end{align*}

    Now we can solve for $g(x)$. We need to compare three cases: $\mu > 0, \mu = 0, \mu < 0$.

    @@@ $\mu > 0$

    This case yields $g(x)$ in the following form.

    \begin{align*}
        g(x) = A \sinh\left(\sqrt{\mu}x\right) + B \cosh\left(\sqrt{\mu} x\right)
    \end{align*}

    Using the Boundary Conditions we get the following values for $A$ and $B$.

    \begin{align*}
        g(0) &= 0 = B\\
        g(L) &= 0 = A \sinh\left(\sqrt{\mu}L\right)\\
        &\Rightarrow A = 0\\
    \end{align*}

    With final solution of the following form.

    \begin{align*}
        g(x) &= 0
    \end{align*}

    Therefore $\mu \not>0$, as there is no non-trivial solution.

    @@@ $\mu = 0$

    This is the simple case.

    \begin{align*}
        g(x) &= Ax + B\\
    \end{align*}

    With Boundary Conditions we get values for $A$ and $B$.

    \begin{align*}
        g(0) &= 0 = B\\
        g(L) &= 0 = AL\\
        &\Rightarrow A = 0
    \end{align*}

    Yielding the final equation:

    \begin{align*}
        g(x) = 0
    \end{align*}

    Therefore $\mu \neq 0$, as there are no non-trivial solutions.

    @@@ $\mu < 0$. Let $\mu = -\lambda^2$.

    In this case, $g(x)$ has the following form.

    \begin{align*}
        g(x) &= A \sin\left( \lambda x \right) + B \cos\left( \lambda x \right)
    \end{align*}

    Using Boundary Conditions we can determine $A$ and $B$.

    \begin{align*}
        g(0) &= 0 = B\\
        g(L) &= 0 = A \sin\left( \lambda L \right)\\
    \end{align*}

    The above equation holds when $\lambda$ is of the following form.

    \begin{align*}
        \lambda L = n \pi \Rightarrow \lambda_n = \frac{n\pi}{L}
    \end{align*}

    This yields the eigenvalues above and the following eigenfunctions

    \begin{align*}
        \phi_n(x) = \sin\left(\lambda_n x\right) = \sin\left(\frac{n\pi x}{L}\right)
    \end{align*}

    @@ This yields the form of the final solution:

    \begin{align*}
        u(x, t) = \sum_{n=1}^\infty \left[ \phi\left(\lambda\right) f(t)\right] 
    \end{align*}

    Using our eigenvalues and eigenfunctions we get:

    \begin{align*}
        u(x, t) = \sum_{n=1}^\infty \left[ a_n \sin\left(\frac{n\pi x}{L}\right) e^{-kt\frac{n\pi}{L}}\right] 
    \end{align*}

    Adding our Steady-State Solution we get the following form.

    \begin{align*}
        u(x, t) = 100 + \sum_{n=1}^\infty \left[ a_n \sin\left(\frac{n\pi x}{L}\right) e^{-kt\frac{n\pi}{L}}\right] 
    \end{align*}

    We can calculate the values for $a_n$ using the Fourier Series equations.

    \begin{align*}
        u(x, 0) \Rightarrow -95 = 100 + \sum_{n=1}^\infty \left[ a_n \sin\left(\frac{n\pi x}{L}\right)\right]
    \end{align*}

    And

    \begin{align*}
        a_n &= \frac{2}{L} \int_0^L f(x) \sin\left(\frac{n\pi x}{L}\right) \, dx\\
        &= \frac{2}{L} \int_0^L -95 \sin\left(\frac{n\pi x}{L}\right) \, dx\\
        &= \frac{1}{n\pi} \pren{190 \cos(n\pi) - 190}
    \end{align*}

    Finally,

    \begin{align*}
        u(x, t) = 100 + \sum_{n=1}^\infty \left[ \pren{\frac{1}{n\pi} \pren{190 \cos(n\pi) - 190}} \sin\left(\frac{n\pi x}{L}\right) e^{-kt\frac{n\pi}{L}}\right] 
    \end{align*}

    @@ Which we can plot.

\simpleweave

\includegraphics[width= 4in]{/home/zoe/classwork/2015b/appm4350/homeworks/figures/hw5_figure6_1.pdf}

\nosimpleweave

    @@ To determine when it's cooked, we've established that it is cooked when it has reached $75^\circ \, C$
    everywhere in the potato. We know that the center is the last affected area, so we can simply determine where the
    centerline ($x = 0.5$) is equal to $75$. This yields the following equation.

    \begin{align*}
        u(0.5, t) = 100 + \sum_{n=1}^\infty \left[ \pren{\frac{1}{n\pi} \pren{190 \cos(n\pi) - 190}} \sin\left(\frac{n\pi 0.5}{L}\right) e^{-kt\frac{n\pi}{L}}\right] 
    \end{align*}

    Plotting again we see our centerline.

\simpleweave

\includegraphics[width= 4in]{/home/zoe/classwork/2015b/appm4350/homeworks/figures/hw5_figure7_1.pdf}

\nosimpleweave

    We can solve numerically for how long this takes.

\weave

\begin{minted}[mathescape, fontsize=\small, xleftmargin=0.5em]{python}
data = potato[int(len(x) / 2), :]
print(t[np.where(data >= 75)[0][0]])
\end{minted}
\begin{minted}[fontsize=\small, xleftmargin=0.5em, mathescape, frame = leftline]{text}
95

\end{minted}

\noweave

Therefore the potato will be cooked after approximately $t=95 s$.

    @ The boiling temperature of water is $100^\circ C$ at sea level. In Boulder, the air pressure is lower, and water
    boils at about $94^\circ C$. Does it take longer or shorter to cook a slice (1 cm thick) of potato in Boulder, CO
    than it does in Boston, MA? By how much?\\

    We can simply recalculate everything using $94^\circ C$ as our base temperature. It follows that our steady state is
    now $94$, meaning our final solution is now the following. Our $a_n$ calculation also changes.

    \begin{align*}
        u(x, t) = 94 + \sum_{n=1}^\infty \left[ \pren{\frac{1}{n\pi} \pren{178 \cos(n\pi) - 178}} \sin\left(\frac{n\pi x}{L}\right) e^{-kt\frac{n\pi}{L}}\right] 
    \end{align*}

    Redoing the above calulations, we can produce the same plots.

\simpleweave

\includegraphics[width= 4in]{/home/zoe/classwork/2015b/appm4350/homeworks/figures/hw5_figure9_1.pdf}

\nosimpleweave

    We examine the centerline again.

\simpleweave

\includegraphics[width= 4in]{/home/zoe/classwork/2015b/appm4350/homeworks/figures/hw5_figure10_1.pdf}

\nosimpleweave

    We can solve numerically for how long this takes.

\weave

\begin{minted}[mathescape, fontsize=\small, xleftmargin=0.5em]{python}
data = potato[int(len(x) / 2), :]
print(t[np.where(data >= 75)[0][0]])
\end{minted}
\begin{minted}[fontsize=\small, xleftmargin=0.5em, mathescape, frame = leftline]{text}
110

\end{minted}

\noweave

    This takes $110s$ to cook, which is about $15s$ longer.

    @ Suppose you deep-fried the potato (in hot oil) instead of boiling it (in water). Now the temperature at the
    boundaries might be $200^\circ C$ instead of $100^\circ C$. How would the cooking time change if the thickness
    remains $1 cm$? \\

    Again, we perform the same calculations.

    \begin{align*}
        u(x, t) = 200 + \sum_{n=1}^\infty \left[ \pren{\frac{1}{n\pi} \pren{390 \cos(n\pi) - 390}} \sin\left(\frac{n\pi x}{L}\right) e^{-kt\frac{n\pi}{L}}\right] 
    \end{align*}

    Redoing the above calulations, we can produce the same plots.

\simpleweave

\includegraphics[width= 4in]{/home/zoe/classwork/2015b/appm4350/homeworks/figures/hw5_figure12_1.pdf}

\nosimpleweave

    We examine the centerline again.

\simpleweave

\includegraphics[width= 4in]{/home/zoe/classwork/2015b/appm4350/homeworks/figures/hw5_figure13_1.pdf}

\nosimpleweave

    We can solve numerically for how long this takes.

\weave

\begin{minted}[mathescape, fontsize=\small, xleftmargin=0.5em]{python}
data = potato[int(len(x) / 2), :]
print(t[np.where(data >= 75)[0][0]])
\end{minted}
\begin{minted}[fontsize=\small, xleftmargin=0.5em, mathescape, frame = leftline]{text}
45

\end{minted}

\noweave

    This only takes $45s$! So oil is the best way to go\ldots

    \end{easylist}

    \newpage

    \subsection{Round Potato}
    What if you didn't cut the potato at all? The one-dimensional heat equation is irrelevant for a three-dimensional
    potato. In spherical coordinates, the three-dimensional heat equation is

    \[
        u_t = k \left[ \frac{1}{r^2} {\left(r^2 u_r\right)}_r + \frac{1}{r^2 \sin \theta} {(\sin\theta u_\theta)}_\theta + \frac{1}{{(r \sin \theta)}^2} u_{\phi\phi} \right]
    \]

    We also need that the boundary conditions on the surface of the sphere are spatially uniform, and that the initial
    conditions are radially symmetric in the interior of the sphere.  Then the problem becomes

    \begin{align*}
        u_t = \frac{k}{r^2} {(r^2 u_r)}_r \qquad & 0 < r < R, t > 0, k > 0\\
        u(0, t) = \text{Bounded}, u(R, t) = 100 \qquad & t > 0\\
        u(r, 0) = 5 \qquad & 0 \le r \le R\\
    \end{align*}

    \begin{easylist}[enumerate]
    @ Find the steady-state solution of this problem.  Let $v(r, t)$ denote the solution of the transient problem, so
    $u(r, t) = $steady-state$+ v(r, t)$. Write down the equations for $v(r, t)$.

    As before, set $u_t = 0$. Therefore we get the simple problem.

    \begin{align*}
        u_t = 0 &= \frac{k}{r^2} {(r^2u_r)}_r\\
        \overline{u}(x) = A + \frac{B}{r}\\
    \end{align*}

    Using the boundary conditions we get the final form.

    \begin{align*}
        \overline{u}(x) &= 100\\
    \end{align*}

    Again, intuitively this makes sense. The potato will cook.

    Now we can determine the transient solution.

    Let us assume that the solution of $u(x, t)$ has the following form.

    \begin{align*}
        u(x, t) = \overline{u}(x, t) + v(x, t)
    \end{align*}

    Substituting back into the original equation we get

    \begin{align*}
        {[\overline{u} + v]}_t = \frac{k}{r^2} {[r^2 (\overline{u} + v)_r]}_r
    \end{align*}

    Distributing the Linear Operator $\partial$ we get

    \begin{align*}
        \overline{u}_t + v_t &= k\overline{u}_{rr} + \frac{2k\overline{u}_r}{r} + kv_{rr} + \frac{2kv_r}{r}
    \end{align*}

    We know the form of the steady state solution, therefore the equation we can solve is 

    \begin{align*}
        v_t &= kv_{rr} + \frac{2kv_r}{r}
    \end{align*}

    In order to apply separation of variables we need to adjust our Initial Condition and Boundary Conditions a little.
    We can rewrite these conditions as the following.

    \begin{align*}
        u(0, t) &= \beta\\
        u(R, t) &= 100\\
        u(r, 0) &= f(x) - 100 = -95
    \end{align*}

    @ In the transient problem, set

    \[
        v(r, t) = \frac{w(r, t)}{r}
    \]

    Write down the equations for $w(r, t)$.\\

    Using the Substitution, we get the following equations.

    \begin{align*}
        v_t &= kv_{rr} + \frac{2kv_r}{r}\\
        \frac{w_t(r, t)}{r} &= k\pren{\frac{w(r, t)}{r}}_{rr} + \frac{2k\pren{\frac{w(r, t)}{r}}_r}{r}\\
        \frac{w_t(r, t)}{r} &= k\pren{\frac{w_r(r, t)}{r} - \frac{w(r, t)}{r^2}}_r + \frac{2k\pren{\frac{w_r(r, t)}{r} - \frac{w(r, t)}{r^2}}}{r}\\
        \frac{w_t(r, t)}{r} &= \pren{\frac{k}{r} \left(w_{rr}{\left (r , t\right )} - \frac{2}{r} w_r{\left (r, t \right)} + \frac{2}{r^{2}} w{\left (r, t \right )}\right)} +
                                \frac{2k\pren{\frac{w_r(r, t)}{r} - \frac{w(r, t)}{r^2}}}{r}\\
        \frac{w_t(r, t)}{r} &= \frac{k}{r} w_{rr}{\left (r , t\right )} - \frac{2k}{r^2} w_r{\left (r, t \right)} + \frac{2k}{r^{3}} w{\left (r, t \right )} +
                                \frac{2k\pren{\frac{w_r(r, t)}{r} - \frac{w(r, t)}{r^2}}}{r}\\
        w_t(r, t) &= k w_{rr}{\left (r , t\right )} - \frac{2k}{r} w_r{\left (r, t \right)} + \frac{2k}{r^{2}} w{\left (r, t \right )} +
                                2k\pren{\frac{w_r(r, t)}{r} - \frac{w(r, t)}{r^2}}\\
        w_t(r, t) &= k w_{rr}{\left (r , t\right )} - \frac{2k}{r} w_r{\left (r, t \right)} + \frac{2k}{r^{2}} w{\left (r, t \right )} +
                                \frac{2kw_r(r, t)}{r} - \frac{2kw(r, t)}{r^2}\\
        w_t(r, t) &= k w_{rr}{\left (r , t\right )}\\
    \end{align*}

    Now we need to convert our initial condition.

    \begin{align*}
        w(r, 0) &= -95 r\\
    \end{align*}

    @ Find $w(r, t)$ in the usual way.\\

    Ok\ldots. Time for separation of variables.

    Assume that $w(r, t)$ is of the form $g(r)f(t)$. If we rewrite with this substitution, we get the following steps.

    \begin{align*}
        w_t(r, t) &= k w_{rr}{\left (r , t\right )}\\
        g(r)f^\prime(t) &= k \ddot{g}(r)f(t)\\
        \frac{f^\prime(t)}{kf(t)} &= \frac{\ddot{g}(r)}{g(r)} = \mu\\
    \end{align*}

    This yields two ODEs.

    \[
    \begin{cases}
        f^\prime(t) &= \mu k f(t)\\
        \ddot{g}(r) &= \mu g(r)
    \end{cases}
    \]

    The solution to $f(t)$ is easy.

    \begin{align*}
        f(t) &= e^{\mu k t}
    \end{align*}

    And we can set based on our BCs and IC that $g(0) = \beta r$ and $g(R) = 100 r$.

    We now need to compare the three cases for $\mu$.

    First, $\mu > 0$. This yields

    \begin{align*}
        g(r) &= A \sinh\pren{\sqrt{\mu}r} + B \cosh\pren{\sqrt{\mu}r}
    \end{align*}

    With BCs

    \begin{align*}
        g(0) &= 0 = B\\
        g(R) &= 0 = A \sinh\pren{\sqrt{\mu}R}
    \end{align*}

    There is no non-trivial solution.

    Now, $\mu = 0$:

    \begin{align*}
        g(r) &= Ar + B\\
        g(0) &= 0 = B\\
        g(R) &= 0 = AR\\
    \end{align*}

    There is no non-trivial solution.

    Finally, $\mu < 0$. Let $\mu = -\lambda^2$.

    \begin{align*}
        g(r) &= A \sin\pren{\lambda r} + B \cos\pren{\lambda r}\\
        g(0) &= 0 = B\\
        g(R) &= A \sin\pren{\lambda R}\\
    \end{align*}

    The above statement is true when $\lambda R = n \pi$, yielding our eigenvalues and functions.

    \begin{align*}
        \lambda_n &= \frac{n\pi}{R}\\
        \phi_n(r) &= \sin\pren{\lambda_n r}
    \end{align*}

    Yielding our penultimate equation:

    \begin{align*}
        w(r, t) &= \sum_{n=1}^\infty a_n \sin\pren{\frac{n\pi r}{R}} e^{-kt \pren{\frac{n\pi}{R}}^2}
    \end{align*}

    To find $a_n$, we use the Fourier Series formula.

    \begin{align*}
        a_n &= \frac{2}{R} \int_0^R w(r, 0) \cdot \sin\pren{\frac{n\pi r}{R}} \, dx\\
        &= \frac{2}{R} \int_0^R -95r \cdot \sin\pren{\frac{n\pi r}{R}} \, dx\\
        &= - \frac{190 R}{n^{2} \pi^{2}} \left(- n \pi \cos{\left (n \pi \right )} + \sin{\left (n \pi \right )}\right)
    \end{align*}

    With final equation:

    \begin{align*}
        w(r, t) &= \sum_{n=1}^\infty \pren{- \frac{190 R}{n^{2} \pi^{2}} \left(- n \pi \cos{\left (n \pi \right )} + \sin{\left (n \pi \right )}\right)} \sin\pren{\frac{n\pi r}{R}} e^{-kt \pren{\frac{n\pi}{R}}^2}
    \end{align*}

    @ Based on $w(r, t)$, find $v(r, t)$ and then $u(r, t)$, the physically measurable temperature.  For any $t > 0$,
    where in the potato does the minimum temperature occur? Using L'Hopital's Rule as needed, approximately when does
    the minimum temperature in the potato reach $75^\circ C$? How long does it take to cook the uncut potato?\\

    Based on the Substitution, we can see $v(r, t)$ is equal to the following.

    \begin{align*}
        v(r, t) &= \frac{1}{r} \sum_{n=1}^\infty \pren{- \frac{190 R}{n^{2} \pi^{2}} \left(- n \pi \cos{\left (n \pi \right )} + \sin{\left (n \pi \right )}\right)} \sin\pren{\frac{n\pi r}{R}} e^{-kt \pren{\frac{n\pi}{R}}^2}
    \end{align*}

    Yielding

    \begin{align*}
        u(r, t) &= -95 + \frac{1}{r} \sum_{n=1}^\infty \pren{- \frac{190 R}{n^{2} \pi^{2}} \left(- n \pi \cos{\left (n \pi \right )} + \sin{\left (n \pi \right )}\right)} \sin\pren{\frac{n\pi r}{R}} e^{-kt \pren{\frac{n\pi}{R}}^2}
    \end{align*}

    We can examine when $r = 0$ to determine when the potato is cooked, as this is the innermost temperature.

\simpleweave

\includegraphics[width= 4in]{/home/zoe/classwork/2015b/appm4350/homeworks/figures/hw5_figure15_1.pdf}

\nosimpleweave

    The potato will be cooked after approximately $30s$.
\end{easylist}


\end{document}
