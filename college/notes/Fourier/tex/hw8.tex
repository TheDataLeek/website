\documentclass[10pt]{article}

%%%%%%%%%%%%%%%%%%%%%%%%%%%%%%%%%%%%%%%%%%%%%%%%%%%%%%%%%%%%%%%%%%%%%%%%%%%%%%%%
% LaTeX Imports
%%%%%%%%%%%%%%%%%%%%%%%%%%%%%%%%%%%%%%%%%%%%%%%%%%%%%%%%%%%%%%%%%%%%%%%%%%%%%%%%
\usepackage{amsfonts}                                                   % Math fonts
\usepackage{amsmath}                                                    % Math formatting
\usepackage{amssymb}                                                    % Math formatting
\usepackage{amsthm}                                                     % Math Theorems
\usepackage{arydshln}                                                   % Dashed hlines
\usepackage{attachfile}                                                 % AttachFiles
\usepackage{cancel}                                                     % Cancelled math
\usepackage{caption}                                                    % Figure captioning
\usepackage{color}                                                      % Nice Colors
\input{./lib/dragon.inp}                                                % Tikz dragon curve
\usepackage[ampersand]{easylist}                                        % Easy lists
\usepackage{fancyhdr}                                                   % Fancy Header
\usepackage[T1]{fontenc}                                                % Specific font-encoding
%\usepackage[margin=1in, marginparwidth=2cm, marginparsep=2cm]{geometry} % Margins
\usepackage{graphicx}                                                   % Include images
\usepackage{hyperref}                                                   % Referencing
\usepackage[none]{hyphenat}                                             % Don't allow hyphenation
\usepackage{lipsum}                                                     % Lorem Ipsum Dummy Text
\usepackage{listings}                                                   % Code display
\usepackage{marginnote}                                                 % Notes in the margin
\usepackage{microtype}                                                  % Niceness
\usepackage{lib/minted}                                                 % Code display
\usepackage{./lib/mlptikz}                                              % Tikz mlp
\usepackage{multirow}                                                   % Multirow tables
\usepackage{pdfpages}                                                   % Include pdfs
\usepackage{pgfplots}                                                   % Create Pictures
\usepackage{rotating}                                                   % Figure rotation
\usepackage{setspace}                                                   % Allow double spacing
\usepackage{subcaption}                                                 % Figure captioning
\usepackage{tikz}                                                       % Create Pictures
\usepackage{tocloft}                                                    % List of Equations
%%%%%%%%%%%%%%%%%%%%%%%%%%%%%%%%%%%%%%%%%%%%%%%%%%%%%%%%%%%%%%%%%%%%%%%%%%%%%%%%
% Package Setup
%%%%%%%%%%%%%%%%%%%%%%%%%%%%%%%%%%%%%%%%%%%%%%%%%%%%%%%%%%%%%%%%%%%%%%%%%%%%%%%%
\hypersetup{%                                                           % Setup linking
    colorlinks=true,
    linkcolor=black,
    citecolor=black,
    filecolor=black,
    urlcolor=black,
}
\RequirePackage[l2tabu, orthodox]{nag}                                  % Nag about bad syntax
\renewcommand*\thesection{\arabic{section} }                             % Reset numbering
\renewcommand{\theFancyVerbLine}{ {\arabic{FancyVerbLine} } }              % Needed for code display
\renewcommand{\footrulewidth}{0.4pt}                                    % Footer hline
\setcounter{secnumdepth}{3}                                             % Include subsubsections in numbering
\setcounter{tocdepth}{3}                                                % Include subsubsections in toc
%%%%%%%%%%%%%%%%%%%%%%%%%%%%%%%%%%%%%%%%%%%%%%%%%%%%%%%%%%%%%%%%%%%%%%%%%%%%%%%%
% Custom commands
%%%%%%%%%%%%%%%%%%%%%%%%%%%%%%%%%%%%%%%%%%%%%%%%%%%%%%%%%%%%%%%%%%%%%%%%%%%%%%%%
\newcommand{\nvec}[1]{\left\langle #1 \right\rangle}                    %  Easy to use vector
\newcommand{\inprod}[2]{\left\langle \vec{#1}, \vec{#2} \right\rangle}  %  Easy to use inner product
\newcommand{\norm}[1]{\lvert \lvert \vec{#1} \rvert \rvert}             %  Easy to use norm
\newcommand{\ma}[0]{\mathbf{A} }                                         %  Easy to use vector
\newcommand{\mb}[0]{\mathbf{B} }                                         %  Easy to use vector
\newcommand{\abs}[1]{\left\lvert #1 \right\rvert}                       %  Easy to use abs
\newcommand{\pren}[1]{\left( #1 \right)}                                %  Big parens
\let\oldvec\vec
\renewcommand{\vec}[1]{\mathbf{#1} }                            %  Vector Styling
\newtheorem{thm}{Theorem}                                               %  Define the theorem name
\theoremstyle{definition}
\newtheorem{definition}{Definition}                                     %  Define the definition name
\newtheorem{ex}{Example}                                                %  Define the example name
\definecolor{bg}{rgb}{0.95,0.95,0.95}
\newcommand{\java}[4]{\vspace{10pt}\inputminted[firstline=#2,
                                 lastline=#3,
                                 firstnumber=#2,
                                 gobble=#4,
                                 frame=single,
                                 label=#1,
                                 bgcolor=bg,
                                 linenos]{java}{#1} }
\newcommand{\python}[4]{\vspace{10pt}\inputminted[firstline=#2,
                                 lastline=#3,
                                 firstnumber=#2,
                                 gobble=#4,
                                 frame=single,
                                 label=#1,
                                 bgcolor=bg,
                                 linenos]{python}{#1} }
\newcommand{\js}[4]{\vspace{10pt}\inputminted[firstline=#2,
                                 lastline=#3,
                                 firstnumber=#2,
                                 gobble=#4,
                                 frame=single,
                                 label=#1,
                                 bgcolor=bg,
                                 linenos]{js}{#1} }
%%%%%%%%%%%%%%%%%%%%%%%%%%%%%%%%%%%%%%%%%%%%%%%%%%%%%%%%%%%%%%%%%%%%%%%%%%%%%%%%
% Beginning of document items - headers, title, toc, etc...
%%%%%%%%%%%%%%%%%%%%%%%%%%%%%%%%%%%%%%%%%%%%%%%%%%%%%%%%%%%%%%%%%%%%%%%%%%%%%%%%
\pagestyle{fancy}                                                       %  Establishes that the headers will be defined
\fancyhead[LE,LO]{Matrix Methods Notes}                                  %  Adds header to left
\fancyhead[RE,RO]{Zoe Farmer}                                       %  Adds header to right
\cfoot{\mlptikz[size=0.25in, text=on, textposx=0, textposy=0, textvalue=\thepage, textscale=0.75in]{applejack} }
\lfoot{APPM 3310}
\rfoot{Beylkin}
\title{Matrix Methods Notes}
\author{Zoe Farmer}

%%%%%%%%%%%%%%%%%%%%%%%%%%%%%%%%%%%%%%%%%%%%%%%%%%%%%%%%%%%%%%%%%%%%%%%%%%%%%%%%
% Beginning of document items - headers, title, toc, etc...
%%%%%%%%%%%%%%%%%%%%%%%%%%%%%%%%%%%%%%%%%%%%%%%%%%%%%%%%%%%%%%%%%%%%%%%%%%%%%%%%
\pagestyle{fancy}                                                       %  Establishes that the headers will be defined
\fancyhead[LE,LO]{Fourier Series}                                  %  Adds header to left
\fancyhead[RE,RO]{Zoe Farmer}                                       %  Adds header to right
\cfoot{\thepage}
\lfoot{APPM 4350}
\rfoot{Mark Hoefer}
\title{Fourier Series Homework Eight}
\author{Zoe Farmer}
%%%%%%%%%%%%%%%%%%%%%%%%%%%%%%%%%%%%%%%%%%%%%%%%%%%%%%%%%%%%%%%%%%%%%%%%%%%%%%%%
% Beginning of document items - headers, title, toc, etc...
%%%%%%%%%%%%%%%%%%%%%%%%%%%%%%%%%%%%%%%%%%%%%%%%%%%%%%%%%%%%%%%%%%%%%%%%%%%%%%%%
\begin{document}



\maketitle

\section{Ambulance Problem}

You are standing in front of your home and begin to hear an ambulance some distance away. Because you have been enjoying
APPM 4350/5350 so much, you think to yourself, ``How can I model the sound coming from this ambulance?'' Being a good
applied mathematician, you decide to start with the simplest model available, the one-dimensional wave equation. But the
ambulance is moving and you are not so you consider the moving boundary value problem

\mn{%
    u_{tt} &= c_s^2 u_{xx} \qquad t > 0 x \in (c_a(t), \infty) \nonumber\\
    u(c_a(t), t) &= p(t) \qquad t > 0 \\
    u(x, 0) &= u_t(x, 0) = 0 \qquad x > 0 \nonumber
}\label{eq:mvbdd}

Where $c_s > 0$ is the speed of sound in air, $x = c_a(t)$ is the position of the ambulance at time $t > 0$, $u(x, t)$ is
the air pressure, and $p(t)$ is the pressure disturbance emitted by the ambulance's siren. If the ambulance is
approaching you, then $c_a > 0$, if it is moving away from you, then $c_a < 0$. Although this may be an unrealistic
model of sound propagation in three spatial dimensions, you make the very realistic assumption that $\abs{c_a} < c_s$,
i.e., the ambulance does not travel at supersonic speeds. You assume that the air is initially motionless.

\begin{easylist}[enumerate]
    @ Solve this moving boundary value problem.

    Since the ambulance is not moving faster than the speed of sound we can make the base assumption that we will hear
    $p(t)$, just at a different time then it is emitted.

    @ What are the restrictions on $p(t)$ to ensure that your solution satisfies~\eqref{eq:mvbdd}?

    @ You now model the ambulance's siren as a pressure disturbance with frequency $f_a > 0$, $p(t)=p_0\sin(2\pi f_at)$.
    Suppose your home is at $x = L > 0$. Determine the frequency you hear (before the ambulance arrives). How does this
    frequency differ from $f_a$ for $c_a > 0$? For $c_a < 0$? This is called the doppler effect.
\end{easylist}

\newpage
\section{Problem 2.5.1.b \& f}

Solve Laplace's Equation inside a rectangle $0 \le x \le L$, $0 \le y \le H$, with the following boundary conditions.

For both, we seek the separated solution,

\[
    u(x, y) = F(x)G(y)
\]

Yielding two ODEs

\[
    \begin{cases}
        F^{\prime\prime}(x) = -\mu F(x)\\
        G^{\prime\prime}(y) = \mu F(y)
    \end{cases}
\]

\begin{easylist}[enumerate]
    @ \ms{%
        u_x(0, y) &= g(y)\\
        u_x(L, y) &= 0\\
        u(x, 0) &= 0\\
        u(x, H) &= 0
    }

    @@ Since $u(x, 0) = u(x, H) = 0$, we can ignore these sides, as the result would be zero.

    @@ Examining the side $u_x(0, y) = g(y)$ first, we know, using our boundary conditions, that $G(0) = G(H) = 0$. This
    means that if $\mu = -\lambda^2$

    \ms{%
        G(y) = A \sin(\lambda y) + B \cos(\lambda y)
    }

    Yielding

    \ms{%
        \phi_n(y) &= \sin \lambda_n y\\
        \lambda_n &= \frac{n \pi}{H}
    }

    Therefore,

    \ms{%
        F(x) = \cosh\bren{\frac{n\pi}{H}\pren{x-L}}
    }

    Meaning that for this first side,

    \ms{%
        u_1(x, y) &= \sum_{n=1}^\infty \bren{A_n \cosh\bren{\frac{n\pi}{H}\pren{x-L}} \sin\pren{\frac{n\pi y}{H}}}\\
        u_1^\prime(0, y) = g(y) &= \sum_{n=1}^\infty
            \bren{A_n \cosh\bren{\frac{n\pi}{H}\pren{-L}} \sin\pren{\frac{n\pi y}{H}}}^\prime\\
        g(y) &= \sum_{n=1}^\infty
            \bren{A_n \frac{n \pi}{H} \sinh{\pren{\frac{n \pi}{H} \pren{-L}}} \sin\pren{\frac{n\pi y}{H}}}
    }

    Therefore, since this is just a Fourier series,

    \ms{%
        A_n \frac{n \pi}{H} \sinh{\pren{\frac{n \pi}{H} \pren{-L}}} &=
            \frac{2}{H} \int_0^L g(y) \sin\pren{\frac{n\pi y}{H}} \, dy\\
        A_n &= \frac{2}{n \pi \sinh{\pren{\frac{n \pi}{H} \pren{-L}}}}
            \int_0^L g(y) \sin\pren{\frac{n\pi y}{H}} \, dy
    }

    Yielding our final solution for $u_1(x, y)$

    \ms{%
        u_1(x, y) &= \sum_{n=1}^\infty \bren{\pren{
        \frac{2}{n \pi \sinh{\pren{\frac{n \pi}{H} \pren{-L}}}}
            \int_0^L g(y) \sin\pren{\frac{n\pi y}{H}} \, dy
        }\cosh\bren{\frac{n\pi}{H}\pren{x-L}} \sin\pren{\frac{n\pi y}{H}}}\\
    }

    @@ We now can examine the case where $u_x(L, y) = 0$. Similar to above, we get

    \ms{%
        \phi_n(y) &= \sin\pren{\lambda_n y}\\
        \lambda_n &= \frac{n\pi}{H}\\
        F(x) &= \cosh\bren{\frac{n\pi x}{H}}
    }

    Yielding this solution to be

    \ms{%
        u_2(x, y) &= \sum_{n=1}^\infty
        \bren{A_n \cosh\bren{\frac{n\pi x}{H}} \sin\pren{\frac{n\pi y}{H}}}
    }

    With $A_n$

    \ms{%
        0 &= \sum_{n=1}^\infty \bren{A_n \frac{n \pi}{H} \sinh{\pren{\frac{n \pi L}{H}}} \sin\pren{\frac{n\pi y}{H}}}
    }

    Again, this is just a Fourier series, yielding $A_n = 0$

    @@ Therefore our final solution is of the form $u_1 + u_2$

    \ms{%
        u_1(x, y) &= \sum_{n=1}^\infty \bren{\pren{
        \frac{2}{n \pi \sinh{\pren{\frac{n \pi}{H} \pren{-L}}}}
            \int_0^L g(y) \sin\pren{\frac{n\pi y}{H}} \, dy
        }\cosh\bren{\frac{n\pi}{H}\pren{x-L}} \sin\pren{\frac{n\pi y}{H}}}\\
    }

    @ \ms{%
        u(0, y) &= f(y)\\
        u(L, y) &= 0\\
        u_y(x, 0) &= 0\\
        u_y(x, H) &= 0
    }

    We can examine this problem one side at a time.

    @@ $u(0, y) = f(y)$

    For $G(y)$ we know

    \ms{%
        \phi_n(y) &= \sin(\lambda_n y)\\
        \lambda_n &= \frac{n\pi}{H}
    }

    Yielding $F(x)$

    \ms{%
        F(x) = \sinh\bren{\frac{n\pi}{H} \pren{x - L}}
    }

    Yielding the solution $u_1(x, y)$

    \ms{%
        u_1(x, y) &= \sum_{n=1}^\infty A_n \sinh\bren{\frac{n\pi}{H} (x - L)} \sin\pren{\frac{n\pi y}{H}}\\
        A_n &= \frac{-2}{H\sinh\bren{\frac{L n\pi}{H}}} \int_0^H f(y) \sin\pren{\frac{n\pi y}{H}} \, dy\\
        u_1(x, y) &= \sum_{n=1}^\infty
            \pren{
                \frac{-2}{H\sinh\bren{\frac{L n\pi}{H}}} \int_0^H f(y) \sin\pren{\frac{n\pi y}{H}} \, dy
            } \sinh\bren{\frac{n\pi}{H} (x - L)} \sin\pren{\frac{n\pi y}{H}}\\
    }

    @@ $u(L, y) = 0$

    This one is easy, as $u_2(x, y) = 0$ for any $x$ or $y$.

    @@ $u_x(x, 0) = 0$

    For this we examine $F(x)$, and get

    \ms{%
        \phi_n(x) &= \sin(\lambda_n x)\\
        \lambda_n &= \frac{n\pi}{L}
    }

    Examining $G(y)$ we get

    \ms{%
        G(y) = \sinh\bren{\frac{n\pi}{L} \pren{y - H}}
    }

    Yielding solution

    \ms{%
        u_3(x, y) &= \sum_{n=1}^\infty
            A_n \sinh\bren{\frac{n\pi}{L} \pren{y - H}} \sin\pren{\frac{n\pi x}{L}}\\
        u_3^\prime(x, y) &= \sum_{n=1}^\infty
            A_n \frac{n\pi}{L} \cosh\bren{\frac{n\pi}{L} \pren{y - H}} \sin\pren{\frac{n\pi x}{L}}\\
        u_3^\prime(x, 0) &= \sum_{n=1}^\infty
            A_n \frac{n\pi}{L} \cosh\bren{\frac{Hn\pi}{L}} \sin\pren{\frac{n\pi x}{L}}\\
        A_n &= 0\\
        u_3(x, y) &= 0
    }

    @@ $u_x(x, H) = 0$

    This is the same as above, yielding final solution

    \ms{%
        u(x, y) &= \sum_{n=1}^\infty
            \pren{
                \frac{-2}{H\sinh\bren{\frac{L n\pi}{H}}} \int_0^H f(y) \sin\pren{\frac{n\pi y}{H}} \, dy
            } \sinh\bren{\frac{n\pi}{H} (x - L)} \sin\pren{\frac{n\pi y}{H}}\\
        }
\end{easylist}

\newpage
\section{Problem 2.5.8.a \& b}

Solve Laplace's equation inside a circular annulus $(a < r< b)$ subject to the boundary conditions.

\begin{easylist}[enumerate]
    @ \ms{%
        u(a, \theta) &= f(\theta)\\
        u(b, \theta) &= g(\theta)
    }

    @ \ms{%
        u_r(a, \theta) &= 0\\
        u(b, \theta) &= g(\theta)
    }
\end{easylist}

\end{document}
