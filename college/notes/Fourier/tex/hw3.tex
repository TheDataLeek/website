\documentclass[10pt]{article}

%%%%%%%%%%%%%%%%%%%%%%%%%%%%%%%%%%%%%%%%%%%%%%%%%%%%%%%%%%%%%%%%%%%%%%%%%%%%%%%%
% LaTeX Imports
%%%%%%%%%%%%%%%%%%%%%%%%%%%%%%%%%%%%%%%%%%%%%%%%%%%%%%%%%%%%%%%%%%%%%%%%%%%%%%%%
\usepackage{amsfonts}                                                   % Math fonts
\usepackage{amsmath}                                                    % Math formatting
\usepackage{amssymb}                                                    % Math formatting
\usepackage{amsthm}                                                     % Math Theorems
\usepackage{arydshln}                                                   % Dashed hlines
\usepackage{attachfile}                                                 % AttachFiles
\usepackage{cancel}                                                     % Cancelled math
\usepackage{caption}                                                    % Figure captioning
\usepackage{color}                                                      % Nice Colors
\input{./lib/dragon.inp}                                                % Tikz dragon curve
\usepackage[ampersand]{easylist}                                        % Easy lists
\usepackage{fancyhdr}                                                   % Fancy Header
\usepackage[T1]{fontenc}                                                % Specific font-encoding
%\usepackage[margin=1in, marginparwidth=2cm, marginparsep=2cm]{geometry} % Margins
\usepackage{graphicx}                                                   % Include images
\usepackage{hyperref}                                                   % Referencing
\usepackage[none]{hyphenat}                                             % Don't allow hyphenation
\usepackage{lipsum}                                                     % Lorem Ipsum Dummy Text
\usepackage{listings}                                                   % Code display
\usepackage{marginnote}                                                 % Notes in the margin
\usepackage{microtype}                                                  % Niceness
\usepackage{lib/minted}                                                 % Code display
\usepackage{multirow}                                                   % Multirow tables
\usepackage{pdfpages}                                                   % Include pdfs
\usepackage{pgfplots}                                                   % Create Pictures
\usepackage{rotating}                                                   % Figure rotation
\usepackage{setspace}                                                   % Allow double spacing
\usepackage{subcaption}                                                 % Figure captioning
\usepackage{tikz}                                                       % Create Pictures
\usepackage{tocloft}                                                    % List of Equations
%%%%%%%%%%%%%%%%%%%%%%%%%%%%%%%%%%%%%%%%%%%%%%%%%%%%%%%%%%%%%%%%%%%%%%%%%%%%%%%%
% Package Setup
%%%%%%%%%%%%%%%%%%%%%%%%%%%%%%%%%%%%%%%%%%%%%%%%%%%%%%%%%%%%%%%%%%%%%%%%%%%%%%%%
\hypersetup{%                                                           % Setup linking
    colorlinks=true,
    linkcolor=black,
    citecolor=black,
    filecolor=black,
    urlcolor=black,
}
\RequirePackage[l2tabu, orthodox]{nag}                                  % Nag about bad syntax
\renewcommand*\thesection{\arabic{section} }                             % Reset numbering
\renewcommand{\theFancyVerbLine}{ {\arabic{FancyVerbLine} } }              % Needed for code display
\renewcommand{\footrulewidth}{0.4pt}                                    % Footer hline
\setcounter{secnumdepth}{3}                                             % Include subsubsections in numbering
\setcounter{tocdepth}{3}                                                % Include subsubsections in toc
%%%%%%%%%%%%%%%%%%%%%%%%%%%%%%%%%%%%%%%%%%%%%%%%%%%%%%%%%%%%%%%%%%%%%%%%%%%%%%%%
% Custom commands
%%%%%%%%%%%%%%%%%%%%%%%%%%%%%%%%%%%%%%%%%%%%%%%%%%%%%%%%%%%%%%%%%%%%%%%%%%%%%%%%
\newcommand{\nvec}[1]{\left\langle #1 \right\rangle}                    %  Easy to use vector
\newcommand{\ma}[0]{\mathbf{A} }                                         %  Easy to use vector
\newcommand{\mb}[0]{\mathbf{B} }                                         %  Easy to use vector
\newcommand{\abs}[1]{\left\lvert #1 \right\rvert}                       %  Easy to use abs
\newcommand{\pren}[1]{\left( #1 \right)}                                %  Big parens
\let\oldvec\vec
\renewcommand{\vec}[1]{\oldvec{\mathbf{#1} } }                            %  Vector Styling
\newtheorem{thm}{Theorem}                                               %  Define the theorem name
\newtheorem{definition}{Definition}                                     %  Define the definition name
\definecolor{bg}{rgb}{0.95,0.95,0.95}
\newcommand{\java}[4]{\vspace{10pt}\inputminted[firstline=#2,
                                 lastline=#3,
                                 firstnumber=#2,
                                 gobble=#4,
                                 frame=single,
                                 label=#1,
                                 bgcolor=bg,
                                 linenos]{java}{#1} }
\newcommand{\python}[4]{\vspace{10pt}\inputminted[firstline=#2,
                                 lastline=#3,
                                 firstnumber=#2,
                                 gobble=#4,
                                 frame=single,
                                 label=#1,
                                 bgcolor=bg,
                                 linenos]{python}{#1} }
\newcommand{\js}[4]{\vspace{10pt}\inputminted[firstline=#2,
                                 lastline=#3,
                                 firstnumber=#2,
                                 gobble=#4,
                                 frame=single,
                                 label=#1,
                                 bgcolor=bg,
                                 linenos]{js}{#1} }
%%%%%%%%%%%%%%%%%%%%%%%%%%%%%%%%%%%%%%%%%%%%%%%%%%%%%%%%%%%%%%%%%%%%%%%%%%%%%%%%
% Beginning of document items - headers, title, toc, etc...
%%%%%%%%%%%%%%%%%%%%%%%%%%%%%%%%%%%%%%%%%%%%%%%%%%%%%%%%%%%%%%%%%%%%%%%%%%%%%%%%
\pagestyle{fancy}                                                       %  Establishes that the headers will be defined
\fancyhead[LE,LO]{Computer Systems Notes}                                  %  Adds header to left
\fancyhead[RE,RO]{Zoe Farmer}                                       %  Adds header to right
\cfoot{ \thepage }
\lfoot{CSCI 2400}
\rfoot{Han}
\title{Computer Systems Notes}
\author{Zoe Farmer}

%%%%%%%%%%%%%%%%%%%%%%%%%%%%%%%%%%%%%%%%%%%%%%%%%%%%%%%%%%%%%%%%%%%%%%%%%%%%%%%%
% Beginning of document items - headers, title, toc, etc...
%%%%%%%%%%%%%%%%%%%%%%%%%%%%%%%%%%%%%%%%%%%%%%%%%%%%%%%%%%%%%%%%%%%%%%%%%%%%%%%%
\pagestyle{fancy}                                                       %  Establishes that the headers will be defined
\fancyhead[LE,LO]{Fourier Series}                                  %  Adds header to left
\fancyhead[RE,RO]{Zoe Farmer}                                       %  Adds header to right
\cfoot{\thepage}
\lfoot{APPM 4350}
\rfoot{Mark Hoefer}
\title{Fourier Series Homework Three}
\author{Zoe Farmer}
%%%%%%%%%%%%%%%%%%%%%%%%%%%%%%%%%%%%%%%%%%%%%%%%%%%%%%%%%%%%%%%%%%%%%%%%%%%%%%%%
% Beginning of document items - headers, title, toc, etc...
%%%%%%%%%%%%%%%%%%%%%%%%%%%%%%%%%%%%%%%%%%%%%%%%%%%%%%%%%%%%%%%%%%%%%%%%%%%%%%%%
\begin{document}




\maketitle

\begin{easylist}[enumerate]
    @ Consider the following functions, defined for all real $x$, $-\infty < x < \infty$,

    \begin{align*}
        f(x) = \begin{cases}
            \cos(\sqrt{x}), \quad &x > 0\\
            \cos(\sqrt{-x}), \quad &x < 0\\
        \end{cases}\\
        f(0) = 1\\
        g(x) = \begin{cases}
            \sin(\sqrt{x}), \quad &x > 0\\
            \sin(\sqrt{-x}), \quad &x < 0\\
        \end{cases}\\
        g(0) = 1\\
        h(x) = \exp(\cos(x))\\
        j(x) = \cos\left(\frac{\pi}{2}e^x\right)\\
    \end{align*}

    @@ For each function, sketch the graph of the function over an open interval that includes $x=0$ and at least two
    zeros of the function on each side of $x=0$. If this is impossible explain why.\\

    We cannot include any zeros of $h(x)$, as it never is less than zero.

\weave

\begin{minted}[mathescape, fontsize=\small, xleftmargin=0.5em]{python}
def f(x):
    return np.array([np.cos(np.sqrt(x0))
        if x0 != 0 else 1 for x0 in np.abs(x)])
def g(x):
    return np.array([np.sin(np.sqrt(x0))
        if x0 != 0 else 1 for x0 in np.abs(x)])
h = lambda x: np.exp(np.cos(x))
j = lambda x: np.cos( (np.pi * np.exp(x)) / 2 )

w = np.arange(-60, 60, 0.01)
x = np.arange(-100, 100, 0.01)
y = np.arange(-10, 10, 0.01)
z = np.arange(-5, 5, 0.01)

fig, axarr = plt.subplots(2, 2, figsize=(8,8))
axarr[0, 0].plot(w, f(w))
axarr[0, 0].set_title(r'$f(x)$')
axarr[0, 1].plot(x, g(x))
axarr[0, 1].set_title(r'$g(x)$')
axarr[1, 0].plot(y, h(y))
axarr[1, 0].set_title(r'$h(x)$')
axarr[1, 1].plot(z, j(z))
axarr[1, 1].set_title(r'$j(x)$')
plt.show()
\end{minted}
\includegraphics[width= 4in]{/home/zoe/classwork/2015b/appm4350/homeworks/figures/hw3_figure2_1.pdf}

\noweave

    @@ Is the function odd, even, or neither? Justify.

    @@@ $f(x)$ is Even, as it is symmetrical about the $y$ axis.

    @@@ $g(x)$ is also Even, because again it is symmetrical about the $y$ axis.

    @@@ $h(x)$ is Even, for the same reason.

    @@@ $j(x)$ is neither, as it has no symmetry, either about the $y$ axis or reflected across the origin.

    @@ Is the function periodic in $x$? What is its period?

    @@@ $f(x)$ is not periodic.

    @@@ $g(x)$ is not periodic.

    @@@ $h(x)$ is periodic, with period $2 \pi$.

    @@@ $j(x)$ is not periodic.

    @@ Is the function continuous in $x$? If not, where are the discontinuities?

    @@@ $f(x)$ is continuous.

    @@@ $g(x)$ is not continuous, as there exists a discontinuity at $x=0$.

    @@@ $h(x)$ is continuous.

    @@@ $j(x)$ is continuous.

    @@ Is the function piecewise continuous for all real $x$? Justify.

    @@@ $f(x)$ is POC as it is also continuous.

    @@@ $g(x)$ is POC, since it only has 1 discontinuity.

    @@@ $h(x)$ is POC as it is also continuous.

    @@@ $j(x)$ is POC as it is also continuous.

    @@ Is the function piecewise smooth for all real $x$? Justify.

    @@@ $f(x)$ is POS as it is also continuous.

    @@@ $g(x)$ is POS smooth, as you can differentiate the left and right hand sides separately and obtain the
    derivatives.

    @@@ $h(x)$ is POS as it is also continuous.

    @@@ $j(x)$ is POS as it is also continuous.

    \newpage
    @ Consider $f(x)$, given above, but defined only on $-\pi^2 \le x \le \pi^2$.

\weave

\begin{minted}[mathescape, fontsize=\small, xleftmargin=0.5em]{python}
x = np.arange(-np.pi**2, np.pi**2, 0.01)
fig = plt.figure()
ax = fig.add_axes([0.1, 0.1, 0.8, 0.8])
ax.plot(x, f(x))
plt.grid()
plt.show()
\end{minted}
\includegraphics[width= 4in]{/home/zoe/classwork/2015b/appm4350/homeworks/figures/hw3_figure3_1.pdf}

\noweave

    @@ Write down the formulae that define each of its Fourier coefficients in terms of an integral.

    \begin{align*}
        a_0 &= \frac{1}{\pi^2} \int_0^{\pi^2} \cos(\sqrt{x}) \, dx\\
        a_n &= \frac{2}{\pi^2} \int_0^{\pi^2} \cos(\sqrt{x}) \cos\left(\frac{nx}{\pi}\right) \, dx\\
        b_n &= \frac{2}{\pi^2} \int_0^{\pi^2} \cos(\sqrt{x}) \sin\left(\frac{nx}{\pi}\right) \, dx\\
    \end{align*}

    @@ It's not hard to find upper bounds on these coefficients. Based on (a), find an upper bound for $\abs{a_0}$ and a
    single bound that is valid for all $a_n$ and $b_n$.\\

    We know that $\abs{\cos(y)} \le 1$ and $\abs{\sin(y)} \le 1$ for any value of $y$, therefore we can rewrite our
    equations as such.

    \begin{align*}
        a_0 &= \frac{1}{\pi^2} \int_0^{\pi^2} 1 \, dx\\
        a_n &= \frac{2}{\pi^2} \int_0^{\pi^2} 1 \, dx\\
        b_n &= \frac{2}{\pi^2} \int_0^{\pi^2} 1 \, dx\\
    \end{align*}

    We can now solve for each upper bound.

    \begin{align*}
        \abs{a_0} &\le \frac{1}{\pi^2} \int_0^{\pi^2} 1 \, dx\\
                    &\le 1\\
    \end{align*}

    We want to add the two to get the maximum.

    \begin{align*}
        \abs{a_n, b_n} &\le \frac{2}{\pi^2} \int_0^{\pi^2} 1 \, dx\\
        & \le 2
    \end{align*}

    We can cheat a little and test our bounds numerically.

\weave

\begin{minted}[mathescape, fontsize=\small, xleftmargin=0.5em]{python}
a0 = np.abs((1 / (np.pi**2)) * si.quad(
        lambda x: np.cos(np.sqrt(x)), 0, np.pi**2)[0])
print('{} <= 1 -- {}'.format(a0, a0 <= 1))
for n in range(1, 6):
    an = np.abs((2 / (np.pi**2)) * si.quad(
            lambda x: np.cos(np.sqrt(x)) *
            np.cos(n * x / np.pi), 0, np.pi**2)[0])
    bn = np.abs((2 / (np.pi**2)) * si.quad(
            lambda x: np.cos(np.sqrt(x)) *
            np.sin(n * x / np.pi), 0, np.pi**2)[0])
    upper = (2 / (np.pi**2)) * si.quad(
            lambda x: 1, 0, np.pi**2)[0]
    print('{}, {} <= {} -- {}'.format(an, bn, upper,
            an <= upper and bn <= upper))
\end{minted}
\begin{minted}[fontsize=\small, xleftmargin=0.5em, mathescape, frame = leftline]{text}
0.4052847345693511 <= 1 -- True
0.7698141397489887, 0.6266813971790594 <= 2.0 -- True
0.24450593198589418, 0.591569145824067 <= 2.0 -- True
0.10815652587394427, 0.025050862270111277 <= 2.0 -- True
0.06215487191116709, 0.31263128215295743 <= 2.0 -- True
0.039615895819941696, 0.005443619048802969 <= 2.0 -- True

\end{minted}

\noweave

    @@ Following the procedure used to prove the Riemann-Lebesgue Lemma, show that $a_n \to 0$ and $b_n \to 0$, as
    $n\to\infty.$\\

    Let us examine the integrals.

    \begin{align*}
        a_n &= \frac{2}{\pi^2} \int_0^{\pi^2} \cos(\sqrt{x}) \cos\left(\frac{nx}{\pi}\right) \, dx\\
        b_n &= \frac{2}{\pi^2} \int_0^{\pi^2} \cos(\sqrt{x}) \sin\left(\frac{nx}{\pi}\right) \, dx\\
    \end{align*}

    As with the Riemann-Lebesgue Lemma, these two integrals are nearly identical so we will merely examine $a_n$.

    \begin{align*}
        a_n &= \frac{2}{\pi^2} \int_0^{\pi^2} \cos(\sqrt{x}) \cos\left(\frac{nx}{\pi}\right) \, dx\\
        &= \frac{2}{\pi^2} \left[ \frac{\pi}{n} \cos(\sqrt{x}) \sin\left(\frac{nx}{\pi}\right) + \frac{\pi}{2n} \int_0^{\pi^2} \sin\left(\frac{nx}{\pi}\right) \sin(\sqrt{x}) \, dx \right]\\
    \end{align*}

    We know that the integral is bounded and is less than or equal to $\pi^2$.

    \begin{align*}
        a_n &\le \frac{2}{n\pi} \left[ \cos(\sqrt{x}) \sin\left(\frac{nx}{\pi}\right) + \frac{\pi^2}{2} \right]\\
    \end{align*}

    Now since everything is being multiplied by $\frac{2}{n\pi}$, as $n \to \infty$, it is clear that $a_n \to 0$.

    @@ With this limited amount of information about the Fourier coefficients for $f(x)$, is it possible to say whether
    the Fourier series for $f(x)$ converges on $-\pi^2 \le x \le \pi^2$, and to what it converges? If so, where and to
    what does it converge? If not, why not?\\

    Yes there is enough information here to determine that the Fourier Series converges to the function, $f(x)$. We also
    know based on our previous bounds that it has pointwise convergence.

    \newpage
    @ Let

    \[
        k(x) = \begin{cases}
            \cos\left(\frac{\pi x}{L}\right) \quad &0 < x < L\\
            0 \quad & -L \le x \le 0
        \end{cases}
    \]

    @@ Sketch $k(x)$ on $-L \le x < L$. Is $k(x)$ continuous on $-L \le x < L$? Is it piecewise smooth on $-L \le x <
    L$? For each question, answer why or why not.\\

    We will graph this with $L=10$.

\weave

\begin{minted}[mathescape, fontsize=\small, xleftmargin=0.5em]{python}
L = 10
def k(x):
    def k_point(x):
        if x <= 0:
            return 0
        else:
            return np.cos(np.pi * x / L)
    if type(x) in [int, float]:
        return k_point(x)
    else:
        return np.array([k_point(x0) for x0 in x])
def bn_gen(n):
    bn = n * (1 + np.cos(n * np.pi)) / (np.pi * (n**2 - 1))
    return bn
def fourier(x, N):
    f = np.zeros(len(x))
    for i in range(len(x)):
        f[i] +=  np.cos((np.pi * x[i]) / L) / 2
        for n in range(2, N + 1):
            f[i] += bn_gen(n) * np.sin(n * np.pi * x[i] / L)
    return f
x = np.arange(-L, L, 0.01)
plt.figure(figsize=(8,8))
plt.plot(x, k(x))
plt.plot(x, fourier(x, 2))
plt.plot(x, fourier(x, 5))
plt.plot(x, fourier(x, 10))
plt.plot(x, fourier(x, 200))
plt.show()
\end{minted}
\includegraphics[width= 4in]{/home/zoe/classwork/2015b/appm4350/homeworks/figures/hw3_figure5_1.pdf}

\noweave

    $k(x)$ is not continuous on the interval as it contains one jump discontinuity.

    On the other hand, it is piecewise continuous as it contains a finite number of jump discontinuities (1).

    @@ Find its Fourier Coefficients.

    \begin{align*}
        a_0 &= \frac{1}{2L} \left[ \int_{-L}^0 0 \, dx + \int_0^L \cos\left(\frac{\pi x}{L}\right) \, dx \right]\\
            &= \frac{1}{2L} \int_0^L \cos\left(\frac{\pi x}{L}\right) \, dx\\
            &= \frac{1}{2L} \frac{L}{\pi} \sin\left(\frac{\pi x}{L}\right) \bigg|_0^L\\
            &= \frac{1}{2\pi} \sin\left(\frac{\pi x}{L}\right) \bigg|_0^L\\
            &= \frac{1}{2\pi} \left[ \sin(\pi) - \sin(0) \right]\\
            &= 0
    \end{align*}

    \begin{align*}
        a_n &= \frac{1}{L} \left[ \int_{-L}^0 0 \cos\left(\frac{nx\pi}{L}\right) \, dx + \int_0^L \cos\left(\frac{\pi x}{L}\right) \cos\left(\frac{nx\pi}{L}\right) \, dx \right]\\
            &= \frac{1}{L} \int_0^L \cos\left(\frac{\pi x}{L}\right) \cos\left(\frac{nx\pi}{L}\right) \, dx\\
            &= \frac{1}{2L} \left[\int_0^L \cos\left(\frac{\pi x}{L}(1 - n)\right) \, dx + \int_0^L \cos\left(\frac{\pi x}{L} (1 + n)\right) \, dx \right]\\
            &= \frac{1}{2L} \left[\frac{L}{\pi(1-n)}\sin\left(\frac{\pi x}{L}(1 - n)\right)\bigg|_0^L + \frac{L}{\pi(1+n)}\sin\left(\frac{\pi x}{L} (1 + n)\right)\bigg|_0^L \right]\\
            &= 0
    \end{align*}

    \begin{align*}
        b_n &= \frac{1}{L} \left[ \int_{-L}^0 0 \sin\left(\frac{nx\pi}{L}\right) \, dx + \int_0^L \cos\left(\frac{\pi x}{L}\right) \sin\left(\frac{nx\pi}{L}\right) \, dx \right]\\
            &= \frac{1}{2L} \left[ \int_0^L \sin\left(\frac{\pi x}{L}(1+n)\right) \, dx - \int_0^L \sin\left(\frac{\pi x}{L}(1-n)\right) \, dx \right]\\
            &= \frac{1}{2L} \left[ -\frac{L}{\pi(1+n)} \cos\left(\frac{\pi x}{L}(1+n)\right)\bigg|_0^L + \frac{L}{\pi(1-n)} \cos\left(\frac{\pi x}{L}(1-n)\right)\bigg|_0^L \right]\\
            &= \frac{1}{2\pi} \left[ -\frac{1}{1+n} (\cos(\pi(1+n)) - 1) + \frac{1}{1-n} ( \cos(\pi(1-n)) - 1) \right]\\
            &= \frac{1}{2\pi} \left[ \frac{\cos(\pi(1+n)) + 1}{1+n} - \frac{\cos(\pi(1-n)) + 1}{1-n}  \right]\\
            &= -\frac{4 n \cos^2\left(\frac{\pi n}{2}\right)}{\pi -\pi n^2}
    \end{align*}

    These coefficients are only valid from $n > 1$, so we manually determine the $n = 1$ case to be

    \begin{align*}
        a_n &= \frac{1}{2}\\
        b_n &= 0\\
    \end{align*}

    @@ Does the Fourier series converge on $-L \le x < L$? Why or why not? If it converges, does it converge uniformly?
    Why or why not?

    It does not converge on $-L \le x < L$, as $k(x)$ is not piecewise smooth.

    \newpage
    @ Convergence

    @@ Consider a Fourier Series:

    \[
        FS(x) = a_0 + \sum_{n=1}^\infty \left[ a_n \cos\left(\frac{nx\pi}{L}\right) + b_n \sin\left(\frac{nx\pi}{L}\right) \right]
    \]

    State the definitions of pointwise convergence, uniform convergence, and absolute convergence for such a series on
    $-L < x < L$. Be precise.\\

    \begin{definition}[Pointwise Convergence]
        If the Fourier series converges to its target function, $T(x)$ for every $x$ in some open interval, $X_1<x<X_2$,
        then the Fourier series converges pointwise to $T(x)$ in $X_1<x<X_2$.
    \end{definition}

    \begin{definition}[Uniform Convergence]
        A Fourier series converges uniformly to $T(x)$ in an interval if for every tolerance $\epsilon > 0$, there is a
        finite, positive integer, $N(\epsilon) < \infty$ such that for every $N > N(\epsilon)$

        \[
            \abs{S_N(x_0) - T(x_0)} < \epsilon
        \]
    \end{definition}

    \begin{definition}[Absolute Convergence]
        A Fourier series converges absolutely if for every $\epsilon > 0$, there is a finite, positive integer
        $M(\epsilon) < \infty$ such that

        \[
            \sum_{n=M+1}^\infty \abs{a_n} + \sum_{n=M+1}^\infty \abs{b_n} < \epsilon
        \]
    \end{definition}

    @@ Among these three kinds of convergence, which ones imply which others?\\

    Absolute implies Uniform implies Pointwise.

    @@ Are there conditions, either on the series itself or on the function $f(x)$ that generated the series, that
    guarantee that a Fourier series converges pointwise?\\

    The function $f(x)$ must be piecewise smooth in order for the series to converge pointwise.

    @@ Suppose we have a fixed Fourier series, but we know nothing about the function that generates the series. If the
    Fourier series converges absolutely, for all real $x$, what information does that tell us about the function to
    which the series converges, even without plotting partial sums of the Fourier series?\\

    This tells us two things. First, the series converges uniformly. Second, the function is continuous for all real $x$.
\end{easylist}


\end{document}
