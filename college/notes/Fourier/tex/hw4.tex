\documentclass[10pt]{article}

%%%%%%%%%%%%%%%%%%%%%%%%%%%%%%%%%%%%%%%%%%%%%%%%%%%%%%%%%%%%%%%%%%%%%%%%%%%%%%%%
% LaTeX Imports
%%%%%%%%%%%%%%%%%%%%%%%%%%%%%%%%%%%%%%%%%%%%%%%%%%%%%%%%%%%%%%%%%%%%%%%%%%%%%%%%
\usepackage{amsfonts}                                                   % Math fonts
\usepackage{amsmath}                                                    % Math formatting
\usepackage{amssymb}                                                    % Math formatting
\usepackage{amsthm}                                                     % Math Theorems
\usepackage{arydshln}                                                   % Dashed hlines
\usepackage{attachfile}                                                 % AttachFiles
\usepackage{cancel}                                                     % Cancelled math
\usepackage{caption}                                                    % Figure captioning
\usepackage{color}                                                      % Nice Colors
\input{./lib/dragon.inp}                                                % Tikz dragon curve
\usepackage[ampersand]{easylist}                                        % Easy lists
\usepackage{fancyhdr}                                                   % Fancy Header
\usepackage[T1]{fontenc}                                                % Specific font-encoding
%\usepackage[margin=1in, marginparwidth=2cm, marginparsep=2cm]{geometry} % Margins
\usepackage{graphicx}                                                   % Include images
\usepackage{hyperref}                                                   % Referencing
\usepackage[none]{hyphenat}                                             % Don't allow hyphenation
\usepackage{lipsum}                                                     % Lorem Ipsum Dummy Text
\usepackage{listings}                                                   % Code display
\usepackage{marginnote}                                                 % Notes in the margin
\usepackage{microtype}                                                  % Niceness
\usepackage{lib/minted}                                                 % Code display
\usepackage{multirow}                                                   % Multirow tables
\usepackage{pdfpages}                                                   % Include pdfs
\usepackage{pgfplots}                                                   % Create Pictures
\usepackage{rotating}                                                   % Figure rotation
\usepackage{setspace}                                                   % Allow double spacing
\usepackage{subcaption}                                                 % Figure captioning
\usepackage{tikz}                                                       % Create Pictures
\usepackage{tocloft}                                                    % List of Equations
%%%%%%%%%%%%%%%%%%%%%%%%%%%%%%%%%%%%%%%%%%%%%%%%%%%%%%%%%%%%%%%%%%%%%%%%%%%%%%%%
% Package Setup
%%%%%%%%%%%%%%%%%%%%%%%%%%%%%%%%%%%%%%%%%%%%%%%%%%%%%%%%%%%%%%%%%%%%%%%%%%%%%%%%
\hypersetup{%                                                           % Setup linking
    colorlinks=true,
    linkcolor=black,
    citecolor=black,
    filecolor=black,
    urlcolor=black,
}
\RequirePackage[l2tabu, orthodox]{nag}                                  % Nag about bad syntax
\renewcommand*\thesection{\arabic{section} }                             % Reset numbering
\renewcommand{\theFancyVerbLine}{ {\arabic{FancyVerbLine} } }              % Needed for code display
\renewcommand{\footrulewidth}{0.4pt}                                    % Footer hline
\setcounter{secnumdepth}{3}                                             % Include subsubsections in numbering
\setcounter{tocdepth}{3}                                                % Include subsubsections in toc
%%%%%%%%%%%%%%%%%%%%%%%%%%%%%%%%%%%%%%%%%%%%%%%%%%%%%%%%%%%%%%%%%%%%%%%%%%%%%%%%
% Custom commands
%%%%%%%%%%%%%%%%%%%%%%%%%%%%%%%%%%%%%%%%%%%%%%%%%%%%%%%%%%%%%%%%%%%%%%%%%%%%%%%%
\newcommand{\nvec}[1]{\left\langle #1 \right\rangle}                    %  Easy to use vector
\newcommand{\ma}[0]{\mathbf{A} }                                         %  Easy to use vector
\newcommand{\mb}[0]{\mathbf{B} }                                         %  Easy to use vector
\newcommand{\abs}[1]{\left\lvert #1 \right\rvert}                       %  Easy to use abs
\newcommand{\pren}[1]{\left( #1 \right)}                                %  Big parens
\let\oldvec\vec
\renewcommand{\vec}[1]{\oldvec{\mathbf{#1} } }                            %  Vector Styling
\newtheorem{thm}{Theorem}                                               %  Define the theorem name
\newtheorem{definition}{Definition}                                     %  Define the definition name
\definecolor{bg}{rgb}{0.95,0.95,0.95}
\newcommand{\java}[4]{\vspace{10pt}\inputminted[firstline=#2,
                                 lastline=#3,
                                 firstnumber=#2,
                                 gobble=#4,
                                 frame=single,
                                 label=#1,
                                 bgcolor=bg,
                                 linenos]{java}{#1} }
\newcommand{\python}[4]{\vspace{10pt}\inputminted[firstline=#2,
                                 lastline=#3,
                                 firstnumber=#2,
                                 gobble=#4,
                                 frame=single,
                                 label=#1,
                                 bgcolor=bg,
                                 linenos]{python}{#1} }
\newcommand{\js}[4]{\vspace{10pt}\inputminted[firstline=#2,
                                 lastline=#3,
                                 firstnumber=#2,
                                 gobble=#4,
                                 frame=single,
                                 label=#1,
                                 bgcolor=bg,
                                 linenos]{js}{#1} }
%%%%%%%%%%%%%%%%%%%%%%%%%%%%%%%%%%%%%%%%%%%%%%%%%%%%%%%%%%%%%%%%%%%%%%%%%%%%%%%%
% Beginning of document items - headers, title, toc, etc...
%%%%%%%%%%%%%%%%%%%%%%%%%%%%%%%%%%%%%%%%%%%%%%%%%%%%%%%%%%%%%%%%%%%%%%%%%%%%%%%%
\pagestyle{fancy}                                                       %  Establishes that the headers will be defined
\fancyhead[LE,LO]{Computer Systems Notes}                                  %  Adds header to left
\fancyhead[RE,RO]{Zoe Farmer}                                       %  Adds header to right
\cfoot{ \thepage }
\lfoot{CSCI 2400}
\rfoot{Han}
\title{Computer Systems Notes}
\author{Zoe Farmer}

%%%%%%%%%%%%%%%%%%%%%%%%%%%%%%%%%%%%%%%%%%%%%%%%%%%%%%%%%%%%%%%%%%%%%%%%%%%%%%%%
% Beginning of document items - headers, title, toc, etc...
%%%%%%%%%%%%%%%%%%%%%%%%%%%%%%%%%%%%%%%%%%%%%%%%%%%%%%%%%%%%%%%%%%%%%%%%%%%%%%%%
\pagestyle{fancy}                                                       %  Establishes that the headers will be defined
\fancyhead[LE,LO]{Fourier Series}                                  %  Adds header to left
\fancyhead[RE,RO]{Zoe Farmer}                                       %  Adds header to right
\cfoot{\thepage}
\lfoot{APPM 4350}
\rfoot{Mark Hoefer}
\title{Fourier Series Homework Four}
\author{Zoe Farmer}
%%%%%%%%%%%%%%%%%%%%%%%%%%%%%%%%%%%%%%%%%%%%%%%%%%%%%%%%%%%%%%%%%%%%%%%%%%%%%%%%
% Beginning of document items - headers, title, toc, etc...
%%%%%%%%%%%%%%%%%%%%%%%%%%%%%%%%%%%%%%%%%%%%%%%%%%%%%%%%%%%%%%%%%%%%%%%%%%%%%%%%
\begin{document}



\maketitle

\begin{easylist}[enumerate]
    @ \textit{Steady State Problems}

    Let $u(x, t)$ denote the temperature in a metal rod of length $L$, with thermal conductivity $K_0 > 0$, and thermal
    diffusivity $k >0$. The rod has uniform cross-section along the rod, and it is insulated on its sides. The rod
    experiences non-uniform internal heating of the form $Q(x) = \frac{Mx}{L}$, for $0 < x < L$, where $M>0$ is a
    positive constant. The governing equation for $u(x, t)$, the temperature of the rod, is

    \[
        u_t = ku_{xx} + \frac{Mx}{L} \qquad 0 < x < L \qquad t > 0
    \]

    The rod is subject to boundary conditions at each end of the rod. Three sets of boundary conditions are given below.

    \begin{align*}
        (i) \begin{cases}
            u(0, t) = 20^\circ C\\
            u_x(L, t) = 0
        \end{cases}\\
        (ii) \begin{cases}
            u_x(0, t) = 0\\
            u(L, t) = 20^\circ C
        \end{cases}\\
        (iii) \begin{cases}
            u(0, t) = 20^\circ C\\
            u(L, t) = 0^\circ C
        \end{cases}
    \end{align*}

    In all three cases, the initial temperature is constant at $20^\circ C$.

    @@ For each set of boundary conditions, find the solution of the steady state problem.\\

    @@@ We can write our equation as the following.

    \[
        k u_{xx} + \frac{Mx}{L} = 0
    \]

    Taking the anti-derivatives yields the following equations.

    \begin{align*}
        u_{xx} &= \frac{-Mx}{Lk}\\
        u_x &= \frac{-Mx^2}{2Lk} + A\\
        u &= \frac{-Mx^3}{6Lk} + Ax + B\\
    \end{align*}

    We can use our boundary conditions to determine the coefficients.

    \begin{align*}
        u(0, t) \Rightarrow B = 20
    \end{align*}

    And,

    \begin{align*}
        u_x(L, t) \Rightarrow \frac{-ML^2}{2Lk} + A &= 0\\
         A &= \frac{ML}{2k}\\
    \end{align*}

    Yielding the final equation of

    \[
        \boxed{%
            u(x, t) = \frac{-Mx^3}{6Lk} + \frac{MLx}{2k} + 20\\
        }
    \]

    @@@ Using the same general equation for $u$ as above, we can simply determine the coefficients again.

    \begin{align*}
        u_x(0,t) \Rightarrow A = 0
    \end{align*}

    And,

    \begin{align*}
        u(L,t) \Rightarrow -\frac{ML^2}{6k} + B &= 20\\
        B &= 20 + \frac{ML^2}{6k}\\
    \end{align*}

    Yielding the final equation of

    \[
        \boxed{%
            u(x, t) = \frac{-Mx^3}{6Lk} + \frac{ML}{6k} + 20\\
        }
    \]

    @@@ And again, we can simply use the equation with the initial conditions, yielding the coefficients to be

    \begin{align*}
        A &= \frac{ML}{6k} - \frac{20}{L}\\
        B&=20
    \end{align*}

    Yielding the final equation to be

    \[
        \boxed{%
            u(x, t) = \frac{-Mx^3}{6Lk} + \left(\frac{ML}{6k} + \frac{20}{L}\right)x + 20
        }
    \]

    @@ For the first two, sketch the steady-state solutions as functions of $x$. Choose a numerical value for $ML^2 / k$
    for this sketch, and state the chosen value. Identify which sketch goes to which set of boundary conditions.\\



    For the below plots, I chose the following constants.

    \begin{align*}
        L &= 10\\
        M &= 3\\
        k &= 7\\
    \end{align*}

    @@@

\weave

\begin{minted}[mathescape, fontsize=\small, xleftmargin=0.5em]{python}
u = lambda x: (-M * x**3 / (6 * L * k)) + (M * L * x / (2 * k)) + 20
plt.figure()
plt.plot(x, u(x))
plt.show()
\end{minted}
\includegraphics[width= 4in]{/home/zoe/classwork/2015b/appm4350/homeworks/figures/hw4_figure1_1.pdf}

\noweave

    @@@

\weave

\begin{minted}[mathescape, fontsize=\small, xleftmargin=0.5em]{python}
u = lambda x: (-M * x**3 / (6 * L * k)) + (M * L**2 / (6 * k)) + 20
plt.figure()
plt.plot(x, u(x))
plt.show()
\end{minted}
\includegraphics[width= 4in]{/home/zoe/classwork/2015b/appm4350/homeworks/figures/hw4_figure2_1.pdf}

\noweave

    @@@ We will also plot the third, as it is a simple check of my above solution.

\weave

\begin{minted}[mathescape, fontsize=\small, xleftmargin=0.5em]{python}
u = lambda x: ((-M * x**3 / (6 * L * k)) +
                    (M * L * x / (6 * k)) -
                    (20 * x / L) + 20)
plt.figure(figsize=(16,8))
plt.plot(x, u(x))
plt.show()
\end{minted}
\includegraphics[width= 4in]{/home/zoe/classwork/2015b/appm4350/homeworks/figures/hw4_figure3_1.pdf}

\noweave

    @@ Find the steady state heat flux for each problem. Which way does the heat flow in each case?\\

    This is simply the derivative.

    @@@ Examining the derivative,

    \[
        u_x(x, t) = \frac{-Mx^2}{2Lk} + \frac{ML}{2k}
    \]

    we see that the flux is equal to $u_x(0)$, which is $ML / 2k$, a positive quantity. Therefore, heat flows to the
    right.

    @@@ Again, we can examine the derivative,

    \[
        u_x(x, t) = \frac{-Mx^2}{2Lk}
    \]

    This time, we can plug in $L$, and get $-ML / 2k$, a negative value. Therefore, heat flows to the left.

    @@@ One last time\ldots

    \[
        u_x(x, t) = \frac{-Mx^2}{2Lk} + \frac{ML}{6k} + \frac{20}{L}
    \]

    Again, we'll use $L$, getting $-ML / (3k) + 20 / L$. Since $M > 0$, and $L > 0$, this is also a negative value,
    meaning heat flows to the left.

    @@ The first two differ only in terms of which boundary condition goes at which end. But the solution of one of the
    problems has a significantly larger change of temperature from one end of the rod to the other. Between these two,
    which problem has the bigger change of temperature from one end of the rod to the other? What is the physical
    mechanism that makes it bigger?\\

    $(i)$ has the larger change. This is a result of more heat being added at the insulated end.

    (We can see this by graphing $Q(x)$.

\simpleweave

\includegraphics[width= 2in]{/home/zoe/classwork/2015b/appm4350/homeworks/figures/hw4_figure4_1.pdf}

\nosimpleweave

    @@ For the third case, what is the critical value of $ML^2/k$ at which the maximum temperature in the rod moves from
    one end of the rod to the interior? Justify.\\

    Using the equation we determined above,

    \[
        u(x, t) = \frac{-Mx^3}{6Lk} + \frac{MLx}{6k} + \frac{20x}{L} + 20
    \]

    As $M / k$ grows, this critical point changes.

    We can prove this by letting $n = ML^2 / k$.

    \begin{align*}
        u(x, t) &= \frac{-nx^3}{6L^3} + \frac{nx}{6L} + \frac{20x}{L} + 20\\
    \end{align*}

    We can show this transition with the following plot, letting $L = 10$.

\simpleweave

\includegraphics[width= 6in]{/home/zoe/classwork/2015b/appm4350/homeworks/figures/hw4_figure5_1.pdf}

\nosimpleweave

    Taking the derivative with respect to $n$ yields the following equation.

    \begin{align*}
        u(x, t) &= \frac{-nx^3}{6L^3} + \frac{nx}{6L} + \frac{20x}{L} + 20\\
    \end{align*}

    Using this we see that the critical point is $n = L^3/k$.

    @@ For the third case, write the entire solution of the problem as $u(x, t) = \overline{u}(x) + v(x, t)$, where
    $\overline{u}(x)$ is the solution of the steady-state problem, and $v(x, t)$ is the solution of the transient
    problem. Write down explicitly the set of equations that define $v(x, t)$, and show how you arrived at them.\\

    We already know $\overline{u}(x)$, which we will denote as $u_{ss}(x)$. We now substitute $u(x, t)$ into the heat
    equation.

    \begin{align*}
        u_t &= ku_{xx} + \frac{Mx}{L}\\
        {(v + u_{ss})}_t &= k{(v + u_{ss})}_{xx} + \frac{Mx}{L}\\
        v_t &= kv_{xx}\\
    \end{align*}

    Now we use our boundary conditions.

    \begin{align*}
        \alpha u(0, t) + \beta u(L, t)\\
        \alpha v(0, t) + \alpha u_{ss}(0, t) + \beta v(L, t) + \beta u_{ss}(L, t)\\
        \alpha v(0, t) + \beta v(L, t)\\
    \end{align*}

    Yielding our final solution as the following.

    \[
        u(x, t) = \underbrace{\frac{-Mx^3}{6Lk} + \frac{MLx}{6k} + \frac{20x}{L} + 20}_{\overline{u}(x, t)} +
            \underbrace{\frac{ML}{k}}_{v(x, t)}
    \]

    \newpage
    @ \textit{Heat equation in an insulated rod}

    Consider the following IC/BC problem for the heat equation.

    \begin{align*}
        u_t = k u_{xx} \qquad &0 < x < L \qquad &t > 0\\
        u(x, 0) = f(x) \qquad &0 < x < L \qquad &t = 0\\
        u_x(0, t) = 0, u_x(L, t) = 0 \qquad &&t > 0\\
    \end{align*}

    @@ Does this problem have a steady-state solution? If so, what is it? Is it unique? Justify.\\

    Yes, this problem has a steady solution since we have boundary conditions that are independent of time. This
    solution is not unique however, as any constant temperature is a solution for the problem.

    @@ Using separation of variables, determine a series solution of the transient problem. The coefficients can involve
    explicit un-evaluated integrals.\\

    First we establish $f$ and $g$ as functions of $t$ and $x$, respectively.

    \begin{align*}
        u(x, t) &= f(t) g(x)\\
        u_t(x, t) &= f_t(t) g(x)\\
        u_{xx}(x, t) &= f(t) g_{xx}(x)
    \end{align*}

    The heat equation implies the following.

    \[
        f_t(t) g(x) = k f(t) g_{xx}(x)
    \]

    We can divide both sides by $f(t)g(x)$.

    \[
        \frac{f_t(t)}{kf(t)} = \frac{g_{xx}(x)}{g(x)} = -\lambda
    \]

    Yielding two ordinary differential equations.

    \begin{align*}
        f_t(t) = -\lambda k f(t)\\
        g_{xx}(x) = -\lambda g(x)
    \end{align*}

    We also know that the product solutions must satisfy the boundary conditions.

    \begin{align*}
        g_x(0)f(t) = 0 \Rightarrow g_x(0) = 0\\
        g_x(L)f(t) = 0 \Rightarrow g_x(L) = 0\\
    \end{align*}

    Yielding eigenvalues and functions,

    \begin{align*}
        \lambda_n &= {\left(\frac{n\pi}{L}\right)}^2\\
        \phi_n &= \cos\left(\frac{n\pi x}{L}\right)\\
        n &= 0, 1, 2, 3, 4, \ldots\\
    \end{align*}

    Therefore our solutions will be

    \[
        u_n (x, t) = A_n \cos\left(\frac{n\pi x}{L}\right) \exp\left(-k{\left(\frac{n\pi}{L}\right)}^2 t\right)
    \]

    With solution

    \[
        u (x, t) = \sum_{n=0}^\infty \left[ A_n \cos\left(\frac{n\pi x}{L}\right) \exp\left(-k{\left(\frac{n\pi}{L}\right)}^2 t\right) \right]
    \]

    Using the initial condition, we see that

    \[
        u (x, 0) = \sum_{n=0}^\infty \left[ A_n \cos\left(\frac{n\pi x}{L}\right) \right]
    \]

    Leaving us with a cosine series, where $A_n$ is given by

    \[
        A_n = \begin{cases}
            \frac{1}{L} \int_0^L f(x) \, dx \qquad &n = 0\\
            \frac{2}{L} \int_0^L f(x) \cos\left(\frac{n\pi x}{L} \right) \, dx \qquad &else\\
        \end{cases}
    \]

    @@ What is $\lim_{t\to\infty} u(x, t)$? Interpret the result physically. How does your answer reflect upon what you
    found in part (a)?\\

    As $t$ approaches $\infty$, the heat throughout the rod will approach the total heat contained. This makes sense
    with our answer in part (a), as it will eventually reach a point where the system is no longer changing.

    @@ Determine the solution explicitly when $f(x) = x^2 (1 - x^2), L = 1, k = 1$.\\



    We can examine $f(x)$.

\simpleweave

\includegraphics[width= 3in]{/home/zoe/classwork/2015b/appm4350/homeworks/figures/hw4_figure6_1.pdf}

\nosimpleweave

    Now we can simply plug it into our equation from before.

    \begin{align*}
        A_n &= \begin{cases}
            \frac{1}{L} \int_0^L f(x) \, dx \qquad &n = 0\\
            \frac{2}{L} \int_0^L f(x) \cos\left(\frac{n\pi x}{L} \right) \, dx \qquad &else\\
        \end{cases}\\
        &= \begin{cases}
            \frac{1}{L} \int_0^L x^2 (1 - x^2) \, dx \qquad &n = 0\\
            \frac{2}{L} \int_0^L x^2 (1 - x^2) \cos\left(\frac{n\pi x}{L} \right) \, dx \qquad &else\\
        \end{cases}\\
    \end{align*}

    This evaluates to 

    \begin{align*}
        A_n &= \begin{cases}
            \frac{1}{L} \int_0^L x^2 (1 - x^2) \, dx \qquad &n = 0\\
            \frac{2}{L} \int_0^L x^2 (1 - x^2) \cos\left(\frac{n\pi x}{L} \right) \, dx \qquad &else\\
        \end{cases}\\
            &= \begin{cases}
            -\frac{L^4}{5} + \frac{L^2}{2} \qquad &n = 0\\
            \frac{2L^2}{n^5\pi^5}\left(24L^2n\pi \cos(n\pi)+2n^3\pi^3(-2L^2+1)\cos(n\pi)\right) \qquad &else\\
        \end{cases}\\
    \end{align*}

\simpleweave

\includegraphics[width= 6in]{/home/zoe/classwork/2015b/appm4350/homeworks/figures/hw4_figure7_1.pdf}

\nosimpleweave

    \newpage
    @ \textit{A Final Problem on Fourier Series}

    \[
        f(x) = \sum_{n=1}^\infty \left[ \frac{1}{n^2} \sin\left(\frac{n^2 \pi x}{L}\right) \right]
    \]

    @@ Show that the series converges absolutely. Conclude that $f(x)$ is continuous for all real $x$, and that it is
    periodic, with period $2L$.\\

    We need to show that

    \[
        \sum_{n=M+1}^\infty \abs{a_n} < \epsilon
    \]

    The steps are as follows.

    \begin{align*}
        \sum_{n=M+1}^\infty \abs{a_n}\\
        \sum_{n=M+1}^\infty \abs{\frac{1}{n^2}}\\
        \sum_{n=M+1}^\infty \frac{1}{n^2} \le \frac{\pi}{6} < \epsilon\\
    \end{align*}

    @@ Formally differentiate $f(x)$ by differentiating within the summation. Show that the formal series for
    $f^\prime(x)$ converges for no real $x$. Explain why.\\

    \begin{align*}
        f(x) &= \sum_{n=1}^\infty \left[ \frac{1}{n^2} \sin\left(\frac{n^2 \pi x}{L}\right) \right]\\
        f^\prime(x) &= \sum_{n=1}^\infty \left[ \frac{d}{dx} \left( \frac{1}{n^2} \sin\left(\frac{n^2 \pi x}{L}\right)\right) \right]\\
        f^\prime(x) &= \sum_{n=1}^\infty \left[ \frac{1}{n^2} \left( \frac{d}{dx} \sin\left(\frac{n^2 \pi x}{L}\right)\right) \right]\\
        f^\prime(x) &= \sum_{n=1}^\infty \left[ \frac{1}{n^2} \left( \frac{n^2 \pi \cos\left(\frac{n^2 \pi x}{L}\right)}{L} \right) \right]\\
        f^\prime(x) &= \sum_{n=1}^\infty \left[ \frac{\pi \cos\left(\frac{n^2 \pi x}{L}\right)}{L} \right]\\
        f^\prime(x) &= \frac{\pi}{2L} \sum_{n=1}^\infty \left[ \exp\left(\frac{in^2x\pi}{L}\right) + \exp\left(\frac{-in^2x\pi}{L}\right) \right]\\
    \end{align*}

    In order for this to converge, the following inequality must hold.

    \[
        \abs{\frac{\pm in^2x\pi}{L}} < 1
    \]

    For each case, $x$ must hold the value of

    \[
        \frac{1}{i} \cdot y
    \]

    where $\abs{y} \le 1$. Because of this complex part, this there is no \textit{real} value of $x$ for which this
    holds.

\end{easylist}


\end{document}
