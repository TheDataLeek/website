\documentclass[10pt]{article}

%%%%%%%%%%%%%%%%%%%%%%%%%%%%%%%%%%%%%%%%%%%%%%%%%%%%%%%%%%%%%%%%%%%%%%%%%%%%%%%%
% LaTeX Imports
%%%%%%%%%%%%%%%%%%%%%%%%%%%%%%%%%%%%%%%%%%%%%%%%%%%%%%%%%%%%%%%%%%%%%%%%%%%%%%%%
\usepackage{amsfonts}                                                   % Math fonts
\usepackage{amsmath}                                                    % Math formatting
\usepackage{amssymb}                                                    % Math formatting
\usepackage{amsthm}                                                     % Math Theorems
\usepackage{arydshln}                                                   % Dashed hlines
\usepackage{attachfile}                                                 % AttachFiles
\usepackage{cancel}                                                     % Cancelled math
\usepackage{caption}                                                    % Figure captioning
\usepackage{color}                                                      % Nice Colors
\input{./lib/dragon.inp}                                                % Tikz dragon curve
\usepackage[ampersand]{easylist}                                        % Easy lists
\usepackage{fancyhdr}                                                   % Fancy Header
\usepackage[T1]{fontenc}                                                % Specific font-encoding
%\usepackage[margin=1in, marginparwidth=2cm, marginparsep=2cm]{geometry} % Margins
\usepackage{graphicx}                                                   % Include images
\usepackage{hyperref}                                                   % Referencing
\usepackage[none]{hyphenat}                                             % Don't allow hyphenation
\usepackage{lipsum}                                                     % Lorem Ipsum Dummy Text
\usepackage{listings}                                                   % Code display
\usepackage{marginnote}                                                 % Notes in the margin
\usepackage{microtype}                                                  % Niceness
\usepackage{lib/minted}                                                 % Code display
\usepackage{./lib/mlptikz}                                              % Tikz mlp
\usepackage{multirow}                                                   % Multirow tables
\usepackage{pdfpages}                                                   % Include pdfs
\usepackage{pgfplots}                                                   % Create Pictures
\usepackage{rotating}                                                   % Figure rotation
\usepackage{setspace}                                                   % Allow double spacing
\usepackage{subcaption}                                                 % Figure captioning
\usepackage{tikz}                                                       % Create Pictures
\usepackage{tocloft}                                                    % List of Equations
%%%%%%%%%%%%%%%%%%%%%%%%%%%%%%%%%%%%%%%%%%%%%%%%%%%%%%%%%%%%%%%%%%%%%%%%%%%%%%%%
% Package Setup
%%%%%%%%%%%%%%%%%%%%%%%%%%%%%%%%%%%%%%%%%%%%%%%%%%%%%%%%%%%%%%%%%%%%%%%%%%%%%%%%
\hypersetup{%                                                           % Setup linking
    colorlinks=true,
    linkcolor=black,
    citecolor=black,
    filecolor=black,
    urlcolor=black,
}
\RequirePackage[l2tabu, orthodox]{nag}                                  % Nag about bad syntax
\renewcommand*\thesection{\arabic{section} }                             % Reset numbering
\renewcommand{\theFancyVerbLine}{ {\arabic{FancyVerbLine} } }              % Needed for code display
\renewcommand{\footrulewidth}{0.4pt}                                    % Footer hline
\setcounter{secnumdepth}{3}                                             % Include subsubsections in numbering
\setcounter{tocdepth}{3}                                                % Include subsubsections in toc
%%%%%%%%%%%%%%%%%%%%%%%%%%%%%%%%%%%%%%%%%%%%%%%%%%%%%%%%%%%%%%%%%%%%%%%%%%%%%%%%
% Custom commands
%%%%%%%%%%%%%%%%%%%%%%%%%%%%%%%%%%%%%%%%%%%%%%%%%%%%%%%%%%%%%%%%%%%%%%%%%%%%%%%%
\newcommand{\nvec}[1]{\left\langle #1 \right\rangle}                    %  Easy to use vector
\newcommand{\inprod}[2]{\left\langle \vec{#1}, \vec{#2} \right\rangle}  %  Easy to use inner product
\newcommand{\norm}[1]{\lvert \lvert \vec{#1} \rvert \rvert}             %  Easy to use norm
\newcommand{\ma}[0]{\mathbf{A} }                                         %  Easy to use vector
\newcommand{\mb}[0]{\mathbf{B} }                                         %  Easy to use vector
\newcommand{\abs}[1]{\left\lvert #1 \right\rvert}                       %  Easy to use abs
\newcommand{\pren}[1]{\left( #1 \right)}                                %  Big parens
\let\oldvec\vec
\renewcommand{\vec}[1]{\mathbf{#1} }                            %  Vector Styling
\newtheorem{thm}{Theorem}                                               %  Define the theorem name
\theoremstyle{definition}
\newtheorem{definition}{Definition}                                     %  Define the definition name
\newtheorem{ex}{Example}                                                %  Define the example name
\definecolor{bg}{rgb}{0.95,0.95,0.95}
\newcommand{\java}[4]{\vspace{10pt}\inputminted[firstline=#2,
                                 lastline=#3,
                                 firstnumber=#2,
                                 gobble=#4,
                                 frame=single,
                                 label=#1,
                                 bgcolor=bg,
                                 linenos]{java}{#1} }
\newcommand{\python}[4]{\vspace{10pt}\inputminted[firstline=#2,
                                 lastline=#3,
                                 firstnumber=#2,
                                 gobble=#4,
                                 frame=single,
                                 label=#1,
                                 bgcolor=bg,
                                 linenos]{python}{#1} }
\newcommand{\js}[4]{\vspace{10pt}\inputminted[firstline=#2,
                                 lastline=#3,
                                 firstnumber=#2,
                                 gobble=#4,
                                 frame=single,
                                 label=#1,
                                 bgcolor=bg,
                                 linenos]{js}{#1} }
%%%%%%%%%%%%%%%%%%%%%%%%%%%%%%%%%%%%%%%%%%%%%%%%%%%%%%%%%%%%%%%%%%%%%%%%%%%%%%%%
% Beginning of document items - headers, title, toc, etc...
%%%%%%%%%%%%%%%%%%%%%%%%%%%%%%%%%%%%%%%%%%%%%%%%%%%%%%%%%%%%%%%%%%%%%%%%%%%%%%%%
\pagestyle{fancy}                                                       %  Establishes that the headers will be defined
\fancyhead[LE,LO]{Matrix Methods Notes}                                  %  Adds header to left
\fancyhead[RE,RO]{Zoe Farmer}                                       %  Adds header to right
\cfoot{\mlptikz[size=0.25in, text=on, textposx=0, textposy=0, textvalue=\thepage, textscale=0.75in]{applejack} }
\lfoot{APPM 3310}
\rfoot{Beylkin}
\title{Matrix Methods Notes}
\author{Zoe Farmer}

%%%%%%%%%%%%%%%%%%%%%%%%%%%%%%%%%%%%%%%%%%%%%%%%%%%%%%%%%%%%%%%%%%%%%%%%%%%%%%%%
% Beginning of document items - headers, title, toc, etc...
%%%%%%%%%%%%%%%%%%%%%%%%%%%%%%%%%%%%%%%%%%%%%%%%%%%%%%%%%%%%%%%%%%%%%%%%%%%%%%%%
\pagestyle{fancy}                                                       %  Establishes that the headers will be defined
\fancyhead[LE,LO]{Fourier Series}                                  %  Adds header to left
\fancyhead[RE,RO]{Zoe Farmer}                                       %  Adds header to right
\cfoot{\thepage}
\lfoot{APPM 4350}
\rfoot{Mark Hoefer}
\title{Fourier Series Homework One}
\author{Zoe Farmer}
%%%%%%%%%%%%%%%%%%%%%%%%%%%%%%%%%%%%%%%%%%%%%%%%%%%%%%%%%%%%%%%%%%%%%%%%%%%%%%%%
% Beginning of document items - headers, title, toc, etc...
%%%%%%%%%%%%%%%%%%%%%%%%%%%%%%%%%%%%%%%%%%%%%%%%%%%%%%%%%%%%%%%%%%%%%%%%%%%%%%%%
\begin{document}




\maketitle

\begin{easylist}[enumerate]
    @ \textit{Orthogonality}
        The real-valued, Fourier basis functions for functions defined on $- \pi < x < \pi $ are the following.
        \[
            \left[
                1, {\left\{ \sin(nx), \cos(nx) \right\}}_{n=1}^\infty
            \right]
        \]
    @@ Show by direct calculation that for any two positive integers $(n, m)$:

    @@@ \[ \int_{-\pi}^\pi \left[ \sin(nx) \sin(mx) \right] \, dx = 0 \quad \text{if } n \neq m \]

    \begin{align*}
        \int_{-\pi}^\pi \left[ \sin(nx) \sin(mx) \right] \, dx &=& 0\\
        \frac{1}{2} \left[ \int_{-\pi}^\pi \cos(x [n - m]) \, dx - \int_{-\pi}^\pi \cos(x [n + m]) \, dx \right] &=& 0\\
        \frac{1}{2} \left[ \frac{\sin(x[n - m])}{n - m}\bigg|_{-\pi}^\pi - \frac{\sin(x[n + m])}{n + m} \bigg|_{\pi}^\pi\right] &=& 0\\
        \frac{1}{2} \left[ \frac{\sin(\pi [n - m])}{n - m} - \frac{\sin(-\pi [n - m])}{n - m} -
            \frac{\sin(\pi[n + m])}{n + m} + \frac{\sin(-\pi[n + m])}{n + m}\right] &=& 0\\
        \frac{1}{2} &=&  0\\
        0 &=& 0\\
    \end{align*}

    @@@ \[ \int_{-\pi}^\pi \left[ \cos(nx) \cos(mx) \right] \, dx = 0 \quad \text{if } n \neq m \]

    \begin{align*}
        \int_{-\pi}^\pi \left[ \cos(nx) \cos(mx) \right] \, dx &=& 0\\
        \frac{1}{2} \left[ \int_{-\pi}^\pi \cos(nx - mx) \, dx + \int_{-\pi}^\pi \cos(nx + mx) \, dx \right] &=& 0\\
        \frac{1}{2} \left[ \frac{\sin(\pi [n - m])}{n - m} - \frac{\sin(-\pi [n - m])}{n - m} +
            \frac{\sin(\pi[n + m])}{n + m} + \frac{\sin(-\pi[n + m])}{n + m}\right] &=& 0\\
        \frac{1}{2} &=& 0\\
        0 &=& 0
    \end{align*}

    @@@ \[ \int_{-\pi}^\pi \left[ 1 \cdot \sin(mx) \right] \, dx = 0 \]

    \begin{align*}
        \int_{-\pi}^\pi \left[ 1 \cdot \sin(mx) \right] \, dx &=& 0\\
        \int_{-\pi}^\pi \sin(mx) \, dx &=& 0\\
        -\frac{\cos(mx)}{m} \bigg|_{-\pi}^\pi &=& 0\\
        -\frac{-1}{m} + \frac{-1}{m} &=& 0\\
        \frac{1}{m} - \frac{1}{m} &=& 0\\
        0 &=& 0\\
    \end{align*}

    @@@ \[ \int_{-\pi}^\pi \left[ \sin(nx) \sin(nx) \right] \, dx = \pi \]

    \begin{align*}
        \int_{-\pi}^\pi \left[ \sin(nx) \sin(nx) \right] \, dx &=& \pi\\
        \int_{-\pi}^\pi \sin^2(nx) \, dx &=& \pi\\
        \int_{-\pi}^\pi \left[ \frac{1 - \cos(2x)}{2} \right] \, dx &=& \pi\\
        \frac{1}{2} \left[ \int_{-\pi}^\pi 1 - \int_{-\pi}^\pi \cos(2x) \, dx \right] &=& \pi\\
        \frac{1}{2} \left[ (2 \pi) - \left( \frac{1}{2} \left[ \sin(2 \pi) - \sin(-2 \pi) \right] \right) \right] &=& \pi\\
        \frac{1}{2} ( 2 \pi ) &=& \pi\\
        \pi &=& \pi
    \end{align*}

    @@@ \[ \int_{-\pi}^\pi \left[ 1 \cdot 1 \right] \, dx = 2 \pi \]

    \begin{align*}
        \int_{-\pi}^\pi \left[ 1 \cdot 1 \right] \, dx &=& 2 \pi\\
        \int_{-\pi}^\pi 1 \, dx &=& 2 \pi\\
        \pi - (- \pi) &=& 2 \pi\\
        2 \pi &=& 2\pi
    \end{align*}

    @@@ \[ \int_{-\pi}^\pi \left[ \sin(nx) \cos(mx) \right] \, dx = 0 \]

    \begin{align*}
        \int_{-\pi}^\pi \left[ \sin(nx) \cos(mx) \right] \, dx &=& 0\\
        \frac{1}{2} \left[ \int_{-\pi}^\pi \sin(x(n + m)) \, dx + \int_{-\pi}^\pi \sin(x(n - m)) \, dx \right] &=& 0\\
        \frac{1}{2} \left[ \frac{-\cos(x(n+m))}{n+m} \bigg|_{-\pi}^\pi +
            \frac{\cos(x(n-m))}{m-n} \bigg|_{-\pi}^\pi \right] &=& 0\\
        \frac{1}{2} \left[ \frac{-1}{n+m} + \frac{1}{n+m} + \frac{-1}{m-n} + \frac{1}{m-n} \right] &=& 0\\
        0 &=& 0\\
    \end{align*}

    @@@ \[ \int_{-\pi}^\pi \left[ 1 \cdot \cos(mx) \right] \, dx = 0 \]

    \begin{align*}
        \int_{-\pi}^\pi \left[ 1 \cdot \cos(mx) \right] \, dx &=& 0\\
        \int_{-\pi}^\pi \cos(mx) \, dx &=& 0\\
        \frac{-\sin(mx)}{m} \bigg|_{-\pi}^\pi &=& 0\\
        \frac{0}{m} - \frac{0}{m} &=& 0\\
        0 &=& 0\\
    \end{align*}

    @@@ \[ \int_{-\pi}^\pi \left[ \cos(nx) \cos(nx) \right] \, dx = \pi \]

    \begin{align*}
        \int_{-\pi}^\pi \left[ \cos(nx) \cos(nx) \right] \, dx &=& \pi\\
        \int_{-\pi}^\pi \left[ \frac{1 + \cos(2nx)}{2} \right] \, dx &=& \pi\\
        \int_{-\pi}^\pi \frac{1}{2} \, dx + \int_{-\pi}^\pi \frac{\cos(2nx)}{2} \, dx &=& \pi\\
        \pi + \frac{\sin (2 \pi  n)}{2 n} &=& \pi\\
        \pi &=& \pi\\
    \end{align*}

    @@ For functions defined on $-L < y < L$, the Fourier basis functions are:

    \[
        \left[
            1,
            {\left\{
                \sin \left( \frac{m \pi y}{L} \right),
                \cos \left( \frac{m \pi y}{L} \right)
            \right\}}_{n=1}^\infty
        \right]
    \]

    Write down the relations among these functions, corresponding to the relations in (a), in the same order. Then show
    the validity of any one of the first three relations and any one of the last three, by direct calculation.\\

    @@@ \[ \int_{-L}^L \left[ \sin\left(\frac{m \pi x}{L} \right) \sin\left(\frac{n \pi x}{L} \right) \right] \, dx = 0 \quad \text{if } n \neq m \]

    $\sin\left(\frac{m \pi y}{L} \right)$ and $\sin\left(\frac{n \pi y}{L} \right)$ are orthogonal if $n \neq m$.

    \begin{align*}
        \int_{-L}^L \left[ \sin\left(\frac{m \pi x}{L} \right) \sin\left(\frac{n \pi x}{L} \right) \right] \, dx &=& 0\\
        \frac{1}{2} \int_{-L}^L \left[ \cos\left(\frac{m \pi x}{L} - \frac{nx\pi}{L}\right) - \cos\left(\frac{n \pi x}{L} + \frac{mx\pi}{L}\right) \right] \, dx &=& 0\\
        \frac{1}{2} \left[ \int_{-L}^L \cos\left(\frac{\pi x}{L}(m - n) \right) \, dx -
            \int_{-L}^L \cos\left(\frac{\pi x}{L} (m + n) \right) \, dx \right] &=& 0\\
        \frac{1}{2} \left[ \frac{\sin\left( \frac{\pi x}{L} (m - n) \right)}{\frac{\pi}{L}(m - n)} \bigg|_{-L}^L -
            \frac{\sin\left( \frac{\pi x}{L} (m + n) \right)}{\frac{\pi}{L}(m + n)} \bigg|_{-L}^L \right] &=& 0\\
        \frac{L \left(\frac{\sin (\pi (m-n))}{m-n}-\frac{\sin (\pi (m+n))}{m+n}\right)}{\pi } &=& 0\\
        \frac{L(0 - 0)}{\pi} &=& 0\\
        0 &=& 0\\
    \end{align*}

    @@@ \[ \int_{-L}^L \left[ \cos\left(\frac{n \pi x}{L}\right) \cos\left(\frac{m \pi x}{L}\right) \right] \, dx = 0 \quad \text{if } n \neq m \]

    $\cos\left(\frac{n \pi x}{L}\right)$ and $\cos\left(\frac{m \pi x}{L}\right)$ are orthogonal if $n \neq m$.

    @@@ \[ \int_{-L}^L \left[ 1 \cdot \sin\left(\frac{mx\pi}{L}\right) \right] \, dx = 0 \]

    $1$ and $\sin\left(\frac{mx\pi}{L}\right)$ are orthogonal.

    @@@ \[ \int_{-L}^L \left[ \sin\left(\frac{nx\pi}{L}\right) \sin\left(\frac{nx\pi}{L}\right) \right] \, dx = L \]

    $\sin\left(\frac{nx\pi}{L}\right)$ and $\sin\left(\frac{nx\pi}{L}\right)$ are the same, so when combined with the
    inner product, they yield the length of the vector.

    @@@ \[ \int_{-L}^L \left[ 1 \cdot 1 \right] \, dx = 2 L \]

    Similar to above, $1$ and $1$ when combined with the inner product yield the length of the vector.

    @@@ \[ \int_{-L}^L \left[ \sin\left(\frac{nx\pi}{L}\right) \cos\left(\frac{mx\pi}{L}\right) \right] \, dx = 0 \]

    $\sin\left(\frac{nx\pi}{L}\right)$ and $\cos\left(\frac{mx\pi}{L}\right)$ are orthogonal.

    @@@ \[ \int_{-L}^L \left[ 1 \cdot \cos\left(\frac{mx\pi}{L}\right) \right] \, dx = 0 \]

    $1$ and $\cos\left(\frac{mx\pi}{L}\right)$ are orthogonal.

    @@@ \[ \int_{-L}^L \left[ \cos\left(\frac{nx\pi}{L}\right) \cos\left(\frac{nx\pi}{L}\right) \right] \, dx = L \]

    As the other above questions, $\cos\left(\frac{nx\pi}{L}\right)$ and $\cos\left(\frac{nx\pi}{L}\right)$ when
    combined with the inner product yields the length of the vector.

    \begin{align*}
        \int_{-L}^L \left[ \cos\left(\frac{nx\pi}{L}\right) \cos\left(\frac{nx\pi}{L}\right) \right] \, dx &=& L\\
        \int_{-L}^L \left[ \frac{1 + \cos\left(\frac{2nx\pi}{L}\right) }{2} \right] \, dx &=& L\\
        \int_{-L}^L \frac{1}{2} \, dx + \int_{-L}^L \frac{\cos\left(\frac{2nx\pi}{L}\right)}{2} \, dx &=& L\\
        L + \frac{L \sin (2 \pi  n)}{2 \pi  n} &=& L\\
        L + 0 &=& L\\
        L &=& L\\
    \end{align*}

    \newpage
    @ If $g(y)$ is defined on $-L \le y \le L$ and if its Fourier series converges to $g(y)$ for all $y$ in $(-L, L)$,
    then its Fourier series has the form

    \[
        g(y) = a_0 + \sum_{m=1}^\infty \left[ a_m \cos \left( \frac{m \pi y}{L} \right) \right] +
            \sum_{m=1}^\infty \left[ b_m \sin \left( \frac{m \pi y}{L} \right) \right]
    \]

    for some set of coefficients, $[a_0, {\{a_m\}}_{m=1}^\infty, {\{b_m\}}_{m=1}^\infty]$.

    @@ Use your results from 1(b) to derive an appropriate formula for each of $[a_0, {\{a_m\}}_{m=1}^\infty,
    {\{b_m\}}_{m=1}^\infty]$ as an integral involving $g(y)$, over $-L \le y \le L$.\\

    Since the coefficients that are bounded on $[-\pi, \pi]$ are defined as the inner product of each basis and the
    function we can define ours as the same.

    \begin{align*}
        \langle 1, g(y) \rangle &=& \left\langle 1 ,
            a_0 + \sum_{m=1}^\infty \left[ a_m \cos \left( \frac{m \pi y}{L} \right) \right] +
            \sum_{m=1}^\infty \left[ b_m \sin \left( \frac{m \pi y}{L} \right) \right] \right\rangle\\
        &=& \left\langle 1 , a_0 \right\rangle + \sum_{m=1}^\infty \left[ a_m \left\langle 1, \cos \left( \frac{m \pi y}{L} \right) \right\rangle\right] +
            \sum_{m=1}^\infty \left[ b_m \left\langle 1, \sin \left( \frac{m \pi y}{L} \right) \right\rangle \right] \\
        &=& \left\langle 1 , a_0 \right\rangle + \sum_{m=1}^\infty \left[ a_m \left( \frac{2 L \sin (\pi  m)}{\pi m} \right) \right] +
            \sum_{m=1}^\infty \left[ b_m \cdot 0 \right] \\
        &=& \int_{L}^L a_0 \, dy\\
        &=& 2L a_0
    \end{align*}

    Therefore,

    \begin{align*}
        a_0 &=& \frac{1}{2L} \int_{-L}^L g(y) \, dy
    \end{align*}

    Now we can calculate $a_m$ in a similar way.

    \begin{align*}
        \left\langle \cos \left( \frac{ ny\pi }{L} \right) , g(y) \right\rangle &=& \left\langle \cos \left( \frac{ ny\pi }{L} \right),
            a_0 + \sum_{m=1}^\infty \left[ a_m \cos \left( \frac{m \pi y}{L} \right) \right] +
            \sum_{m=1}^\infty \left[ b_m \sin \left( \frac{m \pi y}{L} \right) \right] \right\rangle\\
        &=& \int_{-L}^L \cos \left( \frac{ ny\pi }{L} \right) \cdot\\
            &&\left[ a_0 + \sum_{m=1}^\infty \left[ a_m \cos \left( \frac{m \pi y}{L} \right) \right] +
            \sum_{m=1}^\infty \left[ b_m \sin \left( \frac{m \pi y}{L} \right) \right] \right] \, dy\\
        &=& \int_{-L}^L \left( a_0 \cos\left(\frac{ny\pi}{L}\right) \right) \, dy +\\
            &&\sum_{m=1}^\infty \left[ a_m \int_{-L}^L \cos\left(\frac{ny\pi}{L}\right) \cos \left( \frac{m \pi y}{L} \right) \, dy \right] +\\
            &&\sum_{m=1}^\infty \left[ b_m \int_{-L}^L \cos\left(\frac{ny\pi}{L}\right) \sin \left( \frac{m \pi y}{L} \right) \, dy \right]\\
        &=& \sum_{m=1}^\infty \left[ a_m \int_{-L}^L \cos\left(\frac{ny\pi}{L}\right) \cos \left( \frac{m \pi y}{L} \right) \, dy \right]\\
        &=& a_m L\\
    \end{align*}

    Therefore,

    \begin{align*}
        a_n &=& \frac{1}{L}\int_{-L}^L g(y) \cos \left( \frac{ny\pi}{L} \right) \, dy
    \end{align*}

    And finally $b_m$.

    \begin{align*}
        \left\langle \sin \left( \frac{ ny\pi }{L} \right) , g(y) \right\rangle &=& \left\langle \sin \left( \frac{ ny\pi }{L} \right),
            a_0 + \sum_{m=1}^\infty \left[ a_m \cos \left( \frac{m \pi y}{L} \right) \right] +
            \sum_{m=1}^\infty \left[ b_m \sin \left( \frac{m \pi y}{L} \right) \right] \right\rangle\\
        &=& \int_{-L}^L \sin \left( \frac{ ny\pi }{L} \right) \cdot\\
            &&\left[ a_0 + \sum_{m=1}^\infty \left[ a_m \cos \left( \frac{m \pi y}{L} \right) \right] +
            \sum_{m=1}^\infty \left[ b_m \sin \left( \frac{m \pi y}{L} \right) \right] \right] \, dy\\
        &=& \int_{-L}^L \left( a_0 \sin\left(\frac{ny\pi}{L}\right) \right) \, dy +\\
            &&\sum_{m=1}^\infty \left[ a_m \int_{-L}^L \sin\left(\frac{ny\pi}{L}\right) \cos \left( \frac{m \pi y}{L} \right) \, dy \right] +\\
            &&\sum_{m=1}^\infty \left[ b_m \int_{-L}^L \sin\left(\frac{ny\pi}{L}\right) \sin \left( \frac{m \pi y}{L} \right) \, dy \right]\\
        &=& \sum_{m=1}^\infty \left[ b_m \int_{-L}^L \sin\left(\frac{ny\pi}{L}\right) \sin \left( \frac{m \pi y}{L} \right) \, dy \right]\\
        &=& b_m L\\
    \end{align*}

    Therefore,

    \begin{align*}
        b_n &=& \frac{1}{L} \int_{-L}^L g(y) \sin \left( \frac{ny\pi}{L} \right) \, dy
    \end{align*}

    Let's test to see if our coefficients are correct. Let $f(x) = \abs{x}$.

\hfill\begin{minipage}{\dimexpr\textwidth-1cm}

\begin{minted}[mathescape, fontsize=\small, xleftmargin=0.5em]{python}
def fourier_func(f, l, x, n):
    # first term is value, second is error
    # In form: a0 + avals + bvals
    a0 = ((1 / (2 * l)) * si.quad(lambda z: f(z), -l, l)[0])
    an = np.array([sum([(1 / l) * si.quad(lambda z: f(z) *
                            np.cos(i * np.pi * z / l), -l, l)[0] *
                        np.cos(i * np.pi * x0 / l)
                            for i in range(1, n + 1)]) for x0 in x])
    bn = np.array([sum([(1 / l) * si.quad(lambda z: f(z) *
                            np.sin(i * np.pi * z / l), -l, l)[0] *
                        np.sin(i * np.pi * x0 / l)
                            for i in range(1, n + 1)]) for x0 in x])
    res = a0 + an + bn
    return res
l = 10
x = np.arange(-l, l, 0.1)
f = lambda x: np.abs(x)
plt.figure()
plt.plot(x, f(x), label=r'$f(x)=|x|$')
plt.plot(x, fourier_func(f, l, x, 1), label='1 Term Expansion')
plt.plot(x, fourier_func(f, l, x, 2), label='2 Term Expansion')
plt.plot(x, fourier_func(f, l, x, 5), label='5 Term Expansion')
plt.plot(x, fourier_func(f, l, x, 10), label='10 Term Expansion')
plt.legend(loc=1)
plt.xlim(-15, 15)
plt.ylim(0, 15)
plt.show()
\end{minted}
\includegraphics[width= \linewidth]{./figures/hw1_figure2_1.pdf}

\xdef\tpd{\the\prevdepth}
\end{minipage}\\

    \textbf{Awesome.}

    @@ Set $g(y) = y^2 - L^2$, for $-L \le y \le L$. Find its Fourier series explicitly.\\

    This is merely an extension of what we've done.

    First we calculate $a_0$.

    \begin{align*}
        a_0 &=& \frac{1}{2L} \int_{-L}^L g(y) \, dy\\
        &=& \frac{1}{2L} \int_{-L}^L \left( y^2 - L^2 \right) \, dy\\
        &=& -\frac{2 L^2}{3}
    \end{align*}

    Now we find $a_n$.

    \begin{align*}
        a_n &=& \frac{1}{L} \int_{-L}^L g(y) \cos \left( \frac{n y \pi}{L} \right) \, dy\\
        &=& \frac{1}{L} \int_{-L}^L (y^2 - L^2) \cos \left( \frac{n y \pi}{L} \right) \, dy\\
        &=& \frac{4 L^2 (\pi  n \cos (\pi  n)-\sin (\pi n))}{\pi ^3 n^3}\\
        &=& \frac{4 L^2 \cos (\pi  n)}{\pi^2 n^2}\\
    \end{align*}

    And finally, $b_n$.

    \begin{align*}
        a_n &=& \frac{1}{L} \int_{-L}^L g(y) \sin \left( \frac{n y \pi}{L} \right) \, dy\\
        &=& \frac{1}{L} \int_{-L}^L (y^2 - L^2) \sin \left( \frac{n y \pi}{L} \right) \, dy\\
        &=& 0\\
    \end{align*}

    This yields the resulting series:

    \begin{align*}
        g(y) &\approx& a_0 +
            \sum_{m=1}^\infty a_m \cos \left( \frac{m \pi y}{L} \right) +
            \sum_{m=1}^\infty b_m \sin \left( \frac{m \pi y}{L} \right)\\
        &\approx& -\frac{2 L^2}{3} + \sum_{m=1}^\infty \left[ \frac{4 L^2 \cos (\pi  m)}{\pi^2 m^2} \cos \left( \frac{m \pi y}{L} \right) \right]\\
    \end{align*}

    We can test this for $L=10$.\\

\hfill\begin{minipage}{\dimexpr\textwidth-1cm}

\begin{minted}[mathescape, fontsize=\small, xleftmargin=0.5em]{python}
l = 10
x = np.arange(-l, l, 0.1)
f = lambda x: x**2 - l**2
def calculated_fourier(y, n):
    a0 = -(2 * l**2 / 3)
    an = np.array([sum([((4 * l**2 * np.cos(m * np.pi)) /
                            (m**2 * np.pi**2)) *
                        np.cos((m * np.pi * y0) / l)
                        for m in range(1, n + 1)]) for y0 in y])
    res = a0 + an
    return res
plt.figure()
plt.plot(x, f(x), label=r'$f(x) = y^2 - {10}^2$')
plt.plot(x, fourier_func(f, l, x, 1), label='1 Term Expansion')
plt.plot(x, fourier_func(f, l, x, 2), label='2 Term Expansion')
plt.plot(x, fourier_func(f, l, x, 5), label='5 Term Expansion')
plt.plot(x, fourier_func(f, l, x, 10), label='10 Term Expansion')
plt.plot(x, calculated_fourier(x, 10),
            label='Calculated Fourier Expansion')
plt.legend()
plt.xlim(-15, 15)
plt.ylim(-120, 75)
plt.show()
\end{minted}
\includegraphics[width= \linewidth]{./figures/hw1_figure3_1.pdf}

\xdef\tpd{\the\prevdepth}
\end{minipage}\\

    \textbf{Awesome.}

    \newpage
    @ Infinite series convergence.

    @@ In problem 2(b), you found the Fourier series for $g(y) = y^2 - L^2$ on $-L \le y \le L$. Write down
    $\{y^2-L^2=\text{Your Fourier Series}\}$ from 2(b).

    \[
        g(y) \approx -\frac{2 L^2}{3} + \sum_{m=1}^\infty \left[ \frac{4 L^2 \cos (\pi  m)}{\pi^2 m^2} \cos \left( \frac{m \pi y}{L} \right) \right]\\
    \]

    @@ Give the first six non-zero terms of your series for $-L \le x < L$ explicitly.

    \begin{align*}
        g(y) &\approx& -\frac{2 L^2}{3} + \sum_{m=1}^6 \left[ \frac{4 L^2 \cos (\pi  m)}{\pi^2 m^2} \cos \left( \frac{m \pi y}{L} \right) \right]\\
        &\approx& -\frac{2 L^2}{3} +
            \frac{-4L^2}{\pi^2}\cos\left(\frac{\pi y}{L}\right) +
            \frac{L^2}{\pi^2}\cos\left(\frac{2\pi y}{L}\right) +\\
            &&\frac{-4L^2}{\pi^2 9}\cos\left(\frac{3\pi y}{L}\right) +
            \frac{L^2}{\pi^2 4}\cos\left(\frac{4\pi y}{L}\right) +\\
            &&\frac{-4L^2}{\pi^2 25}\cos\left(\frac{5\pi y}{L}\right) +
            \frac{L^2}{\pi^2 9}\cos\left(\frac{6\pi y}{L}\right)\\
    \end{align*}

    @@ Evaluate the left and right sides of your answer to 3(a) at $y=0$, and obtain an explicit numerical value for
    $\sum_{n=1}^\infty \left[ \frac{{(-1)}^{n+1}}{n^2} \right]$. Check the validity of your result by calculating the
    first few partial sums numerically, to see if they are approaching your exact value for the sum.

    @@@

    \begin{align*}
        y^2 - L^2 &\approx& -\frac{2 L^2}{3} + \sum_{m=1}^\infty \left[ \frac{4 L^2 \cos (\pi  m)}{\pi^2 m^2} \cos \left( \frac{m \pi y}{L} \right) \right]\\
        -L^2 &\approx& -\frac{2 L^2}{3} + \sum_{m=1}^\infty \left[ \frac{4 L^2 \cos (\pi  m)}{\pi^2 m^2} \right]\\
        -L^2 &\approx& -\frac{2 L^2}{3} + \frac{4L^2}{\pi^2} \sum_{m=1}^\infty \left[ \frac{{(-1)}^m}{m^2} \right]\\
    \end{align*}

    @@@

    \[
        \sum_{n=1}^\infty \left[ \frac{{(-1)}^{n}}{n^2} \right] = -\frac{\pi^2}{12}
    \]

    @@@ We can combine these results.

    \begin{align*}
        -L^2 &\approx& -\frac{2 L^2}{3} + \frac{4L^2}{\pi^2} \sum_{m=1}^\infty \left[ \frac{{(-1)}^m}{m^2} \right]\\
        -L^2 &\approx& -\frac{2 L^2}{3} + \frac{4L^2}{\pi^2} \left( -\frac{\pi^2}{12} \right)\\
        -L^2 &\approx& -\frac{2 L^2}{3} + -\frac{L^2}{3}\\
        -L^2 &\approx& -L^2\\
    \end{align*}

    @@ Evaluate your answer to 3(a) at $y=L$, and obtain an explicit formula for $\sum_{n=1}^\infty \frac{1}{n^2}$.
    Check your result by calculating the first few partial sums.

    \begin{align*}
        L^2 - L^2 &\approx& -\frac{2 L^2}{3} + \sum_{m=1}^\infty \left[ \frac{4 L^2 \cos (\pi  m)}{\pi^2 m^2} \cos \left( \frac{m \pi L}{L} \right) \right]\\
        0 &\approx& -\frac{2 L^2}{3} + \frac{4L^2}{\pi^2} \sum_{m=1}^\infty \left[ \frac{\cos^2(\pi  m)}{m^2} \right]\\
        0 &\approx& -\frac{2 L^2}{3} + \frac{4L^2}{\pi^2} \left( \frac{\pi^2}{6} \right)\\
        0 &\approx& -\frac{2 L^2}{3} + \frac{2L^2}{3}\\
        0 &\approx& 0\\
    \end{align*}

    @@ Add your results from parts (b) and (c) to obtain an explicit formula for

    \begin{align*}
        \sum_{n=1}^\infty \frac{1}{{(2n - 1)}^2} &=& 1 + \frac{1}{9} + \frac{1}{25} + \cdots\\
        &=& \frac{\pi^2}{8}
    \end{align*}

    @@ Evaluate your answer to 3(a) at $y = L / 2$. What happens?

    \begin{align*}
        g\left(\frac{L}{2}\right) &\approx& -\frac{2 L^2}{3} + \sum_{m=1}^\infty \left[ \frac{4 L^2 \cos (\pi  m)}{\pi^2 m^2} \cos \left( \frac{m \pi {\left(\frac{L}{2}\right)}}{L} \right) \right]\\
        &\approx& -\frac{2 L^2}{3} + \frac{4L^2}{\pi^2} \sum_{m=1}^\infty \left[ \frac{\cos (\pi  m)}{m^2} \cos \left( \frac{m \pi}{2} \right) \right]\\
        &\approx& -\frac{2 L^2}{3} + \frac{4L^2}{\pi^2} \left( \frac{-\pi^2}{48} \right)\\
        &\approx& -\frac{2 L^2}{3} + \frac{-L^2}{12}\\
        &\approx& -\frac{3 L^2}{4}\\
    \end{align*}

    @@ What happens if you evaluate your answer to 3(a) at $y=L/3$?

    \begin{align*}
        g\left(\frac{L}{3}\right) &\approx& -\frac{2 L^2}{3} + \sum_{m=1}^\infty \left[ \frac{4 L^2 \cos (\pi  m)}{\pi^2 m^2} \cos \left( \frac{m \pi {\left(\frac{L}{3}\right)}}{L} \right) \right]\\
        &\approx& -\frac{2 L^2}{3} + \frac{4L^2}{\pi^2} \sum_{m=1}^\infty \left[ \frac{\cos (\pi  m)}{m^2} \cos \left( \frac{m \pi {\left(\frac{L}{3}\right)}}{L} \right) \right]\\
        &\approx& -\frac{2 L^2}{3} + \frac{4L^2}{\pi^2} \sum_{m=1}^\infty \left[ \frac{\cos (\pi  m)}{m^2} \cos \left( \frac{m \pi}{3} \right) \right]\\
    \end{align*}

    A nightmare\ldots

    \newpage
    @ In this problem you will investigate implications on the coefficients of Fourier series when a function $f(x)$
    defined on $-\pi \le x < \pi$ exhibits symmetry.

    @@ Prove that if $f(x)$ is an odd function, then its Fourier series consists only of sine terms, called a sine
    series.\\

    Let us establish some function $f(x)$ such that $f$ is defined on $-\pi \le x < \pi$. By direct calculation we will
    show that such a function, if it is odd, has only sine terms in its Fourier Series Expansion.\\

    The Fourier Series equation is defined as the following.

    \[
        f(x) \approx a_0 + \sum_{n=1}^\infty \left[ a_n \cos(nx) \right] + \sum_{n=1}^\infty \left[ b_n \sin(nx) \right]
    \]

    With coefficients defined as

    \begin{align*}
        a_0 &=& \frac{1}{2 \pi} \int_{-\pi}^\pi f(x) \, dx\\
        a_n &=& \frac{1}{\pi} \int_{-\pi}^\pi f(x) \cos(nx) \, dx\\
        b_n &=& \frac{1}{\pi} \int_{-\pi}^\pi f(x) \sin(nx) \, dx\\
    \end{align*}

    Since we know that $f(x)$ is odd, we can simplify our coefficients somewhat.

    \begin{align*}
        a_0 &=& 0\\
        a_n &=& 0\\
        b_n &=& \frac{1}{\pi} \int_{-\pi}^\pi \underbrace{f(x) \sin(nx)}_{\text{Even Function}} \, dx\\
    \end{align*}

    We can rewrite our original Fourier Expansion.

    \[
        f(x) \approx \sum_{n=1}^\infty \left[ \left( \frac{1}{\pi} \int_{-\pi}^\pi f(x) \sin(nx) \, dx \right) \sin(nx) \right]
    \]

    Because $\cos(nx)$ will either be $-1$ or $1$, this can be reduced to only contain sine terms.

    This, of course, is also extensible to the interval $[-L, L]$.

    @@ Suppose $f(x)$ is defined on for $0 \le x \le \pi$. Consider the odd extension of $f(x)$, $f_{odd}(x)$, as
    follows.

    \[
        f_{odd}(x) = \begin{cases}
            f(x) \qquad &0 \le x \le \pi\\
            -f(-x) \qquad &-\pi \le x < 0\\
        \end{cases}
    \]

    By part (a), we know that this odd function will have a sine series expansion. We will call this the sine series of
    $f(x)$. Write a formula for the sine series coefficients ${\{b_m\}}_{m=1}^\infty$ in terms of $f(x)$. What is the
    value of the sine series evaluated at $x=0$ and $x=\pi$?\\

    Writing the formula for the coefficients is simple, as we've already done it in the previous part.

    \[
         {\{b_n\}}_{n=1}^\infty = \frac{1}{\pi} \int_{-\pi}^\pi f(x) \sin(nx) \, dx
    \]

    We can find this precise value when $x = 0$ and when $x = \pi$.

    @@@ $x = 0$
    \begin{align*}
         {\{b_n\}}_{n=1}^\infty &=& \frac{1}{\pi} \int_{-\pi}^\pi f(0) \sin(0) \, dx\\
         &=& 0\\
    \end{align*}

    @@@ $x = \pi$
    \begin{align*}
         {\{b_n\}}_{n=1}^\infty &=& \frac{1}{\pi} \int_{-\pi}^\pi f(\pi) \sin(n\pi) \, dx\\
         &=& 0\\
    \end{align*}

    @@ Using your result from part (b), determine the sine series expansion for the functions $f(x)=\cos(x)$ and
    $g(x)=\sin(x)$ each restricted to the domain $0 \le x \le \pi$. Plot some partial sums of these sine series and
    describe their behavior qualitatively as the number of terms is increased.\\

    We simply just use our formula from before.

    @@@ $f(x) = \cos(x)$

    \begin{align*}
        {\{b_n\}}_{n=1}^\infty &=& \frac{1}{\pi} \int_0^\pi \cos(x) \sin(nx) \, dx\\
        &=& \frac{n (\cos (\pi  n)+1)}{\pi \left(n^2-1\right)}
    \end{align*}

    These terms will get smaller and smaller, which makes sense, as the sine term is trying to fit to a cosine term.

    And hey! Let's plot that.\\

\hfill\begin{minipage}{\dimexpr\textwidth-1cm}

\begin{minted}[mathescape, fontsize=\small, xleftmargin=0.5em]{python}
x = np.arange(2, 100, dtype=int)
def terms(n):
    res = []
    for m in n:
        res.append(sum([-((i * (np.cos(np.pi * i) + 1)) / (np.pi *
(i**2 - 1))) for i in range(2, m + 1)]))
    return np.array(res)
plt.figure()
plt.scatter(x, terms(x))
plt.show()
\end{minted}
\includegraphics[width= \linewidth]{./figures/hw1_figure4_1.pdf}

\xdef\tpd{\the\prevdepth}
\end{minipage}\\

    @@@ $f(x) = \sin(x)$

    \begin{align*}
        {\{b_n\}}_{n=1}^\infty &=& \frac{1}{\pi} \int_0^\pi \sin(x) \sin(nx) \, dx\\
        &=& -\frac{\sin (\pi  n)}{\pi  \left(n^2-1\right)}
        &=& 0\\
    \end{align*}

    We're not going to plot this one, because it's always zero. Intuitively this makes a lot of sense. Since our Fourier
    Series solely contains sine terms, than we should really only need one, leaving the rest to be zero. This is evident
    in the below graphs that show that for any $n$, the Fourier Expansion is accurate.

    @@@ We will examine the Fourier Expansion.

\hfill\begin{minipage}{\dimexpr\textwidth-1cm}

\begin{minted}[mathescape, fontsize=\small, xleftmargin=0.5em]{python}
l = np.pi
x = np.arange(-l, l, 0.01)
f1 = lambda x: np.cos(x)
f2 = lambda x: np.sin(x)
f, axarr = plt.subplots(2)
axarr[0].plot(x, f1(x))
axarr[0].plot(x, fourier_func(f1, l, x, 1))
axarr[0].plot(x, fourier_func(f1, l, x, 2))
axarr[0].plot(x, fourier_func(f1, l, x, 5))
axarr[0].plot(x, fourier_func(f1, l, x, 10))
axarr[1].plot(x, f2(x))
axarr[1].plot(x, fourier_func(f2, l, x, 1))
axarr[1].plot(x, fourier_func(f2, l, x, 2))
axarr[1].plot(x, fourier_func(f2, l, x, 5))
axarr[1].plot(x, fourier_func(f2, l, x, 10))
plt.show()
\end{minted}
\includegraphics[width= \linewidth]{./figures/hw1_figure5_1.pdf}

\xdef\tpd{\the\prevdepth}
\end{minipage}\\
\end{easylist}

\end{document}
