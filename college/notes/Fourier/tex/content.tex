\section{Review}

    \subsection{Series}

    \[
        \sum_{n=0}^\infty r^n = 0 \qquad 0 < r < 1
    \]

    \subsection{Ordinary Differential Equations}

    \begin{thm}[Picard's Theorem]
        If an $n$th order ODE has $n$ initial conditions, then the solutions exists in a neighborhood of initial conditions.
        The solution is unique when

        \[
            \frac{\partial f}{\partial y}(x_0, y_0) \neq 0
        \]

        TL;DR An $n$th order equation needs $n$ initial conditions.
    \end{thm}

    TODO: Flesh out ODE review

    \subsection{Calculus 3}

    TODO: Flesh out calc 3 review

\section{Fourier Series}

\begin{thm}[Fourier Series Expansion on $-\pi \le x \le \pi$]
    If we change basis of functions we can write them as a Fourier Series with the following equation on the interval
    $[-\pi, \pi]$.

    \begin{equation}
    a_0 + \sum_{n=1}^\infty \left[ a_n \cos(nx) + b_n \sin(nx) \right]
    \end{equation}

    These coefficients are as follows for $[-\pi, \pi]$.

    \begin{align*}
        a_0 &= \frac{1}{2\pi} \int_{-\pi}^\pi f(x) \, dx\\
        a_n &= \frac{1}{\pi} \int_{-\pi}^\pi f(x) \cos(nx) \, dx\\
        b_n &= \frac{1}{\pi} \int_{-\pi}^\pi f(x) \sin(nx) \, dx\\
    \end{align*}

    This function is defined everywhere and is $2\pi$ periodic.
\end{thm}

\begin{thm}[Fourier Series Expansion on $-L \le x \le L$]
    Same equation, but different coefficients.

    \begin{align*}
        a_0 &= \frac{1}{2L} \int_{-L}^L f(x) \, dx\\
        a_n &= \frac{1}{L} \int_{-L}^L f(x) \cos\left(\frac{nx\pi}{L}\right) \, dx\\
        b_n &= \frac{1}{L} \int_{-L}^L f(x) \sin\left(\frac{nx\pi}{L}\right) \, dx\\
    \end{align*}
\end{thm}

    \subsection{Piecewise Continuity}
    A function is \textbf{Piecewise Smooth} on some interval if it can be broken up into pieces such that in each piece
    the function is continuous and its derivative is also continuous. The entire function does not need to be
    continuous, but can only have a finite number of jump discontinuities\footnote{Jump discontinuities are when the
    left and right limits both exist and are unequal.}.

    \subsection{Convergence for Fourier Series}
    \begin{thm}[Fourier's Theorem]
        If $f(x)$ is piecewise smooth on the interval $-L \le x \le L$, then the Fourier series of $f(x)$ converges

        \begin{easylist}[enumerate]
            @ to the periodic extension of $f(x)$, where the periodic extension is continuous;
            @ to the average of the two limits, usually

            \[
                \frac{1}{2} [ f(x_+) + f(x_-) ]
            \]

            where the periodic extension has a jump discontinuity.
        \end{easylist}
    \end{thm}

    \subsection{Cosine and Sine Series}

        \subsubsection{Sine Series}
        An odd function is defined as $f(-x) = f(x)$\footnote{In other words, reflected about the $y$ axis}. We can
        calculate the Fourier Series of odd functions.

        \begin{align*}
            a_0 &= \frac{1}{2L} \int_{-L}^L f(x) \, dx = 0\\
            a_n &= \frac{1}{L} \int_{-L}^L f(x) \cos\left(\frac{nx\pi}{L}\right) \, dx = 0\\
            b_n &= \frac{2}{L} \int_0^L f(x) \sin\left(\frac{nx\pi}{L}\right) \, dx\\
        \end{align*}

        However this doesn't happen often, and instead we usually just find the odd extension of a function.

        \begin{align*}
            f(x) &\sim \sum_{n=1}^\infty \left[ B_n \sin\left(\frac{nx\pi}{L}\right) \right]\\
            B_n &= \frac{2}{L} \int_0^L f(x) \sin\left(\frac{nx\pi}{L}\right) \, dx
        \end{align*}

        This is the Fourier sine series on the interval $0 \le x \le L$.

        \subsubsection{Cosine Series}
        We can apply similar methods to even functions, where $f(-x) = f(x)$.

        \begin{align*}
            f(x) &\sim \sum_{n=0}^\infty \left[ a_n \cos\left(\frac{nx\pi}{L}\right) \right]\\
            a_0 &= \frac{1}{L} \int_0^L f(x) \, dx\\
            a_n &= \frac{2}{L} \int_0^L f(x) \cos\left(\frac{n\pi x}{L}\right)\\
            b_n &= 0\\
        \end{align*}

        Similarly, we can find the even extension of a given function.

        \begin{align*}
            f(x) &\sim \sum_{n=0}^\infty \left[ A_n \cos\left(\frac{nx\pi}{L}\right) \right]\\
            A_0 &= \frac{1}{L} \int_0^L f(x) \, dx\\
            A_n &= \frac{2}{L} \int_0^L f(x) \cos\left(\frac{nx\pi}{L}\right) \, dx
        \end{align*}

        \subsubsection{Even and Odd Parts}
        \begin{thm}
            The Fourier series of $f(x)$ equals the Fourier sine series of $f_o(x)$ plus the Fourier cosine series of
            $f_e(x)$, where $f_e(x) = \frac{1}{2} [f(x) + f(-x)]$, and $f_o(x) = \frac{1}{2} [f(x) - f(-x)]$.
        \end{thm}

