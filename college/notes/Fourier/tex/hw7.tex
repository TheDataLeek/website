\documentclass[10pt]{article}

%%%%%%%%%%%%%%%%%%%%%%%%%%%%%%%%%%%%%%%%%%%%%%%%%%%%%%%%%%%%%%%%%%%%%%%%%%%%%%%%
% LaTeX Imports
%%%%%%%%%%%%%%%%%%%%%%%%%%%%%%%%%%%%%%%%%%%%%%%%%%%%%%%%%%%%%%%%%%%%%%%%%%%%%%%%
\usepackage{amsfonts}                                                   % Math fonts
\usepackage{amsmath}                                                    % Math formatting
\usepackage{amssymb}                                                    % Math formatting
\usepackage{amsthm}                                                     % Math Theorems
\usepackage{arydshln}                                                   % Dashed hlines
\usepackage{attachfile}                                                 % AttachFiles
\usepackage{cancel}                                                     % Cancelled math
\usepackage{caption}                                                    % Figure captioning
\usepackage{color}                                                      % Nice Colors
\input{./lib/dragon.inp}                                                % Tikz dragon curve
\usepackage[ampersand]{easylist}                                        % Easy lists
\usepackage{fancyhdr}                                                   % Fancy Header
\usepackage[T1]{fontenc}                                                % Specific font-encoding
%\usepackage[margin=1in, marginparwidth=2cm, marginparsep=2cm]{geometry} % Margins
\usepackage{graphicx}                                                   % Include images
\usepackage{hyperref}                                                   % Referencing
\usepackage[none]{hyphenat}                                             % Don't allow hyphenation
\usepackage{lipsum}                                                     % Lorem Ipsum Dummy Text
\usepackage{listings}                                                   % Code display
\usepackage{marginnote}                                                 % Notes in the margin
\usepackage{microtype}                                                  % Niceness
\usepackage{lib/minted}                                                 % Code display
\usepackage{multirow}                                                   % Multirow tables
\usepackage{pdfpages}                                                   % Include pdfs
\usepackage{pgfplots}                                                   % Create Pictures
\usepackage{rotating}                                                   % Figure rotation
\usepackage{setspace}                                                   % Allow double spacing
\usepackage{subcaption}                                                 % Figure captioning
\usepackage{tikz}                                                       % Create Pictures
\usepackage{tocloft}                                                    % List of Equations
%%%%%%%%%%%%%%%%%%%%%%%%%%%%%%%%%%%%%%%%%%%%%%%%%%%%%%%%%%%%%%%%%%%%%%%%%%%%%%%%
% Package Setup
%%%%%%%%%%%%%%%%%%%%%%%%%%%%%%%%%%%%%%%%%%%%%%%%%%%%%%%%%%%%%%%%%%%%%%%%%%%%%%%%
\hypersetup{%                                                           % Setup linking
    colorlinks=true,
    linkcolor=black,
    citecolor=black,
    filecolor=black,
    urlcolor=black,
}
\RequirePackage[l2tabu, orthodox]{nag}                                  % Nag about bad syntax
\renewcommand*\thesection{\arabic{section} }                             % Reset numbering
\renewcommand{\theFancyVerbLine}{ {\arabic{FancyVerbLine} } }              % Needed for code display
\renewcommand{\footrulewidth}{0.4pt}                                    % Footer hline
\setcounter{secnumdepth}{3}                                             % Include subsubsections in numbering
\setcounter{tocdepth}{3}                                                % Include subsubsections in toc
%%%%%%%%%%%%%%%%%%%%%%%%%%%%%%%%%%%%%%%%%%%%%%%%%%%%%%%%%%%%%%%%%%%%%%%%%%%%%%%%
% Custom commands
%%%%%%%%%%%%%%%%%%%%%%%%%%%%%%%%%%%%%%%%%%%%%%%%%%%%%%%%%%%%%%%%%%%%%%%%%%%%%%%%
\newcommand{\nvec}[1]{\left\langle #1 \right\rangle}                    %  Easy to use vector
\newcommand{\ma}[0]{\mathbf{A} }                                         %  Easy to use vector
\newcommand{\mb}[0]{\mathbf{B} }                                         %  Easy to use vector
\newcommand{\abs}[1]{\left\lvert #1 \right\rvert}                       %  Easy to use abs
\newcommand{\pren}[1]{\left( #1 \right)}                                %  Big parens
\let\oldvec\vec
\renewcommand{\vec}[1]{\oldvec{\mathbf{#1} } }                            %  Vector Styling
\newtheorem{thm}{Theorem}                                               %  Define the theorem name
\newtheorem{definition}{Definition}                                     %  Define the definition name
\definecolor{bg}{rgb}{0.95,0.95,0.95}
\newcommand{\java}[4]{\vspace{10pt}\inputminted[firstline=#2,
                                 lastline=#3,
                                 firstnumber=#2,
                                 gobble=#4,
                                 frame=single,
                                 label=#1,
                                 bgcolor=bg,
                                 linenos]{java}{#1} }
\newcommand{\python}[4]{\vspace{10pt}\inputminted[firstline=#2,
                                 lastline=#3,
                                 firstnumber=#2,
                                 gobble=#4,
                                 frame=single,
                                 label=#1,
                                 bgcolor=bg,
                                 linenos]{python}{#1} }
\newcommand{\js}[4]{\vspace{10pt}\inputminted[firstline=#2,
                                 lastline=#3,
                                 firstnumber=#2,
                                 gobble=#4,
                                 frame=single,
                                 label=#1,
                                 bgcolor=bg,
                                 linenos]{js}{#1} }
%%%%%%%%%%%%%%%%%%%%%%%%%%%%%%%%%%%%%%%%%%%%%%%%%%%%%%%%%%%%%%%%%%%%%%%%%%%%%%%%
% Beginning of document items - headers, title, toc, etc...
%%%%%%%%%%%%%%%%%%%%%%%%%%%%%%%%%%%%%%%%%%%%%%%%%%%%%%%%%%%%%%%%%%%%%%%%%%%%%%%%
\pagestyle{fancy}                                                       %  Establishes that the headers will be defined
\fancyhead[LE,LO]{Computer Systems Notes}                                  %  Adds header to left
\fancyhead[RE,RO]{Zoe Farmer}                                       %  Adds header to right
\cfoot{ \thepage }
\lfoot{CSCI 2400}
\rfoot{Han}
\title{Computer Systems Notes}
\author{Zoe Farmer}

%%%%%%%%%%%%%%%%%%%%%%%%%%%%%%%%%%%%%%%%%%%%%%%%%%%%%%%%%%%%%%%%%%%%%%%%%%%%%%%%
% Beginning of document items - headers, title, toc, etc...
%%%%%%%%%%%%%%%%%%%%%%%%%%%%%%%%%%%%%%%%%%%%%%%%%%%%%%%%%%%%%%%%%%%%%%%%%%%%%%%%
\pagestyle{fancy}                                                       %  Establishes that the headers will be defined
\fancyhead[LE,LO]{Fourier Series}                                  %  Adds header to left
\fancyhead[RE,RO]{Zoe Farmer}                                       %  Adds header to right
\cfoot{\thepage}
\lfoot{APPM 4350}
\rfoot{Mark Hoefer}
\title{Fourier Series Homework Seven}
\author{Zoe Farmer}
%%%%%%%%%%%%%%%%%%%%%%%%%%%%%%%%%%%%%%%%%%%%%%%%%%%%%%%%%%%%%%%%%%%%%%%%%%%%%%%%
% Beginning of document items - headers, title, toc, etc...
%%%%%%%%%%%%%%%%%%%%%%%%%%%%%%%%%%%%%%%%%%%%%%%%%%%%%%%%%%%%%%%%%%%%%%%%%%%%%%%%
\begin{document}



\maketitle

\section{The Motion of a Piano String}
In class, we analyzed the motion of a guitar string, after it was plucked at a point ($x=d$) on a string of length $L$
with $0 < d < L$. We assumed that the initial displacement of the string was nonzero, but the initial velocity was zero.

A piano wire is activated when a hammer of width $w$ strikes the wire sharply. We may approximate the result by assuming
that the hammer gives the wire an initial velocity, but no initial displacement.

Let $u(x, t)$ denote the transverse displacement of a stretched piano wire. Assume that $u(x, t)$ satisfies

\ms{%
    u_{tt}    &= c^2 u_{xx} \qquad & 0 < x < L \qquad & t > 0 \quad & c^2 > 0\\
    u(0, t)   &= u(L, t) = 0       &           \qquad & t > 0       & \\
    u(x, 0)   &= 0                 & 0 < x < L \qquad & t = 0       & \\
    u_t(x, 0) &= V(x)              & 0 < x < L \qquad & t = 0       &
}

Where

\ms{%
    V(x) = \begin{cases}
        0 \qquad & 0 < x < \frac{L}{2} - \frac{w}{2}\\
        1 & \frac{L}{2} - \frac{w}{2} < x < \frac{L}{2} + \frac{w}{2}\\
        0 & \frac{L}{2} + \frac{w}{2} < x < L
    \end{cases}
}

The center of the hammer is assumed to be at $\frac{L}{2}$ to simplify.

\begin{easylist}[enumerate]
    @ Find $u(x, t)$ in the form of a Fourier Series.

    From the book, we can write the solution in the form

    \[
        u(x, t) = \sum_{n=1}^\infty
            \pren{%
                A_n \sin\bren{\frac{n\pi x}{L}} \cos\bren{\frac{n\pi ct}{L}}
                    + B_n \sin\bren{\frac{n\pi x}{L}} \sin\bren{\frac{n\pi ct}{L}}
                }
    \]

    Where

    \ms{%
        A_n &= \frac{2}{L} \int_0^L f(x) \sin\pren{\frac{n\pi x}{L}} \, dx\\
        &= 0\\
        B_n \frac{n\pi c}{L} &= \frac{2}{L} \int_0^L g(x) \sin\pren{\frac{n\pi x}{L}} \, dx\\
        &= \frac{2}{L} \int_{\frac{L}{2} - \frac{w}{2}}^{\frac{L}{2} + \frac{w}{2}} \sin\pren{\frac{n\pi x}{L}} \, dx\\
        &= \frac{4}{n\pi}\sin\pren{\frac{n\pi}{2}} \sin\pren{\frac{n\pi w}{2L}}
    }

    Yielding final solution

    \[
        u(x, t) = \sum_{n=1}^\infty
            \pren{%
                    \frac{4}{n\pi}\sin\pren{\frac{n\pi}{2}} \sin\pren{\frac{n\pi w}{2L}}
                    \sin\pren{\frac{n\pi x}{L}} \sin\pren{\frac{n\pi ct}{L}}
                }
    \]
    
    Which we can plot:

\simpleweave

\includegraphics[width= 4in]{/home/zoe/classwork/2015b/appm4350/homeworks/figures/hw7_figure1_1.pdf}

\nosimpleweave

    @ In your solution, what happens if $L$ is an integer multiple of $w$?

    Let $m$ be some non-zero integer. Using the substitution $L = mw$ we get the following:

    \ms{%
        u(x, t) &= \sum_{n=1}^\infty
            \pren{%
                    \frac{4}{n\pi}\sin\pren{\frac{n\pi}{2}} \sin\pren{\frac{n\pi w}{2mw}}
                    \sin\pren{\frac{n\pi x}{mw}} \sin\pren{\frac{n\pi ct}{mw}}
                }\\
        &= \sum_{n=1}^\infty
            \pren{%
                    \frac{4}{n\pi}\sin\pren{\frac{n\pi}{2}} \sin\pren{\frac{n\pi}{2m}}
                    \sin\pren{\frac{n\pi x}{mw}} \sin\pren{\frac{n\pi ct}{mw}}
                }
    }

    Plotting, we see that the behavior is chaotic.

    % Add picture! It's the chaotic one.
\simpleweave

\includegraphics[width= 4in]{/home/zoe/classwork/2015b/appm4350/homeworks/figures/hw7_figure2_1.pdf}

\nosimpleweave
    % Note, not 100% on these answers
    @ Is $u(x, t)$ continuous in $x$ for $0 < x < L$, $t > 0$? Why or Why Not?

    Yes, as it is simply a sum of continuous functions.

    @ Is $u(x, t)$ continuous in $t$ for $0 < x < L$, $t > 0$? Why or Why Not?

    Yes for the same reasons.

    @ Is $u_t(x, t)$ continuous in $x$ for $0 < x < L$, $t > 0$? Why or Why Not?

    Yes, for the same reasons.

    @ Is $u_t(x, t)$ continuous in $t$ for $0 < x < L$, $t > 0$? Why or Why Not?

    Yes for the same reasons.

    @ Suppose that a piano string and a guitar string are both tuned to play $A$ ($100 \, Hz$). In addition to the
    fundamental tone (at $110 \, c/s$), each string also generates higher harmonics. Do the two strings generate the
    same set of frequencies in their higher harmonics, or is there a difference?

    The two strings \textit{should} produce the same harmonics, as they are identical save the method in which they are
    played. The only difference between the strings is positioning, and play style. As the prompt notes, this will
    impact the amplitudes of the harmonics, however the frequencies should remain the same.
\end{easylist}

\newpage
\section{A Damped Guitar String}

The solution of the wave equation in 1D oscillates forever, unlike real guitar strings. A variation that approximates
damping is

\ms{%
    u_{tt} + \beta u_t &= c^2 u_{xx} \qquad & 0 < x < L \qquad & t > 0 \quad & c^2 > 0, \beta \ge 0\\
    u(0, t)            &= u(L, t) = 0       &           \qquad & t > 0       & \\
    u_t(x, 0)          &= 0                 & 0 < x < L \qquad & t = 0       & \\
    u(x, 0)            &= U(x)              & 0 < x < L \qquad & t = 0       &
}

Where

\ms{%
    U(x) = \begin{cases}
        a\pren{\frac{x}{d}} \qquad & 0 \le x \le d\\\\
        a \pren{\frac{L - x}{L - d}} & d < x < L
    \end{cases}
}

And where $u(x, t)$ represents the transverse displacement of the stretched string. The added constant is usually small,

\[
    0 \le \beta << \frac{c\pi}{L}
\]

\begin{easylist}[enumerate]
    @ Solve this problem, finding $u(x, t)$ in terms of a Fourier series, with explicit Fourier coefficients. Check by
    setting $\beta = 0$. Show that for $\beta > 0$, $u(x, t) \to 0$ as $t \to \infty$, at every $x \in (0, L)$.

    % https://www.math.hmc.edu/~ajb/PCMI/lecture7.pdf
    @@ We know that whatever solution we come to will have the form

    \ms{%
        u(x, t) = \sum_{n=1}^\infty f_n(t) \sin\pren{\frac{n\pi x}{L}}
    }

    @@ Using our equation, we get the following equation for $f_n$.

    \ms{%
        f^{\prime\prime}_n + \beta f^\prime_n + \pren{\alpha_n}^2 f_n = 0
    }

    Where $\alpha_n = \frac{cn\pi}{L}$, and

    \ms{%
        \lambda_n = \frac{\beta}{2} \pm \sqrt{\pren{\frac{\beta}{2}}^2 - \pren{\alpha_n}^2}
    }

    @@ Since $\beta$ is usually very small,

    \ms{%
        f_n(t) = e^{-\frac{\beta}{2} t}\pren{%
            a_n \cos\pren{\omega_n t} +
            b_n \sin\pren{\omega_n t}
        }
    }

    Where

    \ms{%
        \omega_n = \sqrt{\pren{\alpha_n^2 - \frac{\beta}{2}}^2}
    }

    @@ This yields our ``final'' solution in the form

    \ms{%
        u(x, t) = \sum_{n=1}^\infty
        \bren{%
            \pren{%
                a_n \cos\pren{\omega_n t} +
                b_n \sin\pren{\omega_n t}
            } \sin\pren{\frac{n\pi x}{L}}
        }
    }

    @@ With initial conditions we can determine $a_n$ and $b_n$.

    \ms{%
        u(x, 0) = U(x) &= \sum_{n=1}^\infty a_n \sin\pren{\frac{n\pi x}{L}}\\
        u_t(x, 0) = 0 &= \sum_{n=1}^\infty \pren{\omega_n b_n - \frac{\beta}{2} a_n}\sin\pren{\frac{n \pi x}{L}}
    }

    @@ Given that this is just a Fourier Series, we find the values for $a_n$ and $b_n$.

    \ms{%
        a_n &= \frac{2}{L} \int_0^L U(x) \sin\pren{\frac{n\pi x}{L}} \, dx\\
        &= \frac{2}{L} \bren{%
            \int_0^d a \frac{x}{d} \sin\pren{\frac{n\pi x}{L}} \, dx +
            \int_d^L a \pren{\frac{L - x}{L - d}} \sin\pren{\frac{n\pi x}{L}} \, dx
        }\\
        &= \frac{2 L a}{d n^{2} \pi^{2} \pren{L - d}}
        \pren{L \sin{\pren{\frac{d n}{L} \pi }} - d \sin{\pren{n \pi }}}\\
        b_n &= \frac{2}{\omega L} \int_0^L \frac{\beta}{2} U(x) \sin\pren{\frac{n\pi x}{L}} \, dx\\
        &= \frac{\beta}{\omega L} \bren{%
            \int_0^d a \frac{x}{d} \sin\pren{\frac{n\pi x}{L}} \, dx +
            \int_d^L a \pren{\frac{L - x}{L - d}} \sin\pren{\frac{n\pi x}{L}} \, dx
        }\\
        &= \frac{2 L a\beta}{d \omega n^{2} \pi^{2} \pren{L - d}}
        \pren{L \sin{\pren{\frac{d n}{L} \pi }} - d \sin{\pren{n \pi }}}\\
    }

    @ The lowest Fourier mode in the Fourier series should have the form
    \[
        \cren{\text{Decaying Amplitude in $t$}} \cdot
        \cren{\text{Oscillatory Function of $t$}} \cdot
        \cren{\text{Oscillatory Function of $x$}}
    \]

    Recall that the sound from a plucked guitar string lasts 5 to 10 seconds. For a guitar string tuned so that its
    fundamental tone is $A$ again, if the damping is such that the amplitude decays to half of its initial value in two
    seconds, find the correct value of $\beta$. Include units. How many full cycles does the string complete during
    those two seconds?

    $\beta = 0.0005$.

    @ For $\beta > 0$, is the frequency of the $n = 1$ mode higher, lower, or the same as it would have been with
    $\beta=0$. What effect does $\beta > 0$ have on the pitch of higher harmonics? Could you hear this effect? What
    would you hear?

    With $\beta = 0$ the tone becomes higher and we would hear a difference.
\end{easylist}

\newpage
\section{D'Alembert's Solution of the Wave Equation in 1D}

Edward Engineer conducts an experiment in the ITLL on a process that can be approximated well by the wave equation in
1-D. The experiment takes place in a long tube, long enough that if the initial disturbance is localized (so nonzero
only in $0<x<L$ for a reasonably small $L$), then the entire process being measured will have finished before a nonzero
signal could reach each either end of the tube and return. Therefore Edward (legitimately) assumes that the boundaries
are infinitely far away.

Before the experiment begins, he carefully records $U(x)$, the initial distribution of $u(x, 0)$.  However, he gets
distracted and forgets to record $V(x)$, the initial distribution of $u_t(x, 0)$.

Then he gets back on track, and records all of the appropriate data during the experiment itself. While carrying out the
experiment, he notices that during the entire experiment, the signal propagates only to the right, with no signal moving
to the left. (Additional information: The experiment was conducted on Wednesday morning, during 8 -- 10 AM.  Edward was
wearing a blue shirt.)

Can you help Edward out? What must $V(x)$ have been in order for the experiment to proceed as it did? Justify your
answer -- Edward's grade hangs in the balance.\\

In order to make this work, $f(x-ct)$ needs to be cancelled out. Therefore,

\ms{%
    -\frac{1}{2c} \int_{x-ct}^{x+ct} g(y) \, dy &= \frac{1}{2} f(x - ct)\\
    -\frac{1}{c} \int_{x-ct}^{x+ct} g(y) \, dy &= f(x - ct)\\
    \int_{x-ct}^{x+ct} g(y) \, dy &= -c f(x - ct)\\
    g(y) &= -c f^\prime(x - ct)\\
}
\end{document}
