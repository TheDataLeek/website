\documentclass[10pt]{article}

%%%%%%%%%%%%%%%%%%%%%%%%%%%%%%%%%%%%%%%%%%%%%%%%%%%%%%%%%%%%%%%%%%%%%%%%%%%%%%%%
% LaTeX Imports
%%%%%%%%%%%%%%%%%%%%%%%%%%%%%%%%%%%%%%%%%%%%%%%%%%%%%%%%%%%%%%%%%%%%%%%%%%%%%%%%
\usepackage{amsfonts}                                                   % Math fonts
\usepackage{amsmath}                                                    % Math formatting
\usepackage{amssymb}                                                    % Math formatting
\usepackage{amsthm}                                                     % Math Theorems
\usepackage{arydshln}                                                   % Dashed hlines
\usepackage{attachfile}                                                 % AttachFiles
\usepackage{cancel}                                                     % Cancelled math
\usepackage{caption}                                                    % Figure captioning
\usepackage{color}                                                      % Nice Colors
\input{./lib/dragon.inp}                                                % Tikz dragon curve
\usepackage[ampersand]{easylist}                                        % Easy lists
\usepackage{fancyhdr}                                                   % Fancy Header
\usepackage[T1]{fontenc}                                                % Specific font-encoding
%\usepackage[margin=1in, marginparwidth=2cm, marginparsep=2cm]{geometry} % Margins
\usepackage{graphicx}                                                   % Include images
\usepackage{hyperref}                                                   % Referencing
\usepackage[none]{hyphenat}                                             % Don't allow hyphenation
\usepackage{lipsum}                                                     % Lorem Ipsum Dummy Text
\usepackage{listings}                                                   % Code display
\usepackage{marginnote}                                                 % Notes in the margin
\usepackage{microtype}                                                  % Niceness
\usepackage{lib/minted}                                                 % Code display
\usepackage{multirow}                                                   % Multirow tables
\usepackage{pdfpages}                                                   % Include pdfs
\usepackage{pgfplots}                                                   % Create Pictures
\usepackage{rotating}                                                   % Figure rotation
\usepackage{setspace}                                                   % Allow double spacing
\usepackage{subcaption}                                                 % Figure captioning
\usepackage{tikz}                                                       % Create Pictures
\usepackage{tocloft}                                                    % List of Equations
%%%%%%%%%%%%%%%%%%%%%%%%%%%%%%%%%%%%%%%%%%%%%%%%%%%%%%%%%%%%%%%%%%%%%%%%%%%%%%%%
% Package Setup
%%%%%%%%%%%%%%%%%%%%%%%%%%%%%%%%%%%%%%%%%%%%%%%%%%%%%%%%%%%%%%%%%%%%%%%%%%%%%%%%
\hypersetup{%                                                           % Setup linking
    colorlinks=true,
    linkcolor=black,
    citecolor=black,
    filecolor=black,
    urlcolor=black,
}
\RequirePackage[l2tabu, orthodox]{nag}                                  % Nag about bad syntax
\renewcommand*\thesection{\arabic{section} }                             % Reset numbering
\renewcommand{\theFancyVerbLine}{ {\arabic{FancyVerbLine} } }              % Needed for code display
\renewcommand{\footrulewidth}{0.4pt}                                    % Footer hline
\setcounter{secnumdepth}{3}                                             % Include subsubsections in numbering
\setcounter{tocdepth}{3}                                                % Include subsubsections in toc
%%%%%%%%%%%%%%%%%%%%%%%%%%%%%%%%%%%%%%%%%%%%%%%%%%%%%%%%%%%%%%%%%%%%%%%%%%%%%%%%
% Custom commands
%%%%%%%%%%%%%%%%%%%%%%%%%%%%%%%%%%%%%%%%%%%%%%%%%%%%%%%%%%%%%%%%%%%%%%%%%%%%%%%%
\newcommand{\nvec}[1]{\left\langle #1 \right\rangle}                    %  Easy to use vector
\newcommand{\ma}[0]{\mathbf{A} }                                         %  Easy to use vector
\newcommand{\mb}[0]{\mathbf{B} }                                         %  Easy to use vector
\newcommand{\abs}[1]{\left\lvert #1 \right\rvert}                       %  Easy to use abs
\newcommand{\pren}[1]{\left( #1 \right)}                                %  Big parens
\let\oldvec\vec
\renewcommand{\vec}[1]{\oldvec{\mathbf{#1} } }                            %  Vector Styling
\newtheorem{thm}{Theorem}                                               %  Define the theorem name
\newtheorem{definition}{Definition}                                     %  Define the definition name
\definecolor{bg}{rgb}{0.95,0.95,0.95}
\newcommand{\java}[4]{\vspace{10pt}\inputminted[firstline=#2,
                                 lastline=#3,
                                 firstnumber=#2,
                                 gobble=#4,
                                 frame=single,
                                 label=#1,
                                 bgcolor=bg,
                                 linenos]{java}{#1} }
\newcommand{\python}[4]{\vspace{10pt}\inputminted[firstline=#2,
                                 lastline=#3,
                                 firstnumber=#2,
                                 gobble=#4,
                                 frame=single,
                                 label=#1,
                                 bgcolor=bg,
                                 linenos]{python}{#1} }
\newcommand{\js}[4]{\vspace{10pt}\inputminted[firstline=#2,
                                 lastline=#3,
                                 firstnumber=#2,
                                 gobble=#4,
                                 frame=single,
                                 label=#1,
                                 bgcolor=bg,
                                 linenos]{js}{#1} }
%%%%%%%%%%%%%%%%%%%%%%%%%%%%%%%%%%%%%%%%%%%%%%%%%%%%%%%%%%%%%%%%%%%%%%%%%%%%%%%%
% Beginning of document items - headers, title, toc, etc...
%%%%%%%%%%%%%%%%%%%%%%%%%%%%%%%%%%%%%%%%%%%%%%%%%%%%%%%%%%%%%%%%%%%%%%%%%%%%%%%%
\pagestyle{fancy}                                                       %  Establishes that the headers will be defined
\fancyhead[LE,LO]{Computer Systems Notes}                                  %  Adds header to left
\fancyhead[RE,RO]{Zoe Farmer}                                       %  Adds header to right
\cfoot{ \thepage }
\lfoot{CSCI 2400}
\rfoot{Han}
\title{Computer Systems Notes}
\author{Zoe Farmer}

%%%%%%%%%%%%%%%%%%%%%%%%%%%%%%%%%%%%%%%%%%%%%%%%%%%%%%%%%%%%%%%%%%%%%%%%%%%%%%%%
% Beginning of document items - headers, title, toc, etc...
%%%%%%%%%%%%%%%%%%%%%%%%%%%%%%%%%%%%%%%%%%%%%%%%%%%%%%%%%%%%%%%%%%%%%%%%%%%%%%%%
\pagestyle{fancy}                                                       %  Establishes that the headers will be defined
\fancyhead[LE,LO]{Homework 4}                                  %  Adds header to left
\fancyhead[RE,RO]{Zoe Farmer}                                       %  Adds header to right
\cfoot{ \thepage }
\lfoot{APPM 4570}
\rfoot{Hagar}
\title{Homework 4}
\author{Zoe Farmer}
%%%%%%%%%%%%%%%%%%%%%%%%%%%%%%%%%%%%%%%%%%%%%%%%%%%%%%%%%%%%%%%%%%%%%%%%%%%%%%%%
% Beginning of document items - headers, title, toc, etc...
%%%%%%%%%%%%%%%%%%%%%%%%%%%%%%%%%%%%%%%%%%%%%%%%%%%%%%%%%%%%%%%%%%%%%%%%%%%%%%%%
\begin{document}

\maketitle

\begin{easylist}[enumerate]
    @ For independent random variables $X$ and $Y$, we know that $f(x,y)=f_1(x) f_2(y)$ where $f_1$ is the marginal
    density of $X$ and $f_2$ is the marginal density of $Y$. Using this, show that:
    @@ $E(XY) = E(X)E(Y)$
    @@@ We can rewrite this using the definition of expected value:

    \[
        \begin{aligned}
            \int^\infty_{-\infty}\int^\infty_{-\infty} xy f_1(x) f_2(y) \, dx \, dy =
            \int_{-\infty}^\infty x f_1(x) \, dx \int_{-\infty}^\infty y f_2(y) \, dy\\
            \text{Which can be rewritten as}\\
            \int_{-\infty}^\infty x f_1(x) \, dx \int_{-\infty}^\infty y f_2(y) \, dy =
            \int_{-\infty}^\infty x f_1(x) \, dx \int_{-\infty}^\infty y f_2(y) \, dy\\
        \end{aligned}
    \]

    This is all dependent on that fact that (a) the two random variables are independent, as well as the fact that the
    integrals are only dependent on one variable at a time.

    @@ $\Var(X + Y) = \Var(X) + \Var(Y)$
    @@@ 

    \[
        \begin{aligned}
            \Var(X + Y) &=& E\left[ {\left( X + Y - E(X + Y) \right)}^2 \right]\\
                    &=& E \left[ {\left( X + Y - E(X) - E(Y) \right)}^2 \right]\\
                    &=& E \left[ {\left( X - E(X) \right)}^2 \right] + E \left[ {\left( Y - E(Y) \right)}^2 \right]\\
                    & & + \cancel{2 E \left[ {\left( X - E(X) \right)}{\left( Y - E(Y) \right)} \right]}\\
                    \Var(X + Y) &=& {E\left[ {\left( X - E(X) \right)}^2\right] + E\left[ {\left( Y - E(Y)
                    \right)}^2\right]}_\blacksquare
        \end{aligned}
    \]

    @ At time $t=0$, a lab technician starts an experiment where 20 identical components are tested. The lifetime
    distribution of each component is exponential with mean lifetime $w$ hours. The technician leaves the test facility
    right away, and comes back in 24 hours to find 15 components still running (and 5 of them failed). Based on this
    sample find the MLE for $w$.
    @@ If the components have an exponential distribution, than we can determine the pdf

    \[
        f(x; \lambda=w) =
        \begin{cases}
            \lambda e^{-\lambda x}\\
            0
        \end{cases} \Rightarrow
        \begin{cases}
            w e^{-w x} &\to x \ge 0\\
            0 &\to Otherwise
        \end{cases}
    \]

    Given that we do not know the exact lifetimes of each component, we need to take a slightly different approach from
    the standard. We know that since the lifetime of these components is based on the exponential distribution, that the
    probability that any given component will survive until the twenty-fourth hour is

    \[ P(X > 24) = 1 - P(X \le 24) = e^{-\lambda 24} \]

    And while we don't know the rate parameter, we do know the mean and the relationship between the
    two\footnote{$\lambda = w^{-1}$}, yielding

    \[ P(X > 24) = e^{-w^{-1} 24} = e^\frac{-24}{w} \]

    Now that this has been established we can set up the probability that these components were to fail after 24 hours,
    by letting $Y$ be the number of failed components. In fact, this equation can also be thought of as the likelihood
    function, because we know how many failed, which ones failed, and which ones survived. Let $p$ be the probability
    that any given component fails.

    \[
        \begin{aligned}
            P(Y=y=5) &=&
            \begin{cases}
                {\left( p \right)}^y
                    {\left( 1 - p \right)}^{20-y} &\to y \in \{0, 1, \ldots, 19, 20\}\\
                0 &\to Otherwise
            \end{cases}\\
            &=& \begin{cases}
                {\left( p \right)}^5
                    {\left( 1 - p\right)}^{15} &\to y \in \{0, 1, \ldots, 19, 20\}\\
                0 &\to Otherwise
            \end{cases}
        \end{aligned}
    \]

    We can take this likelihood function and turn it into the log-likelihood function, yielding

    \[
        \begin{aligned}
            L &=& {\left( p \right)}^5 {\left( 1-p\right)}^{15}\\
            \ln L &=& \ln \left( {\left( p \right)}^5 {\left( 1-p\right)}^{15} \right)
        \end{aligned}
    \]

    Which we can take the derivative of with respect to $p$, set equal to zero, and solve.

    \[
        \begin{aligned}
            \frac{\partial \ln L}{\partial p} = \frac{5 - 20p}{p-p^2} &=& 0\\
            \hat{p} &=& \frac{1}{4}\\
            \text{We know that } p &=& 1 - e^\frac{-24}{w}\\
            \hat{w} &=& \frac{-24}{\ln(1-p)}\\
            \text{Solving for } \hat{w} &=& \frac{-24}{\ln \left( \frac{3}{4} \right)} \approx \boxed{83.4254}
        \end{aligned}
    \]

    We aren't quite done, as we haven't converted this back to the form of the exponential distribution. When we do
    this we see that $\lambda = 0.0119868$.

    @ Aphid infestation of fruit trees is usually controlled either via pesticides or via ladybug inundation. In a
    particular area, 2 different (and well isolated) groves, with 15 fruit trees each, are selected for an experiment.
    The trees in both groves are of the same age, roughly the same size, and can be assumed to be independent. One grove
    is sprayed with pesticides, and one is flooded with ladybugs. The fruit yield in pounds for each tree is given.
    @@ Plot the histograms of the yields in the two groves.
    @@@ Pictured in Figure~\ref{fig:hist_groves}. The top row has been created with binsize 10, and the bottom with
    binsize 5, while the left column is the pesticide distribution with the right being the ladybugs distribution.

        \begin{figure}[!ht]
            \centering
            \includegraphics[scale=0.5]{./img/hist_groves.png}
            \caption{Histograms of the Groves of Trees with Normal Overlay}
            \label{fig:hist_groves}
        \end{figure}

    @@ Comment on the histogram shapes. Which densities do they resemble? In particular, do they appear normal?
    @@@ Both distributions are limited in that we only have 15 data points to use to determine any information about
    that data. This means that any and all decisions made must be taken lightly, for we simply do not have enough data
    to come to any concrete observations. That being said, both distributions look very similar to some derivation of
    the normal distribution. Given how probability distributions work however, it is also a complete possibility that
    we've received the more unique entries, and it would be improper to assume wholeheartedly that the data is
    normal.\newline

    This all being said however, given the relative lack of data we will proceed to assume that both distributions are
    normal. Please reference Figure~\ref{fig:hist_groves} in order to see the similarities between the collected data
    and the normal distribution.
    @@ Find the sample means of yields for the two groves.
    @@@ We can use {\ttfamily R}'s summary function to determine the necessary elements.

        \begin{verbatim}
                pest            lady
                Min.   :22.85   Min.   :35.52
                1st Qu.:34.58   1st Qu.:41.88
                Median :38.50   Median :46.98
                Mean   :39.28   Mean   :47.78
                3rd Qu.:44.55   3rd Qu.:50.47
                Max.   :55.57   Max.   :70.17
                sd     :8.511   sd     :8.711
                var    :72.44   var    :75.89
        \end{verbatim}

    @@ Using your conclusion in (b) and answer in (c), provide the two 95\% confidence intervals for the true mean
    yields for trees under the two treatments.
    @@@ Let's first examine the pesticide tree distribution. We've determined the sample mean is $\overline{x} = 39.28$,
    and the sample standard deviation to be $s = 8.511$. Since our $n$ is relatively small, we'll use the $t$
    distribution with $\alpha = 0.5$ in order to find our confidence interval, which is defined as

    \[
        \left( \overline{x} - t_{\alpha/2,n-1} \cdot \frac{s}{\sqrt{n} },
        \overline{x} + t_{\alpha/2,n-1} \cdot \frac{s}{\sqrt{n} } \right)
    \]

    We can now substitue the known values.

    \[
        \left(
            39.28 - 2.144787 \cdot \frac{8.511}{\sqrt{15} },
            39.28 + 2.144787 \cdot \frac{8.511}{\sqrt{15} },
        \right)
    \]

    Therefore our interval is equal to $\boxed{(34.56197, 43.98893)}$.\newline

    For the second distribution we can use the exact same process.

    \[
        \left(
            47.78 - 2.144787 \cdot \frac{8.711}{\sqrt{15} },
            47.78 + 2.144787 \cdot \frac{8.711}{\sqrt{15} },
        \right)
    \]

    For which we determine our interval to be $\boxed{(42.95220, 52.60113)}$.

    @@ Interpret one of the confidence intervals you constructed.
    @@@ Examining the pesticide distribution we can interpret exactly what our interval indicates. If we were to
    reproduce this experiment thousands of times, our interval tells us that 95\% of the time, the true mean would fall
    within our interval. In other words, since our interval will vary from sample to sample, 95\% of the time when we
    calculate our interval we'll capture the true mean.
    @@ Find the sample variances for the yields in the two groves.
    @@@ Please reference the above summary for the variances. Note, the variance is simply the standard deviation
    squared, all of which can be calculated either by hand or by using built in {\ttfamily R} functions {\ttfamily sd}
    and {\ttfamily var}.
    @@ We have learned how to construct confidence intervals for the variance of a normal distribution -- we do that
    using a chi-square distribution. For the grove(s) with approximately normal yield, construct the 95\% confidence
    interval for the true variance of yield.
    @@@ We've assumed that both distributions are to be normal\footnote{Or normal enough for rough approximations given
    the relative lack of data supplied}, and thereby we must examine both. The process is the same for both however, so
    we'll first determine the equation, and then substitute the values.\newline

    The 95\% confidence interval for the variance of a population is defined as

    \[
        \left(
            \frac{(n - 1) s^2}{\chi^2_{\alpha/2, n-1} },
            \frac{(n - 1) s^2}{\chi^2_{1 - \alpha/2, n-1} }
        \right)
    \]

    To which we can insert the known quantities and determine the bounds.

    \[
        \left(
            \frac{14 {\{8.511, 8.711\} }^2}{\chi^2_{\alpha/2, n-1} },
            \frac{14 {\{8.511, 8.711\} }^2}{\chi^2_{1 - \alpha/2, n-1} }
        \right)
    \]

    Which tells us the pesticide distribution variance bounds to be $\boxed{(4.562216, 21.170025)}$, and the ladybug
    distribution variance bounds to be $\boxed{(4.669635, 21.668483)}$. Note that these distributions are very similar,
    and so are the sample standard deviations.
\end{easylist}

\newpage


\end{document}
