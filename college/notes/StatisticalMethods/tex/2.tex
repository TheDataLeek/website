\documentclass[10pt]{article}

%%%%%%%%%%%%%%%%%%%%%%%%%%%%%%%%%%%%%%%%%%%%%%%%%%%%%%%%%%%%%%%%%%%%%%%%%%%%%%%%
% LaTeX Imports
%%%%%%%%%%%%%%%%%%%%%%%%%%%%%%%%%%%%%%%%%%%%%%%%%%%%%%%%%%%%%%%%%%%%%%%%%%%%%%%%
\usepackage{amsfonts}                                                   % Math fonts
\usepackage{amsmath}                                                    % Math formatting
\usepackage{amssymb}                                                    % Math formatting
\usepackage{amsthm}                                                     % Math Theorems
\usepackage{arydshln}                                                   % Dashed hlines
\usepackage{attachfile}                                                 % AttachFiles
\usepackage{cancel}                                                     % Cancelled math
\usepackage{caption}                                                    % Figure captioning
\usepackage{color}                                                      % Nice Colors
\input{./lib/dragon.inp}                                                % Tikz dragon curve
\usepackage[ampersand]{easylist}                                        % Easy lists
\usepackage{fancyhdr}                                                   % Fancy Header
\usepackage[T1]{fontenc}                                                % Specific font-encoding
%\usepackage[margin=1in, marginparwidth=2cm, marginparsep=2cm]{geometry} % Margins
\usepackage{graphicx}                                                   % Include images
\usepackage{hyperref}                                                   % Referencing
\usepackage[none]{hyphenat}                                             % Don't allow hyphenation
\usepackage{lipsum}                                                     % Lorem Ipsum Dummy Text
\usepackage{listings}                                                   % Code display
\usepackage{marginnote}                                                 % Notes in the margin
\usepackage{microtype}                                                  % Niceness
\usepackage{lib/minted}                                                 % Code display
\usepackage{multirow}                                                   % Multirow tables
\usepackage{pdfpages}                                                   % Include pdfs
\usepackage{pgfplots}                                                   % Create Pictures
\usepackage{rotating}                                                   % Figure rotation
\usepackage{setspace}                                                   % Allow double spacing
\usepackage{subcaption}                                                 % Figure captioning
\usepackage{tikz}                                                       % Create Pictures
\usepackage{tocloft}                                                    % List of Equations
%%%%%%%%%%%%%%%%%%%%%%%%%%%%%%%%%%%%%%%%%%%%%%%%%%%%%%%%%%%%%%%%%%%%%%%%%%%%%%%%
% Package Setup
%%%%%%%%%%%%%%%%%%%%%%%%%%%%%%%%%%%%%%%%%%%%%%%%%%%%%%%%%%%%%%%%%%%%%%%%%%%%%%%%
\hypersetup{%                                                           % Setup linking
    colorlinks=true,
    linkcolor=black,
    citecolor=black,
    filecolor=black,
    urlcolor=black,
}
\RequirePackage[l2tabu, orthodox]{nag}                                  % Nag about bad syntax
\renewcommand*\thesection{\arabic{section} }                             % Reset numbering
\renewcommand{\theFancyVerbLine}{ {\arabic{FancyVerbLine} } }              % Needed for code display
\renewcommand{\footrulewidth}{0.4pt}                                    % Footer hline
\setcounter{secnumdepth}{3}                                             % Include subsubsections in numbering
\setcounter{tocdepth}{3}                                                % Include subsubsections in toc
%%%%%%%%%%%%%%%%%%%%%%%%%%%%%%%%%%%%%%%%%%%%%%%%%%%%%%%%%%%%%%%%%%%%%%%%%%%%%%%%
% Custom commands
%%%%%%%%%%%%%%%%%%%%%%%%%%%%%%%%%%%%%%%%%%%%%%%%%%%%%%%%%%%%%%%%%%%%%%%%%%%%%%%%
\newcommand{\nvec}[1]{\left\langle #1 \right\rangle}                    %  Easy to use vector
\newcommand{\ma}[0]{\mathbf{A} }                                         %  Easy to use vector
\newcommand{\mb}[0]{\mathbf{B} }                                         %  Easy to use vector
\newcommand{\abs}[1]{\left\lvert #1 \right\rvert}                       %  Easy to use abs
\newcommand{\pren}[1]{\left( #1 \right)}                                %  Big parens
\let\oldvec\vec
\renewcommand{\vec}[1]{\oldvec{\mathbf{#1} } }                            %  Vector Styling
\newtheorem{thm}{Theorem}                                               %  Define the theorem name
\newtheorem{definition}{Definition}                                     %  Define the definition name
\definecolor{bg}{rgb}{0.95,0.95,0.95}
\newcommand{\java}[4]{\vspace{10pt}\inputminted[firstline=#2,
                                 lastline=#3,
                                 firstnumber=#2,
                                 gobble=#4,
                                 frame=single,
                                 label=#1,
                                 bgcolor=bg,
                                 linenos]{java}{#1} }
\newcommand{\python}[4]{\vspace{10pt}\inputminted[firstline=#2,
                                 lastline=#3,
                                 firstnumber=#2,
                                 gobble=#4,
                                 frame=single,
                                 label=#1,
                                 bgcolor=bg,
                                 linenos]{python}{#1} }
\newcommand{\js}[4]{\vspace{10pt}\inputminted[firstline=#2,
                                 lastline=#3,
                                 firstnumber=#2,
                                 gobble=#4,
                                 frame=single,
                                 label=#1,
                                 bgcolor=bg,
                                 linenos]{js}{#1} }
%%%%%%%%%%%%%%%%%%%%%%%%%%%%%%%%%%%%%%%%%%%%%%%%%%%%%%%%%%%%%%%%%%%%%%%%%%%%%%%%
% Beginning of document items - headers, title, toc, etc...
%%%%%%%%%%%%%%%%%%%%%%%%%%%%%%%%%%%%%%%%%%%%%%%%%%%%%%%%%%%%%%%%%%%%%%%%%%%%%%%%
\pagestyle{fancy}                                                       %  Establishes that the headers will be defined
\fancyhead[LE,LO]{Computer Systems Notes}                                  %  Adds header to left
\fancyhead[RE,RO]{Zoe Farmer}                                       %  Adds header to right
\cfoot{ \thepage }
\lfoot{CSCI 2400}
\rfoot{Han}
\title{Computer Systems Notes}
\author{Zoe Farmer}

%%%%%%%%%%%%%%%%%%%%%%%%%%%%%%%%%%%%%%%%%%%%%%%%%%%%%%%%%%%%%%%%%%%%%%%%%%%%%%%%
% Beginning of document items - headers, title, toc, etc...
%%%%%%%%%%%%%%%%%%%%%%%%%%%%%%%%%%%%%%%%%%%%%%%%%%%%%%%%%%%%%%%%%%%%%%%%%%%%%%%%
\pagestyle{fancy}                                                       %  Establishes that the headers will be defined
\fancyhead[LE,LO]{Homework 2}                                  %  Adds header to left
\fancyhead[RE,RO]{Zoe Farmer}                                       %  Adds header to right
\cfoot{\mlptikz[size=0.25in, text=on, textposx=0, textposy=0, textvalue=\thepage, textscale=0.75in]{applejack} }
\lfoot{APPM 4570}
\rfoot{Hagar}
\title{Homework 2}
\author{Zoe Farmer}
%%%%%%%%%%%%%%%%%%%%%%%%%%%%%%%%%%%%%%%%%%%%%%%%%%%%%%%%%%%%%%%%%%%%%%%%%%%%%%%%
% Beginning of document items - headers, title, toc, etc...
%%%%%%%%%%%%%%%%%%%%%%%%%%%%%%%%%%%%%%%%%%%%%%%%%%%%%%%%%%%%%%%%%%%%%%%%%%%%%%%%
\begin{document}

\maketitle

\begin{easylist}[enumerate]
    @ One out of 500 people has disease $D$. You're a doctor and suspect that one of your patients may have that
    disease. You order a diagnostic test $T$ for that patient. The test is not perfect - it accurately claims the
    disease is ``present'' in 80\% of the patients who actually have it, and accurately declares the disease as
    ``absent'' in 70\% of the patients who indeed don't have the disease.
    @@ What is the probability that the test result comes back negative (the test says ``absence of disease'')? Does
    this mean your patient definitely does not have the disease?
    @@@ For this problem we need to consider the two separate cases where the individual has the disease, and where he
    doesn't. This can be expressed as a problem of conditional probability. Let $E_1$ be the event that he has the
    disease, $E_2$ the event that he doesn't, $E_3$ the event that the test comes back positive, and $E_4$ the event
    that it comes back negative.\newline

    We're only interested in how likely it is for the test to come back negative, but we need to first establish a
    couple probabilities.

        \[ \begin{aligned}
            P(E_1) &=& 1/500\\
            P(E_2) &=& 499/500\\
            P(E_3|E_1) &=& 8/10\\
            P(E_3|E_2) &=& 3/10\\
            P(E_4|E_1) &=& 2/10\\
            P(E_4|E_2) &=& 7/10\\
        \end{aligned} \]

    We can now set up an equality and solve for $P(E_4)$.

        \[ \begin{aligned}
            P(E_4) &=& P(E_4|E_2)P(E_2) + P(E_4|E^\prime_2)P(E^\prime_2)\\
                   &=& P(E_4|E_2)P(E_2) + P(E_4|E^\prime_2)(1 - P(E_2))\\
                   &=& P(E_4|E_2)P(E_2) + P(E_4|E_1)(1 - P(E_2))\\
                   &=& \frac{7}{10} \cdot \frac{499}{500} + \frac{2}{10}\left(1 - \frac{499}{500}\right)\\
                   &=& \frac{699}{1000} = \boxed{0.699}
        \end{aligned} \]

    If this test were to come back negative, it doesn't mean that the patient doesn't have the disease. In fact, since
    the test is always \textit{probably} correct, there is always a chance that it was wrong. However we address this
    issue in the next problem, and it turns out that if the test returns negative there is only a slim chance that the
    individual has the disease.

    @@ If the test is negative, how likely is he to actually have disease $D$?
    @@@ We are now interested in the probability that he has the disease, given it returns negative. Expressed in the
    above notation, we are looking for $P(E_1|E_4)$. In the previous problem we solved for $P(E_4)$, allowing us to
    rewrite the problem below.

        \[\begin{aligned}
            P(E_1|E_4) &=& \frac{P(E_1 \cap E_4)}{P(E_4)}\\
                       &=& \frac{P(E_1)P(E_4|E_1)}{P(E_4)}\\
                       &=& \frac{\frac{1}{500} \cdot \frac{2}{10} }{0.699}\\
                       &=& \frac{2}{3495} = \boxed{0.000572246}
        \end{aligned}\]

    @@ How about if the test comes back positive?
    @@@ Now we're looking at the probability that he has the disease given that the test returns positive. This can be
    solved in a very similar manner as the previous problem, however we don't initially know the value of $P(E_3)$. This
    can be solved in the same method as in the first problem.

        \[ \begin{aligned}
            P(E_3) &=& P(E_3|E_2)P(E_2) + P(E_3|E^\prime_2)P(E^\prime_2)\\
                   &=& P(E_3|E_2)P(E_2) + P(E_3|E^\prime_2)(1 - P(E_2))\\
                   &=& P(E_3|E_2)P(E_2) + P(E_3|E_1)(1 - P(E_2))\\
                   &=& \frac{3}{10} \cdot \frac{499}{500} + \frac{8}{10} \left( 1 - \frac{499}{500} \right)\\
                   &=& \frac{301}{1000} = \boxed{0.301}
        \end{aligned} \]

        Now that we have $P(E_3)$ we can determine $P(E_1|E_3)$.

        \[\begin{aligned}
            P(E_1|E_3) &=& \frac{P(E_1 \cap E_3)}{P(E_3)}\\
                       &=& \frac{P(E_1)P(E_3|E_1)}{P(E_3)}\\
                       &=& \frac{\frac{1}{500} \cdot \frac{8}{10} }{0.301}\\
                       &=& \frac{8}{1505} = \boxed{0.00531561}
        \end{aligned}\]

    @@ What do you think about the accuracy of this test?
    @@@ When it comes to negative answers, this test is very accurate. If there is a negative result, there exists only
    a \textit{very} slim chance that the individual actually has the disease. On the flip side, if the test comes back
    positive, there is still a very slim chance the individual has the disease. In other words, the test is fantastic at
    determining if the individual \textit{doesn't} have the disease, but absolute rubbish at determining if he does.

    @ Give examples to justify your answers
    @@ If events $J$ and $K$ are independent, what can we say about $J^\prime$ and $K$? How about $J^\prime$ and
    $K^\prime$?
    @@@ If $J$ and $K$ are independent, we know that $K$ is dependent on $J^\prime$ and that $J^\prime$ and $K^\prime$
    remain independent. This can be illustrated with a simple example.  Let's say we're rolling 1D6. There are six
    outcomes, $\{1, 2, 3, 4, 5, 6\}$, and let use assume that $J=\{1\}$ and $K=\{2\}$. This means that $J^\prime=\{2, 3,
    4, 5, 6\}$ and $K^\prime = \{1, 3, 4, 5, 6\}$. We are using a fair die, therefore $K$ and $J$ are independent,
    however $K$ can only occur if $J$ occurs, and vice-versa. However $J^\prime$ and $K^\prime$ are not mutually
    dependent.
    @@ If events $A$, $B$ and $C$ are mutually disjoint, what can we say about $A^\prime$, $B^\prime$, and $C^\prime$?
    Are $A$ and $C$ independent? Are $A^\prime$ and $C^\prime$ independent?
    @@@ If $A$, $B$, and $C$ are mutually disjoint, we know that their complements are no longer mutually disjoint. We
    also know that $A$ and $C$ are independent, and that $A^\prime$ and $C^\prime$ are neither dependent nor
    independent. We can use our dice example from before, with the sample space remaining the same. Let $A = \{1\}$,
    $B=\{2\}$, and $C=\{3\}$. $A$, $B$, and $C$ are independent from one another, and their complements overlap. We also
    know that the complements are independent as well, since no outcome relies on another.
    @@ If $E$ is independent of $F$, and $F$ is independent of $G$, is $E$ independent of $G$?
    @@@ Not necessarily, because $E$ could be dependent on $G$. An easy example is when $G$ is the event that an
    individual will die in a car accident, $E$ is the event that the individual doesn't wear a seatbelt, and $F$ is the
    event that the sky is blue. Both $E$ and $G$ are independent of $F$, however they are not independent of one
    another.

    @ Three fair dice are rolled together. Let $X$ be the random variable denoting the sum of all three top faces after
    they're rolled.
    @@ Write out the pmf and cdf of $X$.
    @@@ Please refer to Table~\ref{table:pmf}.

        \begin{table}[!ht]
            \centering
            \begin{tabular}{|l|l|l|}
                \hline
                $x$ & $p(x)$ & cmf\\
                \hline
                3 & 0.00462963 & 0.00462963\\
                4 & 0.01388889 & 0.0185185\\
                5 & 0.02777778 & 0.0462963\\
                6 & 0.04629630 & 0.0925926\\
                7 & 0.06944444 & 0.162037\\
                8 & 0.09722222 & 0.259259\\
                9 & 0.11574074 & 0.375\\
                10 & 0.12500000 & 0.5\\
                11 & 0.12500000 & 0.625\\
                12 & 0.11574074 & 0.740741\\
                13 & 0.09722222 & 0.837963\\
                14 & 0.06944444 & 0.907407\\
                15 & 0.04629630 & 0.953704\\
                16 & 0.02777778 & 0.981481\\
                17 & 0.01388889 & 0.99537\\
                18 & 0.00462963 & 1.0\\
                \hline
            \end{tabular}
            \caption{pmf and cdf of $X$}
            \label{table:pmf}
        \end{table}

    @@ What is the expected value of $X$?
    @@@ To find the expected value, we simply sum the values of $x$ multiplied by their corresponding $p(x)$.
        \[ \sum^{18}_{i=3} i \cdot p(i) = \boxed{10.5} \]
    @@ What is the variance of $X$?
    @@@ Again, this is simply
        \[ \sum^{18}_{i=3} {(i - 10.5)}^2 \cdot p(i) = \boxed{8.75} \]

    @ A certain lab machine has 12 rings. With each use, these rings can fail, and an oil leak occurs. The probability
    of any ring failing during machine use is 2\%. The rings are independent, and the failure of one ring does not
    impact the probability of failing for other rings. The machine gets serviced after each use, so any damaged rings
    are repaired after each use.\newline

    If 6 or more rings fail, the entire machine will shut down. What is the probability that this happens within the
    first 10 uses?
    @@ We can use the Binomial distribution to solve this problem. Recall the formula for the Binomial distribution is
        \[ P(x) = \binom{n}{x} \cdot p^x \cdot {(1 - p)}^n \]
        Therefore if we determine the probability of how many parts fail, we can determine the probability that it will
        happen within the first 10 trials. This summation yields, where $p=0.02$, and $n=10$.
        \[ \sum^{10}_{x=6} \binom{n}{x} \cdot p^x \cdot {(1 - p)}^n = \boxed{1.204005 \times 10^{-8} } \]
        In other words, not very likely\ldots

    @ A traffic office wishes to monitor the number of vehicles crossing a certain bridge in the city in any given day.
    What family of distributions will they most likely be observing?
    @@ Given that we do now know the city, nor the bridge in question, we have to make some assumptions. First, let's
    assume that the city is a relatively populated one, and that second, the bridge is the busiest bridge in the city,
    and that almost everyone crosses it in any given day.\footnote{This is a little ridiculous, but it gets the point
    across} Despite its busyness, this bridge will have certain characteristics that group it with a particular family
    of statistical distributions.\newline

    Since we are interested in how many events occur within a specified time period, we can immediately see the link
    between the scenario and the Poisson distribution, which is characterized by focusing on the probability of certain
    events occurring in a certain time duration independently of previous events.

    @ If they estimate that an average of 30 cars cross the bridge per day, what member of the family of distributions
    are they most likely working with?
    @@ Given this new piece of information, it immediately becomes apparent that if we continue to assume that this is
    the busiest bridge in the city, than the city is very very small.\newline

    We now know that we are working with the Poisson distribution with a daily average of 30 cars, which can be
    expressed as $\mathrm{Pois}(30)$, or

        \[ f(k;\lambda) = \mathrm{Pr}(X=k) = \frac{\lambda^k e^{-\lambda} }{k!} \]

    @ Under the distribution above, what is the probability that less than 10 cars will cross the bridge on any given
    day?
    @@ The equation to determine $p(k)$ for a Poisson distribution is

        \[ f(k;\lambda) = \mathrm{Pr}(X=k) = \frac{\lambda^k e^{-\lambda} }{k!} \]

        If we allow $k$ to equal $[1, 2, 3, \ldots, 10]$ we can determine the cdf and thereby the answer to the
        question.

        \[ \sum^{10}_{k=1} \frac{\lambda^k e^{-30} }{k!} = \boxed{2.234878 \times 10^{-5} } \]
\end{easylist}
\end{document}
