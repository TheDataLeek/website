\documentclass[10pt]{article}\usepackage[]{graphicx}\usepackage[]{xcolor}
%% maxwidth is the original width if it is less than linewidth
%% otherwise use linewidth (to make sure the graphics do not exceed the margin)
\makeatletter
\def\maxwidth{ %
  \ifdim\Gin@nat@width>\linewidth
    \linewidth
  \else
    \Gin@nat@width
  \fi
}
\makeatother

\definecolor{fgcolor}{rgb}{0.345, 0.345, 0.345}
\newcommand{\hlnum}[1]{\textcolor[rgb]{0.686,0.059,0.569}{#1} }%
\newcommand{\hlstr}[1]{\textcolor[rgb]{0.192,0.494,0.8}{#1} }%
\newcommand{\hlcom}[1]{\textcolor[rgb]{0.678,0.584,0.686}{\textit{#1} } }%
\newcommand{\hlopt}[1]{\textcolor[rgb]{0,0,0}{#1} }%
\newcommand{\hlstd}[1]{\textcolor[rgb]{0.345,0.345,0.345}{#1} }%
\newcommand{\hlkwa}[1]{\textcolor[rgb]{0.161,0.373,0.58}{\textbf{#1} } }%
\newcommand{\hlkwb}[1]{\textcolor[rgb]{0.69,0.353,0.396}{#1} }%
\newcommand{\hlkwc}[1]{\textcolor[rgb]{0.333,0.667,0.333}{#1} }%
\newcommand{\hlkwd}[1]{\textcolor[rgb]{0.737,0.353,0.396}{\textbf{#1} } }%

\usepackage{framed}
\makeatletter
\newenvironment{kframe}{%
 \def\at@end@of@kframe{}%
 \ifinner\ifhmode%
  \def\at@end@of@kframe{\end{minipage} }%
  \begin{minipage}{\columnwidth}%
 \fi\fi%
 \def\FrameCommand##1{\hskip\@totalleftmargin \hskip-\fboxsep
 \colorbox{shadecolor}{##1}\hskip-\fboxsep
     % There is no \\@totalrightmargin, so:
     \hskip-\linewidth \hskip-\@totalleftmargin \hskip\columnwidth}%
 \MakeFramed {\advance\hsize-\width
   \@totalleftmargin\z@ \linewidth\hsize
   \@setminipage} }%
 {\par\unskip\endMakeFramed%
 \at@end@of@kframe}
\makeatother

\definecolor{shadecolor}{rgb}{.97, .97, .97}
\definecolor{messagecolor}{rgb}{0, 0, 0}
\definecolor{warningcolor}{rgb}{1, 0, 1}
\definecolor{errorcolor}{rgb}{1, 0, 0}
\newenvironment{knitrout}{}{} % an empty environment to be redefined in TeX

\usepackage{alltt}

%%%%%%%%%%%%%%%%%%%%%%%%%%%%%%%%%%%%%%%%%%%%%%%%%%%%%%%%%%%%%%%%%%%%%%%%%%%%%%%%
% LaTeX Imports
%%%%%%%%%%%%%%%%%%%%%%%%%%%%%%%%%%%%%%%%%%%%%%%%%%%%%%%%%%%%%%%%%%%%%%%%%%%%%%%%
\usepackage{amsfonts}                                                   % Math fonts
\usepackage{amsmath}                                                    % Math formatting
\usepackage{amssymb}                                                    % Math formatting
\usepackage{amsthm}                                                     % Math Theorems
\usepackage{arydshln}                                                   % Dashed hlines
\usepackage{attachfile}                                                 % AttachFiles
\usepackage{cancel}                                                     % Cancelled math
\usepackage{caption}                                                    % Figure captioning
\usepackage{color}                                                      % Nice Colors
\input{./lib/dragon.inp}                                                % Tikz dragon curve
\usepackage[ampersand]{easylist}                                        % Easy lists
\usepackage{fancyhdr}                                                   % Fancy Header
\usepackage[T1]{fontenc}                                                % Specific font-encoding
%\usepackage[margin=1in, marginparwidth=2cm, marginparsep=2cm]{geometry} % Margins
\usepackage{graphicx}                                                   % Include images
\usepackage{hyperref}                                                   % Referencing
\usepackage[none]{hyphenat}                                             % Don't allow hyphenation
\usepackage{lipsum}                                                     % Lorem Ipsum Dummy Text
\usepackage{listings}                                                   % Code display
\usepackage{marginnote}                                                 % Notes in the margin
\usepackage{microtype}                                                  % Niceness
\usepackage{lib/minted}                                                 % Code display
\usepackage{./lib/mlptikz}                                              % Tikz mlp
\usepackage{multirow}                                                   % Multirow tables
\usepackage{pdfpages}                                                   % Include pdfs
\usepackage{pgfplots}                                                   % Create Pictures
\usepackage{rotating}                                                   % Figure rotation
\usepackage{setspace}                                                   % Allow double spacing
\usepackage{subcaption}                                                 % Figure captioning
\usepackage{tikz}                                                       % Create Pictures
\usepackage{tocloft}                                                    % List of Equations
%%%%%%%%%%%%%%%%%%%%%%%%%%%%%%%%%%%%%%%%%%%%%%%%%%%%%%%%%%%%%%%%%%%%%%%%%%%%%%%%
% Package Setup
%%%%%%%%%%%%%%%%%%%%%%%%%%%%%%%%%%%%%%%%%%%%%%%%%%%%%%%%%%%%%%%%%%%%%%%%%%%%%%%%
\hypersetup{%                                                           % Setup linking
    colorlinks=true,
    linkcolor=black,
    citecolor=black,
    filecolor=black,
    urlcolor=black,
}
\RequirePackage[l2tabu, orthodox]{nag}                                  % Nag about bad syntax
\renewcommand*\thesection{\arabic{section} }                             % Reset numbering
\renewcommand{\theFancyVerbLine}{ {\arabic{FancyVerbLine} } }              % Needed for code display
\renewcommand{\footrulewidth}{0.4pt}                                    % Footer hline
\setcounter{secnumdepth}{3}                                             % Include subsubsections in numbering
\setcounter{tocdepth}{3}                                                % Include subsubsections in toc
%%%%%%%%%%%%%%%%%%%%%%%%%%%%%%%%%%%%%%%%%%%%%%%%%%%%%%%%%%%%%%%%%%%%%%%%%%%%%%%%
% Custom commands
%%%%%%%%%%%%%%%%%%%%%%%%%%%%%%%%%%%%%%%%%%%%%%%%%%%%%%%%%%%%%%%%%%%%%%%%%%%%%%%%
\newcommand{\nvec}[1]{\left\langle #1 \right\rangle}                    %  Easy to use vector
\newcommand{\inprod}[2]{\left\langle \vec{#1}, \vec{#2} \right\rangle}  %  Easy to use inner product
\newcommand{\norm}[1]{\lvert \lvert \vec{#1} \rvert \rvert}             %  Easy to use norm
\newcommand{\ma}[0]{\mathbf{A} }                                         %  Easy to use vector
\newcommand{\mb}[0]{\mathbf{B} }                                         %  Easy to use vector
\newcommand{\abs}[1]{\left\lvert #1 \right\rvert}                       %  Easy to use abs
\newcommand{\pren}[1]{\left( #1 \right)}                                %  Big parens
\let\oldvec\vec
\renewcommand{\vec}[1]{\mathbf{#1} }                            %  Vector Styling
\newtheorem{thm}{Theorem}                                               %  Define the theorem name
\theoremstyle{definition}
\newtheorem{definition}{Definition}                                     %  Define the definition name
\newtheorem{ex}{Example}                                                %  Define the example name
\definecolor{bg}{rgb}{0.95,0.95,0.95}
\newcommand{\java}[4]{\vspace{10pt}\inputminted[firstline=#2,
                                 lastline=#3,
                                 firstnumber=#2,
                                 gobble=#4,
                                 frame=single,
                                 label=#1,
                                 bgcolor=bg,
                                 linenos]{java}{#1} }
\newcommand{\python}[4]{\vspace{10pt}\inputminted[firstline=#2,
                                 lastline=#3,
                                 firstnumber=#2,
                                 gobble=#4,
                                 frame=single,
                                 label=#1,
                                 bgcolor=bg,
                                 linenos]{python}{#1} }
\newcommand{\js}[4]{\vspace{10pt}\inputminted[firstline=#2,
                                 lastline=#3,
                                 firstnumber=#2,
                                 gobble=#4,
                                 frame=single,
                                 label=#1,
                                 bgcolor=bg,
                                 linenos]{js}{#1} }
%%%%%%%%%%%%%%%%%%%%%%%%%%%%%%%%%%%%%%%%%%%%%%%%%%%%%%%%%%%%%%%%%%%%%%%%%%%%%%%%
% Beginning of document items - headers, title, toc, etc...
%%%%%%%%%%%%%%%%%%%%%%%%%%%%%%%%%%%%%%%%%%%%%%%%%%%%%%%%%%%%%%%%%%%%%%%%%%%%%%%%
\pagestyle{fancy}                                                       %  Establishes that the headers will be defined
\fancyhead[LE,LO]{Matrix Methods Notes}                                  %  Adds header to left
\fancyhead[RE,RO]{Zoe Farmer}                                       %  Adds header to right
\cfoot{\mlptikz[size=0.25in, text=on, textposx=0, textposy=0, textvalue=\thepage, textscale=0.75in]{applejack} }
\lfoot{APPM 3310}
\rfoot{Beylkin}
\title{Matrix Methods Notes}
\author{Zoe Farmer}

%%%%%%%%%%%%%%%%%%%%%%%%%%%%%%%%%%%%%%%%%%%%%%%%%%%%%%%%%%%%%%%%%%%%%%%%%%%%%%%%
% Beginning of document items - headers, title, toc, etc...
%%%%%%%%%%%%%%%%%%%%%%%%%%%%%%%%%%%%%%%%%%%%%%%%%%%%%%%%%%%%%%%%%%%%%%%%%%%%%%%%
\pagestyle{fancy}                                                       %  Establishes that the headers will be defined
\fancyhead[LE,LO]{Homework 5}                                  %  Adds header to left
\fancyhead[RE,RO]{Zoe Farmer}                                       %  Adds header to right
\cfoot{ \thepage }
\lfoot{APPM 4570}
\rfoot{Hagar}
\title{Homework 5}
\author{Zoe Farmer}
%%%%%%%%%%%%%%%%%%%%%%%%%%%%%%%%%%%%%%%%%%%%%%%%%%%%%%%%%%%%%%%%%%%%%%%%%%%%%%%%
% Beginning of document items - headers, title, toc, etc...
%%%%%%%%%%%%%%%%%%%%%%%%%%%%%%%%%%%%%%%%%%%%%%%%%%%%%%%%%%%%%%%%%%%%%%%%%%%%%%%%
\IfFileExists{upquote.sty}{\usepackage{upquote} }{}
\begin{document}

\maketitle




\begin{easylist}[enumerate]
    @ Hexavalent Chromium has been identified as an inhalation carcinogen and an air toxin of concern in a number of
    different locales. The article gave the data given on both indoor and outdoor concentration (nanograms/meter cubed)
    for a sample of houses selected from a certain region.

\begin{knitrout}
\definecolor{shadecolor}{rgb}{0.969, 0.969, 0.969}\color{fgcolor}\begin{kframe}
\begin{alltt}
     \hlstd{hc_data_raw} \hlkwb{<-} \hlkwd{as.list}\hlstd{(}\hlkwd{as.data.frame}\hlstd{(}\hlkwd{t}\hlstd{(}\hlkwd{read.table}\hlstd{(}
                        \hlstr{'./Hw5_prob1_data.txt'}\hlstd{,}
                        \hlkwc{header}\hlstd{=F,} \hlkwc{sep}\hlstd{=}\hlstr{','}\hlstd{))))}
     \hlstd{hcd} \hlkwb{<-} \hlkwd{data.frame}\hlstd{(}\hlkwc{indoor}\hlstd{=hc_data_raw}\hlopt{$}\hlstd{V1,}
                       \hlkwc{outdoor}\hlstd{=hc_data_raw}\hlopt{$}\hlstd{V2,}
                       \hlkwc{diff}\hlstd{=(hc_data_raw}\hlopt{$}\hlstd{V1} \hlopt{-} \hlstd{hc_data_raw}\hlopt{$}\hlstd{V2))}
\end{alltt}
\end{kframe}
\end{knitrout}


    @@ Calculate a confidence interval for the population mean difference between indoor and outdoor concentrations
    using a confidence level of 95\%, and interpret the resulting interval. Use your confidence interval to carry out a
    hypothesis testing if indoor and outdoor concentrations are the same.

    @@@ We can use {\ttfamily R} to perform these calculations. Note, since our dataset has $33
    < 40$ entries, we need to use the $t$ distribution, defined below.

    \[
        \overline{X} \pm t_{\alpha / 2, n - 1} \cdot \frac{sd}{\sqrt{n} }
    \]

\begin{knitrout}
\definecolor{shadecolor}{rgb}{0.969, 0.969, 0.969}\color{fgcolor}\begin{kframe}
\begin{alltt}
             \hlstd{alpha}    \hlkwb{<-} \hlnum{0.05}
             \hlstd{n}        \hlkwb{<-} \hlkwd{length}\hlstd{(hcd}\hlopt{$}\hlstd{diff)}
             \hlstd{t}        \hlkwb{<-} \hlkwd{qt}\hlstd{(alpha} \hlopt{/} \hlnum{2}\hlstd{, n} \hlopt{-} \hlnum{1}\hlstd{)}
             \hlstd{diffsd}   \hlkwb{<-} \hlkwd{sd}\hlstd{(hcd}\hlopt{$}\hlstd{diff)}
             \hlstd{diffmean} \hlkwb{<-} \hlkwd{mean}\hlstd{(hcd}\hlopt{$}\hlstd{diff)}
             \hlstd{lower}    \hlkwb{<-} \hlstd{diffmean} \hlopt{-} \hlkwd{abs}\hlstd{(t} \hlopt{*} \hlstd{(diffsd} \hlopt{/} \hlkwd{sqrt}\hlstd{(n)))}
             \hlstd{upper}    \hlkwb{<-} \hlstd{diffmean} \hlopt{+} \hlkwd{abs}\hlstd{(t} \hlopt{*} \hlstd{(diffsd} \hlopt{/} \hlkwd{sqrt}\hlstd{(n)))}
\end{alltt}
\end{kframe}
\end{knitrout}


    Using these results, we obtain a confidence interval of $\boxed{\left( -0.5611, -0.2868 \right)}$. We
    can now use this interval to perform hypothesis testing, assuming that the null hypothesis is $indoor = outdoor$.
    If we use the difference between the two datasets as our test statistic, normalize it, and then compare it versus
    the normal distribution, we can determine our p-values. Note, since our hypothesis deals with equality, our testing
    must be two-tailed. If we let $X$ be the difference between the two datasets, then our corresponding z test is

    \[
        Z = \frac{X - \mu}{\sigma/\sqrt{n} }
    \]

\begin{knitrout}
\definecolor{shadecolor}{rgb}{0.969, 0.969, 0.969}\color{fgcolor}\begin{kframe}
\begin{alltt}
             \hlstd{normaldiff} \hlkwb{<-} \hlstd{(hcd}\hlopt{$}\hlstd{diff} \hlopt{-} \hlstd{diffmean)}\hlopt{/}\hlstd{(diffsd} \hlopt{/} \hlkwd{sqrt}\hlstd{(n))}
             \hlstd{p} \hlkwb{<-} \hlkwd{data.frame}\hlstd{(}\hlkwc{vals}\hlstd{=}\hlnum{2} \hlopt{*} \hlstd{(}\hlnum{1} \hlopt{-} \hlkwd{pnorm}\hlstd{(}\hlkwd{abs}\hlstd{(normaldiff))))}
             \hlkwd{plot_p_vals}\hlstd{(p)}
\end{alltt}
\end{kframe}\begin{figure}[H]


{\centering \includegraphics[width=\maxwidth]{figure/hw5_1a_pvals} 

}

\caption[P-Values]{P-Values\label{fig:hw5.1a.pvals} }
\end{figure}


\end{knitrout}


    If you'll note, our p-values are highly weighted towards zero. In fact, most elements fall below 0.05. This
    indicates strong evidence against the null hypothesis.

    @@ If a 34th house were to be randomly selected from the population, between what values would you predict the
    difference in concentrations to lie?
    @@@ WE IGNORE THIS ONE!
    @@ Construct the 95\% confidence intervals for the indoor and for the outdoor true concentration. Do these CIs
    overlap, and what does that imply about the hypothesis test in part (a)? Write a paragraph justifying your answer.
    @@@ Constructing a confidence interval for the two datasets is straightforward.

\begin{knitrout}
\definecolor{shadecolor}{rgb}{0.969, 0.969, 0.969}\color{fgcolor}\begin{kframe}
\begin{alltt}
             \hlstd{indoor_sd} \hlkwb{<-} \hlkwd{sd}\hlstd{(hcd}\hlopt{$}\hlstd{indoor)}
             \hlstd{indoor_mean} \hlkwb{<-} \hlkwd{mean}\hlstd{(hcd}\hlopt{$}\hlstd{indoor)}
             \hlstd{indoor_err} \hlkwb{<-} \hlkwd{abs}\hlstd{(t} \hlopt{*} \hlstd{(indoor_sd} \hlopt{/} \hlkwd{sqrt}\hlstd{(n)))}
             \hlstd{indoor_ci} \hlkwb{<-} \hlkwd{c}\hlstd{(indoor_mean} \hlopt{-} \hlstd{indoor_err,}
                             \hlstd{indoor_mean} \hlopt{+} \hlstd{indoor_err)}

             \hlstd{outdoor_sd} \hlkwb{<-} \hlkwd{sd}\hlstd{(hcd}\hlopt{$}\hlstd{outdoor)}
             \hlstd{outdoor_mean} \hlkwb{<-} \hlkwd{mean}\hlstd{(hcd}\hlopt{$}\hlstd{outdoor)}
             \hlstd{outdoor_err} \hlkwb{<-} \hlkwd{abs}\hlstd{(t} \hlopt{*} \hlstd{(outdoor_sd} \hlopt{/} \hlkwd{sqrt}\hlstd{(n)))}
             \hlstd{outdoor_ci} \hlkwb{<-} \hlkwd{c}\hlstd{(outdoor_mean} \hlopt{-} \hlstd{outdoor_err,}
                              \hlstd{outdoor_mean} \hlopt{+} \hlstd{outdoor_err)}
\end{alltt}
\end{kframe}
\end{knitrout}


    This results in the intervals being:

    \[
        \begin{aligned}
            \text{Indoor} = \left[ 0.1854, 0.2765 \right]\\
            \text{Outdoor} = \left[ 0.5201, 0.7896 \right]
        \end{aligned}
    \]

    If you'll note, these two confidence intervals \textit{do not} overlap! Now, since the given dataset is relatively
    small, our confidence interval may not contain the true mean of the data. This is unlikely. Therefore, it follows
    that the hypothesis previously stated (that the difference of the two datasets would be equal) should not be
    possible since the means are so far away.


    @ Back to the ladybugs! Recall that aphid infestation of fruit trees is usally controlled either via pesticides or
    via ladybug innundation. In a particular area, 2 different (and well isolated) groves, with 15 fruit trees each are
    selected for an experiment. The trees in both groves are of the same age, roughly the same size, and can be assumed
    to be independent. One grove is sprayed with pesticides, and one is flooded with ladybugs. The fruit yield(in
    pounds) for each tree is given.

    @@ Assume that these two samples come from a normal distribution. Find the sample means of yields for the two
    groves, and provide the two 95\% confidence intervals for the true mean yields for trees under the two treatments.

    @@@ Defining our confidence interval using the $t$ distribution, we obtain

    \[
        \left( \overline{X} \pm t_{\alpha / 2, n - 1} \frac{\sigma}{\sqrt{n} } \right)
    \]

\begin{knitrout}
\definecolor{shadecolor}{rgb}{0.969, 0.969, 0.969}\color{fgcolor}\begin{kframe}
\begin{alltt}
             \hlstd{lbd} \hlkwb{<-} \hlkwd{read.csv}\hlstd{(}
                     \hlstr{'./treedata.txt'}\hlstd{,}
                     \hlkwc{sep}\hlstd{=}\hlstr{','}\hlstd{)}
             \hlstd{n}   \hlkwb{<-} \hlkwd{length}\hlstd{(lbd}\hlopt{$}\hlstd{pest)}
             \hlstd{t}   \hlkwb{<-} \hlkwd{qt}\hlstd{(alpha} \hlopt{/} \hlnum{2}\hlstd{,} \hlkwc{df}\hlstd{=n} \hlopt{-} \hlnum{1}\hlstd{)}
             \hlstd{pm}  \hlkwb{<-} \hlkwd{mean}\hlstd{(lbd}\hlopt{$}\hlstd{pest)}
             \hlstd{lm}  \hlkwb{<-} \hlkwd{mean}\hlstd{(lbd}\hlopt{$}\hlstd{lady)}
             \hlstd{per} \hlkwb{<-} \hlkwd{abs}\hlstd{(t}\hlopt{*}\hlstd{(}\hlkwd{sd}\hlstd{(lbd}\hlopt{$}\hlstd{pest)}\hlopt{/}\hlkwd{sqrt}\hlstd{(n)))}
             \hlstd{plw} \hlkwb{<-} \hlstd{pm} \hlopt{-} \hlstd{per}
             \hlstd{pup} \hlkwb{<-} \hlstd{pm} \hlopt{+} \hlstd{per}
             \hlstd{ler} \hlkwb{<-} \hlkwd{abs}\hlstd{(t}\hlopt{*}\hlstd{(}\hlkwd{sd}\hlstd{(lbd}\hlopt{$}\hlstd{lady)}\hlopt{/}\hlkwd{sqrt}\hlstd{(n)))}
             \hlstd{llw} \hlkwb{<-} \hlstd{lm} \hlopt{-} \hlstd{ler}
             \hlstd{lup} \hlkwb{<-} \hlstd{lm} \hlopt{+} \hlstd{ler}
\end{alltt}
\end{kframe}
\end{knitrout}


    We find the sample mean of yield for the Pesticide grove to be $39.2754$, and the sample mean of yield for the
    Ladybug grove to be $47.7767$. Using these with the assumption of normality we can determine the 95\% confidence
    intervals for the sample mean for both groves, using the $t$ distribution since $n=15$. The pesticide grove mean
    confidence interval is identified to be $\left( 34.562, 43.9889 \right)$, and the ladybug grove mean
    confidence interval is identified as $\left( 42.9522, 52.6011 \right)$.

    @@ What are the sample variances for the yields in the two groves? Test the equality of these two variance using a
    significance level of 5\%.
    @@@ Identifying the variances is relatively straightforward.

\begin{knitrout}
\definecolor{shadecolor}{rgb}{0.969, 0.969, 0.969}\color{fgcolor}\begin{kframe}
\begin{alltt}
             \hlstd{vars} \hlkwb{<-} \hlkwd{var}\hlstd{(lbd)}
\end{alltt}
\end{kframe}
\end{knitrout}


    Which results in

    \begin{table}[H]
        \centering
        \begin{tabular}{|l|l|l|}
            \hline
            & {\ttfamily pest} & {\ttfamily lady}\\
            \hline
            {\ttfamily pest} & $72.4448$ & $-4.0252$\\
            \hline
            {\ttfamily lady} & $-4.0252$ & $75.8964$\\
            \hline
        \end{tabular}
        \caption{Covariances of Tree Data}
        \label{table:treevars}
    \end{table}

    Now we can test the variance's equality. Let our hypothesis be that the two variances are equal, and our test
    statistic equal to

    \[
        \frac{s_1}{s_2}
    \]

    We can now use the F distribution\footnote{$ F_{\alpha, m - 1, n - 1} $} to determine our p-values.

\begin{knitrout}
\definecolor{shadecolor}{rgb}{0.969, 0.969, 0.969}\color{fgcolor}\begin{kframe}
\begin{alltt}
             \hlstd{alpha} \hlkwb{<-} \hlnum{0.05}
             \hlstd{s1} \hlkwb{<-} \hlstd{vars[}\hlnum{1}\hlstd{]}
             \hlstd{s2} \hlkwb{<-} \hlstd{vars[}\hlnum{4}\hlstd{]}
             \hlstd{f} \hlkwb{<-} \hlstd{s1} \hlopt{/} \hlstd{s2}
             \hlstd{lF} \hlkwb{<-} \hlkwd{qf}\hlstd{(alpha} \hlopt{/} \hlnum{2}\hlstd{, n} \hlopt{-} \hlnum{1}\hlstd{, n} \hlopt{-} \hlnum{1}\hlstd{)}
             \hlstd{uF} \hlkwb{<-} \hlkwd{qf}\hlstd{(}\hlnum{1} \hlopt{-} \hlstd{alpha} \hlopt{/} \hlnum{2}\hlstd{, n} \hlopt{-} \hlnum{1}\hlstd{, n} \hlopt{-} \hlnum{1}\hlstd{)}
\end{alltt}
\end{kframe}
\end{knitrout}


    Using these values we see that $0.3357 \le 0.9545 \le 2.9786$, therefore the variances are probably
    equal.

    @@ Depending on your answer, test the equality of the two mean yields.
    @@@ Examining the yields, we see that we can test mean equality using a pooled t test procedure. Let $X$ be the
    difference in the means, $\mu_1 - \mu_2$. We can also calculate our shared variance

\begin{knitrout}
\definecolor{shadecolor}{rgb}{0.969, 0.969, 0.969}\color{fgcolor}\begin{kframe}
\begin{alltt}
             \hlstd{diff} \hlkwb{<-} \hlstd{lbd}\hlopt{$}\hlstd{pest} \hlopt{-} \hlstd{lbd}\hlopt{$}\hlstd{lady}
             \hlstd{mu} \hlkwb{<-} \hlkwd{mean}\hlstd{(diff)}
             \hlstd{c} \hlkwb{<-} \hlstd{(n} \hlopt{-} \hlnum{1}\hlstd{)} \hlopt{/} \hlstd{(}\hlnum{2}\hlopt{*}\hlstd{n}  \hlopt{-} \hlnum{2}\hlstd{)}
             \hlstd{s} \hlkwb{<-} \hlstd{c} \hlopt{*} \hlkwd{sd}\hlstd{(lbd}\hlopt{$}\hlstd{pest)} \hlopt{+} \hlstd{c} \hlopt{*} \hlkwd{sd}\hlstd{(lbd}\hlopt{$}\hlstd{lady)}
             \hlstd{z} \hlkwb{<-} \hlstd{(diff} \hlopt{-} \hlstd{mu)} \hlopt{/} \hlkwd{sqrt}\hlstd{(s)}
             \hlstd{p} \hlkwb{<-} \hlkwd{data.frame}\hlstd{(}\hlkwc{vals}\hlstd{=}\hlnum{2}\hlopt{*}\hlstd{(}\hlnum{1} \hlopt{-} \hlkwd{pnorm}\hlstd{(}\hlkwd{abs}\hlstd{(z))))}
             \hlkwd{plot_p_vals}\hlstd{(p)}
\end{alltt}
\end{kframe}\begin{figure}[H]


{\centering \includegraphics[width=\maxwidth]{figure/hw5_2c} 

}

\caption[P-Values]{P-Values\label{fig:hw5.2c} }
\end{figure}


\end{knitrout}


    Note, almost all values calculated have probability of appearing less than 5\% of the time. This indicates that the
    means are probably unequal.

    @ A controlled clinical trial was run to investigate the effect of the drug stelazine on chronic schizophrenics.
    The data is given in the spreadsheet and shows the change in the patients behavior rating score after 3 months. Use
    appropriate hypothesis tests to answer the following two questions, being sure to complete each of the four steps:

    (i) State the two hypotheses being tested.

    (ii) Compute the appropriate test statistic, degrees of freedom, and p-value.

    (iii) Decide whether to reject the null using a significance level of 0.05

    (iv) Conclude by writing a short sentence explaining your decision.

\begin{knitrout}
\definecolor{shadecolor}{rgb}{0.969, 0.969, 0.969}\color{fgcolor}\begin{kframe}
\begin{alltt}
     \hlstd{a_st} \hlkwb{<-} \hlkwd{c}\hlstd{(}\hlnum{0.8}\hlstd{,} \hlnum{0.1}\hlstd{,} \hlnum{0.55}\hlstd{,} \hlnum{0.6}\hlstd{,} \hlnum{0.34}\hlstd{,} \hlnum{1.42}\hlstd{,} \hlnum{1.74}\hlstd{,} \hlopt{-}\hlnum{0.29}\hlstd{,} \hlnum{0.53}\hlstd{)}
     \hlstd{a_pl} \hlkwb{<-} \hlkwd{c}\hlstd{(}\hlopt{-}\hlnum{0.4}\hlstd{,} \hlnum{0.4}\hlstd{,} \hlopt{-}\hlnum{0.1}\hlstd{,} \hlopt{-}\hlnum{0.9}\hlstd{,} \hlnum{0.2}\hlstd{,} \hlnum{0.78}\hlstd{,} \hlnum{0.3}\hlstd{,} \hlnum{0.64}\hlstd{,} \hlnum{0.42}\hlstd{)}
     \hlstd{a_df} \hlkwb{<-} \hlkwd{c}\hlstd{(}\hlnum{1.2}\hlstd{,} \hlopt{-}\hlnum{0.3}\hlstd{,} \hlnum{0.65}\hlstd{,} \hlnum{1.5}\hlstd{,} \hlnum{0.14}\hlstd{,} \hlnum{0.64}\hlstd{,} \hlnum{1.44}\hlstd{,} \hlopt{-}\hlnum{0.93}\hlstd{,} \hlnum{0.11}\hlstd{)}

     \hlstd{b_st} \hlkwb{<-} \hlkwd{c}\hlstd{(}\hlopt{-}\hlnum{0.45}\hlstd{,} \hlnum{0.15}\hlstd{,} \hlopt{-}\hlnum{0.19}\hlstd{,} \hlnum{0.12}\hlstd{,} \hlnum{0.03}\hlstd{,} \hlopt{-}\hlnum{0.15}\hlstd{,} \hlnum{0.48}\hlstd{)}
     \hlstd{b_pl} \hlkwb{<-} \hlkwd{c}\hlstd{(}\hlnum{0.01}\hlstd{,} \hlnum{0.14}\hlstd{,} \hlopt{-}\hlnum{0.55}\hlstd{,} \hlopt{-}\hlnum{0.05}\hlstd{,} \hlnum{0.04}\hlstd{,} \hlopt{-}\hlnum{1.16}\hlstd{,} \hlopt{-}\hlnum{0.16}\hlstd{)}
     \hlstd{b_df} \hlkwb{<-} \hlkwd{c}\hlstd{(}\hlopt{-}\hlnum{0.46}\hlstd{,} \hlnum{0.01}\hlstd{,} \hlnum{0.36}\hlstd{,} \hlnum{0.17}\hlstd{,} \hlopt{-}\hlnum{0.01}\hlstd{,} \hlnum{1.01}\hlstd{,} \hlnum{0.64}\hlstd{)}
\end{alltt}
\end{kframe}
\end{knitrout}



    @@ Considering all patients, test whether the average behavior change is higher for patients using stelazine than
    for patients receiving a placebo.
    @@@ If we consider all patients, we can essentially pool everything together, ignoring the artificial difference of
    ward. In this case we assume the null hypothesis to be that stelazine was more beneficial for the patients than the
    placebo, in other words if the mean change in behavior of those on the drug exceeded the mean change of those on the
    placebo. We can express this by examining the difference in behaviors from stelazine patients and their placebo
    pairing. We also assume that the alternate hypothesis is that the placebo was more beneficial than the stelazine.

\begin{knitrout}
\definecolor{shadecolor}{rgb}{0.969, 0.969, 0.969}\color{fgcolor}\begin{kframe}
\begin{alltt}
             \hlstd{st} \hlkwb{<-} \hlkwd{c}\hlstd{(a_st, b_st)}
             \hlstd{pl} \hlkwb{<-} \hlkwd{c}\hlstd{(a_pl, b_pl)}
             \hlstd{diff} \hlkwb{<-} \hlkwd{c}\hlstd{(a_df, b_df)}
             \hlstd{z} \hlkwb{<-} \hlstd{(diff} \hlopt{-} \hlkwd{mean}\hlstd{(diff))} \hlopt{/} \hlkwd{sd}\hlstd{(diff)}
             \hlstd{p} \hlkwb{<-} \hlkwd{data.frame}\hlstd{(}\hlkwc{vals}\hlstd{=}\hlnum{1} \hlopt{-} \hlkwd{pnorm}\hlstd{(z))}
             \hlkwd{plot_p_vals}\hlstd{(p)}
\end{alltt}
\end{kframe}\begin{figure}[H]


{\centering \includegraphics[width=\maxwidth]{figure/hw5_3a} 

}

\caption[P-Values]{P-Values\label{fig:hw5.3a} }
\end{figure}


\end{knitrout}


    Since our p values are normally distributed (or at least appear to be given our dataset), we do not reject the null
    hypotheses.\newline

    In the context of the problem, this indicates that the patients who were administered stelazine improved more than
    those given the placebo.

    @@ Using the differences, S-P, test whether stelazine is more effective on average in one of the wards than in the
    other. Carry out the appropriate test for the equality of variance if necessary. Were we right to combine all
    patients for our answer in (a)?
    @@@ Our null hypothesis in this case is that stelazine in ward A is more effective than stelazine in ward B. If we
    let $\mu_{A,1}$ be the mean change for ward A patients on stelazine, $\mu_{A,2}$ the mean change for the placebo
    patients in ward A, $\mu_{B,1}$ the mean change for stelazine patients in ward B, and $\mu_{B,2}$ the mean change
    for patients using the placebo, then we can define our null to be

    \[
        \begin{aligned}
            ( \mu_{A,1} - \mu_{A,2} ) - ( \mu_{B,1} - \mu_{B,2} ) > 0\\
            \mu_{A,diff} - \mu_{B,diff} > 0
        \end{aligned}
    \]

    Before we examine the differences of our datasets, we can first examine our data itself and determine whether or not
    it is from the same distribution, starting with testing equality of the variances.

\begin{knitrout}
\definecolor{shadecolor}{rgb}{0.969, 0.969, 0.969}\color{fgcolor}\begin{kframe}
\begin{alltt}
             \hlstd{f} \hlkwb{<-} \hlkwd{var}\hlstd{(a_df)} \hlopt{/} \hlkwd{var}\hlstd{(b_df)}
             \hlstd{m} \hlkwb{<-} \hlkwd{length}\hlstd{(a_df)}
             \hlstd{n} \hlkwb{<-} \hlkwd{length}\hlstd{(b_df)}
             \hlstd{alpha} \hlkwb{<-} \hlnum{0.05}
             \hlstd{lower} \hlkwb{<-} \hlkwd{qf}\hlstd{(alpha} \hlopt{/} \hlnum{2}\hlstd{, m} \hlopt{-} \hlnum{1}\hlstd{, n} \hlopt{-} \hlnum{1}\hlstd{)}
             \hlstd{upper} \hlkwb{<-} \hlkwd{qf}\hlstd{(}\hlnum{1} \hlopt{-} \hlstd{alpha} \hlopt{/} \hlnum{2}\hlstd{, m} \hlopt{-} \hlnum{1}\hlstd{, n} \hlopt{-} \hlnum{1}\hlstd{)}
\end{alltt}
\end{kframe}
\end{knitrout}


    Since $0.215 \le 2.9306 \le 5.5996$ with p-value $0.8964$, we know that the variances
    are probably equal, therefore if we assume that the distribution is normal in each ward, we can perform a pooled-t
    test.

\begin{knitrout}
\definecolor{shadecolor}{rgb}{0.969, 0.969, 0.969}\color{fgcolor}\begin{kframe}
\begin{alltt}
             \hlstd{s} \hlkwb{<-} \hlstd{((m} \hlopt{-} \hlnum{1}\hlstd{)}\hlopt{/}\hlstd{(m} \hlopt{+} \hlstd{n} \hlopt{-} \hlnum{2}\hlstd{))} \hlopt{*} \hlkwd{var}\hlstd{(a_df)} \hlopt{+}
                     \hlstd{((n} \hlopt{-} \hlnum{1}\hlstd{)}\hlopt{/}\hlstd{(m} \hlopt{+} \hlstd{n} \hlopt{-} \hlnum{2}\hlstd{))} \hlopt{*} \hlkwd{var}\hlstd{(b_df)}
             \hlstd{z} \hlkwb{<-} \hlstd{(}\hlkwd{mean}\hlstd{(a_df)} \hlopt{-} \hlkwd{mean}\hlstd{(b_df))} \hlopt{/}
                     \hlkwd{sqrt}\hlstd{(s} \hlopt{*} \hlstd{((}\hlnum{1}\hlopt{/}\hlstd{m)} \hlopt{+} \hlstd{(}\hlnum{1}\hlopt{/}\hlstd{n)))}
             \hlstd{p} \hlkwb{<-} \hlnum{2} \hlopt{*} \hlstd{(}\hlnum{1} \hlopt{-} \hlkwd{pnorm}\hlstd{(}\hlkwd{abs}\hlstd{(z)))}
\end{alltt}
\end{kframe}
\end{knitrout}


    The p-value of $0.4775$ informs us that the distributions have not only similar variance, but also similar means,
    and are therefore the same. This also tells us that combining the wards for part (a) was correct, as the patient
    pairs all come from the same distribution.

    @ A study performed in Boston showed a link between soda and weak bones. Among 57 active children who do not drink
    soda, 5 suffered fractures. Among 107 active children who reported drinking soda, 38 suffered fractures. The study
    did not specify how much soda the children drank.

    @@ Using a two-sample test of proportions, determine whether this study provides significant evidence at the 1\%
    level to suggest that the probability of getting a fracture is lower for kids that do not drink soda than it is for
    kids that do drink soda.
    @@@ We can first identify our two estimators for the proportion of fractured children. Let $\hat{p_1}$ the the
    sample proportion of fractured children who do not drink soda, and similarly $\hat{p_2}$ the sample proportion of
    fractured children who do drink soda.

\begin{knitrout}
\definecolor{shadecolor}{rgb}{0.969, 0.969, 0.969}\color{fgcolor}\begin{kframe}
\begin{alltt}
             \hlstd{p1} \hlkwb{<-} \hlnum{5} \hlopt{/} \hlnum{57}
             \hlstd{m}  \hlkwb{<-} \hlnum{57}
             \hlstd{p2} \hlkwb{<-} \hlnum{38} \hlopt{/} \hlnum{107}
             \hlstd{n}  \hlkwb{<-} \hlnum{107}
\end{alltt}
\end{kframe}
\end{knitrout}


    We know that we are interested in the value $\hat{p_1} - \hat{p_2}$, which, if equal to zero, would indicate that
    there is no difference between drinking soda and not. Therefore we are only interested in $\hat{p_1} - \hat{p_2} >
    0$, which we take as our null hypothesis. We use z to determine the validity of our null hypothesis.

    \[
        z = \frac{\hat{p_1} - \hat{p_2} }{\sqrt{\hat{p}\hat{q} \left( \frac{1}{m} + \frac{1}{n} \right)} }
    \]

    Where $\hat{p}$ is

    \[
        \hat{p} = \frac{m}{m + n} \cdot \hat{p_1} + \frac{n}{m + n} \cdot \hat{p_2}
    \]

\begin{knitrout}
\definecolor{shadecolor}{rgb}{0.969, 0.969, 0.969}\color{fgcolor}\begin{kframe}
\begin{alltt}
             \hlstd{phat} \hlkwb{<-} \hlstd{((m)}\hlopt{/}\hlstd{(m} \hlopt{+} \hlstd{n))}\hlopt{*}\hlstd{p1} \hlopt{+} \hlstd{((m)}\hlopt{/}\hlstd{(m} \hlopt{+} \hlstd{n))}\hlopt{*}\hlstd{p2}
             \hlstd{q} \hlkwb{<-} \hlnum{1} \hlopt{-} \hlstd{phat}
             \hlstd{z} \hlkwb{<-} \hlstd{(p1} \hlopt{-} \hlstd{p2)} \hlopt{/} \hlkwd{sqrt}\hlstd{(phat} \hlopt{*} \hlstd{q} \hlopt{*} \hlstd{((}\hlnum{1} \hlopt{/} \hlstd{m)} \hlopt{+} \hlstd{(}\hlnum{1} \hlopt{/} \hlstd{n)))}
\end{alltt}
\end{kframe}
\end{knitrout}


    Using this calculated z we see that

    \[
        z \ge z_{0.01} \Rightarrow -4.5191 \ge -2.3263
    \]

    Since $z$ is not greater than the normal distribution at 0.01, our null can be assumed to be valid, and that the
    probability of getting a fracture is lower for soda-free children.

    @@ Construct a 99\% confidence interval for the difference in the probabilities of getting a fraction for non-cola
    drinking and cola drinking kids. In one of two sentences, interpret this interval for someone who has never taken a
    statistics course.
    @@@ To construct a confidence interval we need the following formula.

    \[
        \hat{p_1} - \hat{p_2} \pm z_{\alpha / 2} \sqrt{\frac{\hat{p_1}\hat{q_1} }{m} + \frac{\hat{p_2}\hat{q_2} }{n} }
    \]

\begin{knitrout}
\definecolor{shadecolor}{rgb}{0.969, 0.969, 0.969}\color{fgcolor}\begin{kframe}
\begin{alltt}
             \hlstd{z} \hlkwb{<-} \hlkwd{qnorm}\hlstd{(}\hlnum{0.005}\hlstd{)}
             \hlstd{err} \hlkwb{<-} \hlkwd{abs}\hlstd{(z} \hlopt{*} \hlkwd{sqrt}\hlstd{(((p1} \hlopt{*} \hlstd{(}\hlnum{1} \hlopt{-} \hlstd{p1))}\hlopt{/}\hlstd{m)} \hlopt{+} \hlstd{((p2} \hlopt{*} \hlstd{(}\hlnum{1} \hlopt{-} \hlstd{p2))}\hlopt{/}\hlstd{n)))}
             \hlstd{lower} \hlkwb{<-} \hlstd{p1} \hlopt{-} \hlstd{p2} \hlopt{-} \hlstd{err}
             \hlstd{upper} \hlkwb{<-} \hlstd{p1} \hlopt{-} \hlstd{p2} \hlopt{+} \hlstd{err}
\end{alltt}
\end{kframe}
\end{knitrout}


    Which yields the confidence interval $\left[ -0.4208, -0.1141 \right]$. To the uninitiated, this
    confidence interval can be interpreted as if we take a group of children partitioned into those who drink soda and
    those who don't, and apply a consistent force to their leg. The ones that drink soda more often will break a bone
    than those who don't. The difference between the proportions of the populations of the children with fractures will
    fall within this interval 99\% of the time.
\end{easylist}

\newpage

\end{document}
