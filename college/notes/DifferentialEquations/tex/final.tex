\section{Linear Systems of Differential Equations}
To define the linear first order differential equations system:

An $n$-dimensional first order differential equations system on an open interval $I$ is one that can be written as a matrix vector equation.

    \begin{equation}\label{eq:desystem_vector_form}
        \vec{x} \prime (t) = A(t) \vec{x}(t) + \vec{f}(t)
    \end{equation}\myequations{Differential Equation Systems Vector Equation Form}

    \begin{itemize}
        \item $A(t)$ is an $n \times n$ matrix of continuous functions on $I$.
        \item $f(t)$ is an $n \times 1$ vector of continuous functions on $I$.
        \item $\vec{x}(t)$ is an $n \times 1$ solution vector.
        \item If $f(t) \equiv 0$, the system is homogeneous.
    \end{itemize}

    \subsection{Graphical Methods}
    We use the phase plane from before to accurately represent these systems.

        \subsubsection{Nullclines}
        The $v$ nullcline is the set of all points with vertical slope which occur on the curve obtained by solving
            \[
                x\prime = f(x, y) = 0
            \]
        The $h$ nullcline is the same except with horizontal slope and is found with
            \[
                y\prime = f(x, y) = 0
            \]
        At the intersection we get a fixed equilibrium point.

        \subsubsection{Eigenvalues}
        Eigenvalues play a large role in phase planes as well. For an autonomous and homogeneous system of differential linear system of equations:

            \begin{itemize}
                \item Trajectories are toward or away based on the sign of the eigenvalue.
                \item Along each eigenvector is the separatria that seperates different curves.
                \item Equilibrium arrives at origin (Symmetric)
                \item Speed is determined by magnitude of the eigenvalues.
            \end{itemize}

    \subsection{Linear Systems with Real Eigenvalues}
    To solve a system in the form

        \[
            \vec{x} = A \vec{x}
        \]

        \begin{enumerate}
            \item Find eigenvalues of $A$.
            \item Find associated eigenvectors.
            \item Solution is in the form (for a $2\times 2$ matrix at least) our solution is in the form:
                \[
                    \vec{x}(t) = c_1 e^{\lambda_1 t} \vec{v}_1 + c_2 e^{\lambda_2 t} \vec{v}_2
                \]
        \end{enumerate}

    If there are insufficient eigenvalues (repeated eigenvalues), follow the method below.

        \begin{enumerate}
            \item Find the one eigenvalue.
            \item Find its eigenvector.
            \item Find $\vec{v}$ such that $(A - \lambda I) \vec{u} = \vec{v}$.
            \item Solution: $\vec{x}(t) = c_1 e^{\lambda t} \vec{v} + c_2 e^{\lambda t} (t \vec{v} + \vec{u})$.
        \end{enumerate}

    \subsection{Non-Real Eigenvalues}
    If we have a matrix $A$ with non-real eigenvalues $\lambda_1, \lambda_2 = \alpha \pm i \beta$, the corresponding eigenvectors are also complex conjugate pairs in the form:

        \[
            \vec{v}_1, \vec{v}_2 = \vec{p} \pm i \vec{q}
        \]

    To solve:

        \begin{enumerate}
            \item For the first eigenvalue, find its eigenvector. The second eigenvector is a pair of the first.
            \item Construct the real and non-real parts:
                \[
                    \begin{cases}
                        \vec{x}_r = e^{\alpha t} ( \cos(\beta t) \vec{p} - \sin(\beta t)\vec{q})\\
                        \vec{x}_i = e^{\alpha t} ( \sin(\beta t) \vec{p} + \cos(\beta t)\vec{q})
                    \end{cases}
                \]
            \item The general solution is defined as
                \[
                    \vec{x}(t) = c_1 \vec{x}_r(t) + c_2 \vec{x}_i(t)
                \]
        \end{enumerate}

        \subsubsection{Interpreting Non-Real Eigenvalues}
        \[
            \left[ \begin{array}{c}
                \vec{x}_r\\
                \vec{x}_i
            \end{array} \right] = e^{\alpha t}
            \left[ \begin{array}{c}
                \cos(\beta t) - \sin(\beta t)\\
                \sin(\beta t) + \cos(\beta t)
            \end{array} \right]
            \left[ \begin{array}{c}
                \vec{p}\\
                \vec{q}
            \end{array} \right]
        \]

        \begin{itemize}
            \item The first variable defines the expansion.
                \begin{itemize}
                    \item If $\alpha > 0 \to$ Growth without bound.
                    \item If $\alpha < 0 \to$ Decay to $0$.
                    \item If $\alpha = 0 \to$ Period solutions.
                \end{itemize}
            \item The second defines rotation.
                \begin{itemize}
                    \item Counterclockwise for $\beta > 0$
                    \item Clockwise for $\beta < 0$
                \end{itemize}
            \item The third defines tilt and shape.
        \end{itemize}

    \subsection{Stability and Linear Classification}
    A constant solution $\vec{x} \equiv \vec{c}$ is called an equilibrium solution. An equilibrium solution in the phase plane is a fixed point.

        \begin{itemize}
            \item If solutions remain close and tend to $\vec{c}$ as $t \to \infty$ we call this asymptotically stable.
            \item If solutions are neither attracted nor repelled, we call this neutrally stable.
            \item If other, it is unstable.
        \end{itemize}

    \subsection{Parameter Plane}
    % Insert this image

    \subsection{Possibilities in the Parameter Plane}
    We have to consider a couple different possibilities.

    \begin{enumerate}
        \item \textbf{Real Distinct Eigenvalues ($\Delta > 0$)}

            When $\Delta = {({\rm Tr}(A))}^2 - 4|A| > 0$ we have real eigenvalues $\lambda_1 \neq \lambda_2$ with corresponding linearly independent eigenvectors $\vec{v}_1$ and $\vec{v}_2$ with general solution

            \[ \vec{x} = c_1 e^{\lambda_1 t} \vec{v}_1 + c_2 e^{\lambda_2 t} \vec{v}_2 \]

            The signs of the eigenvalues direct the trajectory behavior in the phase portrait.

            We can label the eigendirections fast or slow based on the magnitude of the eigenvalues. Whichever it is, the trajectories are parallel to fast and perpendicular to slow.

            Three possibilities

                \begin{itemize}
                    \item Attracting Node ($\lambda_1 < \lambda_2 < 0$)
                    \item Repelling Node ($0 < \lambda_1 < \lambda_2$)
                    \item Saddle Point ($\lambda_1 < 0 < \lambda_2$)
                \end{itemize}

            \item \textbf{Complex Conjugate Eigenvalues ($\Delta < 0$)}

                When $\Delta = {({\rm Tr}(A))}^2 - 4|A| < 0$ we get non-real eigenvalues.

                \[ \lambda_{1,2} = \alpha \pm \beta i \]

                where $\alpha = \frac{ {\rm Tr}(A)}{2}$ and $\beta = \sqrt{-\Delta}$. $\alpha$ and $\beta$ are real. The real solutions are given by:

                \[ \begin{cases}
                        \vec{x}_r = e^{\alpha t} ( \cos(\beta t) \vec{p} - \sin(\beta t)\vec{q})\\
                        \vec{x}_i = e^{\alpha t} ( \sin(\beta t) \vec{p} + \cos(\beta t)\vec{q})
                    \end{cases} \]

                For complex eigenvalues stability behavior depends on the sign of $\alpha$.

                \begin{itemize}
                    \item Attracting Spiral ($\alpha < 0$)
                    \item Repelling Spiral ($\alpha > 0$)
                    \item Center ($\alpha = 0$)
                \end{itemize}

            \item \textbf{Borderline Case: Zero Eigenvalues ($|A| = 0$)}
                If one eigenvalue is zero we get a row of non-isolated fixed points in the eigendirection associated with the eigenvalues, and the phase plane trajectories are all straight lines in direction of other eigenvector.

                If two eigenvalues are zero, there is only one eigenvector, along which we have a row of non-isolated fixed points. Trajectories from any other point in the phase plane must be parallel to the one eigenvector in the direction specified by the system.

            \item \textbf{Borderline Case: Real Repeated Eigenvalues ($\Delta = 0$)}

                In this situation we have two cases to contend with.

                \begin{enumerate}
                    \item \textit{Degenerate Node:} If $\lambda$ has one linearly independent eigenvector we call it degenerate. The sign of $\lambda$ gives its stability.
                    \item \textit{Star Node:} If $\lambda$ has two linearly independent eigenvectors we call it an attracting or repelling star node. The sign of $\lambda$ gives its stability.
                \end{enumerate}

                In both cases, the sign of $\lambda$ gives its stability.

                    \begin{itemize}
                        \item If $\lambda > 0$, trajectories go to infinity, parallel to $\vec{v}$.
                        \item If $\lambda < 0$, trajectories approach the origin parallel to $\vec{v}$.
                        \item If $\lambda = 0$, there exists a line of fixed points at the eigenvector.
                    \end{itemize}
    \end{enumerate}

\section{Non-Linear Systems}
We will be looking at autonomous $2 \times 2$ systems. Note, there is no matrix without linearity.

    \subsection{Properties of Phase Plane Trajectories in Non-Linear $2 \times 2$ Systems}
    \begin{enumerate}
        \item When uniqueness holds, phase plane trajectories cannot cross.
        \item When the given functions $f$ and $g$ are continuous, trajectories are continuous and smooth.
    \end{enumerate}

    \subsection{Equilibria}
    Phase Portraits can have more than one, or none at all. To find a system's equilibria, solve $x\prime$ and $y\prime$ simultaneously.

    \subsection{Nullclines}
    Nullclines in this case are the same as before.

    \subsection{Limit Cycle}
    A limit cycle is a closed curve (representing a periodic solution) to which other solutions tend by winding around more and more closely from either inside or outside.

\section{Linearization}
Just as we've done with calculus, we can linearize the system to understand the behavior at a certain point as well as nearby.

    \[
        \begin{cases}
            x\prime = f(x, y)\\
            y\prime = g(x, y)
        \end{cases}
        \text{ Inserting Equilibrium Point } e \text{ we get }
        \begin{cases}
            f(x_e, y_e)\\
            g(x_e, y_e)
        \end{cases}
    \]

    \begin{thm}[Jacobian]
        For a given system of equations:

            \[
                \begin{cases}
                    x\prime = f(x, y)\\
                    y\prime = g(x, y)
                \end{cases}
            \]

        where $f$ and $g$ are twice differentiable, the linearized system at an equilibrium point $(x_e, y_e)$ translated by $u = x - x_e$ and $v = y - y_e$ is

            \begin{equation}\label{eq:jacobian}
                \left[ \begin{array}{c}
                    u\\
                    v
                \end{array} \right] \prime = J(x_e, y_e) \text{ where } J(x_e, y_e) = 
                \left[ \begin{array}{cc}
                    f_x(x_e, y_e) & f_y(x_e, y_e)\\
                    g_x(x_e, y_e) & g_y(x_e, y_e)
                \end{array} \right]
            \end{equation}

        which is the Jacobian Matrix. If $J$ is non-singular, the linearized point has a unique equilibrium point at $(u, v) = (0,0)$, and the techniques from before can be used to classify behavior.

        For the non-linear system and Jacobian given above, $\lambda_1$ and $\lambda_2$ be real or non-real.\footnote{Where $\lambda$ is the set of eigenvalues of the Jacobian Matrix}
    \end{thm}

    Using the Jacobian Matrix we can determine behavior of a system of differential equations using Table \eqref{table:jacob_values}.

    \begin{table}
        \scalebox{0.8}{
            \begin{tabular}{| p{0.10\textwidth} | p{0.16\textwidth} | p{0.2\textwidth} | p{0.2\textwidth} | p{0.2\textwidth} | p{0.2\textwidth} |}
                \hline
                Type & Eigenvalues & \multicolumn{2}{c|}{Linearized System} & \multicolumn{2}{c|}{Nonlinear System}\\
                & & Geometry & Stability & Geometry & Stability\\
                \hline
                \multirow{3}{0.10\textwidth}{Real Distinct Roots} & $\lambda_1 < \lambda_2 < 0$ & Attracting Node & Asymptotically Stable & Attracting Node & Asymptotically Stable\\
                & $0 < \lambda_2 < \lambda_1$ & Repelling Node & Unstable & Repelling Node & Unstable\\
                & $\lambda_1 < 0 < \lambda_2$ & Saddle & Unstable & Saddle & Unstable\\
                \hline
                \hline
                \multirow{2}{0.10\textwidth}{Real Repeated Roots} & $\lambda_1 = \lambda_2 < 0$ & Attracting Star of Degenerate Node & Asymptotically Stable & Attracting Node or Spiral & Asymptotically Stable\\
                & $\lambda_1 = \lambda_2 > 0$ & Repelling Star or Degenerate Node & Unstable & Repelling Node or Spiral & Unstable\\
                \hline
                \hline
                \multirow{3}{0.10\textwidth}{Complex Conjugate Roots} & $\alpha > 0$ & Repelling Spiral & Unstable & Repelling Spiral & Unstable\\
                & $\alpha < 0$ & Attracting Spiral & Asymptotically Stable & Attracting Spiral & Asymptotically Stable\\
                & $\alpha = 0$ & Center & Stable & Center or Spiral & Uncertain\\
                \hline
            \end{tabular}
        }
        \caption{Table of Behavior Based on the System's Jacobian Matrix Eigenvalues}
        \label{table:jacob_values}
    \end{table}
