\documentclass[12pt]{beamer}

%%%%%%%%%%%%%%%%%%%%%%%%%%%%%%%%%%%%%%%%%%%%%%%%%%%%%%%%%%%%%%%%%%%%%%%%%%%%%%%%
% LaTeX Imports
%%%%%%%%%%%%%%%%%%%%%%%%%%%%%%%%%%%%%%%%%%%%%%%%%%%%%%%%%%%%%%%%%%%%%%%%%%%%%%%%
\usepackage{amsfonts}                                                   % Math fonts
\usepackage{amsmath}                                                    % Math formatting
\usepackage{amssymb}                                                    % Math formatting
\usepackage{amsthm}                                                     % Math Theorems
\usepackage{arydshln}                                                   % Dashed hlines
\usepackage{attachfile}                                                 % AttachFiles
\usepackage{cancel}                                                     % Cancelled math
\usepackage{caption}                                                    % Figure captioning
\usepackage{color}                                                      % Nice Colors
\input{./lib/dragon.inp}                                                % Tikz dragon curve
\usepackage[ampersand]{easylist}                                        % Easy lists
\usepackage{fancyhdr}                                                   % Fancy Header
\usepackage[T1]{fontenc}                                                % Specific font-encoding
%\usepackage[margin=1in, marginparwidth=2cm, marginparsep=2cm]{geometry} % Margins
\usepackage{graphicx}                                                   % Include images
\usepackage{hyperref}                                                   % Referencing
\usepackage[none]{hyphenat}                                             % Don't allow hyphenation
\usepackage{lipsum}                                                     % Lorem Ipsum Dummy Text
\usepackage{listings}                                                   % Code display
\usepackage{marginnote}                                                 % Notes in the margin
\usepackage{microtype}                                                  % Niceness
\usepackage{lib/minted}                                                 % Code display
\usepackage{./lib/mlptikz}                                              % Tikz mlp
\usepackage{multirow}                                                   % Multirow tables
\usepackage{pdfpages}                                                   % Include pdfs
\usepackage{pgfplots}                                                   % Create Pictures
\usepackage{rotating}                                                   % Figure rotation
\usepackage{setspace}                                                   % Allow double spacing
\usepackage{subcaption}                                                 % Figure captioning
\usepackage{tikz}                                                       % Create Pictures
\usepackage{tocloft}                                                    % List of Equations
%%%%%%%%%%%%%%%%%%%%%%%%%%%%%%%%%%%%%%%%%%%%%%%%%%%%%%%%%%%%%%%%%%%%%%%%%%%%%%%%
% Package Setup
%%%%%%%%%%%%%%%%%%%%%%%%%%%%%%%%%%%%%%%%%%%%%%%%%%%%%%%%%%%%%%%%%%%%%%%%%%%%%%%%
\hypersetup{%                                                           % Setup linking
    colorlinks=true,
    linkcolor=black,
    citecolor=black,
    filecolor=black,
    urlcolor=black,
}
\RequirePackage[l2tabu, orthodox]{nag}                                  % Nag about bad syntax
\renewcommand*\thesection{\arabic{section} }                             % Reset numbering
\renewcommand{\theFancyVerbLine}{ {\arabic{FancyVerbLine} } }              % Needed for code display
\renewcommand{\footrulewidth}{0.4pt}                                    % Footer hline
\setcounter{secnumdepth}{3}                                             % Include subsubsections in numbering
\setcounter{tocdepth}{3}                                                % Include subsubsections in toc
%%%%%%%%%%%%%%%%%%%%%%%%%%%%%%%%%%%%%%%%%%%%%%%%%%%%%%%%%%%%%%%%%%%%%%%%%%%%%%%%
% Custom commands
%%%%%%%%%%%%%%%%%%%%%%%%%%%%%%%%%%%%%%%%%%%%%%%%%%%%%%%%%%%%%%%%%%%%%%%%%%%%%%%%
\newcommand{\nvec}[1]{\left\langle #1 \right\rangle}                    %  Easy to use vector
\newcommand{\inprod}[2]{\left\langle \vec{#1}, \vec{#2} \right\rangle}  %  Easy to use inner product
\newcommand{\norm}[1]{\lvert \lvert \vec{#1} \rvert \rvert}             %  Easy to use norm
\newcommand{\ma}[0]{\mathbf{A} }                                         %  Easy to use vector
\newcommand{\mb}[0]{\mathbf{B} }                                         %  Easy to use vector
\newcommand{\abs}[1]{\left\lvert #1 \right\rvert}                       %  Easy to use abs
\newcommand{\pren}[1]{\left( #1 \right)}                                %  Big parens
\let\oldvec\vec
\renewcommand{\vec}[1]{\mathbf{#1} }                            %  Vector Styling
\newtheorem{thm}{Theorem}                                               %  Define the theorem name
\theoremstyle{definition}
\newtheorem{definition}{Definition}                                     %  Define the definition name
\newtheorem{ex}{Example}                                                %  Define the example name
\definecolor{bg}{rgb}{0.95,0.95,0.95}
\newcommand{\java}[4]{\vspace{10pt}\inputminted[firstline=#2,
                                 lastline=#3,
                                 firstnumber=#2,
                                 gobble=#4,
                                 frame=single,
                                 label=#1,
                                 bgcolor=bg,
                                 linenos]{java}{#1} }
\newcommand{\python}[4]{\vspace{10pt}\inputminted[firstline=#2,
                                 lastline=#3,
                                 firstnumber=#2,
                                 gobble=#4,
                                 frame=single,
                                 label=#1,
                                 bgcolor=bg,
                                 linenos]{python}{#1} }
\newcommand{\js}[4]{\vspace{10pt}\inputminted[firstline=#2,
                                 lastline=#3,
                                 firstnumber=#2,
                                 gobble=#4,
                                 frame=single,
                                 label=#1,
                                 bgcolor=bg,
                                 linenos]{js}{#1} }
%%%%%%%%%%%%%%%%%%%%%%%%%%%%%%%%%%%%%%%%%%%%%%%%%%%%%%%%%%%%%%%%%%%%%%%%%%%%%%%%
% Beginning of document items - headers, title, toc, etc...
%%%%%%%%%%%%%%%%%%%%%%%%%%%%%%%%%%%%%%%%%%%%%%%%%%%%%%%%%%%%%%%%%%%%%%%%%%%%%%%%
\pagestyle{fancy}                                                       %  Establishes that the headers will be defined
\fancyhead[LE,LO]{Matrix Methods Notes}                                  %  Adds header to left
\fancyhead[RE,RO]{Zoe Farmer}                                       %  Adds header to right
\cfoot{\mlptikz[size=0.25in, text=on, textposx=0, textposy=0, textvalue=\thepage, textscale=0.75in]{applejack} }
\lfoot{APPM 3310}
\rfoot{Beylkin}
\title{Matrix Methods Notes}
\author{Zoe Farmer}

\usepackage{multimedia}
%%%%%%%%%%%%%%%%%%%%%%%%%%%%%%%%%%%%%%%%%%%%%%%%%%%%%%%%%%%%%%%%%%%%%%%%%%%%%%%%
% Beginning of document items - headers, title, toc, etc...
%%%%%%%%%%%%%%%%%%%%%%%%%%%%%%%%%%%%%%%%%%%%%%%%%%%%%%%%%%%%%%%%%%%%%%%%%%%%%%%%
\title[Ouroboros]{Creating a One Dimensional\\Soliton Gas in Viscous Fluid
Conduits}
\author{Zoe Farmer\\\url{www.dataleek.io}}
\institute{University of Colorado, Boulder\\
            Advisors: Mark Hoefer, Michelle Maiden}
\date{March 4, 2017}


%%%%%%%%%%%%%%%%%%%%%%%%%%%%%%%%%%%%%%%%%%%%%%%%%%%%%%%%%%%%%%%%%%%%%%%%%%%%%%%%
% Custom Beamer Theming
%%%%%%%%%%%%%%%%%%%%%%%%%%%%%%%%%%%%%%%%%%%%%%%%%%%%%%%%%%%%%%%%%%%%%%%%%%%%%%%%
\usetheme{Madrid}
\makeatletter

% http://tex.stackexchange.com/questions/35166/how-can-i-remove-the-institute-from-the-author-footline-on-beamer
% TLDR; Remove the Institution from default Madrid theme by redefining footline
% template.
\setbeamertemplate{footline}
{
  \leavevmode%
  \hbox{%
  \begin{beamercolorbox}[wd=.333333\paperwidth,ht=2.25ex,dp=1ex,center]{author in head/foot}%
      \usebeamerfont{author in head/foot}{\color{cugold}\insertshortauthor}%~~\beamer@ifempty{\insertshortinstitute}{}{(\insertshortinstitute)}
  \end{beamercolorbox}%
  \begin{beamercolorbox}[wd=.333333\paperwidth,ht=2.25ex,dp=1ex,center]{title in head/foot}%
    \usebeamerfont{title in head/foot}\insertshorttitle
  \end{beamercolorbox}%
  \begin{beamercolorbox}[wd=.333333\paperwidth,ht=2.25ex,dp=1ex,right]{date in head/foot}%
    \usebeamerfont{date in head/foot}\insertshortdate{}\hspace*{2em}
    \insertframenumber{} / \inserttotalframenumber\hspace*{2ex} 
  \end{beamercolorbox}}%
  \vskip0pt%
}
\makeatother

% Redefine title page to be a little more square/2d
\setbeamertemplate{title page}
{
  \begin{centering}
    \begin{beamercolorbox}[sep=8pt,center]{title}
      \usebeamerfont{title}\inserttitle\par%
      \ifx\insertsubtitle\@empty%
      \else%
        \vskip0.25em%
        {\usebeamerfont{subtitle}\usebeamercolor[fg]{subtitle}\insertsubtitle\par}%
      \fi%     
    \end{beamercolorbox}%
    \begin{beamercolorbox}[sep=8pt,center]{institute}
      \usebeamerfont{institute}\insertinstitute
    \end{beamercolorbox}
    \vskip1em\par
    \begin{beamercolorbox}[sep=8pt,center]{date}
      \usebeamerfont{date}\insertdate
    \end{beamercolorbox}%\vskip0.5em
    \begin{beamercolorbox}[sep=8pt,center]{author}
      \usebeamerfont{author}\insertauthor
    \end{beamercolorbox}
  \end{centering}
  %\vfill
}
\makeatother

\usecolortheme{wolverine}

%%%%% SET THEME
\beamertemplatenavigationsymbolsempty   % Disable navigation

%%%%% INSERT LOGO
\logo{%
    \vspace{-0.29cm}
    \makebox[0.95\paperwidth]{%
        \includegraphics[scale=0.4]{./img/nsf.png}{\color{cublack}Funded by NSF EXTREEMS-QED}
        \hfill%
        \color{cugold}CU Boulder Applied Math\includegraphics[height=0.8cm]{./img/appm.png}
    }%
}

%%%%% DEFINE COLORS
\definecolor{bgcolor}{RGB}{255,255,240}
\definecolor{cugold}{RGB}{207,184,124}
\definecolor{cublack}{RGB}{0,0,0}
\definecolor{cudarkgray}{RGB}{86,90,92}
\definecolor{culightgray}{RGB}{162,164,163}

%%%%% SET COLORS
\setbeamercolor{palette primary}{bg=cugold,fg=cublack}
\setbeamercolor{palette secondary}{bg=culightgray}
\setbeamercolor{palette tertiary}{bg=cublack}
\setbeamercolor{frametitle}{bg=cugold,fg=cublack}
\setbeamercolor{item projected}{fg=cugold,bg=black}
\setbeamercolor{itemize item}{fg=cublack,bg=black}
\setbeamercolor{itemize subitem}{fg=cublack,bg=black}
\setbeamercolor{itemize subsubitem}{fg=cublack,bg=black}

%%%%% List Styling
\setbeamertemplate{itemize items}[circle]
\setbeamertemplate{enumerate items}[circle]

%%%%%%%%%%%%%%%%%%%%%%%%%%%%%%%%%%%%%%%%%%%%%%%%%%%%%%%%%%%%%%%%%%%%%%%%%%%%%%%%
% Document
%%%%%%%%%%%%%%%%%%%%%%%%%%%%%%%%%%%%%%%%%%%%%%%%%%%%%%%%%%%%%%%%%%%%%%%%%%%%%%%%
\begin{document}
\frame{\titlepage}

\logo{%
    \vspace{-0.3cm}\color{cugold}CU Boulder Applied Math
    \includegraphics[height=0.8cm]{./img/appm.png}
}

\frame{%
    \frametitle{High Level Overview}

    \begin{enumerate}
        \item What are Conduits
        \item What are solitons?
        \item What's a soliton gas?
        \item Simulations...
    \end{enumerate}
}

\frame{%
    \frametitle{The Conduit}

    \begin{columns}[T]
        \column{0.6\textwidth}
        \begin{figure}[H]
            \centering
            \includegraphics[scale=1.27]{./img/solitonsmall.png}
            \includegraphics[scale=0.15]{./img/conduit.png}
        \end{figure}
        \column{0.4\textwidth}
        \vspace{2cm}
        \begin{itemize}
            \item Deformable pipe
            \item Gravity is down
            \item Rises because of buoyancy
            \item Cross sectional area $A$
        \end{itemize}
    \end{columns}
}

\frame{%
    \frametitle{Notes on Solitons}
    Our system is governed by the Conduit Equation,
    \begin{align*}
        A_t + \pren{A^2}_z - \pren{A^2 \pren{A^{-1} A_t}_z}_z = 0
    \end{align*}
    \begin{itemize}
        \item Solitons are \textit{solitary travelling waves}.
        \item Solitons are a special solution with decaying boundary conditions
            to the conduit equation of the form
            \begin{align*}
                A(z, t) = f(\zeta) = f(z - ct)
            \end{align*}
        \item Solitons have nonlinear characteristics, most notably their speed
            is determined by their non-dimensionalized amplitude ($a$).
            \begin{align*}
                c = \frac{a^2 - 2a^2 \ln a - 1}{2a -a^2 - 1}
            \end{align*}
    \end{itemize}
}

\frame{%
    \frametitle{Soliton - Soliton Interactions}

    \begin{itemize}
        \item Two solitons can interact if a bigger one chases a smaller one.
        \item The solitons' speed and amplitude are preserved save for a
            phase-shift
    \end{itemize}

    \begin{center}
        \includegraphics[scale=0.2]{./img/2soli1.png}\\
        \includegraphics[scale=0.2]{./img/2soli2.png}\\
        \includegraphics[scale=0.2]{./img/2soli3.png}
    \end{center}
}

\frame{%
    \frametitle{Particle-like Interactions}

    \begin{figure}[H]
        \centering
        \includegraphics[scale=0.37]{./img/soli2.png}
    \end{figure}
}

\frame{%
    \frametitle{What's a Soliton Gas?}

    \begin{itemize}
        \item A soliton can be thought of as a wave, but also as a particle
            (similar to a photon)
        \item A gas can be thought of as a random collection of particles
            interacting
        \item Thus a soliton gas is a random collection of solitons interacting
        \item Our system is one-dimensional, so we are generating a 1D gas
        \item A soliton gas has inherent random behavior dictated by two random
            variables:
            \begin{enumerate}
                \item Frequency of solitons, $Z$
                \item Soliton amplitude, $A$
            \end{enumerate}
    \end{itemize}
}

\frame{%
    \frametitle{Plotting our Gas}

    \begin{figure}[H]
        \centering
        \includegraphics[scale=0.37]{./img/soligas.png}
    \end{figure}
}

\frame{%
    \frametitle{Properties of a Soliton Gas}

    \begin{itemize}
        \item Soliton gas theory developed for simpler (integrable) systems.
        \item Soliton centers and amplitudes $\Rightarrow$ compound Poisson process
        \item This means over long time, frequency of solitons, $Z \sim
            Poisson(\lambda)$, and A is preserved
        \item Poisson Distribution: Number of events in interval with known
            average rate and mutually independent events.
    \end{itemize}
    \vspace{-.8cm}
    \begin{columns}[T]
        \column{0.5\textwidth}
            \begin{figure}[H]
                \centering
                \includegraphics[scale=0.3]{./img/poisson.png}
            \end{figure}
        \column{0.5\textwidth}
            \vspace{2cm}
            \begin{align*}
                f(k; \lambda) = \frac{\lambda^k e^{-\lambda}}{k!}
            \end{align*}
    \end{columns}
    \vspace{-0.25cm}
    {\tiny Kinetic Equation for a Dense Soliton Gas G. A. El and A. M.
        Kamchatnov, PRL 95, 2005}
}

\frame{%
    \frametitle{Numerical Simulations}

    \begin{itemize}
        \item Before running time-consuming experiments, useful to run numerical
            simulations
        \item Spatial discretization: 4th-order finite differences with periodic
            BC's
        \item Temporal discretization: medium-order adaptive Runge-Kutta
            (\textsc{Matlab}'s ode45.m)
    \end{itemize}
}

\frame{%
    \frametitle{Finite Size Effects}

    \textbf{How can we simulate an infinite conduit?}

    \begin{itemize}
        \item Since we have finite size effects, eventually the simulation on
            $[0,L]$ will tend back to initial conditions. We want to stop before
            then.
        \item Therefore we'll run two simulations simultaneously, one on $[0,L]$
            and the other on $[0,2L]$.
        \item At each timestep we'll check for a compound Poisson gas process
            (``gas metric'') of each. If they differ significantly we restart
            with new initial conditions.
    \end{itemize}

    {\tiny D. S. Agafontsev and V. E.  Zakharov, Nonlinearity 28, 2791 (2015)}
}

\frame{%
    \frametitle{Plot of Initial Conditions}

    \begin{figure}[H]
        \centering
        \includegraphics[scale=0.5]{./img/IC.png}
    \end{figure}
}

\frame{%
    \frametitle{Initial Conditions}

    \begin{itemize}
        \item We have two random variables to simulate.
        \item Very first case is easy
        \begin{itemize}
            \item $Z$ is one per minimum distance with exponentially small
                overlap.
            \item $A$ is $Unif\pren{\cren{2,2.5,3,3.5,4,4.5,5,5.5,6}}$
        \end{itemize}
    \item After restart on $2L$ and $4L$, need to create new IC's
    \item Linear superposition \textit{does not} hold
    \item Simulate with same gas metric as ended with.
    \end{itemize}
}

\frame{%
    \frametitle{What is our Gas Metric?}

    \begin{itemize}
        \item We've established that a soliton gas should have
            Poisson-distributed solitons.
        \item This means that we can look at the problem as a Poisson-Point
            Process, i.e. at any given point in space we should see points
            appear over time.
            \begin{figure}[H]
                \centering
                \includegraphics[scale=0.5]{./img/HPP.png}
            \end{figure}
    \end{itemize}
}

\frame{%
    \frametitle{Meaning....}

    \begin{itemize}
        \begin{columns}[T]
            \column{0.4\textwidth}
            \item Since this is a Poisson-Point Process the gap between points
                is exponentially distributed.
                \begin{align*}
                    f(x; \lambda) = \lambda e^{-\lambda x}
                \end{align*}
            \column{0.4\textwidth}
            \begin{figure}[H]
                \centering
                \includegraphics[scale=0.3]{./img/exponential.png}
            \end{figure}
        \end{columns}
        \item Therefore the gas metric is a measure of how close our gaps of our
            solitons are to the exponential distribution.
        \item We use the residual sum squared on the QQ-plot as a metric of
            ``distance'' from one distribution to the other. This value is our
            gas metric.
    \end{itemize}
}

\frame{%
    \frametitle{QQ Plots}

    \begin{columns}[T]
        \column{0.5\textwidth}
            \textbf{Quantiles}

            If you have a given dataset, a quantile divides the dataset into
            equally sized portions.

            \textbf{QQ Plots}

            Plotting quantiles of one distribution vs. quantiles of another.

            \textbf{Residuals}

            Distance from theoretical results to experimental.

            \textbf{Residual Sum Squared}
            \begin{align*}
                RSS = \sum_{i=1}^n \pren{y_i - f(x_i)}^2
            \end{align*}
        \column{0.5\textwidth}
        \vspace{-0.8cm}
            \begin{figure}[H]
                \centering
                \includegraphics[scale=0.25]{./img/quantiles.png}
            \end{figure}
            \vspace{-1cm}
            \begin{figure}[H]
                \centering
                \includegraphics[scale=0.24]{./img/qqplot.png}
            \end{figure}
    \end{columns}
}

\frame{%
    \frametitle{Leveraging Parallelism}

    The big flaw so far is that we're only looking at a single run of the
    simulation. We could easily get bad results from only a single run.

    \vspace{1cm}
    Let's instead consider running a hundred different simulations
    simultaneously, or even a thousand. We have to adjust our simulation to be
    able to handle running in a massively parallel environment such as the CU
    supercomputer, Summit.
}

\frame{%
    \frametitle{Leveraging Parallelism}

    \begin{itemize}
        \item If we want to run many simulations at once, this problem can be
            described as \textit{embarrassingly parallel} since the simulations
            don't need to talk to each other.
        \item So how can we design a multi-threaded program to take into account
            the availability of tens or hundreds of threads?
        \item Can this be written \textit{safely} so we don't have any undefined
            behavior?
    \end{itemize}
}

\frame{%
    \frametitle{Multi-Threaded Design}

    \begin{figure}[H]
        \centering
        \includegraphics[width=\textwidth]{./img/threads.png}
    \end{figure}
}

\frame{%
    \frametitle{Data Storage}

    \textbf{SQLite database}
    (2gb $\to$ 40mb from 2 experiments over 6 hours)
    \begin{columns}[T]
        \column{0.25\textwidth}
            \begin{itemize}
                \item simulations
                \item parameters
                \item t-values
                \item peaks
                \item gas metrics
            \end{itemize}
        \column{0.75\textwidth}
        \inputminted[baselinestretch=1,fontsize=\tiny]{sql}{output.txt}
    \end{columns}
}

\frame{%
    \frametitle{Next Steps}
    \begin{itemize}
        \item Large scale supercomputer simulations
        \item Experiments for validation of simulations
        \item Research paper
    \end{itemize}
}

\frame{%
    \frametitle{References and Acknowledgements}
    \textbf{References}
    {\tiny
    \begin{itemize}
        \item M. D. Maiden, N. K. Lowman, D. V. Anderson, M. E. Schubert, and M.
            A. Hoefer, Observation of dispersive shock waves, solitons, and
            their interactions in viscous fluid conduits, Physical Review
            Letters 116, 174501 (2016).
        \item N. K. Lowman, M. A. Hoefer, and G. A. El, Interactions of large
            amplitude solitary waves in viscous fluid conduits, Journal of Fluid
            Mechanics 750, 372-384 (2014).
        \item D. S. Agafontsev and V. E. Zakharov, Nonlinearity 28, 2791 (2015).
        \item Kinetic Equation for a Dense Soliton Gas G. A. El and A. M.
            Kamchatnov, PRL 95, 2005
    \end{itemize}
    }
    \textbf{Acknowledgements}
    \begin{itemize}
        \item Mark Hoefer
        \item Michelle Maiden
        \item Funded by NSF EXTREEMS-QED
    \end{itemize}

}

\frame{%
    \frametitle{Environment Details}
    \begin{columns}[T]
        \column{0.5\textwidth}
        \textbf{Viscous Fluid Conduits}
        \begin{itemize}
            \item Two viscous fluids, with inner forming axisymmetric conduit.
            \item Exterior Fluid: $\rho^{(e)}$ density and $\mu^{(e)}$ viscosity
            \item Interior Fluid: $\rho^{(i)}$ density and $\mu^{(i)}$ viscosity
            \item $\rho^{(i)} < \rho^{(e)} \Rightarrow$ buoyant flow
            \item $\mu^{(i)} << \mu^{(e)} \Rightarrow$ minimal drag
            \item $\text{Re} << 1 \Rightarrow$ low Reynold's number (implies
                Laminar flow)
        \end{itemize}
        \column{0.5\textwidth}
        \begin{figure}[H]
            \centering
            \includegraphics[scale=0.15]{./img/conduit.png}
        \end{figure}
    \end{columns}
}


\frame{%
    \frametitle{Integrable System: KDV}

    \begin{align*}
        u_t + u u_x + u_{xxx} = 0
    \end{align*}
}
\end{document}
